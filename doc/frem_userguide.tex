\documentclass[a4wide,12pt]{article}
%\setlength{\marginparwidth}{0pt}%35
%\setlength{\marginparsep}{0pt}%?
%\setlength{\evensidemargin}{0pt}
%\setlength{\oddsidemargin}{0pt}
\usepackage{lmodern}
\usepackage[utf8]{inputenc}
\usepackage[T1]{fontenc}
\usepackage{textcomp}
\usepackage{verbatim}
\usepackage{enumitem}
\usepackage{longtable}
\usepackage{alltt}
\usepackage{ifthen}
% Reduce the size of the underscore
\usepackage{relsize}
\renewcommand{\_}{\textscale{.7}{\textunderscore}}

\newcommand{\guidetitle}[1]{
\title{#1\\ \vspace{2 mm} {\large PsN 4.1.1}}
\date{2014-02-10}
}

\newcommand{\doctitle}[1]{
\title{#1}
\date{2014-02-10}
}


\newenvironment{optionlist}{
\renewcommand{\arraystretch}{1.1}
\setlength{\leftmargini}{2.5cm}
\begin{description}
%\setlength{\itemsep}{0ex}
}
{\end{description}}

\newcommand{\optname}[1]{\item{{\bfseries\texttt-#1}\newline}}
\newcommand{\optdefault}[2]{\item{{\bfseries\texttt-#1}{\mbox{ = \it #2}}\newline}}

\newcommand{\nextopt}{}

\usepackage{amsmath}
\guidetitle{FREM userguide}{2018-08-29}
\usepackage{hyperref}
\begin{document}
\newcommand{\guidetoolname}{frem}

\maketitle
\tableofcontents
\newpage

\section{Introduction}
The frem tool builds a full random effects model for covariate model building. \cite{Karlsson}, \cite{Ivaturi}, \cite{Yun}.
Selection of covariates of interest is made without concern regarding their correlation.
Covariates are entered into the data set as observed variables, and their multivariate distribution are modeled
as random effects. A full covariance matrix between random effects for parameters and covariates is estimated
together with the other model components.

It can be difficult to get successful estimation of the full covariance matrix if all frem components are entered directly.
The frem tool builds the full model in a sequence of steps, using estimates from one step as initial estimates for the next.

Example:
\begin{verbatim}
frem run1.mod -covariates=SEX,DGRP,WGT
\end{verbatim}
%frem run1.mod -invariant=SEX,DGRP

\section{Output}
The final\_models subdirectory of the main run directory will contain the final frem models and their output.
The m1 subdirectory of the frem run directory will contain intermediate models and output, and the frem dataset.

\section{Input and options}
\subsection{Required input}
A model file is required on the command line.

\subsection{Optional input}
\begin{optionlist}
\optname{always\_proposal\_density}
Default set. If set then print alternative proposal density to use as input to sir even if Model 4 covstep is successful.
Unset with -no-always\_proposal\_density.
\nextopt
\optdefault{categorical}{list}
Default empty. A comma-separated list of non-ordered categorical covariates. These will be automatically transformed from
one covariate with Y categories into Y-1 bivariate covariates, if Y>2. Missing values are not counted as a category.
The -categorical list must be a subset of the -covariate list. If any covariate listed with option -categorical is not
in the list -covariates there will be an error message. There must be no covariate listed with both -categorical and -log.
\nextopt
\optname{check}
Default not set. Run evaluation with frem records ignored after frem data set generation, to check ofv is the same.
\nextopt
\optname{cholesky}
Default not set. If set the tool will create and run Model 5 and Model 6.
\nextopt
\optdefault{covariates}{list}
A comma-separated list of covariates, required. Must be the names used in \$INPUT, the data file headers are ignored.
%Optional if -time\_varying is given, not allowed if restarting an old run, otherwise required.
\nextopt
\optdefault{deriv2\_nocommon\_maxeta}{number}
Default 60. If the number of ETAs in model 2 is larger than this number then\\
\$ABBREVIATED DERIV2=NOCOMMON will be set in the control stream.
\nextopt
%\optdefault{dv}{name}
%Default is DV. The name of the dependent variable. This must be the name used in \$INPUT,
%the data file header is ignored.
%\nextopt
\optname{estimate\_covariates}
Default not set. By default frem will not estimate the variances and covariances of the covariates until the final frem model estimation,
but instead evaluate the EBEs (in Model 2) assuming the same variances and covariances as in covariate data.
These estimates should not differ unless covariates have missing data (which might warrant this option to be set).
Setting this means that Model 2 is estimated which in some cases can be time-consuming.
\nextopt
\optname{estimate\_means}
Default set. By default frem will estimate the mean values of the covariates which have any
missing observations in the frem dataset. Note that the frem dataset contains covariate
observations only from the first original data observation record for each individual.
If no covariate observations are missing in the frem dataset then the mean will not be estimated,
even if option -estimate\_means is set. If -no-estimate\_means is set, frem will not estimate
any means even if there are missing observations.
\nextopt
%\optdefault{estimate}{number}
%Default 3. The number (0, 1, 2 or 3) of the last model to estimate in the frem sequence. All models with a higher number
%will still be created, but not estimated. When restarting in an existing frem run directory, estimation of models that have an existing
%lst-file will be skipped even if the -estimate option is set to a higher number.
%\nextopt
%\optname{estimate\_regular\_final\_model}
%Default set.
%If set then script will estimate
%the final regular parameterization frem model, otherwise the model will be
%created but not estimated.
%\nextopt
\optname{fork\_runs}
Default not set. Fork out separate processes for all output parsing. Will only work on unix-like systems.
Only relevant if output objects are very, very large.
\nextopt
%\optdefault{invariant}{list}
%\nextopt
\optdefault{log}{list}
Default empty. A comma-separated list of covariates	that should have the natural logarithm of the input data
set value as the observed value in the DV column of the frem dataset. The -log list must be a subset of the
-covariate list. If any covariate listed with option -log is not in the list -covariates there will be an
error message. There must be no covariate listed with both log and -categorical.
\nextopt
\optdefault{mceta}{number}
Default not set. If NONMEM 7.3 or later is used, and the last estimation record accepts MCETA option, set MCETA to this value
in the last \$EST of Model 2 and later.
\nextopt
\optname{mu}
Default not set. If set then use mu-modelling for frem-added covariate THETAs and ETAs. Pre-existing
mu-modelling is not compatible with reordering of skipped omegas, because frem will then renumber ETAs
but not MU variables, so if mu-modelling is used, the user must ensure that omegas are already ordered
so that any skipped omegas come first.
\nextopt
%\optdefault{occasion}{name}
%Default OCC. The name of the column that defines occasions. Name used in \$INPUT.
%Needed if -time\_varying is set, otherwise ignored.
%\nextopt
%\optdefault{ordered}{?}
%{\Large \texttt{TODO special case for ordered categorical}}
%\nextopt
%\optdefault{parameters\_bov}{list}
%A comma-separated list of model parameters that should have between-occasion variability.
%Required if -time\_varying is set, otherwise not allowed.
%\nextopt
\optname{rescale}
Default set. Rescale covariate ETAs in Model 2 to SD close to 1. Unset with -no-rescale.
% in order to facilitate estimation.
\nextopt
%\optdefault{rplots}{level}
%{\Large \texttt{Later}}
%\nextopt
\optdefault{rse}{number}
Default is 30\%.
If the covariance step of both Model 4 and either/both of Model 1 and Model 2 fails, then this is the guess of the
relative standard error that will be used for parameters for which there is no other information that can be used
to guess the variance needed for a proposal density for a sir run with Model 4.
\nextopt
\optname{imp\_covariance}
Default set. If set run Model 4b with updated inits and additional importance sampling, expectation only,
step before covariance step to attain uncertainty (if Model 4 covariance step fails).
\optname{run\_sir}
Default not set. If set and covariance step of Model 4 fails, run sir to obtain standard errors of parameters.
If covariance step of Model 4 is successful, sir will not be run even if option -run\_sir is set.
\nextopt
%\optdefault{skip\_etas}{non-negative integer}
%The number of leading input model ETAs to \emph{not} include when estimating covariances
%between parameter and covariate ETAs. The default is 0, which means all
%input model ETAs will be correlated to covariates.
%The non-included ETAs, if any, must be the first 1 to 'skip\_etas' ETAs in the model.
%If the input model has BOV, the BOV ETAs must be skipped using this option, so if the model has BOV the ETAs must
%be the the first 1 to 'skip\_etas' ETAs. This option is not allowed in combination with skip\_omegas.
%\nextopt
\optdefault{skip\_omegas}{comma-separated list of record numbers}
A comma-separated list of the \$OMEGA records that should be excluded when covariances
between original model ETAs and new covariate ETAs are estimated. Numbering starts at 1.
Numbering goes strictly by the text \$OMEGA in the control stream, without concern
about the type or size of the record. OMEGAs for BOV ETAs will be automatically skipped,
even if option -skip\_omegas was not set.
\nextopt
%\optdefault{time\_varying}{list}
%A comma-separated list of time-varying covariates.
%Names used in \$INPUT. Optional if -invariant is given, not allowed if restarting an old run, otherwise required.
%\nextopt
%\optname{vpc}
%{\Large \texttt{Later}}
%Default not set. If set then script will create a frem model that can be run with the vpc script (in a separate call to the vpc pscript).
%\nextopt
\end{optionlist}

\subsection{PsN common options}
For a complete list see common\_options.pdf or type psn\_options -h on the command line.

\subsection{Restrictions on the input model}
\begin{itemize}
%	\item Time-varying covariates must only vary between occasions, not within occasions.
	\item There must be no parameter called FREMTYPE in \$INPUT/\$PK/\$PRED/\$ERROR.
	\item Categorical covariates must be bivariate, unless listed with option -categorical.
    \item All input model ETAs that are not to have covariance with covariate ETAs estimated must
    belong to a skipped omega record (either explicitly set with option -skip\_omegas or
    automatically skipped since it is a BOV ETA).
    \item There must be only a single \$PROBLEM.
    \item There may be more than one \$ESTIMATION, but frem will only handle output from the last estimation step.
    \item The input model code must not include covariates that are listed with
    -covariates.
%    \item Only if option -vpc is set: All model parameters (for example CL, V, KA) with ETA number > skip\_etas must be defined using \\
 %   TV$<$par$>$= expression possibly based on THETA, and\\
  %  $<$par$>$= expression using TV$<$par$>$ and BSV ETA
%    \item Parameters listed with option -parameters\_bov must be defined in the mode following pattern:
%    \begin{itemize}
%    \item  {\bf BOV ETAs already in input model:}
%    The BOV ETAs must have ETA numbers that are all greater than the last BSV ETA.
%    The corresponding parameter must be defined using \\
%    $<$par$>$= expression using TV$<$par$>$ and BOV ETA
%    \item {\bf BOV not already input model:}
%    For each parameter listed with option -parameters\_bov,
%    the line defining $<$par$>$
%    must include a BSV-expression. Even if there is no
%    BSV for a parameter listed with option -parameters\_bov,
%    i.e. the value is equal to the typical value, the user must still
%    include an expression with 0 in the place of the BSV eta, using the desired form
%    (additive, proportional, exponential,..).
%    This is necessary to enable PsN to automatically add the BOV code. The 0 must be enclosed in parentheses,
%    otherwise the frem program will not recognize it. Example with placeholder for additive inclusion of BOV:\\
%    $<$par$>$=TV$<$par$>$ + (0) \\
%    Example with placeholder for exponential inclusion of BOV:\\
%    $<$par$>$=TV$<$par$>$*EXP(0)
%    \end{itemize}
%	\item TODO: If -vpc is set: Form of TV$<$par$>$ or THETA directly
\end{itemize}

\section{Workflow}
\begin{enumerate}
\item Input checks, for example that DV is found in \$INPUT.
%Skip ETAs with number <= skip\_etas. For all remaining BOV ETAs, if any,
%check which parameter they are on by searching for expressions of the form\\
%$<$par$>$= expression using TV$<$par$>$ and BOV ETA.
%If the parameter found is not listed with option -parameters\_bov there will be an error message.
%\item Check that there are no non-BOV ETAs after the first non-skipped BOV ETA, otherwise give error message.

%\item Check that the first (non-skipped) parameter ETA is the first in a new \$OMEGA record, otherwise give error message.
\item Unless input model is already run,
run a copy of the input model (Model 1) to get estimates.
%\item Count the number of parameter \$OMEGA blocks, N\_parameter\_blocks,
%that are needed in the frem model:
%Initiate N\_parameter\_blocks = 0. Skip all \$OMEGA for ETA <= skip\_etas.
%Then add 1 for each block \$OMEGA and add N for each diagonal \$OMEGA of size N.
%\item Create frem dataset according to section frem dataset below, include covariate observations 'N\_parameter\_blocks' time(s).
%\emph{Alternative:}
\item Create frem dataset according to section frem dataset below, include covariate observations 1 time.

%\item Create Model 1, model without covariates, with full \$OMEGA block for parameter BSV ETAs but without
%and parameter BOV ETAs (pre-existing BOV ETAs removed), according to section below. Original data set.
%\item Run Model 1.
\item Create Model 2, model with covariates but without
estimation of covariances between parameter and covariate ETAs,
according to section below. frem dataset.
THETA and OMEGA and SIGMA for PK parameters are FIXED in this model.
%Include \$COV.
\item Evaluate Model 2 (estimate if option -estimate\_covariates set).
\item Create Model 3.
%where there are N\_parameter\_blocks full \$OMEGA blocks with all covariances between parameter and covariate ETAs,
%according to section below.
Model 3 is a copy of Model 2, but all non-skipped Model 1 omega plus covariate omega are combined
into single large block. For covariances between parameter omegas that did
not have covariance in Model 1, set %very very small init $10^{-7}$.
an initial estimate corresponding to 1\% correlation.
Initial values for covariances between parameter-covariate ETAs are computed from
phi-file from estimation of Model 2. If any such covariance is exactly 0 then set
an initial estimate corresponding to 1\% correlation.
Finally, ensure that omega block is positive definite.
Set MAXEVAL=0 and remove \$COV.
\item Evaluate Model 3 to get more accurate phi-file ETAs.
\item Recompute correlations
between parameter and covariate ETAs based on phi-file from Model 3 evaluation,
use these to create Model 4.

Model 4 is a copy of model 3, but update parameter-covariate covariances from phi-file from evaluation of Model 3
(a rectangular part of \$OMEGA block).
Leave all parameter-parameter and covariate-covariate covariances as is, but
finally check that updated omega block is positive definite.
Set estimation record back to what it was in Model 1.
Set covariance record back to what it was in Model 2.
Unfix everything that was not fix in Model 1.
Model 4 is the (unestimated) regular parameterization frem model.
\item
Only if option -cholesky is set: Create Model 5 by
Cholesky reparameterizing the \$OMEGA block
for parameter-covariate ETAs in Model 4.
Set to 0 FIX the parameter-parameter correlations that are 0 in Model 1 (separate blocks in Model 1).
FIX everything in model except parameter-covariate correlation THETAs.
\item Only if option -cholesky is set: Estimate Model 5, include \$COV step.
\item Only if option -cholesky is set: Create Model 6
by copying Model 5 and updating initial estimates from
estimation of Model 5, and then un-FIX parameters that
were not FIX in input model (Model 1) and correlations that are not exactly 0.
Model 6 is the (unestimated) Cholesky parameterization frem model.
\item Only if option -cholesky is set:
Estimate Model 6, include \$COV step.
\item %Only if option -estimate\_regular\_final\_model is set:
Estimate Model 4 (frem regular model), include \$COV step.
Estimation is done in parallel with Model 6, if estimation of Model 6 was requested.
%\item If option -vpc is set, the script will also
%run the vpc1 model using estimates from frem model as initial estimates,
%and finally process output to generate a frem vpc model.
\item Only if estimation of Model 4 failed to produce an ofv:
\begin{enumerate}
\item Create Model 7, which is a copy of Model 4 but with MAXEVAL=0.
\item Run (Evaluate) Model 7.
\item If evaluation of Model 7 also failed, then stop frem.
\end{enumerate}
\item Create an approximate covariance matrix, see section Proposal density for sir.
This step is performed even if Model 4 covariance step was successful.
\item Only if covariance step of Model 4 was not successful and option -run\_sir is set:
Run sir with -covmat\_input and Model 4 or, if estimation of Model 4 did not produce ofv, Model 7.
\end{enumerate}



\section{Proposal density for sir}
\begin{enumerate}
\item Create an empty covariance matrix with dimensions for Model 4 estimated parameters.
This matrix will hold the joined proposal covariance matrix.
\item If covariance step of Model 1 was run and was successful: Copy all available
elements to joined matrix. Reorder them if Model 1 was reordered because of skipped omegas.
\item If covariance step of Model 2 was successful: Copy all available
elements to joined matrix
\item For each diagonal element of joined matrix that is still not filled and is
the variance of a Model 4 OMEGA element: If a ``perfect individual'' count $\hat{N}_{x,y}$ can
be computed using formula below, then set variance to
\[
var\left(OMEGA(x,y)\right)\approx \frac{1}{\hat{N}_{x,y}}
\left( OMEGA(x,y)^2 + OMEGA(x,x)\cdot OMEGA(y,y) \right)
\]
where OMEGA estimates are from Model 4 (initial values if estimation failed to produce estimates).
\item For remaining diagonal elements of joined matrix that could not yet be filled,
compute a variance guess based on input option -rse and parameter estimate from Model 4.
\end{enumerate}

Perfect individual counts:
\[
\hat{N}_{x}=1+2\left(\frac{OMEGA(x,x)}{se\left(OMEGA(x,x)\right)}\right)^2
\]
where OMEGA(x,x) estimates and se are from covariance step of Model 1 or Model 2, wherever the
OMEGA was estimated.

\[
\hat{N}_{x,y} = minimum(\hat{N}_{x}, \hat{N}_{y})
\] if available standard errors allow both $\hat{N}_{y}$ and $\hat{N}_{x}$ to be computed.
If only $\hat{N}_{x}$ or $\hat{N}_{y}$ available, use the available. If neither available then
cannot use formula, use rse fallback instead.


\section{Model 1}
The input model, input dataset.

\section{frem Dataset}
Create a new data set.
\begin{enumerate}
	\item In input dataset: Find EVID and/or MDV columns. If neither found and the model
    has a \$PK record then MDV will be added via a NONMEM evaluation step.
    Otherwise, i.e. if neither EVID or MDV found and the model
    has a \$PRED record then all data records are considered observations. Find DV column.
    For each covariate  determine a unique %(set of N\_parameter\_blocks)
    FREMTYPE value %(s)
    > 0.
    FREMTYPE=0 is reserved for original observations.
    \item Create a filter model. This is either a dummy NONMEM model or, if MDV has to be added,
    the original model with MAXEVAL=0 METHOD=ZERO and without \$COV.
	\item Run the filter model where FREMTYPE=0 to filter the original data on any IGNORE/ACCEPT and to get a new data file
    from \$TABLE where the column headers are the same as in \$INPUT but a FREMTYPE column is added with only zeros as values.
    Also add MDV if needed. In the dummy model columns that are DROPped or SKIPped are replaced with an undropped column so that the filtered data set has the same number of columns as the original data set.
    \item In filtered data set,
    find each column listed with -categorical and count the number of categories Y for baseline values.
    A warning is printed if there are individuals who have some covariate values that are different from the
    baseline value. Missing values are not counted as a category.

    \item  In new data file, append columns for Y-1 new bivariate covariate variables for each
          categorical covariate where number of categories Y>2. In subsequent analysis steps, replace old covariate with
          Y-1 new variables.
\item Then loop over each individual:
\begin{enumerate}
	\item Copy first observation line for individual. i.e. first data record where MDV/EVID ==0. If no such line then just skip this individual.
Loop over each covariate:
If covariate value for this individual is missing for first observation, then skip this
covariate line for this individual. Otherwise add one % N\_parameter\_blocks
new line %(s)
just before first observation line for individual.
On each copied line, set value in DV column to covariate value, or ln(value) if this covariate is listed with option -log,
and FREMTYPE to a unique type-value for this covariate. % and parameter block.
If MDV column present then MDV=0, else if EVID present then EVID=0.% if non-missing covariate value, otherwise EVID=2.
Store non-missing covariate values in array, one array per covariate, to be able to compute means and %medians and
variance-covariance matrix later to be used as initial estimates in \$THETA and \$OMEGA.
%	\item Loop over each occasion (if OCC defined).
%Copy first line observation line for this occasion.
%Inner loop over each time-varying covariate:
%\begin{enumerate}
%	\item Add new line as first data set line for this occasion.
%    \item Set DV value to covariate value for this occasion.
%    \item Set FREMTYPE to type-value for this covariate.
%    \item If MDV present then MDV=0 if have covariate value, otherwise MDV=1.
%    If EVID present then EVID=0 if have covariate value, otherwise EVID=2.
%    \item Store non-missing time-varying covariate values (to compute median over occasions for this individual,
%    median is one scalar per covariate.)
%    \item[] (end inner loop over time-varying cov)
%\end{enumerate}
%\item[] (end loop over occasions)
%\item Compute median of time-varying cov over occasions for this individual, median is one scalar per covariate.
%\item Then store non-missing medians in array, one array per time-varying covariate, (to compute
%median of medians later for \$THETA, and variance-covariance matrix  to be used in \$OMEGA).
\item[] (end loop over individuals)
\end{enumerate}

	\item Compute means of %medians of %invariant and time-varying
    covariates. Warn if any mean %median
    has absolute value less than $10^{-2}$ since additive error will be inappropriate for that covariate.
    Also compute variance-covariance matrix %of invariant covariates, and of medians (over occasions) for time-varying. Two separate variance-covariance matrices.
    If missing values then just skip.
\end{enumerate}

This dataset is used for Model 2 and on.

\section{Data check model, frem Dataset}
\subsection{Initiation}
Copy Model 1.

\subsection{DATA changes}
Set \$DATA to frem dataset. Skip old IGNORE/ACCEPT. Set IGNORE=@ IGNORE=(FREMTYPE.GT.0)

\subsection{INPUT changes}
Append extra items to \$INPUT: MDV(possibly) and FREMTYPE.

For safety check, done if option -check is set: Estimate Data check model. OFV should be identical to Model 1 OFV.


\section{Model 2}

%For BOV the script defines a variable BOV$<$par$>$ for each parameter, and then BOV$<$par$>$ is set equal to different ETAs depending on the occasion. Same for BOV$<$covariate$>$. See details below.
%This method is safer to implement than having switch variables for occasions that are 1 or 0 depending on the occasion, and then use a long sum of products between switch variables and ETAs.

%For BOV\_par block, use \$OMEGA estimates from Model 0 if it had any BOV ETAs, and use
%phi-file to compute initial values for covariances.

\subsection{Setup}
%Count number n\_theta of already present THETAS in Model 0. Count number bsv\_parameters, which is
%equal to already present ETAs
%(dimensionality of total \$OMEGAs in Model 0) minus (start\_eta-1).
Count number of n\_eps already present (dimension of \$SIGMA in Model 1), set epsnum=n\_eps+1.
Build Model 2 based on copy of Model 1, after updating initial estimates with final estimates
from estimation of Model 1.

\subsection{DATA changes}
Set \$DATA to frem dataset. Skip old IGNORE/ACCEPT. Set IGNORE=@.

\subsection{INPUT changes}
Append extra items to \$INPUT: MDV(sometimes) and FREMTYPE.

\subsection{OMEGA changes}
%For the set of non-skipped parameter ETAs, change any diagonal \$OMEGA of size N>1
%to N \$OMEGA of size 1.
Automatically add BOV \$OMEGAs to skip\_omegas.
Reorder \$OMEGA based on -skip\_omegas, so that all skipped come first, and renumber ETAs.
All non-skipped diagonal \$OMEGA of size $N>1$
are replaced with $N$ \$OMEGA of size 1.
Auto-set skip\_etas, error message if no ETAs remain.

FIX all \$OMEGAs from Model 1.

%After each of the 'N\_parameter\_blocks' \$OMEGA for non-skipped ETAs, insert
Insert
a full BLOCK(N\_covariates) where initials is,
if option -rescale is set (the default), the correlation matrix of covariates computed during
creation of dataset, or if option -rescale is unset (using -no-rescale),
the variance-covariance matrix of covariates computed during creation of dataset.

If any covariate-covariate correlation is exactly 0 in the data set, change the initial
value in the block to 1\% correlation, or the covariance corresponding to 1\% correlation,
to avoid NONMEM error messages.
%Renumber existing parameter BSV ETA:s accordingly in \$PK/\$PRED.

\subsection{SIGMA changes}
FIX all \$SIGMA from Model 1.
Add\\
\$SIGMA 0.000001 FIX; EPSCOV
%only one sigma even if many N\_parameter\_blocks

\subsection{THETA changes}
FIX all \$THETAs from Model 1.
Loop over all covariates, j=1...N\_covariates. %with inner loop over k=1...N\_parameter\_blocks,
If option -estimate\_means was set (the default)
and any frem dataset observations of covariate j are missing add lines\\
\$THETA init\_j ; TVcov\_j\\
%\$THETA init\_j ; TVcov\_j\_k\\
or else , if either no covariate j observation is missing or
-no-estimate\_means was set, add \\
\$THETA init\_j FIX ; TVcov\_j\\
where init\_j is mean of covariate j computed during creation
of new data set, cov\_j is the name of covariate
j. %and k is the order number of the parameter block.
%If mean is exactly zero set init to a very small positive value.
Store the mapping between the theta label and the theta number.%TVcovname – THETA(n\_theta+j).
%TODO: Possibly set boundaries to max and min for covariate, in that case need to find that during generation of dataset 2.
\subsection{PK/PRED changes}
These changes are only made if option -mu is set.
In \$PK or \$PRED, whichever is present:
At the end append the following code: \\
Loop over covariates j=1...N\_covariates. %parameter blocks k=1...N\_parameter\_blocks,
Add lines, if option -rescale is set (the default):
\begin{verbatim}
MU_<orig_etas+j> = THETA(orig_thetas+j)
COV<orig_etas+j> = MU_<orig_etas+j> + ETA(orig_etas+j)*(sd of baseline covariate)
\end{verbatim}
or, if option -rescale is unset,
\begin{verbatim}
MU_<orig_etas+j> = THETA(orig_thetas+j)
COV<orig_etas+j> = MU_<orig_etas+j> + ETA(orig_etas+j)
\end{verbatim}
or,
\subsection{ERROR/PRED changes}
In \$ERROR or \$PRED, whichever is present:
At the end, after -mu changes if applicable, append the following code: \\
Loop over covariates j=1...N\_covariates, %parameter blocks k=1...N\_parameter\_blocks,
add lines\\
Case 1: option -mu is not set and option -rescale is set (the default):\\
\begin{verbatim}
IF (FREMTYPE .EQ. j*100) THEN
; <cov_j> <sd of baseline covariate>
  Y = THETA(orig_thetas+j) + ETA(orig_etas+j)*(sd of baseline covariate) +EPS(epsnum)
  IPRED = THETA(orig_thetas+j) + ETA(orig_etas+j)*(sd of baseline covariate)
END IF
\end{verbatim}

\noindent Case 2: option -mu is not set and option -rescale is unset:\\
\begin{verbatim}
IF (FREMTYPE .EQ. j*100) THEN
; <cov_j> 1
  Y = THETA(orig_thetas+j) + ETA(orig_etas+j) +EPS(epsnum)
  IPRED = THETA(orig_thetas+j) + ETA(orig_etas+j)
END IF
\end{verbatim}


\noindent Case 3: option -mu is set, option -rescale is either set or not set:\\
\begin{verbatim}
IF (FREMTYPE .EQ. j*100) THEN
; <cov_j> <rescaling>
  Y = COV<orig_etas+j> +EPS(epsnum)
  IPRED = COV<orig_etas+j>
END IF
\end{verbatim}
where $rescaling$ is the sd of baseline covariate if option -rescale is set (the default), otherwise $rescaling$ is 1.

%\emph{Alternative:} Do not loop over N \_parameter\_blocks, insert code only once for each covariate

(do not define a case for FREMTYPE=0, original obs. Leave original code as it is for that)


%\noindent BSV\_$<$cov\_j$>$\_k = ETA(orig\_etas+(j-1)*N\_parameter\_blocks +k)\\
%\noindent TV\_$<$cov\_j$>$\_k = THETA(orig\_thetas+(j-1)*N\_parameter\_blocks +k)
%\noindent TV\_$<$cov\_j$>$ = THETA(orig\_thetas+j)\\
%\noindent if option -rescale is \emph{not} set:\\
%\noindent BSV\_$<$cov\_j$>$ = ETA(orig\_etas+j)\\
%\noindent or if option -rescale \emph{is} set:\\
%\noindent BSV\_$<$cov\_j$>$ = ETA(orig\_etas+j)*\emph{sd of baseline covariate}

%\noindent Loop over parameters\_bov, initialize BOV PK (provided there are any time-varying cov)
%\noindent BOV$<$parameter$>$=0
%\noindent Loop over occasions k=1...N\_occ , one IF block per occasion
%\noindent IF (OCC .EQ. k) THEN\\
%(inner loop over parameters j=1$\ldots$bov\_parameters)\\
%BOV$<$parameter$>$ = ETA(offset+N\_invariant+(k-1)*bov\_parameters+j)\\
%END IF
%\noindent Loop over time-varying covariates, initialize BOV tcov
%BOV$<$covariate$>$=0
%\noindent Loop over occasions k=1...N\_occ , one IF block per occasion
%IF (OCC .EQ. k) THEN
%\noindent (inner loop over time-varying covariates j=1...N\_time-var)
%\noindent BOV<covariate> = ETA(offset+N\_invariant+N\_occ*Nparams+(k-1)*N\_time-var+j)\\
%END IF
%\subsection{PK or PRED changes at beginning, B}
%In \$PK or \$PRED (continuing):
%Loop over parameters\_bov (provided there are any time-varying cov)
%Locate line with
%$<$par$>$ = TV$<$par$>$ (expression containing ETA or (0) as BOV placeholder)
%Add BOV$<$parameter$>$ at appropriate place in $<$par$>$=TV$<$par$>$... expression.



%Loop over time-varying covariates, j=(N\_invariant+1)...(N\_invariant+N\_time-var), use THETA-covname mapping, add lines
%\begin{verbatim}
%Y<j> = THETA(cov) + BOV<cov>
%\end{verbatim}
%Loop over covariates j=1...N\_covariates %, parameter blocks k=1...N\_parameter\_blocks
%\begin{verbatim}
%IF (FREMTYPE .EQ.((j-1)*N_parameter_blocks +k)) THEN
%Y=Y_<cov_j>_k+EPS(epsnum)
%IPRED=Y_<cov_j>_k
%END IF
%\end{verbatim}

\subsection{\$ESTIMATION changes}
Only evaluate model (MAXEVALS=0 for classical methods) unless option -estimate\_covariates has been
set. If the last estimation method is classical, set option NONINFETA=1 in the last estimation record.
If option -mceta=N is set with $N>0$ and the NONMEM version is 7.3 or later and
the last estimation record accepts MCETA, then set option MCETA=N in the last estimation record.
\subsection{\$COV changes}
If there was no \$COV record in Model 1, add
\begin{verbatim}
$COVARIANCE PRINT=R UNCONDITIONAL
\end{verbatim}


\section{Models 3-6}

Described above in workflow

%This models is a copy of Model 2, but each pair of input model BSV and covariate blocks is combined
%into one large full block. Initial values for the new covariances are computed from
%phi-file from estimation of Model 2.
%Set MAXEVAL=0.
\section{Workflow in the case of missing data}

\begin{enumerate}
\item Run frem with option -no-run\_sir and -missing\_data set to missing data identifier.
\item After frem has finished, use the EBE:s (from the phi-file of model\_4.mod in the final\_models subdirectory of frem)
as the values for the missing covariates. These values should be entered in a modified copy of the frem\_dataset.
\item Run sir with model\_4 and modified dataset to get uncertainty, perhaps only one iteration is needed
the covariance step of model\_4 was successful.
\item postprocess using separate perl and R code
\end{enumerate}

\section{Post-processing}

\begin{description}
\item[If Model 4 covstep was not successful:]
Run sir
with model\_4.mod from the final\_models subdirectory of the frem run directory
to obtain a raw\_results file with parameter vectors representing the uncertainty of the parameter-covariate coefficients.
Set sir option\\
-covmat\_input=proposal\_density.cov\\
where
the file proposal\_density.cov
is generated by frem and saved in the frem run directory. Then run postfrem using option -sir\_directory.
Example (assuming outside the frem run directory and sir was started inside the frem directory):
\begin{verbatim}
postfrem -frem_dir=frem_base_lin4 -sir_dir=frem_base_lin4/sir_dir1
\end{verbatim}
The script currently assumes that sir was run with input model final\_models/model\_4.mod from the
frem run directory. If sir was run with model\_7.mod or some other model then the script will not work.

\item[If Model 4 covstep was successful:] Run postfrem without option -sir\_directory.
Example (assuming outside the frem run directory)
\begin{verbatim}
postfrem -frem_dir=frem_base_lin4
\end{verbatim}
\end{description}

\noindent Post-processing output is generated in a folder named postfrem1 (or other number).
In the script it is currently assumed that SEX equal to 1 is male and equal to 2 is female.
Units for continuous covariates are hardcoded as
AGE:years, WGT:kg, LBW:kg and CRCO: ml/min.

The post-processing script does the following:
\begin{enumerate}
\item Reads covariate names, rescaling factors and typical values from final\_models/model\_4.mod,
and parameter labels from the last omega block of the model file.
\item Using the formulas in section ``Conditional multi-variate normal'' below,
compute the parameter-eta variances conditional on knowing
each covariate separately and conditional on knowing all covariates together.
Use the sir raw results vectors to compute the 5\% and 95\% quantiles of the respective distributions.
Also output the parameter-eta variances conditional on no covariates.
In addition, compute the coefficients for adjusting parameter-eta means
conditional on knowing the covariates in each situation, and the full expressions to be used in NONMEM code
including the covariate reference values (means) and covariate names.

\item For each parameter do the following:\\
For each covariate in turn, define a reference individual with the value of this single covariate equal to
the reference value. For continous covariates the reference value is the observed mean, and for categorical covariates
it is the most common value.
If the covariate is continuous, determine the 5\% and 95\% quantile of the observed covariate range
and compute the ratios of the parameter value, conditional on a known covariate value equal to the 5\% and 95\% quantile,
to the parameter value conditional on a known value equal to the observed mean.
For categorical covariates, compute the ratio of the parameter value, conditional on a known value equal to the least
common category, to the parameter value conditional on a known value equal to the most common value.

The above computations are repeated for each sir vector, to obtain the 90\% confidence interval around the observed
ratios.

In the current version of the post-processing script exponentiated parameter etas (log-normally distributed parameters)
are assumed.
\item For each parameter,
a reference individual with known values of all covariates is defined, with continuous covariates equal
to their observed mean, and categorical covariates equal to the most common category.
For each individual, compute the ratio of the parameter value, conditional on all covariate values equal to the individual's
values, to the parameter value conditional on all the reference values.

The above computations are repeated for each sir vector, to obtain the 90\% confidence interval around the observed
ratios.

In the current version of the post-processing script exponentiated parameter etas (log-normally distributed parameters)
are assumed.
\item Runs the R scripts for frem results visualization.
\end{enumerate}

\noindent If no uncertainty information could be read postfrem tries to perform the above for only one sample, the mean
value (typical individual), so as to fall back on mere population predictions. If matrices are not positive-definite
(which may be the case if cov step did not succeed in NONMEM), they are forced such through pushing small eigenvalues
above tolerance value. This is necessary for finding the inverse through current Cholesky decomposition implementation.
The impact of this on the results should be small.

If covariance matrix could be read but is not positive-definite, postfrem can't proceed. This can happen when the cov
step of NONMEM does not terminate appropriately (or the S matrix is used in place of the R matrix).
\texttt{-force\_posdef\_covmatrix} can then be supplied to force eigenvalues to become positive (the impact of this is
unclear and a positive-definite covariance matrix from NONMEM is recommended).

%The standard deviation of $x=TV_{par}\cdot e^{\eta}$, where $TV_{par}$ is constant and $\eta$ is normally distributed
%with mean $\mu$ and variance $\sigma^2$, is
%\[
%sd[x] = TV_{par}\cdot e^{\mu+0.5\sigma^2}\cdot \sqrt{e^{\sigma^2}-1}
%\]
%, and the $CV$ in \% is
%\[
%CV_{x}=\frac{sd[x]}{TV_{par}}= e^{\mu+0.5\sigma^2}\cdot\sqrt{e^{\sigma^2}-1}\cdot 100\%
%\]

%We have $\mu=0$.

\subsection*{Conditional multi-variate normal}

We have a multivariate normally distributed vector
\[
\vec{\eta} = \left(
\begin{array}{c}
\vec{\eta}_{par} \\
\vec{\eta}_{cov}\\
\end{array}
\right)
\]
with sizes
\[
\left(
\begin{array}{c}
n_{par} \times 1 \\
n_{cov} \times 1\\
\end{array}
\right)
\]
where $n_{par}$ is the number of parameters and $n_{cov}$ is the number of covariates related to those parameters.

The partitioned mean vector is
\[
\vec{\mu} = \left(
\begin{array}{c}
\vec{\mu}_{par} \\
\vec{\mu}_{cov}\\
\end{array}
\right)
\]
and the partitioned covariance matrix is
\[
\Sigma=\left(
\begin{array}{cc}
\Sigma_{par,par} & \Sigma_{par,cov}\\
\Sigma_{cov,par} & \Sigma_{cov,cov}\\
\end{array}
\right)
\]
with sizes
\[
\left(
\begin{array}{cc}
n_{par} \times n_{par} & n_{par} \times n_{cov}\\
n_{cov} \times n_{par} & n_{cov} \times n_{cov}\\
\end{array}
\right)
\]

In frem $\vec{\eta}_{cov}$ are the deviations of observed values of the covariates from the observed means.

The distribution of $\vec{\eta}_{par}$ conditional on $\vec{\eta}_{cov}= \vec{a}_{cov}$,
where $\vec{a}_{cov}$ is known,
is
multivariate normal with mean
\[
\tilde{\mu}_{par}=\vec{\mu}_{par}+\Sigma_{par,cov}\Sigma_{cov,cov}^{-1}\left(\vec{a}_{cov}-\vec{\mu}_{cov}\right)
\]
and covariance matrix
\[
\tilde{\Sigma}_{par,par}=\Sigma_{par,par}-\Sigma_{par,cov}\Sigma_{cov,cov}^{-1}\Sigma_{cov,par}
\]

If conditioning on a single know covariate value, then the same formulas are used, but with $\vec{\eta}_{cov}$ a scalar
and $n_{cov}=1$.


\section{Known issues}
\$OMEGA records that are set to 0 FIX must be listed in -skip\_omegas. If 0 FIX \$OMEGA are not skipped then
frem will fail.

Setting option -cholesky will often fail with NONMEM versions older than NM7.3, since the Fortran 95 code line continuation
\& is not recognized by nmtran.

The frem program does not support \$MSFI in the input model, since that is incompatible
with changing/adding \$OMEGA and \$THETA records.

%It can happen that final estimate OMEGA blocks in Model 3 are not positive definite due to rounding errors. Then the scripts
%will halt with an error messages. In that case the user should remove all vpc\_model\_1 output files from
%the m1 directory, open model\_3.ext and carefully add a small value to the diagonal elements of the problematic omega block,
%save the edited model\_3.ext in m1, and restart frem. The program will then read the omega block from model\_3.ext
%without running model 3 and then use that block instead.


\references

\end{document}

\section{Computing SE of correlations}
We use the so-called delta method for computing the uncertainty of correlations based on uncertainty of covariances.
% see output get_eta_eps_correlations

\[
\begin{split}
\hat{corr_{A,B}}&=g(\Phi)\\
&=g\left(\phi_1,\phi_2,\phi_3\right)\\
&=g\left(\hat{cov_{A,B}},\hat{var_{A}},\hat{var_{B}}\right)\\
&=\frac{\hat{cov_{A,B}}}{\sqrt{\hat{var_{A}}}\cdot \sqrt{\hat{var_{B}}}}\\
&=\hat{cov_{A,B}}\cdot \hat{var_{A}}^{-0.5} \cdot \hat{var_{B}}^{-0.5}
\end{split}
\]

\[
\begin{split}
var\left(\hat{corr_{A,B}}\right)&=\sum_{i=1}^{3}\sum_{j=1}^{3} \frac{\partial g(\Phi)}{\partial \phi_{i}} \cdot \frac{\partial g(\Phi)}{\partial \phi_{j}}\cdot cov\left(\phi_{i},\phi_{j} \right)
\end{split}
\]

\[
\begin{split}
\frac{\partial g(\Phi)}{\partial \phi_1}&=\frac{\partial g(\Phi)}{\partial \hat{cov_{A,B}}} =\hat{var_{A}}^{-0.5} \cdot \hat{var_{B}}^{-0.5}\\
\frac{\partial g(\Phi)}{\partial \phi_2}&=\frac{\partial g(\Phi)}{\partial \hat{var_{A}}} = -0.5 \cdot \hat{cov_{A,B}}\cdot \hat{var_{A}}^{-1.5} \cdot \hat{var_{B}}^{-0.5}\\
\frac{\partial g(\Phi)}{\partial \phi_3}&=\frac{\partial g(\Phi)}{\partial \hat{var_{B}}} = -0.5 \cdot \hat{cov_{A,B}}\cdot \hat{var_{A}}^{-0.5} \cdot \hat{var_{B}}^{-1.5}
\end{split}
\]

\[
\begin{split}
&var\left(\hat{corr_{A,B}}\right)=\\
=&\hat{var_{A}}^{-1} \cdot \hat{var_{B}}^{-1}\cdot var\left(\hat{cov_{A,B}} \right)+\\
& 0.25 \cdot \hat{cov_{A,B}}^2\cdot \hat{var_{A}}^{-3} \cdot \hat{var_{B}}^{-1}\cdot var\left(\hat{var_{A}} \right)+\\
& 0.25 \cdot \hat{cov_{A,B}}^2\cdot \hat{var_{A}}^{-1} \cdot \hat{var_{B}}^{-3}\cdot var\left(\hat{var_{B}} \right)+\\
& - \hat{var_{A}}^{-0.5} \cdot \hat{var_{B}}^{-0.5} \cdot \hat{cov_{A,B}}\cdot \hat{var_{A}}^{-1.5} \cdot \hat{var_{B}}^{-0.5} \cdot cov\left(\hat{cov_{A,B}},\hat{var_{A}}\right)+\\
& - \hat{var_{A}}^{-0.5} \cdot \hat{var_{B}}^{-0.5} \cdot \hat{cov_{A,B}}\cdot \hat{var_{A}}^{-0.5} \cdot \hat{var_{B}}^{-1.5} \cdot cov\left(\hat{cov_{A,B}},\hat{var_{B}}\right)+\\
& 0.5 \cdot \hat{cov_{A,B}}\cdot \hat{var_{A}}^{-1.5} \cdot \hat{var_{B}}^{-0.5}\cdot \hat{cov_{A,B}}\cdot \hat{var_{A}}^{-0.5} \cdot \hat{var_{B}}^{-1.5}\cdot cov\left(\hat{var_{A}},\hat{var_{B}}\right)\\
=&\hat{var_{A}}^{-1} \cdot \hat{var_{B}}^{-1}\cdot var\left(\hat{cov_{A,B}} \right)+\\
& 0.25 \cdot \hat{cov_{A,B}}^2\cdot \hat{var_{A}}^{-3} \cdot \hat{var_{B}}^{-1}\cdot var\left(\hat{var_{A}} \right)+\\
& 0.25 \cdot \hat{cov_{A,B}}^2\cdot \hat{var_{A}}^{-1} \cdot \hat{var_{B}}^{-3}\cdot var\left(\hat{var_{B}} \right)+\\
& - \hat{cov_{A,B}} \cdot \hat{var_{A}}^{-2} \cdot \hat{var_{B}}^{-1} \cdot cov\left(\hat{cov_{A,B}},\hat{var_{A}}\right)+\\
& - \hat{cov_{A,B}} \cdot \hat{var_{A}}^{-1} \cdot \hat{var_{B}}^{-2} \cdot cov\left(\hat{cov_{A,B}},\hat{var_{B}}\right)+\\
& 0.5 \cdot \hat{cov_{A,B}}^2\cdot \hat{var_{A}}^{-2} \cdot \hat{var_{B}}^{-2}\cdot cov\left(\hat{var_{A}},\hat{var_{B}}\right)\\
=&\hat{var_{A}}^{-1} \cdot \hat{var_{B}}^{-1}\cdot var\left(\hat{cov_{A,B}} \right)+0.5 \cdot \hat{cov_{A,B}}^2\cdot \hat{var_{A}}^{-1} \cdot \hat{var_{B}}^{-1}\cdot \\
&\left(0.5 \left ( \hat{var_{A}}^{-2}\cdot var\left(\hat{var_{A}} \right)+\hat{var_{B}}^{-2}\cdot var\left(\hat{var_{B}} \right)    \right) +  \hat{var_{A}}^{-1} \cdot \hat{var_{B}}^{-1}\cdot cov\left(\hat{var_{A}},\hat{var_{B}}\right) \right)+\\
& - \hat{cov_{A,B}} \cdot \hat{var_{A}}^{-1} \cdot \hat{var_{B}}^{-1}\left(\hat{var_{A}}^{-1} \cdot cov\left(\hat{cov_{A,B}},\hat{var_{A}}\right)+\hat{var_{B}}^{-1}\cdot cov\left(\hat{cov_{A,B}},\hat{var_{B}}\right) \right)
\end{split}
\]

\[
\begin{split}
&var\left(corr_{ETA-A,ETA-B}\right)=\\
& OM_{(A,A)}^{-1} \cdot OM_{(B,B)}^{-1}\cdot var_{OM(A,B)}+0.5 \cdot OM_{(A,B)}^2\cdot OM_{(A,A)}^{-1} \cdot OM_{(B,B)}^{-1}\cdot \\
&\left(0.5 \left ( OM_{(A,A)}^{-2}\cdot var_{OM(A,A)}+OM_{(B,B)}^{-2}\cdot var_{OM(B,B)}    \right) +  OM_{(A,A)}^{-1} \cdot OM_{(B,B)}^{-1}\cdot cov_{OM(A,A),OM(B,B)} \right)+\\
& - OM_{(A,B)} \cdot OM_{(A,A)}^{-1} \cdot OM_{(B,B)}^{-1}\left(OM_{(A,A)}^{-1} \cdot cov_{OM(A,B),OM(A,A)}+OM_{(B,B)}^{-1}\cdot cov_{OM(A,B),OM(B,B)} \right) =\\
& OM_{(A,A)}^{-1} \cdot OM_{(B,B)}^{-1}\cdot [ \\
& var_{OM(A,B)}+0.5 \cdot OM_{(A,B)}^2\cdot \\
&\left(0.5 \left ( OM_{(A,A)}^{-2}\cdot var_{OM(A,A)}+OM_{(B,B)}^{-2}\cdot var_{OM(B,B)}    \right) +  OM_{(A,A)}^{-1} \cdot OM_{(B,B)}^{-1}\cdot cov_{OM(A,A),OM(B,B)} \right)+\\
& - OM_{(A,B)} \left(OM_{(A,A)}^{-1} \cdot cov_{OM(A,B),OM(A,A)}+OM_{(B,B)}^{-1}\cdot cov_{OM(A,B),OM(B,B)} \right) ]=\\
& \frac{1}{OM_{(A,A)}\cdot OM_{(B,B)}}\cdot [ var_{OM(A,B)} +\\
& +0.5 \cdot OM_{(A,B)}^2\cdot
\left(0.5 \left ( \frac{var_{OM(A,A)}}{OM_{(A,A)}^{2}}  +\frac{var_{OM(B,B)}}{OM_{(B,B)}^{2}}    \right) +  \frac{cov_{OM(A,A),OM(B,B)}}{OM_{(A,A)} \cdot OM_{(B,B)}} \right)+\\
& - OM_{(A,B)} \left(\frac{cov_{OM(A,B),OM(A,A)}}{OM_{(A,A)}}+ \frac{cov_{OM(A,B),OM(B,B)}}{OM_{(B,B)}} \right) ]\\
\end{split}
\]

\[
\begin{split}
var\left(corr_{ETA-A,ETA-B}\right)&=\frac{1}{OM_{(A,A)}\cdot OM_{(B,B)}}\cdot \{\\
 var_{OM(A,B)} + OM_{(A,B)} [ & \frac{OM_{(A,B)}}{2}\cdot
\left[0.5 \left ( \frac{var_{OM(A,A)}}{OM_{(A,A)}^{2}}  +\frac{var_{OM(B,B)}}{OM_{(B,B)}^{2}}    \right) +  \frac{cov_{OM(A,A),OM(B,B)}}{OM_{(A,A)} \cdot OM_{(B,B)}} \right]\\
& - \left(\frac{cov_{OM(A,B),OM(A,A)}}{OM_{(A,A)}}+ \frac{cov_{OM(A,B),OM(B,B)}}{OM_{(B,B)}} \right)]\}\\
\end{split}
\]


\section{Creating frem vpc model}
Needs to be revised.
Done if option -vpc is given to the frem tool. This part of the recipe describes modifications needed to the frem model and data set before running vpc. Definitions of model types (Model 2 and frem Model) and names of \$OMEGA blocks are as in description above.

\subsection{Step 1: Obtain typical parameter values conditional on covariate info}
Create frem\_vpc1.mod

\begin{enumerate}
	\item Take the frem Model. Update inits from estimation, if available. Afterwards FIX \$THETA and \$OMEGA. Unfix all \$SIGMA.
	\item Set IGNORE=@ IGNORE=(FREMTYPE.GT.0)
	\item Remove \$COV.
	\item In \$PK or \$PRED, whichever is present:
Figure out which parameters that have BSV %and/or BOV eta(s)
on them.
%BOV ETAs may not have been present in Model 0, but added in Model 2.
%Parameters may have BOV ETA without BSV eta, and vice versa.
%First take all parameters listed with option -parameters\_bov, since they have had BOV ETAs added. Then find additional
Find parameters
that have BSV ETA %already
in model 1.
To do this: Find all code lines starting with TV$<$par$>=$ in Model 1. Then find the lines starting with the corresponding $<$par$>=$ and
check if there is any ETA in the right hand side expression. If yes, then store the parameter name $<$par$>$.
% If not find THETA number then raise error.
	\item Add \$TABLE with list of all undropped \$INPUT variables plus names of parameters that have BSV eta  + ONEHEADER NOAPPEND NOPRINT FILE=frem\_vpc.dta
	\item Store sequence of headers in \$TAB to be used in \$INPUT in model frem\_vpc2.mod, but in this list prepend parameter names with CTV to separate from other params.
	\item Run model with NM.
\end{enumerate}

\subsection{Step 3:  Calculate parameter OMEGA conditional on covariate OMEGA}
Create frem\_vpc2.mod
\begin{enumerate}
	\item Base this model on Model 1. %1.
    \item Make the same SIGMA changes as in Model 2.
%    \item Make similar OMEGA changes as in Model 2, but no not add any BSV\_cov block or BOV\_cov\_occN blocks.
    \item Do not add any THETAs.
    \item Copy THETA, SIGMA estimates and 1-skip\_etas leading OMEGA estimates from vpc1.
	\item DATA is frem\_vpc.dta
	\item Set IGNORE=@
 %   \item Make the ``PK or PRED changes at beginning, A'' as in Model 2 except skip all lines involving BSV<cov> or
 %   BOV<cov>, and renumber remaining ETAs accordingly.
 %   \item Make the same ``PK or PRED changes at beginning, B'' as in Model 2.
	\item Set \$INPUT to sequence of headers from step 1 (contents of table file frem\_vpc.dta but prepend CTV for params)
	\item Use parameter list from Step 1 to set TV$<$par$>$ = CTV$<$par$>$ for
parameters that have BSV ETA %and/or BOV ETA
on them.
	\item Read final estimated BSV\_all omega blocks from lst-file and compute conditional omega blocks,
    see Appendix “Compute conditional omega block” below. In \$OMEGA replace BSV\_par blocks with
    conditional omega blocks from computation.
%	\item Read final estimated BOV\_full\_occ1 omega block from lst-file from Step 1 compute conditional omega block
%    in the same way as for BSV. In \$OMEGA replace BOV\_par\_occ1 with conditional block.
    \item Remove any \$COV
\end{enumerate}

\subsection{Step 4: Prepare to run  vpc}
Now a regular vpc can be run on frem\_vpc2.mod found in the frem run directory.


\section{Appendix: Compute conditional omega block}
To calculate the conditional covariance matrix,
one inverts the overall covariance matrix,
drops the rows and columns corresponding to the variables being conditioned upon, and then inverts back to get the conditional covariance matrix.
% Should use generalized inverse, but we skip that for now
\begin{math}
\Omega^{-1}\Omega = I
\end{math}

To avoid inverting a full matrix, which is numerically wasteful in terms of time and precision, we instead want to perform a sequence of numerically more stable steps.
We have a symmetric $n\times n$ variance-covariance matrix $\Omega$, and want to obtain the leading upper $k\times k$ submatrix $\Omega^{-1}_{11}$
of $\Omega^{-1}$, where
\[
\Omega^{-1} = \left( \begin{array}{cc}
\Omega^{-1}_{11} & \Omega^{-1}_{12}\\
\Omega^{-1}_{21} & \Omega^{-1}_{22}
\end{array} \right)
\] with sizes \[
\left(
\begin{array}{cc}
k\times k & k\times (n-k)\\
(n-k)\times k & (n-k)\times (n-k)
\end{array}
\right)
\]
\begin{enumerate}
\item Make a Cholesky decomposition of omega, $\Omega=GG^T$
\item Obtain the inverse of $\Omega$ by solving the triangular system $G^{T}G^{-T}=I$ for just the $k$ first cols and using $\Omega^{-1}=G^{-T}G^{-1}$
\item We want the upper $k\times k$ submatrix of $\Omega^{-1}=G^{-T}G^{-1}$.
\end{enumerate}
%Ok to run on block from BOV omega?

\references

\end{document}

\section{Model 1}
This filling in of \$OMEGA is only done if there are invariant covariates listed with the -invariate option.

Copy the input model and change \$OMEGA for skip\_etas+1 to last\_bsv\_eta
from whatever form in Model 0 to a single large BLOCK. If any BOV ETA included in input model,
then replace those ETAs in model code with zeros, and remove corresponding \$OMEGA.
Update initial estimates with final estimates from Model 0.
Initial off-diagonal \$OMEGA elements that are new to Model 1 are set using
correlations computed from the ETA columns in the phi-file of Model 0, and estimated variances.
The large \$OMEGA block of Model 1 is called the BSV\_par block.

Then, if have time-varying covariates:
\begin{itemize}
\item BOV\_par\_occ1: Add one full BLOCK(bov\_parameters) where initials is: diagonal 0.01 and off-diagonals small enough to make
block positive definite.
At each line, possibly add comment with BOV$<$parameter$>$ to be used as label for ETA in PsN
%could in filtering print individual parameters and compute variance??? Use .cov from previous step??? probably not
\item BOV\_par\_occ2-end: Then, for each occasion that is not the first occasion add BLOCK SAME.
\item BOV\_cov\_occ1: Add one full BLOCK(N\_time-var) where initials is variance-covariance matrix of time-varying covariates computed during creation of dataset
At each line, possibly add comment with covariate name to be used as label for ETA in PsN
\item BOV\_cov\_occ2-end: Then, for each occasion that is not the first occasion add BLOCK SAME.
\end{itemize}

\section{Model 2-only time-varying}
Use the frem dataset but in \$DATA set IGNORE for all FREMTYPE for invariant covariates. Build model based on Model 0 (no filling in of BSV block). Add code and parameters according to Model 2 instructions above but set the number of invariant covariates N\_invariant to 0

\section{Model 2-only time-invariant}
Use the frem dataset but in \$DATA set IGNORE for all FREMTYPE for time-varying covariates. Build model based on Model 0 (no filling in of BSV block). Add code and parameters according to Model 2 instructions above but set the number of time-varying covariates N\_invariant to 0.

\section{Model 3}
\subsection{SETUP}
Same as for Model 2 (build on Model 0 or 1)

\subsection{DATA changes}
Same as Model 2.
\subsection{INPUT changes}
Same as Model 2.

\subsection{OMEGA changes}
BSV\_all: One full BLOCK(bsv\_parameters+N\_invariant).
%Initials: EITHER from eta estimates from Model 2 (from .phi-file), compute variance-covariance matrix. OR
Same initials as in Model 2 plus small values to fill in full block. When Model 2 is estimated
the final estimates can be used to set initials in the frem Model.

Then, provided have time-varying covariates:
BOV\_all\_occ1: Add one full BLOCK(bov\_parameters+N\_time-var) where bov\_parameters is number of parameters set with -parameters\_bov option.
%Initials: EITHER from eta estimates from Model 2, compute variance-covariance matrix (.phi-file, must rearrange ETAs to new order in Model 3). OR
Same initials as in Model 2 plus small values to fill in full BLOCK.
BOV\_all\_occ2-end: Then, for each occasion that is not the first occasion add \$OMEGA BLOCK SAME.

\subsection{SIGMA changes}
Same as Model 2.

\subsection{THETA changes}
Same as Model 2.

\subsection{PK or PRED changes at beginning A}
Very similar to Model 2, but different arrangement of loops so that ETA numbering is different.
In $PK or $PRED, whichever is present:
At the very beginning (unless have FREM-ANCHOR analogue to scm- anchor) the following code:
Loop over invariant covariates j=1...N\_inv, add lines
\begin{verbatim}
BSV<cov> = ETA(offset+j)
\end{verbatim}
Loop over parameters\_bov, initialize BOV PK (provided there are any time-varying cov)
\begin{verbatim}
BOV<parameter>=0
\end{verbatim}
Loop over time-varying covariates, initialize BOV tcov
\begin{verbatim}
BOV<covariate>=0
\end{verbatim}
Loop over occasions k=1...N\_occ , one IF block per occasion
\begin{verbatim}
IF (OCC .EQ. k) THEN
\end{verbatim}
[First inner loop, over parameters j=1...bov\_parameters]
\begin{verbatim}
BOV<parameter> = ETA(offset+N_invariant+(k-1)*(bov_parameters+N_timevar)+j)
\end{verbatim}
[Second inner loop (still inside IF BLOCK, but outside first inner loop), over time-varying covariates j=1...N\_variant]
\begin{verbatim}
BOV<covariate> = ETA(offset+N_invariant+(k-1)*(bov_parameters+N_timevar)+bov_parameters+j)
END IF
\end{verbatim}

\subsection{PRED or ERROR changes}
Same as Model 2.

\section{Initial values off-diagonal OMEGA}

Change prog to allow existing BOV etas? Only require that sorted, i.e. 'neither' then BSV then BOV.


Manuscript says ``In FREM due to nature of parameterization via var-cov matrix the parameter-covariate
relationship
takes an exponential form``. Is that not assuming the parameter, e.g. CL, has exponential BSV eta to begin with?

Definitions:
Base frem model: Two omega blocks: One full omega block for model parameter etas and then separate full block for
covariate etas.

Full frem model: Single full omega block for all etas.

We have difficulties:
1) Obtaining good Initial estimates for OMEGA in full frem model based on output from base frem model,
to get successful estimation.

2) Get successful cov step for full frem model after estimating full frem model.
Is Jonas formula for this issue, i.e. getting covariances without running covstep, assuming we have
successful cov step for base frem model, but only estimates for full ferm model?

%variance (\hat Cov(X,Y)) = q(n)/n * (Cov(X,Y)^2 + Var(X) Var(Y))
%Jonas  q(n) = 1 if ML-estiamtes for \hat Cov(X,Y), and converge to 1 when n-> 1 for other estimators.
% N=2\frac{var^2}{SE^2}+1  where var is variance est from NONMEM, SE is SE of variance from NONMEM

Coefficient is computed as
\[
\frac{cov_{par,cov}}{var_{cov}}=\frac{sd_{par}sd_{cov}corr_{par,cov}}{sd_{cov}sd_{cov}}=\frac{sd_{par}}{sd_{cov}}corr_{par,cov}
\]



  cholFullA=chol(FullA);
invGT=  inv(cholFullA'); %compute only leading k block
[q,R]=qr(invGT(:,1:k));
invR=inv(R(1:k,1:k));
result2=invR*invR'

\subsection{Rescaling of ETAs}
Only if option -rescale is set (the default):
For each non-skipped \$OMEGA and new \$OMEGA for covariates:
\begin{enumerate}
\item Find the ETA numbers for this \$OMEGA.
\item Compute the initial values for the standard deviations (SD) as the square root of the diagonal
element(s) of this \$OMEGA.
\item Add a new THETA FIX for each SD. Set label ASD\_(label for diagonal omega)
\item In \$PK or \$PRED, whichever is present:
At the very beginning (unless have FREM-ANCHOR similar to scm- anchor) add the following code: \\
\begin{verbatim}
ASD_ETA_N = THETA(k)
\end{verbatim}
where ASD stands for approximate SD, N is the ETA number, and k is the THETA number.
\item Divide \$OMEGA row and column initial values with for this ETA with the SD
\item In all code records (PK, PRED, ERROR...) replace all instances of ETA(N) with (ETA(N)*ASD\_ETA\_N)
\end{enumerate}
