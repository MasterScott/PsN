\documentclass[a4wide,12pt]{article}
%\setlength{\marginparwidth}{0pt}%35
%\setlength{\marginparsep}{0pt}%?
%\setlength{\evensidemargin}{0pt}
%\setlength{\oddsidemargin}{0pt}
\usepackage{lmodern}
\usepackage[utf8]{inputenc}
\usepackage[T1]{fontenc}
\usepackage{textcomp}
\usepackage{verbatim}
\usepackage{enumitem}
\usepackage{longtable}
\usepackage{alltt}
\usepackage{ifthen}
% Reduce the size of the underscore
\usepackage{relsize}
\renewcommand{\_}{\textscale{.7}{\textunderscore}}

\newcommand{\guidetitle}[1]{
\title{#1\\ \vspace{2 mm} {\large PsN 4.1.1}}
\date{2014-02-10}
}

\newcommand{\doctitle}[1]{
\title{#1}
\date{2014-02-10}
}


\newenvironment{optionlist}{
\renewcommand{\arraystretch}{1.1}
\setlength{\leftmargini}{2.5cm}
\begin{description}
%\setlength{\itemsep}{0ex}
}
{\end{description}}

\newcommand{\optname}[1]{\item{{\bfseries\texttt-#1}\newline}}
\newcommand{\optdefault}[2]{\item{{\bfseries\texttt-#1}{\mbox{ = \it #2}}\newline}}

\newcommand{\nextopt}{}

\guidetitle{EXECUTE user guide}{2015-02-13}


\begin{document}

\maketitle
\newcommand{\guidetoolname}{execute}


\section{Introduction}
The execute script is a PsN tool that allows you to run multiple modelfiles either sequentially or in parallel. It is an nmfe replacement with advanced extra
functionality.

Execute creates subdirectories where it puts NONMEMs input and output files, to make sure that parallel NONMEM runs do not interfere with each other.
The top directory is by default named 'modelfit\_dirX' where 'X' is a number that starts at 1 and is increased by one each time you run execute.
Example
\begin{verbatim}
execute -threads=2 -retries=5 run1.mod 
\end{verbatim}


\section{Input and options}
\subsection{Required input}
A model file is required on the command-line. 

\subsection{Optional input}
All options listed in document common\_options\_defaults\_versions\_psn.pdf apply to execute. 
Those options govern how the NONMEM runs are managed, and hence apply to all PsN scripts.
Since execute does not do much except running a model with NONMEM, 
they have a very short list of options that are unique to them:

\begin{optionlist}
\optname{model\_dir\_name}
Default not used. This option changes the basename of the run directory from modelfit\_dir to $\langle$modelfile$\rangle$.dir. where $\langle$modelfile$\rangle$ 
is the name of the (first) input model file, without the extension. 
The directories will be numbered starting from 1, increasing the number each time execute is run with a model file with the 
same name. If the option directory is used it will override -model\_dir\_name.
\nextopt
\optname{timestamp}
Default not used. This option changes the name of the run directory to $\langle$modelfile$\rangle$-PsN-$\langle$date$\rangle$
where $\langle$modelfile$\rangle$ is the name of the first model file in the list given as arguments, without the extension,
and $\langle$date$\rangle$ is the time and date the run was started. 
Example: directory name run1-PsN-2014-06-12-152502 for a run that was started at 15:25:02 June 12th in year 2014.
If the option directory is used it will override -timestamp.
\nextopt
\optname{prepend\_options\_to\_lst}
Default not used. This option makes PsN prepend the final lst-file (which is copied back to the directory from which execute was called) with the file version\_and\_option\_info.txt which contains run information, including     all actual values of optional PsN options. PsN can still parse the lst-file with the options prepended, so the file can still be used it as input to e.g. sumo, vpc or update\_inits. Disabled with -no-prepend\_options\_to\_lst if set in psn.conf.
\nextopt
\optname{tail\_output}
Default not used. This option only works for execute under Windows. Option -tail\_output specifies that execute should invoke a program (tail) that displays the output file, including the gradients, during minmization. The tail program is started automatically but it is up to the user to terminate the program. For the tail\_output option to work, a third party tail program must be installed. Tail programs that are known to work are WinTail and Tail for Win32.  The latter is recommended and can be downloaded at http://tailforwin32.sourceforge.net. It is also necessary to have correct settings of the variables wintail\_exe, which is the path to the tail program, and wintail\_command, which is the command for the tail program. The defaults, which work for the Tail for Win32 package, are seen below. These two variables can be set to other values in psn.conf.
\nextopt
\optdefault{wintail\_exe}{string}
Default \verb|C:\Program Files\Tail-4.2.12\Tail.exe|. Only for Windows. See tail\_output for description.
\nextopt
\optdefault{wintail\_command}{string}
Default "tail OUTPUT”. Only for Windows. See -tail\_output for description.
\nextopt
\optname{copy\_data}
Default set. Disable with -no-copy\_data. By default, PsN will copy the datafile into NM\_run1 and set a local path in psn.mod, the actual modelfile run with NONMEM. If -no-copy\_data is set, PsN will not copy the data to NM\_run1 and instead set a global path to the datafile in psn.mod.
\nextopt
\end{optionlist}

\subsection{Auto-generated R-plots from PsN}
\newcommand{\rplotsconditions}{The default execute template 
requires the xpose4 R library. It also relies on Xpose-type tables being 
created, such as sdtab, patab and cotab, with the correct run number. 
See the Xpose documentation for requirements on such table files.
If option -subset\_variable\_rplots is used, 
the user must ensure that the subset variable
is printed to one of the xpose tables, for example sdtab, and then
there will be separate plots created for
subsets of the data, via xpose options 'subset' and 'by'. 
If the conditions are not fulfilled then no pdf will be generated,
see the .Rout file in the main run directory for error messages.}
PsN can automatically generate R plots to visualize results for \guidetoolname, using a default template found in the R-scripts subdirectory of the installation directory. The user can also create a custom template, see more details in the section Auto-generated R-plots from PsN in common\_options.pdf.

\rplotsconditions

\begin{optionlist}
\optdefault{rplots}{level}
-rplots<0 means R script is not generated\\ 
-rplots=0 (default) means R script is generated but not run\\ 
-rplots=1 means basic plots are generated\\													  
-rplots=2 means basic and extended plots are generated\\													  
\nextopt
\end{optionlist}

\subsubsection*{Troubleshooting}
If no .pdf was generated even if a template file is available and the appropriate options were set, check the .Rout-file in the main run directory for error messages. If no .Rout-file exists, then check that R is properly installed, and that either command 'R' is available or that R is configured in psn.conf.


\section{Output}
When the NONMEM runs are finished, output and table files will be copied to the directory where execute started in which means that you can normaly ignore the 'modelfit\_dirX' directory. If you need to access any special files you can find them inside 'modelfit\_dirX'. Inside 'modelfit\_dirX' you find a few subdirectories named 'NM\_runY'. For each model file specified on the command line there will be one 'NM\_runY' directory in which the actual NONMEM execution takes place. The order of the 'NM\_runY' directories corresponds to the order of the modelfiles given on the command line. The first run will take place inside 'NM\_run1', the second in 'NM\_run2' etc.



\end{document}
