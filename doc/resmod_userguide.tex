\documentclass[a4wide,12pt]{article}
%\setlength{\marginparwidth}{0pt}%35
%\setlength{\marginparsep}{0pt}%?
%\setlength{\evensidemargin}{0pt}
%\setlength{\oddsidemargin}{0pt}
\usepackage{lmodern}
\usepackage[utf8]{inputenc}
\usepackage[T1]{fontenc}
\usepackage{textcomp}
\usepackage{verbatim}
\usepackage{enumitem}
\usepackage{longtable}
\usepackage{alltt}
\usepackage{ifthen}
% Reduce the size of the underscore
\usepackage{relsize}
\renewcommand{\_}{\textscale{.7}{\textunderscore}}

\newcommand{\guidetitle}[1]{
\title{#1\\ \vspace{2 mm} {\large PsN 4.1.1}}
\date{2014-02-10}
}

\newcommand{\doctitle}[1]{
\title{#1}
\date{2014-02-10}
}


\newenvironment{optionlist}{
\renewcommand{\arraystretch}{1.1}
\setlength{\leftmargini}{2.5cm}
\begin{description}
%\setlength{\itemsep}{0ex}
}
{\end{description}}

\newcommand{\optname}[1]{\item{{\bfseries\texttt-#1}\newline}}
\newcommand{\optdefault}[2]{\item{{\bfseries\texttt-#1}{\mbox{ = \it #2}}\newline}}

\newcommand{\nextopt}{}

\guidetitle{RESMOD user guide}{2017-02-22}

\begin{document}

\maketitle
\newcommand{\guidetoolname}{resmod}


\section{Overview}
The residual modelling tool performs modelling of the conditional weighted residual output of a model run. Different types of models are run and changes in OFV will be recorded together with some interesting parameters.

Example
\begin{verbatim}
resmod run1.mod
\end{verbatim}

\section{Input and options}

\subsection{Required input}
A model file is required on the command-line.

\subsection{Optional input}

\begin{optionlist}
\optdefault{dvid}{DVID}
Name of the dvid column. DVID is default.
\nextopt
\optdefault{idv}{TIME}
Name of the independent variable. Set to TIME by default. If set to PRED
PsN will rename it appropriately so that it can be used in \$INPUT.
\nextopt
    \optdefault{dv}{CWRES}
        Name of the dependent variable. Set to CWRES by default.
    \nextopt
\optdefault{occ}{OCC}
Name of the occasion column. Set to OCC by default.
\nextopt
\optdefault{groups}{4}
      Set the number of groups to use for the time varying models.     
      Quantiles using this number will be calculated.
      The default is 4.
\nextopt
    \optname{iterative}
        Iterate by selecting the best model and rerun all models with the output from that model.
    \nextopt
    \optname{max\_iterations}
        Default is to iterate until all models have been selected, but this option makes it possible
        to stop the iteration earlier.
    \nextopt
\end{optionlist}

\section{Iteration}
If the option -iterate has been selected the model with the lowest dOFV will be selected. The output from this model
will be used to rerun all the models once more etc. The iteration will stop when either the -max\_iterations has been
reached or when all models have been selected. The iteration will be done for each DVID separately.

\section{Output}
The delta OFVs of all models are summarized in the results.csv file together with the interesting parameter estimates.

\section{Residual models}

\subsection{Base model 1}
This model is used as a base model for the eta-on-epsilon, power, time-varying and the AR models.

\begin{verbatim}
Y = THETA(1) + ETA(1) + ERR(1)
\end{verbatim}

\subsection{IIV on RUV}
This model is compared to Base model 1 and is named \verb|IIV_on_RUV| in results.csv. The \%CV for the omega for ETA(2) is presented.

\begin{verbatim}
Y = THETA(1) + ETA(1) + ERR(1) * EXP(ETA(2))
\end{verbatim}

\subsection{Power}
This model is compared to Base model 1 and is named \verb|power| in results.csv. The power (THETA(2)) is presented.

\begin{verbatim}
Y = THETA(1) + ETA(1) + ERR(1)*(IPRED)**THETA(2)
\end{verbatim}

\subsection{Time varying RUV}
This model is compared to Base model 1 and is named \verb|time_varying_RUV| in results.csv. The <idv> below will be replaced with the name of the independent variable. The <q1>, <q2> and <q3> below will be replaced with the quartiles of the independent variable
for all observations combined. I.e. from the data set remove all records for which CWRES = 0 then calculate the quartiles of the idv column. The standard deviations of the epsilons and the time cutoff values are presented in the results.csv.
\begin{verbatim}
Y = THETA(1) + ETA(1) + ERR(4)
IF (<idv>.LT.<q1>) THEN
    Y = THETA(1) + ETA(1) + ERR(1)
            'END IF
IF (<idv>.GE.<q1> .AND. <idv>.LT.<q2>) THEN
    Y = THETA(1) + ETA(1) + ERR(2)
END IF
IF (<idv>.GE.<q2> .AND. <idv>.LT.<q3>) THEN
    Y = THETA(1) + ETA(1) + ERR(3)
END IF
\end{verbatim}

\subsection{Autocorrelation}
This model is compared to Base model 1 and is named \verb|autocorrelation| in results.csv. The autocorrelation (THETA(2)) will be presented. TIME will always be used as idv even if the option -idv is specified.
\begin{verbatim}
"FIRST
" USE SIZES, ONLY: NO
" USE NMPRD_REAL, ONLY: C=>CORRL2
" REAL (KIND=DPSIZE) :: T(NO)
" INTEGER (KIND=ISIZE) :: I,J,L
"MAIN
"C If new ind, initialize loop
" IF (NEWIND.NE.2) THEN
"  I=0
"  L=1
"  OID=ID
" END IF
"C Only if first in L2 set and if observation
"C  IF (MDV.EQ.0) THEN
"  I=I+1
"  T(I)=TIME
"  IF (OID.EQ.ID) L=I
"
"  DO J=1,I
"      C(J,1)=EXP((-0.6931/THETA(2))*(TIME-T(J)))
"  ENDDO
Y = THETA(1) + ETA(1) + EPS(1)
\end{verbatim}

\subsection{T-distribution base model}
This model is used as base model for the t-distribution model.

\begin{verbatim}
    IPRED = THETA(1) + ETA(1)
	W     = THETA(2)
	IWRES=(DV-IPRED)/W
	LIM = 10E-14
	IF(IWRES.EQ.0) IWRES = LIM
	LL=-0.5*LOG(2*3.14159265)-LOG(W)-0.5*(IWRES**2)
	L=EXP(LL)
	Y=-2*LOG(L)
\end{verbatim}

Using the Laplace estimation method.

\subsection{T-distribution 2LL}
Named \verb|tdist_2ll_dfest| in results.csv. The degrees of freedom is estimated and presented in results.csv.

\begin{verbatim}
IPRED = THETA(1) + ETA(1)
W = THETA(2)
DF = THETA(3) ; degrees of freedom of Student distribution
SIG1 = W ; scaling factor for standard deviation of RUV
IWRES = (DV - IPRED) / SIG1
; Nemesapproximation of gamma funtion(2007) for
; first factor of t-distrib(gamma((DF+1)/2))
PHI = (DF + 1) / 2
INN = PHI + 1 / (12 * PHI - 1 / (10 * PHI))
GAMMA = SQRT(2 * 3.14159265 / PHI) * (INN / EXP(1)) ** PHI
; Nemesapproximation of gamma funtion(2007) for
; second factor of t-distrib(gamma(DF/2))
PHI2 = DF / 2
INN2 = PHI2 + 1 / (12 * PHI2 - 1 / (10 * PHI2))
GAMMA2 = SQRT(2*3.14159265/PHI2)*(INN2/EXP(1))**PHI2
; coefficient of PDF of t-distribution
COEFF=GAMMA/(GAMMA2*SQRT(DF*3.14159265))/SIG1
BASE=1+IWRES*IWRES/DF ; base of PDF of t-distribution
POW=-(DF+1)/2 ; power of PDF of t-distribution
L=COEFF*BASE**POW ; PDF oft-distribution
Y=-2*LOG(L)
\end{verbatim}

\subsection{DTBS base model}
This model is used as base model for the DTBS model.
\begin{verbatim}
    IPRT   = THETA(1)*EXP(ETA(1))
	WA     = THETA(2)
	LAMBDA = THETA(3)
	ZETA   = THETA(4)
	IF(IPRT.LT.0) IPRT=10E-14
	W = WA*IPRED**ZETA
	IPRTR = IPRT
	IF (LAMBDA .NE. 0 .AND. IPRT .NE.0) THEN
		IPRTR = (IPRT**LAMBDA-1)/LAMBDA
	ENDIF
	IF (LAMBDA .EQ. 0 .AND. IPRT .NE.0) THEN
		IPRTR = LOG(IPRT)
	ENDIF
	IF (LAMBDA .NE. 0 .AND. IPRT .EQ.0) THEN
		IPRTR = -1/LAMBDA
	ENDIF
	IF (LAMBDA .EQ. 0 .AND. IPRT .EQ.0) THEN
		IPRTR = -1000000000
	ENDIF
	IPRT = IPRTR
	Y = IPRT + ERR(1)*W
	IF(ICALL.EQ.4) Y=EXP(DV)
\end{verbatim}

Lambda and zeta will be fixed to 0.

\subsection{DTBS}
Named \verb|dtbs| in results.csv. The lambda and zeta parameters (THETA(3) and THETA(4)) will be presented.

This model is same as the base model except that all parameters will be estimated.

\subsection{L2}
Name \verb|L2| in results.csv. The L2 model will be tried only if the original model has multiple DVs. The L2 data column will
be taken to be a copy of the TIME column. Resulting delta ofv will be the difference between the sum of the OFVs of the base models for all DVIDs and
the OFV of the L2 model.
\begin{verbatim}
IF(DVID.EQ.1) Y = THETA(1) + ETA(1) + ERR(1)
IF(DVID.EQ.2) Y = THETA(2) + ETA(2) + ERR(2)
\end{verbatim}

\end{document}
