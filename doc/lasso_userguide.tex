\documentclass[a4wide,12pt]{article}
%\setlength{\marginparwidth}{0pt}%35
%\setlength{\marginparsep}{0pt}%?
%\setlength{\evensidemargin}{0pt}
%\setlength{\oddsidemargin}{0pt}
\usepackage{lmodern}
\usepackage[utf8]{inputenc}
\usepackage[T1]{fontenc}
\usepackage{textcomp}
\usepackage{verbatim}
\usepackage{enumitem}
\usepackage{longtable}
\usepackage{alltt}
\usepackage{ifthen}
% Reduce the size of the underscore
\usepackage{relsize}
\renewcommand{\_}{\textscale{.7}{\textunderscore}}

\newcommand{\guidetitle}[1]{
\title{#1\\ \vspace{2 mm} {\large PsN 4.1.1}}
\date{2014-02-10}
}

\newcommand{\doctitle}[1]{
\title{#1}
\date{2014-02-10}
}


\newenvironment{optionlist}{
\renewcommand{\arraystretch}{1.1}
\setlength{\leftmargini}{2.5cm}
\begin{description}
%\setlength{\itemsep}{0ex}
}
{\end{description}}

\newcommand{\optname}[1]{\item{{\bfseries\texttt-#1}\newline}}
\newcommand{\optdefault}[2]{\item{{\bfseries\texttt-#1}{\mbox{ = \it #2}}\newline}}

\newcommand{\nextopt}{}

\guidetitle{LASSO user guide}{2018-03-02}
\usepackage{hyperref}


\begin{document}

\maketitle
\newcommand{\guidetoolname}{lasso}
\tableofcontents
\newpage

\section{Introduction}
Covariate models for population pharmacokinetics and pharmacodynamics are often built with a stepwise covariate modelling procedure (scm tool, also available in PsN). When analysing a small dataset this method may produce a covariate model that suffers from selection bias and poor predictive performance. 

The lasso tool \cite{Ribbing} is a method suggested to remedy these problems. It may also be faster than the scm tool and provide a validation of the covariate model. 

In the lasso tool all covariates must be standardised to have zero mean and standard deviation one. Subsequently, the model containing all potential covariate–parameter relations is fitted with a restriction: the sum of the absolute covariate coefficients must be smaller than a value, t. The restriction will force some coefficients towards zero while the others are estimated with shrinkage. This means in practice that when fitting the model the covariate relations are tested for inclusion at the same time as the included relations are estimated. For a given scm analysis, the model size depends on the P-value required for selection. In the lasso tool the model size instead depends on the value of t which can be estimated using cross-validation.
\\Example:
\begin{verbatim}
lasso run2.mod -relations=CL:CLCR-2,WT-3,,V:WT-2
\end{verbatim}

\section{Input and options}

\subsection{Required input}
A model file is required on the command line. Each parameter, e.g. CL and V, listed with the mandatory option -relations must  have the typical value encoded as TV\emph{parameters} in the model e.g. TVCL and TVV. The covariate effects will be added multiplicatively to the TV variables, see the example lasso model code below. If option -log\_scale is set, then instead the logarithm of the typical value must be encoded as LNTV\emph{parameter}, e.g. LNTVCL, and the covariate effects will be added additively to the LNTV variables.

The lasso tool does not support \$PRIOR in the input model.
\begin{optionlist}
\optdefault{relations}{list}
The relations input is required. The parameter-covariate pairs to test and the parameterizations to use. The parameterizations are:
\begin{enumerate}
	\item categorical covariate (treated as nominal scale),
	\item linear continuous covariate,
	\item piece-wise linear “hockey-stick” for continuous covariate.
\end{enumerate}

The syntax is sensitive and must be followed exactly. Note colons and commas. There must be no spaces in the list of relations. There are double commas before a new parameter. It is optional to input the breakpoint for the piece-wise linear relation (3), the number is then set after a second dash (-) as in WGT-3-45.2. 
\\Example (everything should be on the same line):\\
	-relations= CL:CRCL-2,SEX-1,,V:AGE-3-45.2,SEX-1,,KA:WGT-3,DIU-1
\end{optionlist}
\newpage
\subsection{Optional input}
\begin{optionlist}
\optname{adaptive}
Default not set. If set, run adaptive lasso algorithm. This option will be set automatically if option -adjusted is set. See section Lasso procedure for a description of the extra steps run when this option is set.
\nextopt
\optname{adjusted}
Default not set. If set, run adjusted adaptive lasso algorithm. See section Lasso procedure for a description of the extra steps run when this option is set.
\nextopt
\optdefault{al\_coefficients}{comma-separated list}
Default not set. Only allowed with adjusted adaptive lasso algorithm. A comma-separated list of absolute value of full model coefficient estimate divided by the standard error of the estimate. The values must be in the same order as the coefficients are in the auto-generated
full model. This vector will be used instead of values from an estimation of the full model. Values must only be give for the lasso coefficients, no other THETAs.
\nextopt
\optdefault{convergence}{FIRSTMIN | HALT | REACHMAX}
Default is FIRSTMIN. The convergence criterion.\\ 
The alternatives are:
\begin{itemize} 
	\item FIRSTMIN: Stop when predicted ofv increases from previous value (or when stop\_t is reached). 
	\item HALT: Stop when program crashes or when one model cannot terminate. 
	\item REACHMAX: Stop when program reaches max (stop\_t), then best t-value is selected. 
	\end{itemize}
\nextopt
\optdefault{cutoff}{X}
Default is 0.005. The theta cutoff. If the absolute value of the estimated covariate theta is below cutoff then the theta will be fixed to zero. 
\nextopt
\optdefault{-external\_data}{filename}
Default not set. A filename for external validation of the final model.
\nextopt
\optdefault{groups}{N}
Default 5, must be in the range 2-number of individuals in dataset. The number of validation groups in the cross-validation. The larger the number the longer the cross-validation run-time. The value set for the common option -threads will have no effect if set to something larger than the number of groups (cross-validation data sets).
\nextopt
\optname{log\_scale}
Default not set. If set, covariates will be added to LNTV instead of multiplied with TV. 
\\For example, if the input model has
\begin{verbatim}
LNTVCL=THETA(1)
\end{verbatim}
then the lasso tool will get
\begin{verbatim}
LNTVCL=THETA(1)
LNTVCL=LNTVCL+CLCOV
\end{verbatim}
where CLCOV is a sum of covariate effects.
\nextopt
\optdefault{maxevals}{N}
This common option is important for lasso, since the difficult minimizations in the cross-validation may require many evaluations. In the template psn.conf maxevals is set to 50000. Will only work for classical estimation methods. NONMEM only allows 9999 function evaluations. PsN can expand this limit by adding an MSFO option to \$ESTIMATION. Later when NONMEM hits the max number of function evaluations allowed by NONMEM (9999) PsN will remove initial estimates from the model-file and add \$MSFI and restart NONMEM. This will be repeated until the number of function evaluations specified with option -maxevals has been reached. 

Note: PsN does not change the MAXEVALS setting in the model-file, therefore the number of evaluations set on the command line may be exceeded before PsN does the check if the run should be restarted with msfi or not. 
\nextopt
\optname{normalize}
Default set. If unset with -no-normalize, PsN will not normalize covariates to mean 0 variance 1.
\nextopt
\optdefault{pred\_ofv\_start\_t}{X}
Default not set. The total predicted ofv for the model with t=start\_t.  If the option is not set PsN will run the lasso tool with t=start\_t. A given value will save time. 
\nextopt
\optdefault{retries}{N}
Using this common option PsN can do a retry with cut thetas (only lasso) and/or randomly tweaked inits if minimization is not successful. For extreme examples with lasso option retries may have to be as much as 20. On the other hand, -retries=0 is generally enough if rounding errors are acceptable (see option significant\_digits\_accept).
\nextopt
\optname{run\_final\_model}
Default not set. If set, the input model will be run with covariate relations added and THETAs fixed to estimates from the lasso procedure. 
\nextopt
\optdefault{significant\_digits\_accept}{number}
Default not set. Setting this common option may reduce run times considerably. Normally lasso tries new initial estimates unless MINIMIZATION SUCCESSFUL is found in the NONMEM output file. With the -significant\_digits\_accept, lasso will only rerun if the resulting significant digits is lower than the value specified with this option. It may or may not affect the final results. 
\nextopt
\optdefault{start\_t}{X}
Default is 0. The first t-value. 
\nextopt
\optdefault{step\_t}{X}
Default is 0.05. The step length for t in the cross-validation. The step length can be negative if start\_t is larger than stop\_t. 
\nextopt
\optdefault{stop\_t}{X}
Default is 1. The last t-value. 
\nextopt
\optdefault{stratify\_on}{variable}
Default not set. If the option is set, PsN will try to preserve the relative proportions of the values of this variable when creating the cross-validation datasets. The variable must be in the dataset (not in an extra data file or in the model). 
\nextopt
\end{optionlist}
Note: Do not set common option \emph{\textbf{min\_retries}} with lasso as it may interfere with the handling of runs with rounding errors.
\subsection{PsN common options}
For a complete list see common\_options.pdf or type psn\_options -h on the command line.

\subsection{Minimization and Retries}
The high dimensionality of a lasso model can make minimization more difficult. Therefore lasso has some special methods for handling minimization terminated due to rounding errors, and minimization terminated due to max number of evaluations exceeded.

Retries in lasso work slightly differently from retries in other PsN tools. Before doing a retry in lasso, if minimization is terminated due to rounding errors then PsN will set thetas below cutoff to 0. Initial estimates will not be tweaked at the first retry, when thetas are cut for the first time. Instead the initial estimates will be set to the final estimates of the first try. If repeated retries are required then thetas will be cut again and then initial estimates will be tweaked. As with any PsN tool the total number of retries is governed by the -retries option. 

Picking the best retry results is done as in other PsN tools, see details in common\_options.pdf. It does not matter if some retries have thetas that are set to zero.

\section{Output}
The raw\_results csv-file contains the parameter estimates of the final lasso model(s), as well as the FACTOR and
the final lasso coefficients. The file lasso\_coefficients.csv contains the final lasso coefficients.

The optimal lasso model and the original model with the optimal covariate relations added are in the m1 subdirectory of the run directory. Located directly in the run directory: the file lasso.log contains the sequence of estimation and prediction ofv:s and the corresponding t-values. The files est\_ofv.log and pred\_ofv.log contain the estimation and prediction ofv in table format. The file coeff\_table.log lists the estimated theta values for the covariates for each value of t in the cross-validation.

\section{Reusing old lasso output}

There are no special methods implemented to restart a crashed/stopped lasso run. Try starting with the same command, with -directory set to existing run directory. Do not change option rerun from the default value (1).

Option -pred\_ofv\_start\_t and -start\_t can be set to values known from a previous run. This approach would require the use of a new run directory; i.e. do not set -dir to that of the previous run. The approach useful when one wants to take smaller t-steps in a particular region. Make sure the -seed is the same between the two runs, so that the cross-validation datasets are identical. If the optimal t-value is known set -start\_t equal to -stop\_t. 

\section{Example lasso model}

With the command
\begin{verbatim}
lasso run1.mod -relations=CL:WGT-2,SEX-1,,V:WGT-2,APGR-3-5.5
\end{verbatim}

where the input model file run1.mod is as follows:

\begin{verbatim}
$PROBLEM    PHENOBARB 
$INPUT      ID TIME AMT WGT APGR DV SEX
$DATA       pheno_ch.dta IGNORE=@
$SUBROUTINE ADVAN1 TRANS2
$PK
      TVCL  = THETA(1)
      CL    = TVCL*EXP(ETA(1))
      TVV   = THETA(2)
      V     = TVV*EXP(ETA(2))
      S1    = V
$ERROR      
      W     = THETA(3)
      Y     = F+W*EPS(1)
      IPRED = F          ;  individual-specific prediction
      IRES  = DV-IPRED   ;  individual-specific residual
      IWRES = IRES/W     ;  individual-specific weighted residual
$THETA  (0,0.0105)     ;CL
$THETA  (0,1.0500)     ;V
$THETA  (0,0.4)        ;W
$OMEGA  .4
        .25
$SIGMA  1 FIX
$ESTIMATION MAXEVAL=9999 SIGDIGITS=4 POSTHOC METHOD=1
\end{verbatim}

the initial lasso model becomes (normalization constants depend on statistics computed from run1.mod data file)

\begin{verbatim}
$PROBLEM    PHENOBARB
$INPUT      ID TIME AMT WGT APGR DV SEX
$DATA      pheno_ch.dta IGNORE=@
$SUBROUTINE ADVAN1 TRANS2
$OMEGA  .4
 .25
$PK
;;; LASSO-BEGIN
    TVALUE  = THETA(9)
    ABSSUM = ABS(THETA(4))+ABS(THETA(5))+ABS(THETA(6))+ABS(THETA(7))
    ABSSUM = ABSSUM+ABS(THETA(8))

    RATIO = ABSSUM/TVALUE
    IF (RATIO.GT.5) EXIT 1 1
    FACTOR = EXP(1-RATIO)

    SEX0 = 0
    HAPGR = 0
    IF (SEX.EQ.0) SEX0=1
    IF (APGR.GT.5.50000) HAPGR = APGR-5.50000

    CLWGT = THETA(4)*(WGT-1.52542)/0.70456*FACTOR
    CLSEX0 = THETA(5)*(SEX0-0.49153)/0.50422*FACTOR
    VAPGR = THETA(6)*(APGR-6.42373)/2.23764*FACTOR
    VHAPGR = THETA(7)*(HAPGR-1.49153)/1.33420*FACTOR
    VWGT = THETA(8)*(WGT-1.52542)/0.70456*FACTOR

    VCOV = (VAPGR+1)*(VHAPGR+1)*(VWGT+1)
    CLCOV = (CLWGT+1)*(CLSEX0+1)
;;; LASSO-END

      TVCL  = THETA(1)
      TVCL = TVCL*CLCOV
      CL    = TVCL*EXP(ETA(1))
      TVV   = THETA(2)
      TVV = TVV*VCOV
      V     = TVV*EXP(ETA(2))
      S1    = V
$ERROR
      W     = THETA(3)
      Y     = F+W*EPS(1)
      IPRED = F          ;  individual-specific prediction
      IRES  = DV-IPRED   ;  individual-specific residual
      IWRES = IRES/W     ;  individual-specific weighted residual

$THETA  (0,0.0105) ; CL
$THETA  (0,1.0500) ; V
$THETA  (0,0.4) ; W
$THETA  (-0.33962,0.000100,0.76134) ; TH4 CLWGT
$THETA  (-0.99163,0.000100,1.02583) ; TH5 CLSEX0
$THETA  (-0.62569,0.000100,0.41256) ; TH6 VAPGR
$THETA  (-0.44348,0.000100,0.89452) ; TH7 VHAPGR
$THETA  (-0.33962,0.000100,0.76134) ; TH8 VWGT
$THETA  (-1000000,0.000000) FIX ; TH9 T-VALUE
$SIGMA  1  FIX
$ESTIMATION MAXEVAL=9999 SIGDIGITS=4 POSTHOC METHOD=1
\end{verbatim}

Subsequent models tested during cross-validation have other values of T-VALUE.
If the adaptive or adjusted adaptive lasso is run, fixed adaptive lasso coefficients are included in the lasso model 
code. Example:
\begin{verbatim}
    VWGT = THETA(8)*(WGT-1.52542)/0.70456*FACTOR
\end{verbatim}
is replaced with
\begin{verbatim}
    VWGT = THETA(8)*AL_VWGT *(WGT-1.52542)/0.70456*FACTOR
\end{verbatim}
where AL\_VWGT is the adaptive lasso coefficient for VWGT and is equal to a FIXED \$THETA. 

\section{Lasso procedure}
\begin{enumerate}
\item Read input model. Basic model parameters are \emph{not} updated with final estimates even if input model
lst-file exists. 
\item Read data set and compute mean and standard deviation of covariates in full data set.
\item Create the lasso model:
\begin{itemize}
\item Normalize covariates by subtracting the mean and dividing by the standard deviation, example:
\begin{verbatim}
(WGT-1.52542)/0.70456
\end{verbatim}
\item Add code for each parameter-covariate relation as a lasso THETA to be estimated multiplied with
the normalized covariate multiplied with the lasso-constraint FACTOR variable.
If adaptive lasso is run (option -adaptive and/or -adjusted) then also multiply with a FIXED adaptive lasso (AL)
coefficient. To begin with, the AL coefficients are set equal to 1.
\begin{verbatim}
CLWGT = THETA(4)*AL_CLWGT*(WGT-1.52542)/0.70456*FACTOR
\end{verbatim}
\item Add code for covariates in expressions for typical values of parameters. The default is to include
the relations multiplicatively:
\begin{verbatim}
CLCOV = (CLWGT+1)*(CLSEX0+1)
TVCL  = THETA(1)
TVCL = TVCL*CLCOV
\end{verbatim}
but if option -log\_scale is set then relations are included additively:
\begin{verbatim}
CLCOV = CLWGT+CLSEX0
LNTVCL  = THETA(1)
LNTVCL = LNTVCL+CLCOV
\end{verbatim}
\item Add code for lasso constraint, which involves sum of absolute lasso THETAs (not AL THETAs) and
the t-value.
\end{itemize}
\item Create cross-validation data sets, possibly using stratification. Divide data into N groups.
Create N estimation datasets where estimation dataset j is the total dataset minus group j.
There are N prediction datasets where prediction dataset j is identical to group j. 
\item Only if adjusted adaptive lasso is run: 
\begin{enumerate}
\item Create the full model by copying the lasso model created as above and removing the
lasso constraint. 
\item Add \$COV.
\item Unless AL-coefficients given with option -al\_coefficients:
Run the full model to obtain estimates and standard errors of the lasso \$THETAs,
and compute new AL-coefficients for each parameter-covariate relation 
as $\frac{abs(lasso theta)}{se_{lasso theta}}$
\item Change the AL coefficients in the lasso model from 1 FIXED to these new FIXED values.
\end{enumerate}
\item Run a cross-validation with the lasso model to find the optimal value of t, 
starting at start\_t and ending when the prediction ofv increases compared to the
previous t or when stop\_t is reached.
%\item Set t=start\_t. The default starting position is start\_t=0.
%If option pred\_ofv\_start\_t is given then go to step 6 (change t again), otherwise go to step 7 (run an xv step).
%\item Set t=t+step\_t. 
%\item Run an xv step with lasso model. An xv step consists of N estimations, one for each estimation dataset, and N predictions, one for each prediction dataset. The prediction j is a MAXEVAL=0 run with initial estimates set to the final estimates from estimation run j. Note: if t=0 the lasso model collapses to the base model.
%\item Check convergence. If convergence is not reached, go back to step 6. If converged continue to 9.
\item Run the lasso model with optimal t on the whole (original) dataset. 
\item Compute final lasso coefficients as described below in subsection
``Computing final lasso coefficients''.
\item Create regular model as described below in subsection ``Creating regular model''
\item Only if adaptive lasso, but \emph{not} adjusted adaptive lasso, is run:
\begin{enumerate}
\item Change the AL coefficients in the lasso model from 1 FIXED to the absolute values
of the lasso coefficients computed above (step 8).
\item Run a cross-validation with the updated lasso model to find the optimal value of t, 
starting at start\_t and ending when the prediction ofv increases compared to the
previous t or when stop\_t is reached.
\item Run the lasso model with optimal t on the whole (original) dataset. 
\item Compute final adaptive lasso coefficients as described below in subsection
``Computing final lasso coefficients''.
\item Create regular model as described below in subsection ``Creating regular model''
\end{enumerate}
%\item Add the covariate relations and parameters from the optimal lasso model run (only relations with thetas larger than cutoff) to the input model, fix the selected covariate thetas as the lasso restriction has been removed and run the input model on the whole dataset.
\end{enumerate}

\subsubsection*{Computing final lasso coefficients}
\begin{itemize}
\item If (adjusted) adaptive lasso: Compute final lasso coefficients as
$FACTOR*lassotheta*ALtheta$. If $abs(lassotheta)\cdot ALtheta\leq 0.005$ then set lasso coefficient to 0.
\item If regular lasso:
Compute final lasso coefficients as $FACTOR*lassotheta$. If $abs(lassotheta)\leq 0.005$ then
set lasso coefficient to 0.
\end{itemize}

\subsubsection*{Creating regular model}
Final lasso coefficients, either from regular, adaptive or adjusted adaptive lasso is needed as input. 
\begin{enumerate}
\item Copy the lasso model, and update non-fix parameters with final estimates from
previous estimation of lasso model with optimal t on full data set.
\item Remove lasso constraint from the model by removing code involving the t-value and removing FACTOR.
\item Set the parameter-covariate (lasso) thetas
(e.g. THETA(4) in below example) equal to the corresponding final lasso coefficients
computed as above divided by the standard deviation of the covariate
and FIX them. Remove division by sd from the code, and also remove AL coefficients and FACTOR.
Note 1: if only regular lasso is run then AL coefficients are not
in the model at all. Note 2: Values of AL coefficients, if used, are included in regular model via final lasso coefficients.
Example:
\begin{verbatim}
CLWGT = THETA(4)*AL_CLWGT*(WGT-1.52542)/0.70456*FACTOR
\end{verbatim}
becomes
\begin{verbatim}
CLWGT = THETA(4)*(WGT-1.52542)
\end{verbatim}
where THETA(4) is set to final lasso coefficient for CLWGT divided by 0.70456 and FIXED.
Note: Some lasso coefficients will be 0.
\item Set THETAs for AL coefficients, if any, to 1 FIX. 
\item If an external validation data set was given: Copy regular model, set MAXEVAL=0 and change data set to
external data set.
\item Run regular model with full data set, i.e. reestimate model parameters.
\item If an external validation data set was given: Copy regular model again, update initial estimates
with final estimates from estimation in previous step, set MAXEVAL=0 and change data set to
external data set.
\end{enumerate}

\section{Covariate relations}
\subsection{Linear continuous covariate}
The individual mean for a continuous covariate is calculated as
\[\bar{x}_{j,ind} = \frac{1}{N_j} \sum_{i=1}^N x_{ji}
\]
where $N_j$ is the number of samples for individual j and $x_{ji}$ is the sample value $i$ for individual $j$. Missing values are ignored.

The population mean of a continuous covariate is calculated as the mean of the individual means
\[\bar{x}_{pop} = \frac{1}{M} \sum_{j=1}^M \bar{x}_{j,ind}
\]
where $M$ is the number of individuals. If an individual has all values missing it is ignored.

The population standard deviation for a continuous covariate is calculated as the standard deviation of the individual means.
\[sd_{pop} = \sqrt{\frac{1}{M-1} \sum_{j=1}^M(\bar{x}_{j,ind} - \bar{x}_{pop})^2}
\]
If an individual has all values missing are imputed as having the population mean.

 
\subsection{Piece-wise linear covariates}
Covariates that are set to be piece-wise linear will be added to the model in the same manner as the linear covariates together with an extra dummy covariate which is truncated below its breakpoint according to:
\[   x_{ijk,dummy} = \left\{
        \begin{array}{ll}
            x_{ijk} - breakpoint & \textrm{if} \;  x_{ijk} > breakpoint \\
            0 & \textrm{otherwise} \\
        \end{array} 
    \right. \]

\subsection{The model}
\begin{enumerate}
    \item Calculate the absolute sum of the added thetas $\sum_{j=1}^N |\theta_j|$ 
    \item Calculate the ratio between the absolute sum and the t-value $r = \frac{\sum{|\theta_j|}}{t}$
    \item This ratio should be a positive value less than or equal to one according to the LASSO constraint $\sum{|\theta_i \le t|}$. This is accomplished by letting the variables be affected by a factor $f=e^{1-r}$.
    \item The covariates are standardized to 0 mean and variance 1. This is done by subtracting the population mean and dividing by the population standard deviation.
    \item The variables are added to the model as the standardized covariates multiplied with the factor $f$ and the corresponding theta. $v_k = \theta_k \cdot \frac{x_{ij} - \bar{x}_{pop}}{sd_{pop}} \cdot f$ where $k$ is the number of created variables and $f$ the factor defined above.
    \item The typical values are estimated as $TV_a = TV_a \cdot \prod_{i=1}^N ( 1+v_a)$ where $N$ is the number of created variables for a specific parameter $a$ e.g. CL.
\end{enumerate}

\references

\end{document}
