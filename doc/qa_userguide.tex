\documentclass[a4wide,12pt]{article}
%\setlength{\marginparwidth}{0pt}%35
%\setlength{\marginparsep}{0pt}%?
%\setlength{\evensidemargin}{0pt}
%\setlength{\oddsidemargin}{0pt}
\usepackage{lmodern}
\usepackage[utf8]{inputenc}
\usepackage[T1]{fontenc}
\usepackage{textcomp}
\usepackage{verbatim}
\usepackage{enumitem}
\usepackage{longtable}
\usepackage{alltt}
\usepackage{ifthen}
% Reduce the size of the underscore
\usepackage{relsize}
\renewcommand{\_}{\textscale{.7}{\textunderscore}}

\newcommand{\guidetitle}[1]{
\title{#1\\ \vspace{2 mm} {\large PsN 4.1.1}}
\date{2014-02-10}
}

\newcommand{\doctitle}[1]{
\title{#1}
\date{2014-02-10}
}


\newenvironment{optionlist}{
\renewcommand{\arraystretch}{1.1}
\setlength{\leftmargini}{2.5cm}
\begin{description}
%\setlength{\itemsep}{0ex}
}
{\end{description}}

\newcommand{\optname}[1]{\item{{\bfseries\texttt-#1}\newline}}
\newcommand{\optdefault}[2]{\item{{\bfseries\texttt-#1}{\mbox{ = \it #2}}\newline}}

\newcommand{\nextopt}{}

\guidetitle{QA user guide}{2019-10-15}

\usepackage{hyperref}
\begin{document}

\maketitle
\newcommand{\guidetoolname}{qa}
\tableofcontents
\newpage


\section{Introduction}
Quality assurance (qa) is a tool to assess the quality of a final model. The analysis is not done directly on the final model, but on certain model proxies to speed up the analysis time.
The proxies include a linearized version of the model to be qa:ed as well as additional modeling of the residuals from that model \cite{Khandelwal, Svensson, Ibrahim}. The idea is that these simplified models will provide similar improvement in goodness-of-it (OFV) and similar parameter estimates with a model change as if the same model change had been done directly on the non-linear model. The model proxies however are running much faster and are more robust. 

The suggested use of the resulting report is that first inspect the overview section and there assess whether any change to the model resulted in an improvement of such a size that it would be of interest to learn more about it. If so, there is more information in the subsections. If not, the qa functionality has provided reassurance that the model in a large number of aspects describe the data well. 

The "Structural" section contains information additional information about bias in the population predictions\cite{Ibrahim3} versus different independent variables.  If this bias is low, despite a marked improvement in OFV, an argument can be made about the clinical relevance of the bias. This size and direction of the bias may also suggest a new model structure. Note though that also covariate or stochastic model misspecification can result in bias and may be best to corrected for first. 

The "Parameter Variability Model" section extensions to the IIV model are tested \cite{Svensson, Petersson}. If any model suggests an improvement, the PsN functionality "transform" can easily make the corresponding change to the original model (learn through \verb|transform -help|).

The "Covariates" section explores covariate relations in addition to those already present in the model. A very strong, but correctly implemented covariate in the original model, should not provide a signal in this analysis. Thus for a model with correctly implemented covariates this would provide only marginal improvements. Two methods are used: a univariate estimation of all individual parameter-covariate relations (the typical first step in an SCM \cite{Jonsson2}), and a full random effects model (FREM \cite{Karlsson, Yun, Yngman, Yngman2018}) which estimates simultaneously all parameter-covariate relations using the baseline covariate values. For the FREM the sources of variability is explored via the variability attribution plot\cite{Ueckert2018}.

"Residual Error Model" explores a number of extensions to the original residual error model \cite{Ibrahim, Karlsson2, Karlsson3, Dosne2012}. (For dtbs there is functionality in PsN to transform the original model automatically (see option dtbs after \verb|execute –help|). 

"Influential individuals" are identified through a case-deletion diagnostics routine where the model is reestimated after dropping one subject at a time (subject i). The metric used is the difference in OFV for all subjects except subject i between the model with all subjects included and the model where subject i is dropped. This difference should always be positive as the model will always fit the remaining subjects better when subject I has been omitted. If the difference is >3.84 (the cut-off used for declaring an influential subject) it means that a parameter may have changed significantly (@ p<0.05) by the presence of the subject, hence influential.\cite{Nordgren2018}

"Outliers" are subjects for which the model fits poorly. In this evaluation outlier status is judged based the overall goodness of fit to an individual's data \cite{Largajolli}. For an individual with such a characteristic, it may be further evaluated whether it is due to unusual parameter values and/or outlying observations (see simeval functionality in PsN). The metric presented for outlying subjects is the same as for Influential Individuals. Naturally an individual that is both outlying and influential is of particular concern.

The qa tool avoids a full estimation of the input model and generates the proxies at final estimates of the input model. For an optimal performance of qa, the lst, ext and phi files need to be present.
\newpage

Examples:
\begin{verbatim}
qa run1.mod -continuous_covariates=WGT,AGE -categorical_covariates=SEX,RACE -add_etas=KA
qa run262.mod -continuous=HT,WT,AGE -categorical=SEX -add_etas=V
qa run112d.mod -cont=AGE,CRCL,WT -cat=SEX -dvid=CMT -add_etas=VMAX,KM,CLM,V6 -only=scm,frem
\end{verbatim}

\section{Input and options}

\subsection{Required input}
A model file is required on the command line.

    The qa tool will perform multiple assessments of the input model covering
    different parts of the model.

    Many of the assessments does not need anything but the model from the user,
    but some need specific input to trigger the analysis.

\subsubsection{Covariates}
    To perform investigation of covariate effects the user needs to supply lists
    of the covariates via the \verb|-categorical_covariates| and/or the
    \verb|-continuous_covariates| options.

\subsubsection{Add IIV}
    Inter-individual variability for parameters that does not have it can be
    assessed by listing the interesting parameters with the \verb|-add_etas| option.
    IIV will be added as an exponential and without any covariances.

\subsubsection{Add IOV}
    Inter-occasion variability can be assessed by setting which occasion
    column to use via the \verb|-iov| option. IOV will be added to all parameters
    that have IIV using the same block structure.

\subsubsection{Multiple DVs}
    To assess different DVs separately in the residual and structural models
    use the \verb|-dvid| option.

\subsubsection{Extra independent variable}
    To add the assessment of residual and structural models for one
    extra independent variable other than \verb|TIME|, \verb|TAD| and \verb|PRED| the
    \verb|-resmod_idv| option should be used.



\subsection{Optional input}

\begin{optionlist}
\optname{categorical\_covariates}
A comma-separated list of categorical covariates.
\nextopt
\optname{continuous\_covariates}
A comma-separated list of continuous covariates.
\nextopt
\optname{dvid}
Name of the dvid column.
Will only be used for the structural and residual model analysis.
\nextopt
\optname{resmod\_groups}{10}
Default is 10. Set the number of groups to use for the time varying models.
Quantiles using this number will be calculated. See the resmod user guide
for a detailed explanation.
\nextopt
\optdefault{resmod\_idv}{TIME}
Default is to use TIME,PRED and TAD (if present). Name of the independent variable for the structural and residual model analysis.
This option will change the name of TIME.
\nextopt
\optname{lst\_file}
Default is to use an lst-file with the same name as the model. Set a NONMEM output file to be used for initial estimates.

\nextopt
\optname{occ}
Name of the occasion column. The IOV functionality is not yet implemented.
\nextopt
\optname{nm\_parallel}
	Select a special NONMEM version to be used for the steps
    that run one NONMEM run at a time. Currently linearize
    and frem. The value of the threads option will be used for
    the number of nodes for these tools and threads will be set
    to one.
\nextopt
\optname{nonlinear}
Default is off. Don't linearize the model and use the original non-linear model for all runs.
This option is experimental.
\nextopt
\optname{skip}
Skip one or more sections of the qa procedure. Takes a list of section names to skip. The sections are: scm, frem, cdd, simeval,
transform and resmod.
\nextopt
\optname{only}
Only run one or more sections of the qa procedure. Takes a list of section names to run. The sections are: scm, frem, cdd, simeval, transform and resmod.
\nextopt
\end{optionlist}

\section{Known limitations and problems}

Some limitations and problems will affect the ability to generate any output, other will impact one or some of the different evaluations, but still result in an output report. 

\subsection{General limitations and problems}
\begin{itemize}
	\item There is no support in qa for categorical data, including handling censored continuous data using the M3/M4 likelihood-based methods
	\item There is no support in qa for the first-order (FO) method. Testing with FOCE(I) has been extensive, and sparse with other methods (SAEM, ITS, IMP and IMPMAP). BAYES has not been tested. 
    \item LAPLACE is not supported.
	\item With linearization and eta-epsilon interaction (INTER) individual estimates of empirical Bayes estimates (EBE) of parameters may end up in local minima. The most prominent sign of this is that differences in OFV between two models have a sign, different from that expected, i.e. a smaller model may have a lower OFV than a full model.
	\item Multiple variables, through the option \verb|-dvid|, is not yet implemented in the report although the relevant models are being estimated.
	\item Automatic addition of interoccasion variability, implemented via \verb|-iov| is not yet functional.
    \item Missing values for covariates must be coded using -99
    \item There is no support for setting \verb|F_FLAG|
    \item Mixture models are not directly supported and need some extra manual work. See section below.
\end{itemize}

\subsection{Mixture models}
Mixture models cannot be handled directly by qa, but with some extra steps they can be made to work by preestimating the most likely subpopulation for each individual and use this as a covariate instead of a mixture:
\begin{enumerate}
    \item Get the MIXEST for each individual either from the .phm file or in a table
    \item Add the MIXEST values as a column in your dataset and call it SPOP
    \item Add SPOP to \$INPUT of the model
    \item Remove the \$MIX of the model and all \$THETAS now not longer needed
    \item Replace all the use of MIXNUM in the abbreviated code into SPOP
\end{enumerate}


\subsection{FREM}
\begin{itemize}
	\item For time-varying covariates, only the baseline value will be used.
	\item For multiple categorical covariates, missing combinations may make the model unstable. Such aan example would be the bivariate covariates SEX and GENOTYPE, with the category female-PM missing.
\end{itemize}

\subsection{resmod}
The resmod tool that is used for assessing the residual error model and the structural model will be run multiple times with different independent variables. The different results will be used for assessment of the structural model given different idvs. Which independent variables that will be used is determined by the following rules:
\begin{itemize}
    \item If TIME is in the model (either in \$INPUT or defined in the model) it will be used.
    \item If TAD is in the model it will be used.
    \item PRED will always be used.
    \item If the user has specified an idv different than TIME and TAD it will be used.
\end{itemize}


\section{Results}
The main result file is the PsN\_qa\_plots.pdf that will collect all tables and plots of interest

\section{Technical details}
The boxcox run will boxcox transform all OMEGAs (IIV and IOV). The fullblock run will exclude IOV OMEGAs.

\references

\end{document}
