\documentclass[a4wide,12pt]{article}
%\setlength{\marginparwidth}{0pt}%35
%\setlength{\marginparsep}{0pt}%?
%\setlength{\evensidemargin}{0pt}
%\setlength{\oddsidemargin}{0pt}
\usepackage{lmodern}
\usepackage[utf8]{inputenc}
\usepackage[T1]{fontenc}
\usepackage{textcomp}
\usepackage{verbatim}
\usepackage{enumitem}
\usepackage{longtable}
\usepackage{alltt}
\usepackage{ifthen}
% Reduce the size of the underscore
\usepackage{relsize}
\renewcommand{\_}{\textscale{.7}{\textunderscore}}

\newcommand{\guidetitle}[1]{
\title{#1\\ \vspace{2 mm} {\large PsN 4.1.1}}
\date{2014-02-10}
}

\newcommand{\doctitle}[1]{
\title{#1}
\date{2014-02-10}
}


\newenvironment{optionlist}{
\renewcommand{\arraystretch}{1.1}
\setlength{\leftmargini}{2.5cm}
\begin{description}
%\setlength{\itemsep}{0ex}
}
{\end{description}}

\newcommand{\optname}[1]{\item{{\bfseries\texttt-#1}\newline}}
\newcommand{\optdefault}[2]{\item{{\bfseries\texttt-#1}{\mbox{ = \it #2}}\newline}}

\newcommand{\nextopt}{}

\guidetitle{QA user guide}{2017-06-19}

\begin{document}

\maketitle
\newcommand{\guidetoolname}{qa}


\section{Overview}
Qa is a tool.

Example
\begin{verbatim}
qa pheno.mod -parameters=CL,V -covariates=CL,APGR
\end{verbatim}

\section{Input and options}

\subsection{Required input}
A model file is required on the command-line.

\subsection{Constraints on the input model}
\begin{itemize}
	\item A fixed omega is not allowed if there are non-fixed omegas both before and after it
\end{itemize}



\subsection{Optional input}



\begin{optionlist}
\optname{categorical}
A comma separated list of categorical covariates
\nextopt
\optname{covariates}
A comma separated list of continuous covariates
\nextopt
\optname{dv}{CWRES}
Name of the dependent variable for the structural and residual model analysis. CWRES is default.
\nextopt
\optname{dvid}{DVID}
Name of the dvid column. DVID is default.
Will only be used for the structural and residual model analysis.
\nextopt
\optname{fo}
Set if the FO method should be used for all estimations
of the linearized model. Default is off.
\nextopt
\optname{groups}{10}
Set the number of groups to use for the time varying models.
Quantiles using this number will be calculated.
The default is 10.
\nextopt
\optdefault{idv}{TIME}
Name of the independent variable for the structural and residual model analysis.
Default is to use TIME,PRED and TAD (if present). This option will change the name
of TIME.
\nextopt
\optname{lst\_file}
Set a NONMEM output file to be used for initial estimates.
Default is to use an lst-file with the same name as the model.
\nextopt
\optdefault{occ}{OCC}
Name of the occasion column. OCC is default.
The IOV functionality is not yet implemented.
\nextopt
\optname{parameters}
A comma separated list of parameters
\nextopt
\end{optionlist}


\section{Results}

The main result file is the PsN\_qa\_plots.pdf that will collect all tables and plots of interest

\references

\end{document}
