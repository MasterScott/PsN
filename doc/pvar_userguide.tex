\documentclass[a4wide,12pt]{article}
%\setlength{\marginparwidth}{0pt}%35
%\setlength{\marginparsep}{0pt}%?
%\setlength{\evensidemargin}{0pt}
%\setlength{\oddsidemargin}{0pt}
\usepackage{lmodern}
\usepackage[utf8]{inputenc}
\usepackage[T1]{fontenc}
\usepackage{textcomp}
\usepackage{verbatim}
\usepackage{enumitem}
\usepackage{longtable}
\usepackage{alltt}
\usepackage{ifthen}
% Reduce the size of the underscore
\usepackage{relsize}
\renewcommand{\_}{\textscale{.7}{\textunderscore}}

\newcommand{\guidetitle}[1]{
\title{#1\\ \vspace{2 mm} {\large PsN 4.1.1}}
\date{2014-02-10}
}

\newcommand{\doctitle}[1]{
\title{#1}
\date{2014-02-10}
}


\newenvironment{optionlist}{
\renewcommand{\arraystretch}{1.1}
\setlength{\leftmargini}{2.5cm}
\begin{description}
%\setlength{\itemsep}{0ex}
}
{\end{description}}

\newcommand{\optname}[1]{\item{{\bfseries\texttt-#1}\newline}}
\newcommand{\optdefault}[2]{\item{{\bfseries\texttt-#1}{\mbox{ = \it #2}}\newline}}

\newcommand{\nextopt}{}

\guidetitle{PVAR user guide}

\begin{document}

\maketitle


\section{Introduction}
Parameter variability

Examples
\begin{verbatim}
pvar run1.mod run2.mod
pvar scm_dir1
\end{verbatim}

\section{Input and options}

\subsection{Required input}
Required arguments are either an scm run directory or a list of model files.

\begin{optionlist}

\optdefault{parameters}{CL,V,...}
A comma separated list of parameters.
\nextopt
\end{optionlist}

\subsection{Optional input}

\begin{optionlist}
\optdefault{samples}{n}
Number of simulated datasets to generate. Default 100.
\nextopt
\end{optionlist}

\section{Description}

Parametric variability can calculate how much of the variability that is explained by the parameters of different models.

\section{Algorithm}

\begin{enumerate}
	\item Input: List of models.
	\item For each model:
	\begin{enumerate}
		\item Create two new models. One for EPV and one for PV
		\item If model has been run update the initial values from results.
		\item Remove \$EST
		\item If model already contain \$SIMULATION give a warning but keep \$SIMULATION
		\item Add \$SIMULATION (<seed>) NSUBS=<samples> ONLYSIM
		\item Add \$TABLE NOPRINT NOAPPEND FIRSTONLY ONEHEADER ID FILE=<EPV|PV><n>.tab <par1> <par2> ...
		\item Set \$OMEGA 0 FIX for the EPV model
	\end{enumerate}
	\item Run models
	\item Loop through table-files
\end{enumerate}

\end{document}
