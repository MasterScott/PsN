\documentclass[a4wide,12pt]{article}
%\setlength{\marginparwidth}{0pt}%35
%\setlength{\marginparsep}{0pt}%?
%\setlength{\evensidemargin}{0pt}
%\setlength{\oddsidemargin}{0pt}
\usepackage{lmodern}
\usepackage[utf8]{inputenc}
\usepackage[T1]{fontenc}
\usepackage{textcomp}
\usepackage{verbatim}
\usepackage{enumitem}
\usepackage{longtable}
\usepackage{alltt}
\usepackage{ifthen}
% Reduce the size of the underscore
\usepackage{relsize}
\renewcommand{\_}{\textscale{.7}{\textunderscore}}

\newcommand{\guidetitle}[1]{
\title{#1\\ \vspace{2 mm} {\large PsN 4.1.1}}
\date{2014-02-10}
}

\newcommand{\doctitle}[1]{
\title{#1}
\date{2014-02-10}
}


\newenvironment{optionlist}{
\renewcommand{\arraystretch}{1.1}
\setlength{\leftmargini}{2.5cm}
\begin{description}
%\setlength{\itemsep}{0ex}
}
{\end{description}}

\newcommand{\optname}[1]{\item{{\bfseries\texttt-#1}\newline}}
\newcommand{\optdefault}[2]{\item{{\bfseries\texttt-#1}{\mbox{ = \it #2}}\newline}}

\newcommand{\nextopt}{}

\guidetitle{EXECUTE user guide}
%Kajsa review 2014-03-07

\begin{document}

\maketitle


\section{Introduction}
The linearize script is a PsN tool that allows you to automatically linearize a model and obtain the derivatives dataset to use for further 
model development.

Example
\begin{verbatim}
linearize run1.mod
\end{verbatim}


\section{Input and options}
\subsection{Required input}
A model file is required on the command-line.

\subsection{Optional input}
\begin{optionlist}
\optname{epsilon}
Linearize with respect to epsilon, set by default. Disable with -no-epsilon.
\nextopt
\optname{error}
Only relevant if -no-epsilon is set. 
Use an approximate linearization of the error model instead of an exact.
Alternatives are add (for additive), prop (for proportional) or
propadd (for proportional plus additive).
The error model must be defined in a particular way when this option is used,
see the scm userguide for details.
\nextopt
\optname{foce}
Set by default. Expand around conditional ETA estimates instead of around ETA=0. To expand around ETA=0 set -no-foce.
\nextopt
\end{optionlist}

\section{Output}
Linearized model file with name xx and data file with name nn.

\end{document}
