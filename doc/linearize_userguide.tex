\documentclass[a4wide,12pt]{article}
%\setlength{\marginparwidth}{0pt}%35
%\setlength{\marginparsep}{0pt}%?
%\setlength{\evensidemargin}{0pt}
%\setlength{\oddsidemargin}{0pt}
\usepackage{lmodern}
\usepackage[utf8]{inputenc}
\usepackage[T1]{fontenc}
\usepackage{textcomp}
\usepackage{verbatim}
\usepackage{enumitem}
\usepackage{longtable}
\usepackage{alltt}
\usepackage{ifthen}
% Reduce the size of the underscore
\usepackage{relsize}
\renewcommand{\_}{\textscale{.7}{\textunderscore}}

\newcommand{\guidetitle}[1]{
\title{#1\\ \vspace{2 mm} {\large PsN 4.1.1}}
\date{2014-02-10}
}

\newcommand{\doctitle}[1]{
\title{#1}
\date{2014-02-10}
}


\newenvironment{optionlist}{
\renewcommand{\arraystretch}{1.1}
\setlength{\leftmargini}{2.5cm}
\begin{description}
%\setlength{\itemsep}{0ex}
}
{\end{description}}

\newcommand{\optname}[1]{\item{{\bfseries\texttt-#1}\newline}}
\newcommand{\optdefault}[2]{\item{{\bfseries\texttt-#1}{\mbox{ = \it #2}}\newline}}

\newcommand{\nextopt}{}

\guidetitle{LINEARIZE user guide}{2015-09-16}


\begin{document}

\maketitle
\newcommand{\guidetoolname}{linearize}


\section{Introduction}
The linearize script is a PsN tool that allows you to automatically create a linearized version of a model and obtain the dataset including
individual predictions and derivatives necessary for further estimation of extensions implemented in the linearized model. Command:
\begin{verbatim}
linearize runX.mod
\end{verbatim}

\noindent The linearization was developed with the aim to facilitate the development of nonlinear mixed effects models by establishing a 
diagnostic method for evaluation of stochastic model components. A full description of the methodology and it's performance is presented in
\cite{Svensson}. The online supplementary 
material includes a comprehensive code example. 
The methodology utilizes first order Taylor expansions and substantially shortens run times \cite{Khandelwal}. 
Examples of extensions  that can be evaluated with the linearized model are addition of inter individual or inter occasion variability parameters, 
correlation structures and more complex residual error models. 

Before proceeding with implementation and evaluation of extensions, it is important to check that the OFV value of the nonlinear and linearized 
version of the base model agrees (printed in the command window and in the linlog.txt file generated in the linearization folder). 
If the OFV-values differ more than a few points, this can depend on the occurrence of local minima in the MAP estimation. During the development of the methodology 
this was observed several times for models with a proportional or combined residual error structure. The problems are caused by the shape distortion
of the individual EBE likelihood profiles sometimes are introduced by the interaction term in the linearized model. The problem can easily be solved either by
transformation to log-scale (rendering proportional residual errors additive and thence interaction term is zero) or by use of the MCETA option 
available in NONMEM  from version 7.3.0. MCETA allows the user to define a number of vectors containing random samples drawn from the 
variance-covariance matrix OMEGA to be tested as initial ETA-values. Whichever supplies the lowest OFV will be used in the estimation. With a large 
enough number of initial values tested, the probability of the estimation to end up in a local minimum can be decreased.

If extensions including new random effects should be evaluated, the code for these should be included already in the base nonlinear model but with an 
exceedingly small and fixed OMEGA value (i.e. 0.00001 FIX). This is to enable calculation of the derivative with respect to the new parameter so these can 
be outputted in the generated dataset. If the value is fixed to zero, NONMEM will not calculate the derivative. 

The linearization is a diagnostics to identify the best extensions of a models stochastic components. Values of fixed effects cannot change in a linearized
model and the structural model must be reasonably good to make evaluation of stochastic components meaningful. Once the best extended model is identified, 
it is recommended to implement these in the nonlinear version of the model and reestimate all parameters.


\subsection{Workflow}
\begin{enumerate}
\item Select a base model with an acceptable structural component and potentially the most influential stochastic parameters included.
\item If investigating additional ETAs, include in base model but with OMEGA fixed to small value (i.e. 0.00001 FIX) to get derivatives in output dataset.
\item Run PsN command: linearize runX.mod.
\item Check OFV agreement. Disagreement indicates local minima in MAP estimation (solve by log-transformation to get additive error or use MCETA option
      in \$ESTIMATION).
\item Add extensions to the linearized model manually and evaluate.
\item When decided which extension to include, implement in standard nonlinear format and reestimate.
\end{enumerate}

\section{Input and options}
\subsection{Required input}
A model file (the nonlinear base model) is required on the command-line.

\subsection{Optional input}
\begin{optionlist}
\optname{epsilon}
Linearize with respect to epsilon, set by default. Disable with -no-epsilon.
\nextopt
\optname{error}
Only relevant if -no-epsilon is set. 
Use an approximate linearization of the error model instead of an exact.
Alternatives are add (for additive), prop (for proportional) or
propadd (for proportional plus additive).
The error model must be defined in a particular way when this option is used,
see the scm userguide for details.
\nextopt
\optname{foce}
Set by default. Expand around conditional ETA estimates instead of around ETA=0. To expand around ETA=0 set -no-foce.
\nextopt
\end{optionlist}

\subsection{Some important common PsN options}
There are many options that govern how PsN manages NONMEM runs, and
those options are common to all PsN programs that run NONMEM.
For a complete list of such options see common\_options.pdf, 
or psn\_options -h on the commandline. A selection of
the most important common options is found here.
\begin{optionlist}
\optname{h or -?}
Print the list of available options and exit. 
\nextopt
\optname{help}
With -help all programs will print a longer help message. 
If an option name is given as argument, help will be printed for this option. 
If no option is specified, help text for all options will be printed. 
\nextopt
\optdefault{directory}{'string'}
Default \guidetoolname\_dirN,
where N will start at 1 and
be increased by one each time you run the script. The directory option sets the directory in which PsN 
will run NONMEM and where PsN-generated output files will be stored. 
You do not have to create the directory,  it will be done for you. If you set
-directory to a the name of a directory that already exists, PsN will run in the existing directory.
\nextopt
\optdefault{seed}{'string'}
You can set your own random seed to make PsN runs reproducible.
The random seed is a string, so both -seed=12345 and -seed=JustinBieber are valid.
It is important to know that because of the way the Perl pseudo-random
number generator works, for two similar string seeds the random sequences may be identical. 
This is the case e.g. with the two different seeds 123 and 122. 
Setting the same seed guarantees the same sequence, but setting two slightly different 
seeds does not guarantee two different random sequences, that must be verified.
\nextopt
\optdefault{clean}{'integer'}
Default 1. The clean option can take four different values:  
\begin{description}
\item[0] Nothing is removed 
\item[1] NONMEM binary and intermediate files except INTER are removed, and files specified with option -extra\_files. 
\item[2] model and output files generated by PsN restarts are removed, and data files in the NM\_run directory, and (if option -nmqual is used) the xml-formatted NONMEM output. 
\item[3] All NM\_run directories are completely removed. If the PsN tool has created modelfit\_dir:s inside the main run directory, these  will also be removed. 
\end{description}
\nextopt
\optdefault{nm\_version}{'string'}
Default is 'default'. 
If you have more than one NONMEM version installed you can use option
-nm\_version to choose which one to use, as long as it is 
defined in the [nm\_versions] section in psn.conf, see psn\_configuration.pdf for details. 
You can check which versions are defined, without opening psn.conf, using the command
\begin{verbatim}
psn -nm_versions
\end{verbatim}
\nextopt
\optdefault{threads}{'integer'}
Default 1. Use the threads option to enable parallel execution of multiple models.
This option decides how many models PsN will run at the same time, and it is completely
independent of whether the individual models are run with serial NONMEM or parallel NONMEM.
If you want to run a single model in parallel you must use options -parafile and -nodes.
On a desktop computer it 
is recommended to not set -threads higher the number of CPUs in the system plus one. 
You can specify more threads, 
but it will probably not increase the performance. If you are running on a computer cluster, 
you should consult your 
system administrator to find out how many threads you can specify. 
\nextopt
\optname{version}
Prints the PsN version number of the tool, and then exit. 
\nextopt
\end{optionlist}



\section{Output}
Linearized model file with name runX\_linbase.mod and data file with name runX\_linbase.dta.

\bibliography{PsN}{}
\bibliographystyle{plain}

\end{document}
