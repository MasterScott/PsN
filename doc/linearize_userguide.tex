\documentclass[a4wide,12pt]{article}
%\setlength{\marginparwidth}{0pt}%35
%\setlength{\marginparsep}{0pt}%?
%\setlength{\evensidemargin}{0pt}
%\setlength{\oddsidemargin}{0pt}
\usepackage{lmodern}
\usepackage[utf8]{inputenc}
\usepackage[T1]{fontenc}
\usepackage{textcomp}
\usepackage{verbatim}
\usepackage{enumitem}
\usepackage{longtable}
\usepackage{alltt}
\usepackage{ifthen}
% Reduce the size of the underscore
\usepackage{relsize}
\renewcommand{\_}{\textscale{.7}{\textunderscore}}

\newcommand{\guidetitle}[1]{
\title{#1\\ \vspace{2 mm} {\large PsN 4.1.1}}
\date{2014-02-10}
}

\newcommand{\doctitle}[1]{
\title{#1}
\date{2014-02-10}
}


\newenvironment{optionlist}{
\renewcommand{\arraystretch}{1.1}
\setlength{\leftmargini}{2.5cm}
\begin{description}
%\setlength{\itemsep}{0ex}
}
{\end{description}}

\newcommand{\optname}[1]{\item{{\bfseries\texttt-#1}\newline}}
\newcommand{\optdefault}[2]{\item{{\bfseries\texttt-#1}{\mbox{ = \it #2}}\newline}}

\newcommand{\nextopt}{}

\guidetitle{LINEARIZE user guide}{2018-03-02}
\usepackage{hyperref}

\begin{document}

\maketitle
\newcommand{\guidetoolname}{linearize}
\tableofcontents
\newpage

\section{Introduction}
The linearize tool allows you to automatically create a linearized version of a model and obtain the dataset including
individual predictions and derivatives necessary for further estimation of extensions implemented in the linearized model.\\ 
Example:
\begin{verbatim}
linearize run10.mod
\end{verbatim}

The linearization was developed with the aim to facilitate the development of nonlinear mixed effects models by establishing a diagnostic method for evaluation of stochastic model components. A full description of the methodology and its performance 
is presented in \cite{Svensson}. The online supplementary material includes a comprehensive code example. 
The methodology utilizes first order Taylor expansions and substantially shortens run times. Examples of extensions that can be evaluated with the linearized model are addition of inter individual or inter occasion variability parameters, correlation structures and more complex residual error models.

Before proceeding with implementation and evaluation of extensions, it is important to check that the OFV value of the nonlinear and linearized version of the base model agrees (printed in the command window and in the linlog.txt file generated in the linearization folder). If the OFV-values differ more than a few points, this can depend on the occurrence 
of local minima in the MAP estimation. 

\subsection{Workflow}
\begin{enumerate}
\item Select a base model with an acceptable structural component and potentially the most influential stochastic parameters included.
\item If investigating additional ETAs, include in base model but with OMEGA fixed to small value (i.e. 0.00001 FIX) to get derivatives in output dataset.
\item Run linearize run10.mod.
\item Check OFV agreement. Disagreement indicates local minima in MAP estimation (solve by log-transformation to get additive error or set MCETA=1000 in \$ESTIMATION).
\item Add extensions to the linearized model manually and evaluate.
\item When decided which extension to include, implement in standard nonlinear format and reestimate.
\end{enumerate}

If the original model has already been run the initial values will be updated and MAXEVAL=0 will be used for the derivatives model.
If the original model has a phi file this will be used in a \$ETAS record in the linearized model.

\section{Input and options}
\subsection{Required input}
A model file is required on the command line.

\subsection{Optional input}
The following options are valid but intended only for research and method exploration. It is recommended to not use them.
\begin{optionlist}
\optname{epsilon}
Default set. Linearize with respect to epsilon. Disable with -no-epsilon.
\nextopt
\optdefault{error}{add | prop | propadd}
Only relevant if -no-epsilon is set. Use an approximate linearization of the error model instead of an exact.\\
Alternatives are;
\begin{itemize}
\item add (for additive
\item prop (for proportional)
\item propadd (for proportional plus additive)
\end{itemize}
The error model must be defined in a particular way when this option is used. See the scm\_userguide.pdf for details.
\nextopt
\optname{foce}
Default set. Expand around conditional ETA estimates instead of around ETA=0.  
\nextopt
\optname{keep\_covariance}
The default setting will delete \$COVARIANCE from the bootstrap models, to save run time. If option -keep\_covariance is set, PsN will instead keep \$COVARIANCE.
\nextopt
\optname{nointer}
Default off. Don't use interaction. 
\nextopt
\end{optionlist}

\subsection{PsN common options}
For a complete list see common\_options.pdf or type psn\_options -h on the command line.

\section{Output}
Linearized model file with name run10\_linbase.mod and data file with name run10\_linbase.dta.

\references

\end{document}
