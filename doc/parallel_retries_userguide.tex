\documentclass[a4wide,12pt]{article}
%\setlength{\marginparwidth}{0pt}%35
%\setlength{\marginparsep}{0pt}%?
%\setlength{\evensidemargin}{0pt}
%\setlength{\oddsidemargin}{0pt}
\usepackage{lmodern}
\usepackage[utf8]{inputenc}
\usepackage[T1]{fontenc}
\usepackage{textcomp}
\usepackage{verbatim}
\usepackage{enumitem}
\usepackage{longtable}
\usepackage{alltt}
\usepackage{ifthen}
% Reduce the size of the underscore
\usepackage{relsize}
\renewcommand{\_}{\textscale{.7}{\textunderscore}}

\newcommand{\guidetitle}[1]{
\title{#1\\ \vspace{2 mm} {\large PsN 4.1.1}}
\date{2014-02-10}
}

\newcommand{\doctitle}[1]{
\title{#1}
\date{2014-02-10}
}


\newenvironment{optionlist}{
\renewcommand{\arraystretch}{1.1}
\setlength{\leftmargini}{2.5cm}
\begin{description}
%\setlength{\itemsep}{0ex}
}
{\end{description}}

\newcommand{\optname}[1]{\item{{\bfseries\texttt-#1}\newline}}
\newcommand{\optdefault}[2]{\item{{\bfseries\texttt-#1}{\mbox{ = \it #2}}\newline}}

\newcommand{\nextopt}{}

\guidetitle{PARALLEL\_RETRIES user guide}{2018-01-31}

\begin{document}

\maketitle
\newcommand{\guidetoolname}{parallel\_retries}


\section{Introduction}
The parallel\_retries tool allows you to run multiple copies of a single model with tweaked initial estimates in parallel.
In PsN terminology it is called a retry to (re)run a model with tweaked initial estimates.

The retries are always done serially in execute and other PsN tools. The parallel\_retries tool takes a single model as input, creates 'min\_retries' extra copies of this model with tweaked initial estimates, and then runs the original plus the extra models in parallel in separate NM\_run subdirectories. Results are summarized in raw\_results.csv and parallel\_retries\_results.csv in the run directory. The tool will select the best retry using the same criteria as execute, and copy back output files (NONMEM files, table files) from the selected retry to the input model's directory.

The parallel\_retries tool accepts all options in common\_options.pdf.\\ 

Example:
\begin{verbatim}
parallel_retries run33.mod -min_retries=4 -threads=5 -seed=12345
\end{verbatim}

\section{Input and options}
\subsection{Required input}
A model file is required on the command line. In addition, either -min\_retries or -rawres\_input is required. The tool takes the same input as the execute tool, see execute -h.
\begin{optionlist}
\optdefault{min\_retries}{integer}
The number of model copies with tweaked initial estimates to run. If this option is not used then -rawres\_input must be set.
\nextopt
\optdefault{rawres\_input}{filename}
Default not set. An alternative way to estimate with different initial estimates. Instead of using initial estimates from a random perturbation, take parameter initial values from a raw\_results-like file. The raw results file must contain at least as many samples as the input -samples to parallel\_retries, the labels for  THETA/OMEGA/SIGMA in the file must match the labels in the simulation model given as input to parallel\_retries, the theta columns must be directly followed by the omega columns which must be directly followed by the sigma columns, and the first column must have header model just as a bootstrap raw\_results file. Note that is is possible to generate a file with initial parameter estimates outside of PsN, as long as the file follows the format rules.
\nextopt
\end{optionlist}
\subsection{Optional input}
\begin{optionlist}
\optdefault{in\_filter}{comma separated list of conditions}
Default not set. Only relevant in combination with rawres\_input. The parameter estimates lines in the file can be filtered on values in the different columns. When specifying which column(s) the filtering should be based on, the exact column name must be used, e.g. minimization\_successful. Filtering can only be based on columns with numeric values. The allowed relations are .gt. (greater than), .lt. (less than) and .eq. (equal to). Conditions are separated with commas. If the remaining number of lines after filtering is smaller than -samples, parallel\_retries will stop with an error message. Then the user must either change the filtering rules or change -samples. If the user has created a file with parameter estimates outside of PsN, filtering can be done on any numeric column in that file. Do not set column headers containing .eq. or .lt. or .gt.in the user-generated file as this would interfere with the in\_filter option syntax.

Example (all code on the same line):
\begin{verbatim}
-in_filter=minimization_successful.eq.1,
significant_digits.gt.3.5
\end{verbatim}
\nextopt
\optname{keep\_nm\_output}
Default not set. If set then do not clean nm\_output files for the non-chosen retry models from the main run directory.
\nextopt
\optname{keep\_table\_files}
Default not set. If set then do not clean table files for the non-chosen retry models from the main run directory.
\nextopt
\optname{model\_dir\_name}
Default not set. This option changes the basename of the run directory from parallel\_retries\_dir to $\langle$modelfile$\rangle$.dir. where $\langle$modelfile$\rangle$ is the name of the (first) input model file, without the extension. The directories will be numbered starting from 1, increasing the number each time parallel\_retries is run with a model file with the same name. If the option directory is used it will override -model\_dir\_name.
\nextopt
\optdefault{offset\_rawres}{N}
Default is 1. Only valid in combination with -rawres\_input. The number of result lines to skip in the input raw results file before starting to read final parameter estimates. In a regular bootstrap raw\_results file the first line of estimates refers to the input model with the full dataset, so therefore the default offset is 1.
\nextopt
\optname{prepend\_options\_to\_lst}
Default not set. This option makes PsN prepend the final lst-file (which is copied back to the directory from which parallel\_retries was called) with the file version\_and\_option\_info.txt which contains run information, including all actual values of optional PsN options. PsN can still parse the lst-file with the options prepended, so the file can still be used it as input to e.g. sumo, vpc or update\_inits.\\ Disabled with -no-prepend\_options\_to\_lst if set in psn.conf.
\nextopt
\optdefault{samples}{N}
Default is to use all sets after skipping the first 'offset\_rawres' sets. Only valid in combination with -rawres\_input. The number of parameter sets to use from the rawres\_input file. 
\nextopt
\optname{timestamp}
Default not set. Automatically set the run directory name with a time stamp and the stem of the model name.\\
Example: directory name run1-PsN-2014-06-12-152502 for a run that was started at 15:25:02 June 12th in year 2014. If the option directory is used it will override -timestamp.
\nextopt
\optname{update\_fix}
Default not set. Only relevant in combination with rawres\_input. By default, FIX values in the input model are not updated based on the rawres\_input file, but if -update\_fix is set then also FIX values will be updated.
\nextopt
\end{optionlist}

\subsection{Retries in PsN}
A NONMEM run is not always successful and sometimes it is possible to improve the results of the run. 
A simple procedure is to tweak the initial 
estimates of the parameters and see if changed starting condition gives better results. 
The -tweak\_inits and -retries options turn on this feature in PsN, 
which is off by default. When the feature is on and the minimization fails, PsN will pick a random value within 10\% above or below each initial value, 
and run the model again with these new initial estimates. This is called a retry. PsN can do several retries, and each time the bounds is 
increased by 10\%. Note that PsN always respects the upper and lower bound set in the model file. 

The retries functionality is adapted to control streams with a single \$PROBLEM. If there are more than one \$PROBLEM, PsN will
locate the \emph{first \$PROBLEM that runs estimation in its last \$EST},
i.e. the NONMEM lst-file has the line 
\begin{verbatim}
ESTIMATION STEP OMITTED:                 NO
\end{verbatim}
for the last \$EST in that \$PROBLEM. PsN will use the
termination status and ofv of
that estimation to decide if a retry should be initiated
and also for selecting the best try. Subsequent \$PROBLEMs that also have \$EST will be ignored in the retries procedure.
If no \$PROBLEM runs estimation in the last \$EST, PsN will instead use
the \emph{first \$PROBLEM that runs evaluation in its last \$EST},
i.e. the NONMEM lst-file has the line 
\begin{verbatim}
ESTIMATION STEP OMITTED:                 YES
\end{verbatim}
for the last \$EST in that \$PROBLEM. In this case retries will only be initiated if forced using option -min\_retries, see below,
and the ofv from the evaluation will be used to select the best retry.

If there are multiple \$PROBLEM and PsN concludes that the initial estimates should be tweaked, PsN will first try with the
\$PROBLEM used for decision-making, but if no parameters were found in that \$PROBLEM, e.g. if \$MSFI was used, PsN will
try with the previous \$PROBLEM(s), one by one, until either PsN finds parameters to tweak, or there are no more \$PROBLEMs to try.
No retry will be initiated if PsN cannot find any parameters to tweak.

The control files and outputs for the retries are found in the NM\_run1 subdirectory. The files psn-1.mod and psn-1.lst are for 
the first run which is always performed and not called a retry. psn-2.mod and up are the control files with perturbed initial 
estimates. In the same directory as psn-1.mod, psn-2.mod etc are created, there is a file called stats-runs.csv. In there is a 
set of parameters+values for the set of runs, in the order given by the modelfile numbers. As the last line is 
written "Selected ..." where it says which model was judged as the best of all the retries.

The lst-file for the selected model is then copied to psn.lst (no number) in the same directory, and also copied back 
up to the working directory. Same principle for table files. The file psn-x.mod for the selected model is copied 
to psn.mod (for PsN version 3.x.x and up), but psn.mod is not copied back up.   	

The string MINIMIZATION SUCCESSFUL is important when PsN decides whether to make a retry. With new estimation 
methods in NONMEM7, that string will not appear. The flag for minimization\_successful is set or unset using the following logic:

\begin{enumerate}
\item Only status of last \$EST step is considered, except when last \$EST is IMP with EONLY=1
\item BURN-IN/(REDUCED) STATISTICAL PORTION/OPTIMIZATION NOT TESTED - set
\item BURN-IN/(REDUCED) STATISTICAL PORTION/OPTIMIZATION COMPLETED -  set
\item BURN-IN/(REDUCED) STATISTICAL PORTION/OPTIMIZATION NOT COMPLETED PRIOR TO USER INTERRUPT - set
\item BURN-IN/(REDUCED) STATISTICAL PORTION/OPTIMIZATION NOT COMPLETED - unset
\item If any of the two steps in SAEM failed - unset 
\item If last \$EST is IMP with EONLY=1, the minimization status is determined by the next to last \$EST
\end{enumerate}

\subsubsection{Controlling retries using PsN options}
The retries option, which defaults to 0, controls the maximum number of retries before giving up. Note that it is 
possible to set a different default in the psn.conf, the PsN configuration file. The option -min\_retries, 
with default 0, controls the minimum number of retries that PsN is forced to make, and has precedence over -retries. 
Option -min\_retries is useful if the user suspects that there is a risk of finding local minima.

Normally PsN will be satisfied with a run that has minimization successful, but PsN can be more picky about the 
quality of NONMEM results. If the -picky option is used PsN will do a retry even if the minimization is successful, 
given that a 'picky-message' appears in NONMEMs minimization message, see the list of picky-messages below by the picky option description.

If option -reduced\_model\_ofv is set to some value (option currently only available in execute),
PsN will do a retry if the current ofv is more than 'accepted\_ofv\_difference' worse than
reduced\_model\_ofv.

It is also possible to reduce PsN's requirements of quality. If the minimization is not successful but the significant 
digits are high enough, PsN can skip doing a retry. The limit is set with option -significant\_digits\_accept. This option 
is by default not used, and it has no effect if picky is set at the same time.

Following is the list of condititions at the end of a NONMEM run that will lead PsN to initiate a retry, provided that 
the maximum number of retries set with option -retries has not been reached:

\begin{enumerate}
\item The minimum number of retries set with -min\_retries have not yet been run.
\item Option -reduced\_model\_ofv is set and the run ofv is larger than reduced\_model\_ofv+accepted\_ofv\_difference
\item Option -picky is set and the run has either not finished with MINIMIZATION SUCCESSFUL, or finished with one of the picky-messages listed below.
\item The run has not finished with MINIMIZATION SUCCESSFUL and option -significant\_digits\_accept is 
either not set or the number of significant digits is less than -significant\_digits\_accept.
\item The run has finished fulfilling conditions from MINIMIZATION SUCCESSFUL/signficicant\_digits/picky/reduced\_model\_ofv, 
but the ofv of this run minus 'accepted\_ofv\_difference' is larger than the ofv of a previous run 
that did not fulfill the conditions, indicating that the current run has terminated in a local minimum.
\end{enumerate}

After all retries have finished, PsN will select the best try. 
The rules for selection are significantly changed in PsN 4.3.9 and later, giving the ofv value more weight than in earlier versions.
In most cases the run with the lowest ofv is selected, but there is a 'accepted\_ofv\_difference' preference for runs
that pass the picky test, if option picky was set, or for minimization successful.
The selection procedure is 
\begin{enumerate}
\item If option -picky was set, and the ofv of the best try that passed the picky test is not more than accepted\_ofv\_difference
worse than the best overall ofv, then choose the best try that passed the picky test. 
\item Otherwise, if the ofv of the best try that had minimization successful is not more than accepted\_ofv\_difference
worse than the best overall ofv, then choose the best try that had minimization successful.
\item Otherwise, chose the try with the best overall ofv, if any tries at all gave an ofv.
\item Otherwise, chose the first try.
\end{enumerate}

\begin{optionlist}
\optname{tweak\_inits}
Default set, can be disabled with -no-tweak\_inits. 
If NONMEM terminates nonsuccessfully, PsN can perturb the initial estimates and run NONMEM again. The generation of new initial estimates init\_i for the i:th retry are performed according to init\_i = init\_0 + rand\_uniform(+-degree*init\_0)where init\_0 are the initial estimates of the original run
and degree is set with option -degree, see below. 
The updating procedure makes sure that boundary conditions on the parameters are respected. 
For this option to have effect, the -retries option must be set to a number larger than zero. 
\nextopt
\optdefault{degree}{'number'}
Default 0.1. A number larger than 0. This number decides how big the perturbation will
be in tweak\_inits.
\nextopt
\optdefault{retries}{'integer'}
Default 0. The -retries option tells PsN how many times it shall try to rerun a NONMEM job if it fails according to given criterias. 
The -retries option is only valid together with -tweak\_inits. 
\nextopt
\optdefault{min\_retries}{'integer'}
Default 0. Option min\_retries forces PsN to try make extra retries even if a run has already terminated successfully,
or if estimation is not run at all (e.g. MAXEVAL=0). Precedence over -retries option.  
\nextopt
\optname{picky}
Default not used. The -picky option is only valid together with -tweak\_inits. Normally PsN only tries new initial estimates if 'MINIMIZATION SUCCESSFUL' is not found in the NONMEM output file. With the -picky option, PsN will regard any of the following messages, the 'picky-messages',  as a signal for rerunning:


\begin{verbatim}
0ESTIMATE OF THETA IS NEAR THE BOUNDARY
0PARAMETER ESTIMATE IS NEAR ITS BOUNDARY
0R MATRIX ALGORITHMICALLY SINGULAR
0S MATRIX ALGORITHMICALLY SINGULAR
\end{verbatim}
\nextopt
\optdefault{significant\_digits\_accept}{'number'}
Default not used. Has no effect in combination with -picky. The -significant\_digits\_accept option is only valid together with option -tweak\_inits. Normally PsN tries new initial estimates if 'MINIMIZATION SUCCESSFUL' is not found in the NONMEM output file. With the -significant\_digits\_accept, PsN will only rerun if the resulting significant digits is lower than the value specified with this option. 
\nextopt
\optdefault{accepted\_ofv\_difference}{'number'}
Default 0.5. This option is used by PsN when deciding if a retry should be run, and when selecting the best retry out of the whole set. 
This option decides how much preference should be given to runs that fulfill the picky conditions/have minimization successful 
but a slightly higher ofv (at most accepted\_ofv\_difference) than a run that did not fulfill the conditions.  
\nextopt
\optdefault{reduced\_model\_ofv}{'number'}
This option is used by PsN when deciding if a retry should be run. It is currently only available with the execute program,
so it is strictly speaking not a common option.
A retry will be initiated if the ofv of the current run is more than 'accepted\_ofv\_difference' worse than the ofv
given with this option.
\nextopt
\optname{add\_retries}
Default not set. By default, PsN will never do retries on a model when a run is restarted if the file stats-runs.csv is found in the NM\_run subdirectory, since the existence of this file indicates that all retries have finished and the best try has already been selected. If option -add\_retries is set, PsN will ignore that stats-runs.csv exists, and check again if retries are needed based on the existing files in the NM\_run directory. This makes it possible to restart a run using new settings for retries (e.g. -retries, -min\_retries, -picky) compared to the original run. If the original run had clean$<=1$, PsN will count the previously run tries when relating the number of tries to options -min\_retries and -retries. If clean=2, PsN will not count the previously run tries, but psn.mod which in reality is the finally selected try from the original run will be counted as the first try in the restart.  
\nextopt
\end{optionlist}


\subsection{PsN common options}
For a complete list see common\_options.pdf or type psn\_options -h on the command line.

\section{Output}
When the NONMEM runs are finished, a summary of the results of the retries are found in the raw\_results file and parallel\_retries\_results.csv in the main run directory.

\end{document}
