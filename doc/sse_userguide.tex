\documentclass[a4wide,12pt]{article}
%\setlength{\marginparwidth}{0pt}%35
%\setlength{\marginparsep}{0pt}%?
%\setlength{\evensidemargin}{0pt}
%\setlength{\oddsidemargin}{0pt}
\usepackage{lmodern}
\usepackage[utf8]{inputenc}
\usepackage[T1]{fontenc}
\usepackage{textcomp}
\usepackage{verbatim}
\usepackage{enumitem}
\usepackage{longtable}
\usepackage{alltt}
\usepackage{ifthen}
% Reduce the size of the underscore
\usepackage{relsize}
\renewcommand{\_}{\textscale{.7}{\textunderscore}}

\newcommand{\guidetitle}[1]{
\title{#1\\ \vspace{2 mm} {\large PsN 4.1.1}}
\date{2014-02-10}
}

\newcommand{\doctitle}[1]{
\title{#1}
\date{2014-02-10}
}


\newenvironment{optionlist}{
\renewcommand{\arraystretch}{1.1}
\setlength{\leftmargini}{2.5cm}
\begin{description}
%\setlength{\itemsep}{0ex}
}
{\end{description}}

\newcommand{\optname}[1]{\item{{\bfseries\texttt-#1}\newline}}
\newcommand{\optdefault}[2]{\item{{\bfseries\texttt-#1}{\mbox{ = \it #2}}\newline}}

\newcommand{\nextopt}{}

\guidetitle{SSE user guide}{2018-01-31}

\begin{document}

\maketitle
\newcommand{\guidetoolname}{sse}


\section{Introduction}
The sse (stochastic simulation and estimation) tool is for model comparison and hypothesis testing.
\begin{itemize}
\item First, using the input model, a number of simulated datasets are generated.
\item Second, the input model and a set of alternative models are fitted to the simulated data.
\item Finally, a set of statistical measures are computed for the parameter estimates and objective function values of the various models.
\end{itemize}
Example:
\begin{verbatim}
sse run1.mod -samples=1000 -alternative_models=alt1.mod,alt2.mod
\end{verbatim}

\section{Input and options}
\subsection{Required input}
A model file, the simulation model, and the option -samples are required on the command line.
\begin{optionlist}
\optdefault{samples}{N}
The number of simulated datasets to generate is a required option. N must be greater than or equal to 1. 
\nextopt
\end{optionlist}
\subsection{Optional input}
\begin{optionlist}
\optname{add\_models}
This option tells PsN to add the alternative models listed with option -alternative\_models to an old sse run. All models given via option -alternative\_models will be estimated from scratch, so the alternatives from the old run should not be listed in the input again. The -add\_models option requires that the old sse run has been completed without errors. It is necessary to also set the general PsN option -directory to the sse directory of the old run, e.g. -directory=sse\_dir50. The user must ensure that the -samples option in the new sse run is equal to or smaller than in the old run. 

If some simulated datasets are missing it will result in an error. The simulation model must still be given as input on the command line. If the option -estimate\_simulation is set (it is set by default) old estimation results will be reloaded if they exist, otherwise the simulation model will be estimated. The numbering of the extra models will start at the number of the last old alternative plus 1. Results for the added models are in file sse\_results\_add1.csv.
\nextopt	
\optdefault{alternative\_models}{alt1.mod,alt2.mod, ...}
A comma-separated list of one or more model files to use for estimation with simulated datasets. If not given, only the input model will be fitted to the simulated data. 
\nextopt
\optdefault{append\_columns}{col1,col2,...}
A comma-separated list of one or more data columns to append to the simulated data sets. These variables must be defined in the simulation model, otherwise all the simulation models will crash with an NMtran-error. PsN will not check that the variables are defined, but simply append them to the \$TABLE generating the simulated datasets. 

This option cannot be used in combination with -add\_models, since in that case no simulations will be run. The user must ensure that the estimation models have these columns properly included in \$INPUT, PsN will not add them.  
\nextopt
\optname{estimate\_simulation}
Default set. By default, the simulation model is also used for estimation with the simulated datasets. The resulting OFV values are used as reference when evaluating the estimation results of alternative models. By setting -no-estimate\_simulation the estimation of the simulation model is turned off, and the first alternative model is used as reference instead. Also see -ref\_ofv.

Regardless of how option -no-estimate\_simulation is set, the initial estimates in the simulation model will be used as reference when computing bias, rmse and rsebias for theta, omega and sigma. If simulating with uncertainty then the reference values will be different for each simulated dataset. 
\nextopt
\optdefault{in\_filter}{comma-separated list of conditions}
Default not set. Option is only relevant in combination with -rawres\_input. The parameter estimates lines in the file can be filtered on values in the different columns. When specifying which column(s) the filtering should be based on, the exact column name must be used, e.g. minimization\_successful. Filtering can only be based on columns with numeric values. 

The allowed relations are:
\begin{itemize}
	\item .gt. (greater than)
	\item .lt. (less than)
	\item .eq. (equal to)
\end{itemize}
Conditions are separated with commas. If the remaining number of lines after filtering is smaller than -samples,  sse will stop with an error message. Then the user must either change the filtering rules or change -samples. If the user has created a file with parameter estimates outside of PsN, filtering can be done on any numeric column in that file. Do not set column headers containing .eq. or .lt. or .gt.in the user-generated file as this would interfere with the in\_filter option syntax.
\newpage
Example (there must be no linebreaks in the actual command):
\begin{verbatim}
-in_filter=minimization_successful.eq.1,
significant_digits.gt.3.5
\end{verbatim} \\
\nextopt
\optname{keep\_tables}
By default, all pre-existing \$TABLE will be deleted from the simulation and alternative models, to save disk space. If option -keep\_tables is set, 
PsN will instead keep \$TABLE and number the file names according to the sample number.
\nextopt
\optdefault{offset\_rawres}{N}
Default is 1. Only relevant in combination with rawres\_input. The number of result lines to skip in the input raw results file before starting to read final parameter estimates. In a regular bootstrap raw\_results file the first line of estimates refers to the input model with the full dataset, so therefore the default offset is 1.
\nextopt
\optdefault{out\_filter}{comma-separated list of conditions}
Default not set. The user may choose to only compute results based on estimations which fulfill certain conditions. The default is to only skip runs where the ofv cannot be read from the lst-file or is equal to 0. Filtering of output can be done on any numeric column in a standard sse raw\_results file, for example minimization\_successful, significant\_digits 
and covariance\_step\_successful. 

The allowed relations are:
\begin{itemize}
	\item .gt. (greater than)
	\item .lt. (less than)
	\item .eq. (equal to)
\end{itemize}
If the value in the filter column is 'NA' then that parameter set will be skipped, regardless of the defined filter relation. Conditions are separated with commas. If the remaining number of estimation results after filtering is less than 2, sse will stop with an error message.

Example (there must be no linebreaks in the actual command):
\begin{verbatim}
-out_filter=minimization_successful.eq.1,
significant_digits.gt.3.5
\end{verbatim} \\
\nextopt
\optdefault{parallel\_simulation}{X}
Default is the same value as set for option -threads. The number of parallel processes to start for the simulation step (not the estimation step) on a parallel computer. 
\nextopt
\optname{random\_estimation\_inits}
This option can only be used in combination with -rawres\_input. It turns off simulation with uncertainty and instead uses the parameter values from the rawres\_input file as initial values for the estimation step. When this option is \emph{not} used, the estimation initial values will be the same as the values used for simulating data.
\nextopt
\optdefault{rawres\_input}{filename}
A simple way to simulate with uncertainty. Instead of using identical parameter estimates for simulation of each new dataset, take parameter estimates from a raw\_results.csv file, e.g. from a bootstrap run or the initial\_estimates.csv file from a previous sse run with \$PRIOR in the simulation model. The raw results file must contain at least as many samples as the input -samples to sse, the labels for THETA/OMEGA/SIGMA in the file must match the labels in the simulation model given as input to sse, and the theta columns must be directly followed by the omega columns which must be directly followed by the sigma column. Note that it is possible to generate a file with initial parameter estimates outside of PsN, as long as the file follows the format rules. 
\nextopt
\optdefault{recompute}{raw results filename, including directory name}
Default not set. Setting this option makes PsN recompute output statistics based on the specified raw\_results file. Note that the filename must be given including the directory name. The user may change the -out\_filter settings for the recomputation. Apart from -out\_filter, the input model must be set, as well as the option -samples. Alternative models are not needed, information about them will be read from the raw results file. Option -directory will be ignored, instead the directory specified as part of the file path will be used.
Example:
\begin{verbatim}
-recompute=sse_dir12/raw_results_run1.csv
\end{verbatim}
\nextopt
\optdefault{ref\_ofv}{500}
Instead of using the OFV values from the estimation of a model as reference when evaluating the other estimation results, it is possible to set a fixed reference OFV value. If using ref\_ofv, it is not allowed to also estimate the simulation model.
\nextopt
\optname{update\_fix}
Default not set. Option is only relevant in combination with -rawres\_input. If update\_fix is set, FIXED initial estimates in models will be updated to values from the rawres\_input file. If update\_fix is not set, the default, FIXED initial estimates will not be changed.
\end{optionlist}

\subsection{PsN common options}
For a complete list see common\_options.pdf or type psn\_options -h on the command line.

\subsection{Auto-generated R plots from PsN}
\newcommand{\rplotsconditions}{Parametric power estimation (PPE) allows to generate data for full power versus sample size curve(s) based on a sse with full and reduced models performed with a single sample size. The procedure is described in \cite{Ueckert}. The default sse template will perform all required PPE calculations in R and output the resulting power curve(s). For the template to work, the R libraries ggplot2 and plyr need to be installed. Furthermore, it is recommended 
to indicate the reduced model associated with a specific full model by adding the suffix `red' (i.e. `run1red.mod' is the reduced model associated with the full model `run1.mod'). If no full and reduced model pair with the expected naming scheme is found, matching will occur by position (i.e., estimation model 1 will be taken to be the full model and estimation model 2 the associated reduced model, same for 3 and 4, etc.).}

PsN can automatically generate R plots to visualize results for \guidetoolname, using a default template found in the R-scripts subdirectory of the installation directory. The user can also create a custom template, see more details in the section Auto-generated R-plots from PsN in common\_options.pdf.

\rplotsconditions

\begin{optionlist}
\optdefault{rplots}{level}
-rplots<0 means R script is not generated\\ 
-rplots=0 (default) means R script is generated but not run\\ 
-rplots=1 means basic plots are generated\\													  
-rplots=2 means basic and extended plots are generated\\													  
\nextopt
\end{optionlist}

\subsubsection*{Troubleshooting}
If no .pdf was generated even if a template file is available and the appropriate options were set, check the .Rout-file in the main run directory for error messages. If no .Rout-file exists, then check that R is properly installed, and that either command 'R' is available or that R is configured in psn.conf.


\subsubsection*{Basic R plots}
The basic PPE R plot will be generated if option -rplots is set >0, and the general R plots conditions fulfilled, see above.
The R plot shows, for each of the provided full/reduced model pairs, a color-coded power versus sample size curve together with a shaded region indicating the uncertainty in the power estimate due to Monte-Carlo noise (-samples option in PsN). 
By default power curves are drawn from a sample size of 1 to the samples size of 99\% power. 

\subsubsection*{Extended R plots}
Extended PPE R plots, will be generated if -rplots>1, and are on the second page of the pdf-output. The extended R plots are intended as a diagnostic to evaluate the assumptions of the PPE algorithm. They show separately for each full/reduced model pair the empirical cumulative distribution function (ECDF) of the $\Delta$-OFVs together with a shaded area indicating the 95\% confidence region for the ECDF provided the underlying distributional assumptions of the PPE algorithm are true. The results of the algorithm should be re-evaluated through an alternative method for power estimation when the ECDF curve lies outside the confidence interval. 

\subsubsection*{Special cases}
To simulate odd-type data, relevant code (with a simulation block) is needed in the input model file, as well as a \$SIMULATION row. It should contain one ordinary (from a normal distribution) seed number and one extra seed number (from a uniform distribution) “\$SIMULATION (11111) (11111 UNIFORM)”. The random number generator handled by PsN will replace the seed numbers in a controlled manner (derived from “–seed=” if provided).

Please note that NONPARAMETRIC is not yet completely supported, since it requires additional results handling features.

\subsection{Simulation with uncertainty}
Simulation with uncertainty is supported using three methods. 
\begin{itemize}
	\item One method is to use options -rawres\_input and -offset\_rawres, see above.
	\item The second method is to define \$PRIOR NWPRI in the input model, including the PLEV option required when simulating from a prior. Please note that the update\_inits script has some functionality for automatically adding \$PRIOR NWPRI to a model, based on output from an estimation. However the update\_inits feature is still experimental and the generated model needs to be manually checked.
	\item The third method is to define \$PRIOR TNPRI in the input model, including the PLEV option required when simulating from a prior. If no \$SIM record is present in the input model containing \$PRIOR, PsN will create one and add option TRUE=PRIOR. If PLEV option is not set in \$PRIOR, NONMEM will halt saying that the value of PLEV is inappropriate. When \$PRIOR is present in the input model, PsN will store the initial estimates sampled from the prior in the file initial\_estimates.csv. This file is formatted so that it can be used with option -rawres\_input in a later sse run. The first row, model 0, in initial\_estimates.csv is the initial estimates of the input model, and these will be skipped if -offset\_rawres is equal to the default value.
	
	If \$PRIOR TNPRI is set, then NONMEM requires \$MSFI in the first \$PROB. PsN will automatically copy the msf-file to the run directory, meaning that it is not necessary to set PsN option -extra\_files for the msf-file.
\end{itemize}  


\section{Output}
The output from each sse run is collected in a new directory with a name of the form sse\_dirX, where X is an integer. The most recent sse run is found in the directory with the highest number. The only exception is when the user restarts a run from a specified existing directory using the -directory option, then the results will be saved in that directory. 

When \$PRIOR is present in the input model, PsN will store the initial estimates sampled from the prior in the file initial\_estimates.csv, see Simulation with uncertainty above.

sse creates two files summarizing the results that can be opened in e.g. Excel. The raw\_results.csv file is a standard PsN file containing raw result data for each estimation run (termination status, parameter estimates, uncertainty estimates, etc…). The sse\_results.csv file is a file specific to the sse routine containing summary statistics and comparisons common to all estimation runs. 

The header line for the reference/true values uses complete indexes for OMEGA and SIGMA, e.g OMEGA(2,1), SIGMA(3,3). The header lines for the input model and the alternative models use naïve numbering, for example if OMEGA(1,1), OMEGA(2,2), OMEGA(3,2) and OMEGA(3,3) are defined in the modelfile, they will in the sse output be numbered OM\_1, OM2, OM\_3 and OM\_4. See also section Principles of statistical analyses. 

\subsection{Explanation for data items in sse\_results.csv:}
The 'true' values in the formulas below are always the initial parameter estimates of the simulation model. When simulating without uncertainty the true value is constant over i, but when simulating with uncertainty the true value varies with index i. If for any i the true value is equal to 0 in a computation of a relative value, then that sample is skipped and N is adjusted. 
\emph{sd = standard deviation of estimated parameter $x_i$.}

\[
Skewness = \frac{N}{(N-1)\cdot(N-2)} sd^{-3} \sum_{i=1}^N(x_i - \bar{x})^3
\]
\[
Kurtosis = \frac{N\cdot(N+1)}{(N-1)\cdot(N-2)\cdot(N-3)} sd^{-4} \sum_{i=1}^N(x_i - \bar{x})^4 - \frac{3\cdot(N-1)^2}{(N-2)\cdot(N-3)}
\]
\[
rmse = \sqrt{\frac{1}{N} \sum_i(est_i - true_i)^2}
\]
\[
relative\\_rmse = 100\% \sqrt{\frac{1}{N} \sum_i \frac{(est_i - true_i)^2}{true_i^2}}
\]
\[
bias = \frac{1}{N} \sum_i(est_i - true_i)
\]
\[
relative\verb|_|bias = 100\% \frac{1}{N} \sum_i \frac{est_i - true_i}{true_i}
\]
\[
relative\verb|_|absolute\verb|_|bias = 100\% \frac{1}{N} \sum_i \frac{est_i - true_i}{|true_i|}
\]
\[
Rse_{bias} = relativestandarderror = 100\% \frac{sd}{True \sqrt{N}}
\]


Standard error $CI_{bias}$ = parametric confidence intervals = mean bias +/- Z*rse where Z = [2.58, 1.96, 1.645] for CI = [99\%, 95\%, 90\%]
Items under the OFV statistics heading by default refer to likelihood ratio test comparisons between the input model and the alternative models. The exceptions are described under section Principles of statistical analyses below.


\section{Known problems}
The sse tool will use the wrong simulated data for reestimation if the simulation model file has more than one \$INPUT statement, the table file with simulated data used for reestimation will only include the items from the last \$INPUT.
Note: It works fine to split \$INPUT over several lines, as long as the text \$INPUT only appears once.

The simulated datasets are \$TABLE output from NONMEM, and NON-MEM formats and rounds off values when printing tables. This leads to two known problems.

\begin{enumerate}
	\item In NONMEM 6 1013201 is rounded to 1013200 (five significant digits), and if this makes a significant change to the model estimation, for example if the value is a covariate, then the sse results will be wrong. 
	
	In NONMEM 7 it is possible to set the FORMAT or RFORMAT option in \$TABLE to make sure no important information is lost. With NONMEM 6 the user must make sure the rounding to five significant digits does not harm the results. PsN cannot detect this problem.

	\item A column of integers in the input dataset, e.g. OCC for occasion, will be formatted as a floating point number in table output. When the table output is used as input in the estimation step of sse, statements like IF (OCC .EQ. 1) will not work, because to NONMEM 1 is not equal to 1.000000e00. As a workaround the user must write code that will work both for integers and floating point numbers, for example (IF OCC .GT. 0 AND OCC .LT. 2). Later versions of NONMEM support ID.EQN.1, which will be true both for 1 and 1.000e00.
\end{enumerate}

\section{Recovering a crashed/stopped sse}
If the sse run is halted before the simulations are finished and option -clean is <=2, it is possible to reuse the already finished runs. Run the same command as before, set option -directory if it was not set in the original call. Please note that PsN will print a message “Starting NNN NONMEM executions” that indicates that all simulations are rerun, but the check for reusable results is done after this message is printed. Some copying of results is done, and some existing result files will be overwritten with identical copies. 

If the sse run is halted after the simulations are finished and option -clean is <=2, PsN can reuse the results from both simulations and finished estimations. Run the same command again with -directory set. Do not set the rerun option. PsN will print that all NONMEM executions are started even if results are reused, see above. 

If the sse run is stopped during the estimation phase when -clean=3, partial results can still be recoved but it requires a number of steps.
\begin{enumerate}
\item Change directory to the m1 subdirectory of the sse run directory.
\item Run execute on all .mod-files in the m1-subdirectory that do not already have a .lst-file. Note that execute accepts multiple .mod-files in a single run.
\item Run the rawresults script (no options) in the m1 directory after all execute runs have finished.
This creates a raw\_results file based on the output in the m1 subdirectory.
\item Change directory to outside the main sse directory and run sse with option -recompute (see help text for this option) to compute the final results.
\end{enumerate}

\section{Technical overview of algorithm}
The sse tool creates N number of samples simulation model files that are modified copies of the input model file given on the command line. Then the program creates N*M (M=number of alternative model files + 1 (for the input model file)) estimation model files. Each model file has one estimation copy per simulated dataset. N simulated datasets are generated by running NONMEM once for each simulation model file, and then each simulated dataset is used for parameter estimation M times, once with each model (simulation + alternatives). The estimated parameters and OFV values from the M*N runs are collected in raw\_results.csv together with other NONMEM outputs, and statistical measures are written to sse\_results.csv. 

\subsection{Important features of the N simulation model files generated from the input model}
\begin{itemize}
	\item In \$DATA record IGNORE and ACCEPT statements are kept intact.
	\item If the input model contains a \$SIMULATION record, the record is kept intact, except that the random seeds are set to different values in each copy, option NSUBS/SUBPROBLEMS is removed, and it is made sure that TRUE=PRIOR is set if the input model has \$PRIOR. If there is no existing \$SIMULATION record, the program creates a new one according to \$SIMULATION ($<$random seed$>$) ONLYSIMULATION and adds TRUE=PRIOR if the input model has \$PRIOR.
	\item If any \$ESTIMATION record contains LIKELIHOOD, -2LOGLIKELI-\\HOOD, -2LLIKELIHOOD or LAPLACIAN, then ONLYSIMULATION NOPREDICTION is set in \$SIMULATION.  
	\item All \$ESTIMATION records are removed. 
	\item Any \$COVARIANCE record is removed. 
	\item If option -keep\_tables is not set, old \$TABLE records are removed. Otherwise old \$TABLE records are kept, and the filename in option FILE=$<$filename$>$ will have trailing numberes removed and then be numbered -sim-i, where i is the order number of the simulation model file. 
	\item In the newly generated \$TABLE record for the simulated dataset the option FILE=$<$simdata$>$ is set, where $<$simdata$>$ is a unique name for each simulation model file.
	\item If option -rawres\_input is set, then replace the THETA/OMEGA/SIGMA parameter values in the ith simulation model with the final estimates from the ith results line in the input raw\_results file.
\end{itemize}

\subsection{Important features of the N estimation model files generated from the input model and of the estimation model files generated from alternative model files (N files per alternative model)}
\begin{itemize}
	\item The N model files generated from the input model are numbered with the number of the simulated dataset used as input in the modelfile. The files generated from the alternative models are numbered with X-Y, where X is the order number of the alternative model and Y is the order number of the simulated dataset.
	\item In \$DATA record, IGNORE is set to @ (i.e. ignore text lines), replacing any old IGNORE=$<$single character$>$. All old IGNORE=(list) and ACCEPT statements are kept. The user must be cautious regarding statistics if IGNORE or ACCEPT statements differ between models.
	\item Any \$SIMULATION record is removed from the first \$PROBLEM.
	\item All \$ESTIMATION records are kept. If MSFO=$<$filename$>$ is specified, $<$filename$>$ will first have any trailing numbers removed, and then $<$filename$>$ will have the same numbering appended as the name of the modelfile itself (see above).
	\item If option -keep\_tables is not set, old \$TABLE records are removed. Otherwise, in \$TABLE records of the first \$PROBLEM, the filename in FILE=<filename> will have trailing numbers removed and then be numbered as the modelfile, see above. 
	\item Input is set to $<$simdata$>$ from one of the simulation model files. 
	\item If an alternative model has a second \$PROBLEM (this is not allowed for the input model unless the first \$PROB has \$PRIOR TNPRI), then it is assumed that the first \$PROB has MSFO=$<$filename$>$ in \$ESTIMATION and that the second \$PROBLEM has an \$MSFI record. PsN will set the same numbered filename in MSFO and MSFI. It is also assumed that the second \$PROBLEM has a \$TABLE record (it is not checked that -keep\_tables is set, the user must do that). In the first \$TABLE of the second \$PROBLEM PsN will set FILE=simtabX-Y.dat, where X-Y is numbers as above. The filename in \$DATA of the second \$PROB will be set to the same <simdata> as in the first \$PROBLEM, but IGNORE statements will not be changed. This means that the user must ensure that there is an IGNORE=@ in \$DATA of the second \$PROBLEM. PsN will not change the random seeds in \$SIMULATION, but since the simulated datasets used as input are different, the table output will also be different.
	\item If the input model is to be estimated and it has \$PRIOR set, then option PLEV is removed from \$PRIOR.
\end{itemize}

\subsection{Principles of statistical analyses}

All estimated parameters values; thetas, omegas and sigmas, are compared with the values used for generating the simulated datasets. The matching is based on numbering, so the value of theta 3 used for simulation will be compared with all estimated theta 3. It is up to the user to ensure that the matching is correct, i.e. that the simulation value of CL is always compared with the estimated values of CL. 
If for example OMEGA(1,1), OMEGA(2,2), OMEGA(3,2) and OMEGA(3,3) are defined in the modelfile, they will in the sse output be numbered OM\_1, OM2, OM\_3 and OM\_4, i.e. values are numbered naively without considering if they are on the diagonal or not. Same for SIGMA. 
If OMEGA/SIGMA is diagonal in the original model but of block form in an alternative model, the matching will be incorrect. To avoid that error, use block form also in the original model, setting off-diagonal elements to 0 (NONMEM will keep these fixed). If there are more parameters in an alternative model than in the simulation or vice versa, no matching can be done and the spaces for the comparison statistics will be empty. If parameters are different from a model to another one, numbering should be done considering that, or if not comparisons should not be interpreted. 

sse does not check whether minimization was successful or not. Statistical computations include also parameter estimates from NONMEM runs terminated with e.g. rounding errors, unless the option -out\_filter is used, see above. sse will skip all runs where the ofv cannot be read from the lst-file  or is equal to 0. This means that e.g. the mean may be based on different number of runs (samples) for different alternative models. If the ofv is available but a parameter value is missing, then sse will print a warning and compute the statistics without that value. 

If a run neither minimizes nor terminates, but gets into an infinite loop, it won't be killed by PsN; so, in order for the sse results to be computed, a manual kill can be contemplated.

In sse\_results.csv no individual parameter estimates are reported. Mean, median, sd and other measures are the means etc. over the simulated datasets.

The OFV values are treated differently. By default, the OFV value obtained when estimating parameters using the input model is used as the reference for each simulated dataset, and the OFV values obtained when estimating the alternative models are compared with this reference. There are two alternative methods. The first is to set the option -no-estimate\_simulation without setting -ref\_ofv. In this case the OFV-values from the first alternative model are used as references. The second alternative is to set -no-estimate\_simulation together with -ref\_ofv=X, and then X is used as the reference OFV value for all alternative models.

\references

\end{document}
