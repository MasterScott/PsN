\documentclass[a4wide,12pt]{article}
%\setlength{\marginparwidth}{0pt}%35
%\setlength{\marginparsep}{0pt}%?
%\setlength{\evensidemargin}{0pt}
%\setlength{\oddsidemargin}{0pt}
\usepackage{lmodern}
\usepackage[utf8]{inputenc}
\usepackage[T1]{fontenc}
\usepackage{textcomp}
\usepackage{verbatim}
\usepackage{enumitem}
\usepackage{longtable}
\usepackage{alltt}
\usepackage{ifthen}
% Reduce the size of the underscore
\usepackage{relsize}
\renewcommand{\_}{\textscale{.7}{\textunderscore}}

\newcommand{\guidetitle}[1]{
\title{#1\\ \vspace{2 mm} {\large PsN 4.1.1}}
\date{2014-02-10}
}

\newcommand{\doctitle}[1]{
\title{#1}
\date{2014-02-10}
}


\newenvironment{optionlist}{
\renewcommand{\arraystretch}{1.1}
\setlength{\leftmargini}{2.5cm}
\begin{description}
%\setlength{\itemsep}{0ex}
}
{\end{description}}

\newcommand{\optname}[1]{\item{{\bfseries\texttt-#1}\newline}}
\newcommand{\optdefault}[2]{\item{{\bfseries\texttt-#1}{\mbox{ = \it #2}}\newline}}

\newcommand{\nextopt}{}

\guidetitle{BOOT\_SCM user guide}{2018-03-02}

\usepackage{hyperref}

\begin{document}

\maketitle

\newcommand{\guidetoolname}{boot\_scm}
\tableofcontents
\newpage
\section{Introduction}

The boot\_scm, (bootstrap of the scm) tool is an implementation of the method presented in \cite{Keizer2}.
The program depends heavily on the scm tool, and all scm options except base\_ofv apply also to boot\_scm. Please refer to scm\_userguide.pdf for help on scm options.

Examples
\begin{verbatim}
boot_scm config_run1_nonlinear.scm -samples=100 -seed=12345
boot_scm config_run1_linearize.scm -samples=100  -seed=12345 -methodA
\end{verbatim}

\section{Input and options}

\subsection{Required input}
A configuration file is required on the command line. The format of the configuration file follows the format of 
the scm configuration file exactly. 
The input model must be set in the configuration file, it cannot be given on the boot\_scm command line. 

In addition to the configuration file, one command line option is mandatory:
	
\begin{optionlist}
\optdefault{samples}{N}
Mandatory command line option.
The number of bootstrapped datasets to run the scm on. 
\nextopt
\end{optionlist}

\subsection{Optional input}
These options are specific to boot\_scm, and they can only be given on the command line, not in the configuration file.

\begin{optionlist}
\optdefault{dummy\_covariates}{comma-separated list of covariates}
Default not set. If used, a new column for each listed covariate will be added to the dataset, 													  
containing a randomly permuted copy of the original covariate column and with header X\verb|<name of original covariate>|. 
The dummy covariate will be tested for inclusion in the covariate model exactly like the original covariate. 
However, a known bug is that boot\_scm will not correctly create a dummy covariate based on a time-varying covariate.
\nextopt
\optname{methodA}
Default not set. If the scm option linearize=1 is not set in the scm config file, 
the bootstrap scm non-linear method will be used. If option linearize=1 is set in the scm config file, 
by default the bootstrap scm linear method B (see algorithm description below) will be used. 
If option linearize=1 is set together with option -methodA on the boot\_scm command line (no argument to -methodA) 
then the bootstrap scm linear method A will be used. If linearize=1 is not set and option -methodA is set this will result in an error message.  
Setting linearize=1 in the scm config file by default gives linearization using FOCE, for details see the scm userguide.
\nextopt
\optdefault{missing\_data\_token}{string}
Default is -99. This option sets the string that PsN accepts as missing data, and needs to be set correctly when PsN computes summary statistics for data set columns.
\nextopt
\optname{run\_final\_models}
Default not set. If set then boot\_scm will run the final models from each scm on the original dataset and collect the ofv values 
in the output file ofv\_final.csv   
\nextopt
\optdefault{stratify\_on}{item in \$INPUT}
Default not set. It may be necessary to use stratification in the resampling procedure. For example, if the original 
data consists of two groups of patients - say 10 patients with full pharmacokinetic profiles and 90 patients with sparse 
steady state concentration measurements - it may be wise to restrict the resampling procedure to resample within the two 
groups, producing bootstrap data sets that all contain 10 rich + 90 sparse data patients but with different compositions. 
Set -stratify\_on to the column (the name in \$INPUT in the model) that defines the two groups.
\nextopt
\end{optionlist}

\subsection{PsN common options}
For a complete list see common\_options.pdf or type psn\_options -h on the command line.

\section{Algorithm overview}

If IGNORE/ACCEPT is found in \$DATA (not counting single character IGNORE like e.g. IGNORE=@), 
the data will be filtered using a dummy model run in the preprocess\_data\_dir subdirectory of the boot\_scm directory. The new dataset is called
 filtered.dta. A modified input model called orig\_model\_filtered\_data.mod is created where the new dataset is used.

If -dummy\_covariates is set, a modified input model (based on orig\_model\\\_filtered\_data.mod or on the original input model if no filtering was done) 
called model\_with\_xcov.mod is created where the dummy covariates are added in \$INPUT and \$DATA specifies a new dataset called xcov\_$\langle$old 
data name$\rangle$ where the dummy covariates are added. The new model and dataset is created in preprocess\_data\_dir. 

When using method A or Non-linear (i.e. if option linearize=1 is not set in the scm config file, 
or options linearize=1 and boot\_scm option -methodA are both set): The program creates 'samples' bootstrapped datasets from the possibly pre-processed 
original dataset. Then a regular scm is run on each of these datasets, using the options set in the configuration file. Filtering on IGNORE/ACCEPT is skipped 
in these scm runs, since filtering was done during preprocessing if necessary. 

When using method B (i.e. if option linearize=1 is set in the scm config file but not option -methodA on the boot\_scm command line): 
The tool runs the possibly pre-processed input model with the possibly pre-processed dataset using the options set in the scm configuration 
file and terminates the run directly after the derivatives dataset has been generated. Then 'samples' bootstrapped datasets are created from the 
derivatives dataset. 
A regular scm is run on each bootstrapped dataset, using the options set in the scm configuration file. 

In addition to the options in the scm configuration file, 
the bootstrapped derivatives data is used as input with option -derivatives\_data (this is done automatically, the user should not set this option), 
which makes the scm run faster since the derivatives generation step can be skipped. In these scm runs the filtering on IGNORE/ACCEPT is skipped, 
since filtering was done during pre-processing. 

If there are time-varying covariates (option time\_varying is set in the original configuration file) each scm run will include a run with the 
original, non-linear model on a bootstrapped version of the possibly pre-processed original dataset, 
using the same individuals in each sample as in the bootstrapping of the derivatives dataset. 
This extra run is needed to compute medians for the time-varying covariates. 

If option -run\_final\_models is set: Run the final models from each scm on the original, possibly pre-processed, dataset.

\section{Output}

The file bs\_ids.csv contains one row per bootstrapped dataset and one column per individual in the bootstrapped dataset. 
The value in each column gives the original data ID of that individual.
The file ofv\_final.csv is only created if option run\_final\_models is set. It contains one row per bootstrapped dataset plus one for the original, 
possibly pre-processed, model. It lists the ofvs of the final models from the scm, rerun on the original dataset.
The file covariate\_inclusion.csv has one row per bootstrapped dataset. There is one column per parameter-covariate-state combination possible given 
the test\_relations and valid\_states settings in the configuration file, excluding state 1 (which means 'not included'). 
For each bootstrapped dataset the value in the column is 1 if the relation is included in the final model, and 0 otherwise.

\references

\end{document}
