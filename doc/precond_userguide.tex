\documentclass[a4wide,12pt]{article}
%\setlength{\marginparwidth}{0pt}%35
%\setlength{\marginparsep}{0pt}%?
%\setlength{\evensidemargin}{0pt}
%\setlength{\oddsidemargin}{0pt}
\usepackage{lmodern}
\usepackage[utf8]{inputenc}
\usepackage[T1]{fontenc}
\usepackage{textcomp}
\usepackage{verbatim}
\usepackage{enumitem}
\usepackage{longtable}
\usepackage{alltt}
\usepackage{ifthen}
% Reduce the size of the underscore
\usepackage{relsize}
\renewcommand{\_}{\textscale{.7}{\textunderscore}}

\newcommand{\guidetitle}[1]{
\title{#1\\ \vspace{2 mm} {\large PsN 4.1.1}}
\date{2014-02-10}
}

\newcommand{\doctitle}[1]{
\title{#1}
\date{2014-02-10}
}


\newenvironment{optionlist}{
\renewcommand{\arraystretch}{1.1}
\setlength{\leftmargini}{2.5cm}
\begin{description}
%\setlength{\itemsep}{0ex}
}
{\end{description}}

\newcommand{\optname}[1]{\item{{\bfseries\texttt-#1}\newline}}
\newcommand{\optdefault}[2]{\item{{\bfseries\texttt-#1}{\mbox{ = \it #2}}\newline}}

\newcommand{\nextopt}{}

\guidetitle{PRECOND user guide}

\begin{document}

\maketitle


\section{Introduction}
The precond script creates and runs a reparametrized modelfile given a matrix to use for preconditioning.
Preconditioning could make it possible to run the covariance step for models that would otherwise fail.

Examples
\begin{verbatim}
precond run1.mod -pre=pmatrix.csv
precond run2.mod -pre=psn.cov -cholesky
\end{verbatim}

\section{Input and options}

\subsection{Required input}
Required argument is a model file and a preconditioning matrix.

\begin{optionlist}

\optdefault{pre}{psn.cov}
A file in which to find the matrix for preconditioning. This matrix will be modified before use (see the preconditioning matrix section below) It can be either a comma separated file without header or a NONMEM output covariance matrix.
\nextopt
\end{optionlist}

\subsection{Optional input}

\begin{optionlist}
\optname{cholesky}
Use cholesky decomposition of the preconditioning matrix instead of LU decomposition.
\optname{nodec}
Turn off decomposition of preconditioning matrix.
\optdefault{output\_model}{run1\_repara.mod}
This option will break the normal execution flow and have precond only create the reparametrized model without running it.
The model will be created with the specified name.
\optdefault{cov}{result.cov}
This option will break the normal execution flow and only perform a conversion of a reparametrized covariance matrix.
If this option is set no model will be run.
\optdefault{update\_model}{filename}
Copy the model with updated inital thetas to your work directory
\end{optionlist}

\section{Output}

Using the normal execution flow (if none of -output\_model or -cov are specified) the model will be reparametrized, run and the resulting covariance matrix will be
backconverted and stored as result.cov.

The file update\_model.mod will always be created in the run directory. It is the original model with updated initial thetas after
running the reparametrized model and backconverting the theta estimates.


\section{Preconditioning matrix}

The input preconditioning matrix can be either a .csv file or a NONMEM .cov file. If it is a .csv file it should just have the matrix
rows one on each line and no header.
The input matrix will be modified in the following ways to create
a square lower triangular matrix that will be used for preconditioning of the THETAs:
First it will be either truncated or padded to become a square matrix with same size as the number of thetas in the model.
When padded all new rows will be set to zero except for the diagonal element which will be set to one. If any rows in the preconditioning matrix is found
to be all zeros a one will be placed on the diagonal. To obtain a lower triangular matrix a decomposition of the matrix will be performed. The default is to
use the LU factorization. The -cholesky option changes this to use cholesky decomposition instead. It is also possible to skip the decomposition by
setting the option -nodec in which case the user is responsible for the input matrix.

\section{Description}

This section describes in more detail how the reparametrization is done

\begin{itemize}
    \item The inputs are a model file and a lower triangular square matrix, L (see the section on the Preconditioning matrix above)
    \item Thetas are reparametrized as $\theta_{original} = L \cdot \theta_{estimated}$ For example:
        \begin{verbatim}
        $PK
        THE_1 = L_11 * THETA(1)
        THE_2 = L_21 * THETA(1) + L_22 * THETA(2)
        \end{verbatim}
    \item Use the original parameters in the place of THETAs in all relevant code blocks (currently pk, pred, error, des, aes, aesinitial, mix and infn)
        For example: \verb|CL = THETA(1)| will be replaced with \verb|CL = THE_1|
	\item Initial values of thetas will be transformed
	\item All bounds for thetas in the original model are removed
\end{itemize}


\end{document}
