\documentclass[a4wide,12pt]{article}
%\setlength{\marginparwidth}{0pt}%35
%\setlength{\marginparsep}{0pt}%?
%\setlength{\evensidemargin}{0pt}
%\setlength{\oddsidemargin}{0pt}
\usepackage{lmodern}
\usepackage[utf8]{inputenc}
\usepackage[T1]{fontenc}
\usepackage{textcomp}
\usepackage{verbatim}
\usepackage{enumitem}
\usepackage{longtable}
\usepackage{alltt}
\usepackage{ifthen}
% Reduce the size of the underscore
\usepackage{relsize}
\renewcommand{\_}{\textscale{.7}{\textunderscore}}

\newcommand{\guidetitle}[1]{
\title{#1\\ \vspace{2 mm} {\large PsN 4.1.1}}
\date{2014-02-10}
}

\newcommand{\doctitle}[1]{
\title{#1}
\date{2014-02-10}
}


\newenvironment{optionlist}{
\renewcommand{\arraystretch}{1.1}
\setlength{\leftmargini}{2.5cm}
\begin{description}
%\setlength{\itemsep}{0ex}
}
{\end{description}}

\newcommand{\optname}[1]{\item{{\bfseries\texttt-#1}\newline}}
\newcommand{\optdefault}[2]{\item{{\bfseries\texttt-#1}{\mbox{ = \it #2}}\newline}}

\newcommand{\nextopt}{}

\guidetitle{PRECOND user guide}

\begin{document}

\maketitle


\section{Introduction}
The precond script creates a reparametrized modelfile given a matrix to use for preconditioning.

Examples
\begin{verbatim}
precond run1.mod -pre=pmatrix.csv
precond run2.mod -pre=psn.cov -cholesky 
\end{verbatim}

\section{Input and options}

\subsection{Required input}
Required argument is a model file and a preconditioning matrix.

\begin{optionlist}

\optdefault{pre}{psn.cov}
The name of the matrix used for preconditioning. It can be either a comma separated file without header or a NONMEM output covariance matrix.
\nextopt
\end{optionlist}

\subsection{Optional input}

\begin{optionlist}
\optname{cholesky}
Use cholesky decomposition of the preconditioning matrix instead of LU decomposition.
\optname{nodec}
Turn off decomposition of preconditioning matrix.
\optdefault{output}{run1\_repara.mod}
Set name of output model. Default is to add \_repara.mod
\end{optionlist}

\section{Output}

The output is a reparametrized model file with the suffix \_repara.mod or with the specified filename given by -output option.


\section{Description}





\end{document}
