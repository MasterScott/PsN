\documentclass[a4wide,12pt]{article}
%\setlength{\marginparwidth}{0pt}%35
%\setlength{\marginparsep}{0pt}%?
%\setlength{\evensidemargin}{0pt}
%\setlength{\oddsidemargin}{0pt}
\usepackage{lmodern}
\usepackage[utf8]{inputenc}
\usepackage[T1]{fontenc}
\usepackage{textcomp}
\usepackage{verbatim}
\usepackage{enumitem}
\usepackage{longtable}
\usepackage{alltt}
\usepackage{ifthen}
% Reduce the size of the underscore
\usepackage{relsize}
\renewcommand{\_}{\textscale{.7}{\textunderscore}}

\newcommand{\guidetitle}[1]{
\title{#1\\ \vspace{2 mm} {\large PsN 4.1.1}}
\date{2014-02-10}
}

\newcommand{\doctitle}[1]{
\title{#1}
\date{2014-02-10}
}


\newenvironment{optionlist}{
\renewcommand{\arraystretch}{1.1}
\setlength{\leftmargini}{2.5cm}
\begin{description}
%\setlength{\itemsep}{0ex}
}
{\end{description}}

\newcommand{\optname}[1]{\item{{\bfseries\texttt-#1}\newline}}
\newcommand{\optdefault}[2]{\item{{\bfseries\texttt-#1}{\mbox{ = \it #2}}\newline}}

\newcommand{\nextopt}{}

\setlength{\evensidemargin}{0pt}
\setlength{\oddsidemargin}{0pt}

\guidetitle{SUMO user guide}{2014-05-20}

\begin{document}

\maketitle


\section{Introduction}
The sumo script is used to summarize output from a NONMEM run and print it to screen. It takes a lst-file as input.
Example
\begin{verbatim}
sumo run33.lst
\end{verbatim}


\section{Output}

The above command may produce output similar to this:

\begin{verbatim}
-----------------------------------------------------------------------

run33.lst

Successful minimization                                         [    OK   ]   
No rounding errors                                              [    OK   ]   
No zero gradients                                               [    OK   ]   
No final zero gradients                                         [    OK   ]   
Hessian not reset                                               [    OK   ]   
Parameter(s) near boundary                                      [    WARNING   ]   
OMEGA(2,2)	0.000112	0
Covariance step                                                 [    OK   ]   
Large standard errors found                                     [    WARNING   ]   
OMEGA(1,1)	1.08589
OMEGA(2,2)	1.01786
OMEGA(3,2)	0.848024
OMEGA(3,3)	0.548035
Large condition number                                          [    WARNING   ]   
Correlations                                                    [    OK   ]

    0MINIMIZATION SUCCESSFUL
    NO. OF FUNCTION EVALUATIONS USED:  172
    NO. OF SIG. DIGITS IN FINAL EST.:  3.4

    ETABAR IS THE ARITHMETIC MEAN OF THE ETA-ESTIMATES,
    AND THE P-VALUE IS GIVEN FOR THE NULL HYPOTHESIS THAT THE TRUE MEAN IS 0.

    ETABAR:  -0.41E-01  0.59E-03  0.30E-01
    SE:       0.35E+00  0.25E-02  0.18E+00

    P VAL.:   0.91E+00  0.82E+00  0.87E+00
    Objective function value: 116.966
    Condition number: 1.215e+004

           THETA                          OMEGA                      SIGMA            
THETA1    1.87  (-0.2503)  OMEGA(1,1)   1.277  (-1.086) SIGMA(1,1) 0.478  (-0.2762) 
THETA2  0.0862 (-0.05139)  OMEGA(2,2) 0.01058  (-1.018)                                
THETA3  0.0398 (-0.08266)  OMEGA(3,2)  0.9187  (-0.848)
                           OMEGA(3,3)  0.6768  (-0.548)

-----------------------------------------------------------------------
\end{verbatim}

The first section of the output from sumo contains run diagnostics. They are largely self-explanatory but some of them depends on cut-off values. For example, the diagnostics for a large condition number compares the obtained condition number to a cut-off value, normally 1000. The default values for the cut-off values can be changed by setting them in 
\verb|[default_sumo_options]| in psn.conf. It is possible to override a default value by specifying it as an option to sumo (see below).
\\ 
Next are the NONMEM minimization messages and \$COV messages, including etabars etc. Following this is the objective function value. If \$COV has been requested with the PRINT=E option (so that NONMEM reports eigenvalues) the condition number is also reported (the ratio between the largest and smallest eigenvalue)
The column THETA contains the theta estimates as reported in the NONMEM output file and the numbers in parentheses after the theta estimates, are the relative standard errors (SE/Est). The square roots of the estimates of the diagonal elements of the OMEGA and SIGMA matrices in the NONMEM output file are reported in the OMEGA and SIGMA columns (except when -ci is used, see below). The off-diagonal estimates are reported as the corresponding correlations, i.e. 
$\sqrt{\frac{cov(om1,om2)}{var(om1)*var(om2)}}$. 
The numbers in parentheses after the OMEGA and SIGMA columns are the relative SEs of the \emph{variance} and \emph{covariance} estimates, as taken from the NONMEM output file
(but see option -ci and -sd\_rse below).
Relative standard errors are not displayed unless NONMEM reports the standard errors in the output file, i.e. when the covariance step has been requested and has terminated successfully.


\section{Input and options}

\subsection{Required input}
An lst-file is required on the command-line.

\subsection{Optional input}

\begin{optionlist}
\optdefault{c\_level}{'number'}
Default 95. This option specifies the coverage of the confidence interval computed by the -ci option (see below). Allowed values are 90, 95, 99 and 99.5 
\nextopt
\optname{ci}
Default not set. This option computes and reports the traditional confidence intervals for the parameter estimates instead of the relative standard errors: CI = estimate +/- c\_level*SE. With this option the parameter estimates reported in the OMEGA and SIGMA columns are the numbers given in the NONMEM output file, i.e. no transformation to CVs or correlations is performed. 
\nextopt
\optdefault{condition\_number\_limit}{'number'}
Default 1000. The cut-off for issuing a warning for a high condition number. The test will only be made if the \$COV step has been requested in the NMtran control stream and NONMEM has been asked to report eigenvalues (\$COV PRINT=E). The default is set in psn.conf. 
\nextopt
\optdefault{correlation\_limit}{'number'}
Default 0.9. The cut-off for issuing a warning for high correlations between the parameter estimates. This test will only be made if the \$COV step has been requested in the NMtran control stream. The default is set in psn.conf. 
\nextopt
\optname{csv}
Default not set. Produce output in which the columns of the parameters estimates and standard errors are comma separated. This option can be used if you wish to import the sumo output into a spreadsheet programs, such as MS Excel. (Something like this is probably what you would want to do in that case: sumo -csv run1.lst > sumooutput.csv) 
\nextopt
\optdefault{large\_omega\_cv\_limit}{'number'}
Default 0.5. When the CV (SE/estimate) for a parameter estimates is greater than this a warning will be issued. 
\nextopt
\optdefault{large\_sigma\_cv\_limit}{'number'}
Default 0.3. When the CV (SE/estimate) for a parameter estimates is greater than this a warning will be issued. 
\nextopt
\optdefault{large\_theta\_cv\_limit}{'number'}
Default 0.3. When the CV (SE/estimate) for a parameter estimates is greater than this a warning will be issued. 
\nextopt
\optdefault{max\_problem\_num}{'integer'}
Default 1. The default is for sumo to report the results for the first \$PROBLEM only. Set this option larger than 1 to report results for additional problems. 
\nextopt
\optdefault{max\_subproblem\_num}{'integer'}
Default 1. The default is for sumo to report the results only for the first subproblem of the selected problems. Each problem contains at least one sub-problem. Should \$SIMULATION be used there will be as many sub-problems as the value of NSUBS. 
\nextopt
\optdefault{near\_bound\_sign\_digits}{'integer'}
Default 2. When a parameter estimates is equal to a boundary with these many signi ficant digits, a warning will be issued. The default is for NONEM VI to report if parameter estimates are close to a boundary value but this can be turned off. The boundary test in sumo is independent of the way NONMEM is configured and will report these types of issues in output from NONMEM. 
\nextopt
\optdefault{near\_zero\_boundary\_limit}{'number'}
Default 0.001. If the bound is 0, the boundary test will issue a warning if the estimate is this close to zero. 
\nextopt
\optname{check\_run}
Default set. When unset with -no-check\_run the initial run diagnostic output of sumo will be suppressed. 
\nextopt
\optdefault{precision}{'integer'}
Default 4. Precision in output.  
\nextopt
\optname{sd\_rse}
Default set. If sd\_rse is not set sumo will report the relative standard errors for the random effects parameters (omega and sigma) on the variance scale (SE/variance estimate). If sd\_rse is set, sumo will report these on the (approximate) standard deviation scale: (SE/variance estimate)/2. 
\nextopt
\optdefault{sign\_digits\_off\_diagonals}{'integer'}
Default 2. The boundary test for off-diagonal omega elements are performed by first converting the covariances to the corresponding correlations and then check if they are close to +/-1 with this many significant digits.  
\nextopt
\end{optionlist}

\end{document}
