\documentclass[a4wide,12pt]{article}
%\setlength{\marginparwidth}{0pt}%35
%\setlength{\marginparsep}{0pt}%?
%\setlength{\evensidemargin}{0pt}
%\setlength{\oddsidemargin}{0pt}
\usepackage{lmodern}
\usepackage[utf8]{inputenc}
\usepackage[T1]{fontenc}
\usepackage{textcomp}
\usepackage{verbatim}
\usepackage{enumitem}
\usepackage{longtable}
\usepackage{alltt}
\usepackage{ifthen}
% Reduce the size of the underscore
\usepackage{relsize}
\renewcommand{\_}{\textscale{.7}{\textunderscore}}

\newcommand{\guidetitle}[1]{
\title{#1\\ \vspace{2 mm} {\large PsN 4.1.1}}
\date{2014-02-10}
}

\newcommand{\doctitle}[1]{
\title{#1}
\date{2014-02-10}
}


\newenvironment{optionlist}{
\renewcommand{\arraystretch}{1.1}
\setlength{\leftmargini}{2.5cm}
\begin{description}
%\setlength{\itemsep}{0ex}
}
{\end{description}}

\newcommand{\optname}[1]{\item{{\bfseries\texttt-#1}\newline}}
\newcommand{\optdefault}[2]{\item{{\bfseries\texttt-#1}{\mbox{ = \it #2}}\newline}}

\newcommand{\nextopt}{}

\guidetitle{LLP user guide}{2018-03-02}
\usepackage{hyperref}
\begin{document}

\maketitle
\newcommand{\guidetoolname}{llp}
\tableofcontents
\newpage

\section{Introduction}
The llp (Log Likelihood Profiling) tool is used to calculate confidence intervals of parameter estimates. Without the llp the confidence intervals can be calculated with the standard errors (SE) of the parameters under the assumption that the parameter values are normally distributed. The llp tool, however, makes no assumption of the shape of the distribution and will calculate the confidence intervals for any number of parameters in the model, working with one parameter at a time. By first fitting the original model and then fixing the parameter at values close to the NONMEM estimate, the llp tool obtains the difference in likelihood between the original model and new, reduced model. The logarithm of the difference in likelihood is chi2 distributed and when that value is 3.84, the parameter value is at the 95\% confidence limit. The search for the limit is done on both sides of the original parameter value, and thus the llp tool makes no assumption of symmetry or the parameter value distribution.\\
Examples:
\begin{verbatim}
llp run89.mod -thetas=1,2
\end{verbatim}
This will make the llp tool try to estimate the confidence intervals for thetas one and two of the model in run89.mod. It will base the first guesses on the SE estimates from run89.lst.
\begin{verbatim}
llp run89.mod -thetas=1,2 -rse_thetas=20,30
\end{verbatim}
In this example, we explicitly specify the relative SE which is necessary if we do not have an output file with SE estimates.
\section{Input and options}

\subsection{Required input}
A model file is required on the command line. Then, at least one of the options -omegas, -sigmas or -thetas must be specified. If an lst-file with SE estimates already exists, no more input is needed. Otherwise, for each specified $\langle$parameter$\rangle$ (theta/omega/sigma) there must be a corresponding rse-value given by option -rse\_$\langle$parameter$\rangle$. 

\begin{optionlist}
\optdefault{omegas}{comma-separated list of parameter numbers}
A comma-separated list, specifying the omegas for which the llp should try to assess confidence intervals. The numbers refer to the order number of the initial values in the model file. For example, if first there is a block record with size 2 (3 initial values) and then there is a diagonal record with size 3, then the numbers 4,5 and 6 refer to the diagonal elements.
\nextopt
\optdefault{sigmas}{comma-separated list of parameter numbers}
A comma-separated list, specifying the sigmas for which the llp should try to assess confidence intervals. The numbers refer to the order number of the initial values in the model file. For example, if first there is a block record with size 2 (3 initial values) and then there is a diagonal record with size 3, then the numbers 4, 5 and 6 refer to the diagonal elements.
\nextopt
\optdefault{thetas}{comma-separated list of parameter numbers}
A comma-separated list, specifying the thetas for which the llp should try to assess confidence intervals. 
\nextopt
\end{optionlist}

\subsection{Optional input}

\begin{optionlist}

\optdefault{max\_iterations}{N}
Default value is 10. This number limits the number of search iterations for each interval limit. If the llp has not found the upper limit for a parameter after max\_iteration number of guesses it terminates. 
\nextopt
\optdefault{normq}{X}
Default value is 1.96. The value is used for calculating the first guess of the confidence interval limits. If the SE exist, the guess will be maximum-likelihood estimate (MLE) $\pm$ normq * SE, otherwise it will be MLE $\pm$ normq * rse\_parameter/100 * MLE, where rse\_parameter is rse\_thetas, rse\_omegas or rse\_sigmas (optional input parameters). The default value of normq translates to a 95\% confidence interval assuming normal distribution of the parameter estimates. 
\nextopt
\optdefault{ofv\_increase}{X}
Default value is 3.84. The increase in the objective function value associated with the desired confidence interval. 
\nextopt
\optdefault{outputfile}{filename}
The name of the NONMEM output file. The default name is the name of the model file with '.mod' substituted with '.lst'.\\
Example: if the modelfile is run89.mod, llp will by default look for the outputfile run89.lst. If the name of the output file does not follow this standard, the name must be specifed with this option. 
\nextopt
\optdefault{rse\_omegas}{comma-separated list of relative standard errors}
A comma-separated list specified in percent (\%), for each omega listed by option -omegas. 
\nextopt
\optdefault{rse\_sigmas}{comma-separated list of relative standard errors}
A comma-separated list specified in percent (\%), for each sigma listed by option -sigmas. 
\nextopt
\optdefault{rse\_thetas}{comma-separated list of relative standard errors}
A comma-separated list specified in percent (\%), for each theta listed by option -thetas. 
\nextopt
\optdefault{significant\_digits}{N}
Default 3. Specifies the number of significant digits that are required for the test of the increase in objective function value. With the default 3, the method will stop once the difference in objective function value is between 3.835 and 3.845 if\\ -ofv\_increase is set to its default value 3.84. 
\nextopt
\end{optionlist}

\subsection{PsN common options}
For a complete list see common\_options.pdf or type psn\_options -h on the command line.

\subsection{Auto-generated R plots from PsN}
\newcommand{\rplotsconditions}{The default llp template 
requires the R libraries ggplot2, reshape and plyr.
If the packages are not installed then no pdf will be generated,
see the .Rout file in the main run directory for error messages.}
PsN can automatically generate R plots to visualize results for \guidetoolname, using a default template found in the R-scripts subdirectory of the installation directory. The user can also create a custom template, see more details in the section Auto-generated R-plots from PsN in common\_options.pdf.

\rplotsconditions

\begin{optionlist}
\optdefault{rplots}{level}
-rplots<0 means R script is not generated\\ 
-rplots=0 (default) means R script is generated but not run\\ 
-rplots=1 means basic plots are generated\\													  
-rplots=2 means basic and extended plots are generated\\													  
\nextopt
\end{optionlist}

\subsubsection*{Troubleshooting}
If no .pdf was generated even if a template file is available and the appropriate options were set, check the .Rout-file in the main run directory for error messages. If no .Rout-file exists, then check that R is properly installed, and that either command 'R' is available or that R is configured in psn.conf.


\subsubsection*{Basic R plots}
A basic llp R plot will be generated in file PsN\_llp\_plots.pdf if option -rplots is set >0, and the general R plots conditions fulfilled, see above.
The R plot has one panel per evaluated parameter. The parameter value is on the x-axis, and the delta-ofv relative the input model on the y-axis. The blue line shows the confidence interval computed from the standard error under the assumption of normality, and the red line shows the llp-computed confidence interval. Each black dot represents an evaluated parameter value. The IR-value at the top is the ratio between the center-to-upper-limit-distance and the center-to-lower-limit-distance. 
If the IR is equal to 1 it means the llp-computed confidence interval is symmetric.

\section{Output}

The file llp\_results.csv contains statistics and summaries specific for the llp. \\
The raw\_results.csv file is a standard PsN file containing raw result data for termination status, parameter estimates, uncertainty estimates etc. for all model estimations. If option -rplots>0 the file PsN\_llp\_plots.pdf is also created.

\end{document}
