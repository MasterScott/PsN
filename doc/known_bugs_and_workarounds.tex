\documentclass[a4wide,12pt]{article}
%\setlength{\marginparwidth}{0pt}%35
%\setlength{\marginparsep}{0pt}%?
%\setlength{\evensidemargin}{0pt}
%\setlength{\oddsidemargin}{0pt}
\usepackage{lmodern}
\usepackage[utf8]{inputenc}
\usepackage[T1]{fontenc}
\usepackage{textcomp}
\usepackage{verbatim}
\usepackage{enumitem}
\usepackage{longtable}
\usepackage{alltt}
\usepackage{ifthen}
% Reduce the size of the underscore
\usepackage{relsize}
\renewcommand{\_}{\textscale{.7}{\textunderscore}}

\newcommand{\guidetitle}[1]{
\title{#1\\ \vspace{2 mm} {\large PsN 4.1.1}}
\date{2014-02-10}
}

\newcommand{\doctitle}[1]{
\title{#1}
\date{2014-02-10}
}


\newenvironment{optionlist}{
\renewcommand{\arraystretch}{1.1}
\setlength{\leftmargini}{2.5cm}
\begin{description}
%\setlength{\itemsep}{0ex}
}
{\end{description}}

\newcommand{\optname}[1]{\item{{\bfseries\texttt-#1}\newline}}
\newcommand{\optdefault}[2]{\item{{\bfseries\texttt-#1}{\mbox{ = \it #2}}\newline}}

\newcommand{\nextopt}{}

\guidetitle{Known Bugs and Workarounds}{2018-03-02}
\usepackage{hyperref}

\begin{document}

\maketitle
\tableofcontents
\newpage

\section{Introduction}
Work to fix bugs in PsN is ongoing, but often feature addition is prioritized over bug fixing, especially if there is a known workaround for the bug. This document list the most important bugs, including workarounds when possible, that were known at the time of release of PsN version \psnversion. Please refer to the known bugs list to learn about bugs that have been discovered after the time of release:\\
\texttt{https://uupharmacometrics.github.io/PsN/known\_bugs.html}   

\section{Bugs}
\subsection{Sensitive format for raw results input file in sse, vpc, npc, sir}
The parameter labels in the column headers for THETA, OMEGA and SIGMA must be unique. If two parameters have the same
column header then PsN will crash during reading of the raw results file.

Any column header that contains a comma, e.g. OMEGA(2,1), must be enclosed in double quotes. This is done automatically when PsN creates a raw results file but may be changed if the file is saved in e.g. Excel.

\subsection{Linearized scm issues}
Option -foce, which only applies to linearized scm, must not be changed from the default value, or the scm cannot run. Also, the input model to a linearized scm must \emph{not} have METHOD=ZERO.

\subsection{boot\_scm with dummy covariates that are time-varying}
When randomizing data columns to create dummy covariates (option -dummy\_covariates) for boot\_scm, PsN will not properly
shuffle the time-varying set of values between individuals, but instead set all values to the individual's mean in the randomized copy.

\subsection{Use prior-specific records for \$PRIOR NWPRI}
When a control stream has multiple \$PROBLEM \emph{and} the first \$PROBLEM has \$PRIOR NWPRI \emph{and} the prior parameters are encoded with \$THETA/\$OMEGA/\$SIGMA instead of the prior-specific records \$THETAP, \$THETAPV, \$OMEGAP etc, then PsN will not be able to handle the parameter column headers correctly in the raw\_results file. The workaround is to always use the prior-specific records (\$THETAP etc) for encoding the prior information.

It especially important to use the prior-specific records when running sir, or bootstrap with option -dofv, with a model that has \$PRIOR NWPRI. In those two cases PsN will itself create a control stream with multiple \$PROBLEM.

\subsection{Missing estimates in raw\_results.csv when first \$PROBLEM uses \$MSFI}
When the first \$PROBLEM in the modelfile uses \$MSFI, and \$THETA, \$OMEGA and \$SIGMA are all missing in that \$PROBLEM, then there will be no theta/omega/sigma headers in raw\_results.csv and the parameter estimates will not be printed to the file.

\subsection{Data values with more than five significant digits in sse}
The simulated datasets used in sse are \$TABLE output from NONMEM, and NONMEM rounds off values when printing tables. In NONMEM 6 1013201 is rounded to 1013200 (five significant digits), and if this makes a significant change to the model estimation, for example if the value is a covariate, then the sse results will be wrong. In NONMEM7 it is possible to set the FORMAT option in \$TABLE to make sure no important information is lost. With NONMEM 6 the user must make sure the rounding to five significant digits does not harm the results.

\end{document}
