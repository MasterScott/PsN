\documentclass[a4wide,12pt]{article}
%\setlength{\marginparwidth}{0pt}%35
%\setlength{\marginparsep}{0pt}%?
%\setlength{\evensidemargin}{0pt}
%\setlength{\oddsidemargin}{0pt}
\usepackage{lmodern}
\usepackage[utf8]{inputenc}
\usepackage[T1]{fontenc}
\usepackage{textcomp}
\usepackage{verbatim}
\usepackage{enumitem}
\usepackage{longtable}
\usepackage{alltt}
\usepackage{ifthen}
% Reduce the size of the underscore
\usepackage{relsize}
\renewcommand{\_}{\textscale{.7}{\textunderscore}}

\newcommand{\guidetitle}[1]{
\title{#1\\ \vspace{2 mm} {\large PsN 4.1.1}}
\date{2014-02-10}
}

\newcommand{\doctitle}[1]{
\title{#1}
\date{2014-02-10}
}


\newenvironment{optionlist}{
\renewcommand{\arraystretch}{1.1}
\setlength{\leftmargini}{2.5cm}
\begin{description}
%\setlength{\itemsep}{0ex}
}
{\end{description}}

\newcommand{\optname}[1]{\item{{\bfseries\texttt-#1}\newline}}
\newcommand{\optdefault}[2]{\item{{\bfseries\texttt-#1}{\mbox{ = \it #2}}\newline}}

\newcommand{\nextopt}{}

\guidetitle{CROSSVAL user guide}{2018-01-31}

\begin{document}

\maketitle
\newcommand{\guidetoolname}{crossval}


\section{Overview}

The crossval tool divides the NONMEM data set into 'groups' that are equally sized parts.
For each part i, the remaining groups-1 parts are used for estimation,
and then part i is used for prediction.

Example
\begin{verbatim}
crossval run1.mod -groups=5
\end{verbatim}

\section{Input and options}
	
\subsection{Required input}
A model file is required on the command line together with the -groups option.
\begin{optionlist}
\optname{groups}
The number of cross-validation groups in the data set.
\nextopt		
\end{optionlist}

\subsection{Optional input}
			
\begin{optionlist}
\optname{msf}
Default not set, which means prediction model\\
THETA/OMEGA/SIGMA are updated with final estimates from estimation models.
If -msf is set, instead use MSFO - MSFI to transfer parameter values from estimation models to prediction models.
\nextopt
\end{optionlist}

\subsection{PsN common options}
For a complete list see common\_options.pdf or type psn\_options -h on the command line.

\section{Output}

The output is the file xv\_result.txt with prediction model OFVs and corresponding estimation model OFVs.

\end{document}
