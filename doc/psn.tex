\documentclass[a4wide,12pt]{article}
%\setlength{\marginparwidth}{0pt}%35
%\setlength{\marginparsep}{0pt}%?
%\setlength{\evensidemargin}{0pt}
%\setlength{\oddsidemargin}{0pt}
\usepackage{lmodern}
\usepackage[utf8]{inputenc}
\usepackage[T1]{fontenc}
\usepackage{textcomp}
\usepackage{verbatim}
\usepackage{enumitem}
\usepackage{longtable}
\usepackage{alltt}
\usepackage{ifthen}
% Reduce the size of the underscore
\usepackage{relsize}
\renewcommand{\_}{\textscale{.7}{\textunderscore}}

\newcommand{\guidetitle}[1]{
\title{#1\\ \vspace{2 mm} {\large PsN 4.1.1}}
\date{2014-02-10}
}

\newcommand{\doctitle}[1]{
\title{#1}
\date{2014-02-10}
}


\newenvironment{optionlist}{
\renewcommand{\arraystretch}{1.1}
\setlength{\leftmargini}{2.5cm}
\begin{description}
%\setlength{\itemsep}{0ex}
}
{\end{description}}

\newcommand{\optname}[1]{\item{{\bfseries\texttt-#1}\newline}}
\newcommand{\optdefault}[2]{\item{{\bfseries\texttt-#1}{\mbox{ = \it #2}}\newline}}

\newcommand{\nextopt}{}

\guidetitle{PsN}{2017-09-15}

\usepackage{tabularx}


\begin{document}


\maketitle
\newcommand{\guidetoolname}{<toolname>}

\section{Introduction}
%This document aims to be a starting point for the PsN documentation and also to contain topics that are common to the different PsN tools.
This document aims to contain topics that are common to the different PsN tools.

\section{Metadata}
All PsN tools generate metadata with information about the specific run. The file 'meta.yaml' is, starting with PsN 4.7.8, created in each run directory with the aim of gathering all the interesting metadata in one place. As a yaml-file the meta.yaml is both easy to read for humans and easy to interpret by computer program. Below is a description of all currently available tags in meta.yaml

\begin{center}
    \begin{tabularx}{\linewidth}{ r X }
    \hline
    Tag & Description \\ \hline
    \verb|command_line| & The command line with the full path to the command \\ \hline
    \verb|common_options| & All common options that were use either explicitly on the command line or from psn.conf \\ \hline
    \verb|copied_files| & A list of all files that were copied back to the calling directory \\ \hline
    \verb|finish_time| & Timestamp for the finish of the PsN run as yyyy-mm-dd hh:mm:ss \\ \hline
    \verb|model_files| & An array of the full\_paths to all model files used as input to run \\ \hline
    \verb|NONMEM_directory| & The full path to were the NONMEM version for this run is stored \\ \hline
    \verb|NONMEM_version| & The version of NONMEM \\ \hline
    \verb|PsN_version| & The version of PsN \\ \hline
    \verb|start_time| & Timestamp for the start of the PsN run as yyyy-mm-dd hh:mm:ss \\ \hline
    \verb|tool_name| & The name of the tool without version extensions \mbox{('-4.7.8')} or developer version extension ('-dev') \\ \hline
    \verb|tool_options| & All tool specific options used for this run. \\ \hline
  \end{tabularx}
\end{center}

\section{Non-Supported NM-TRAN constructs}

\subsection{INCLUDE and \$INCLUDE}
The INCLUDE record is not supported. NM-TRAN allows INCLUDE without the dollar-sign and PsN will add this as options or code to the previous record potentially causing strange errors or worse. If \$INCLUDE was used PsN will give a nice error message.

\end{document}
