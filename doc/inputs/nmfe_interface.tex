\begin{optionlist}
\optname{always\_datafile\_in\_nmrun}
Default not set. By default, PsN will often not copy the datafile to the NM\_run subdirectories, but instead include the path to the datafile in \$DATA in the control stream copy inside NM\_run. This is the case in for example the bootstrap and randtest programs, or when option -no-copy\_data is set in the tools that accept that option. If -always\_datafile\_in\_nmrun is set, then PsN will always copy the datafile to NM\_run and set the datafile name without path in \$DATA. This behaviour may be useful when running on a grid where only the contents of NM\_run are available to NONMEM at runtime. Option -always\_datafile\_in\_nmrun will override -no-copy\_data, if -no-copy\_data is set.
\nextopt
\optdefault{extra\_files}{comma-separated list of filenames}
Default not used. If you need extra files in the directory where NONMEM is run you specify them in the comma separated -extra\_files list. It could for example be fortran subroutines you need compiled with NONMEM, or a file with initial estimates for the NONMEM 7 CHAIN command, a defaults.pnm file (NONMEM 7.2 or later), or a file set in \$ETAS with option FILE. 
\nextopt
\optname{nmfe}
Default set. Invoke NONMEM via the nmfe script (or a custom wrapper) from within PsN.  Unless option -nmqual is set, option -nmfe is 
set automatically. Also, -nmfe is set in the default configuration file.
\nextopt
\optdefault{nmfe\_options}{string}
Only relevant if NONMEM 7.2 or later is used. The text set with this option will be copied verbatim to the nmfe script call. 
PsN will not check that the options are allowed. When set on the PsN commandline the string must be enclosed by quotes if it contains any spaces, but when set in psn.conf it must never be enclosed by quotes even if it contains spaces. 

On UNIX type systems, but not on Windows, any parentheses in the nmfe options, for example as in -maxlim=(1,2,3), must be
escaped with backslashes like this: \verb|-maxlim=\(1,2,3\) |
\nextopt
\optname{nmqual}
Default not set. Run an NMQual-installed NONMEM via autolog.pl. Only NMQual 8 is supported. When set, PsN will locate the autolog.pl file and log.xml in the nmqual subdirectory of the NONMEM installation directory, and then run \\
\verb|perl /path/autolog.pl /path/log.xml run ce|\\
\verb| /full/path/workdir psn| (extra NM options)

Important note: The extra NM options are options set with e.g. -nmfe\_options or -nodes, but unless the do-on-run block of the log.xml file is edited to use these extra options, \emph{they will be ignored}. PsN will append them to the autolog.pl call but it is up to log.xml to decide what to do with them.

If the user wants by default to run PsN with the autolog script it is recommended to set nmqual=1 in the 
\verb|[default_options]| section in psn.conf and remove nmfe=1 from the same section. The user should also consider to add log.xml to the -nm\_output list in psn.conf (see help text for -nm\_output).
\nextopt
\optdefault{nodes}{integer}
Only relevant together with option -parafile. 

Appends “[nodes]=option\_value” to the nmfe call. This option acts independently of -threads. There is no adjustment of -nodes based on -threads or vice versa, and if -threads=1 it is still possible to use -nodes=10. 
\nextopt
\optdefault{parafile}{filename}
NONMEM 7.2 (or later version) parafile.

Appends “-parafile=filename” to the nmfe call. 
\nextopt
\end{optionlist}
