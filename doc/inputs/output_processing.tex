\begin{optionlist}
\optdefault{nm\_output}{comma-separated list of file extensions}
Default not used.  NONMEM generates many output files per run. 
PsN will always copy the lst-file back to the calling directory. The option -nm\_output decides which of the more than 10 additional files should also be copied back. The default in the auto-generated psn.conf is ext,cov,cor,coi,phi. 
Note that NM output files which are not copied to the calling directory can still be found inside the run directory
unless -clean is set larger than 2. If the user runs NMQual8 (option -nmqual is set) and wants the log.xml file copied back then the log.xml extension should be included in the -nm\_output list, i.e. nm\_output=ext,cov,cor,coi,phi,log.xml or similar.
\nextopt
\optdefault{extra\_output}{comma-separated list of filenames}
Default not used. If NONMEM generates a file which PsN normally does not copy back to the working directory, specifying a comma-separated list of such files with this options will make PsN copy the listed files. An example is output generated by verbatim code. 
\nextopt
\optname{zip}
Default not set. If this option is set the m1 folder in the run directory will be zipped into the file m1.zip.
This will save space and reduce the number of files generated by PsN. PsN will, also when option -zip is not set,
automatically unzip
zipped m1 folders when restarting/resuming a run using the -directory option.
\nextopt
\optname{cwres}
Default not used. Compute the conditional weighted residuals (CWRES) for a model run. It is also possible for users of NONMEM7 to request CWRES directly in \$TABLE. 
\nextopt
\optname{iofv}
Default not used. Compute the individual contributions to the objective function (written in file iotab$<$N$>$ in NM\_run directory). This option is disabled for NONMEM7 because the additional output phi-file contains individual ofv values. 
\nextopt
\optname{shrinkage}
Default not used. Calculate the shrinkage for the model run.  Shrinkage is calculated as 1-(sd(eta(x))/omega(x)) and measures the shrinkage of the empirical Bayes estimates (EBEs) towards the mean of the expected distribution.  A 'large' shrinkage means that diagnostics using EBEs cannot be trusted. The shrinkage values appear in the file raw\_results.csv. 
PsN does not use the NONMEM7 shrinkage values. When shrinkage is requested, PsN adds two \$TABLE to the modelfile so that NONMEM will output data needed for the shrinkage computation. For eta shrinkage the table requests items ID ETA1 ETA2... and for iwres shrinkage it requests items ID IWRES EVID. 
%From andy 
%There are occasions where NONMEM gets the calculation wrong.  
%Mostly for computing EPS shrinkage if you have a more complicated residual variability model.  
%In previous versions of NONMEM (prior to 7.3) then the ETA shrinkage calculations were often wrong in NONMEM as well (by including individuals with ETA values of zero in the calculation).
\nextopt
\optdefault{mirror\_plots}{'integer'}
Default 0. This command creates a set of simulations from a model file that can then be read into Xpose 4 for mirror plotting. The command requires an integer value -mirror\_plots=XX where XX is an integer representing the number of simulations to perform. This command uses the MSFO file created by runN.mod to get final estimates used in the simulations. If this file is not available run1.mod is run again.  If run times are long, and you did not create an MSFO file with your initial NONMEM run, you can combine the above command with the -mirror\_from\_lst option to avoid running the model again (PsN then reads from the *.lst file to get final parameter estimates for the simulations). 
\nextopt
\optname{mirror\_from\_lst}
Default not used. Can only be used in combination with -mirror\_plots=XX where XX is an integer representing the number of simulations to perform.  These commands create a set of simulations from a model file and output file that can then be read into Xpose 4 for mirror plotting.  The -mirror\_from\_lst option reads from the *.lst file of a NONMEM run to get final parameter estimates for the simulations. 
\nextopt
\optname{prepend\_model\_file\_name}
Default not used. Table files have the name defined by FILE in \$TABLE. 
This option will prepend the model file name, without extension, to
the name set with FILE.
\nextopt
\optname{standardised\_output}
    Default not used. If this option is used a DDMoRe standardised output (SO 0.3.1) xml file will be created for each run. Note that the file format is still under development and that this option should be considered experimental. See the \mbox{nmoutput2so\_userguide.pdf} for more information.
\nextopt
%this must be tested before advertised
%\optdefault{outputfile}{'string'}
%Default modelfilename with the extension substituted with lst. 
%The -outputfile option specifies the output file name for the NONMEM run. Currently this option is only 
%valid when a single model is supplied. 
%\nextopt
\end{optionlist}
