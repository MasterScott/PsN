PsN can automatically transform a model according to the 
Transform Both Sides method.
For details on the theory read Oberg and Davidian: \\
http://onlinelibrary.wiley.com/doi/10.1111/ j.0006-341X.2000.00065.x/abstract.

Bill Frame: http://www.ncbi.nlm.nih.gov/pubmed/19904583
 http://www.thtxinfo.com/
 
 Model must be coded “the Uppsala way”, i.e. with the standard deviation of the residual error expressed as a THETA and SIGMA 1 FIX. 
Also, the user must use untransformed data (e.g. no log-transformation or similar) in the model, and the possible range of IPRED and DV must not include negative values. If IPRED or DV becomes negative there will be a NONMEM error when running the tbs-modified model. 

\begin{optionlist}
\optname{tbs}
Default not set. Invokes Transform Both Sides method, by default using the Box-Cox transformation. 
When -tbs is set, PsN will make the following changes to the model file before running:
\begin{enumerate}
	\item add a THETA representing the Box-Cox LAMBDA parameter to be estimated. Default (can be changed with option -tbs\_lambda) is no lower boundary, initial estimate 1, no upper boundary.
	\item A set of IF-statements will make sure log transformations are made instead of Box-Cox if the LAMBDA estimate is 0. IF statements will also handle the cases IPRED=0 and DV=0.
	\item IPRED will be transformed as 
 IPRED\_trans=(IPRED**LAMBDA-1)/LAMBDA, 
 and the transformed  IPRED used instead of IPRED.  
	\item The IWRES definition is changed to 
 IWRES=((DV**LAMBDA-1)/LAMBDA-IPRED\_trans)/W
	\item Any IPRED dependence in the W definition is removed.
	\item The error model will be changed to an additive error model.
	\item In \$SUB two fortran routines, contra.txt and ccontra\_nm7.txt are added. These files are automatically printed to the run directory. In the file ccontra\_nm7.txt that is created, the x in theta(x) points to the lambda that was added in the model file.
\end{enumerate}
\nextopt
\optname{dtbs}
Default not set. Invokes Dynamic Transform Both Sides method, see 
 
Dosne AG, Keizer RJ, Bergstrand M, Karlsson MO. \emph{A strategy for residual error modeling incorporating both scedasticity of variance and distribution shape}. 
PAGE 21 (2012) Abstr 2527.

The model code is modified the same way as with -tbs, see above, except that the W definition is changed using a new ZETA parameter:
\begin{enumerate}
\item A new parameter ZETA will be defined, to be used in the error model. The THETA for this parameter will be automatically added.
\item Any IPRED term in the W definition will be replaced with IPRED**ZETA.
Any THETAs that are not multiplied with IPRED in the W definition will be set to 0 FIX.
If W does not depend on IPRED at all in the input model, then W = old\_definition will be replaced by
W = (IPRED**ZETA)*old\_definition.
\item Because lambda and zeta are correlated, estimation stability can be increased 
by reparameterizing and estimating DELTA instead of ZETA, with ZETA=LAMBDA+DELTA. 
This reparameterization done when the user sets option -tbs\_delta, see below.
\end{enumerate}
Correspondence between -dtbs parameter values and common error models\\
\begin{tabular}{|l|l|l|l|}
\hline
Error model & LAMBDA & ZETA & DELTA=ZETA-LAMBDA\\
\hline
additive & 1 & 0 & -1 \\
\hline
proportional & 1 & 1 & 0 \\
\hline
additive on log-&  &  &  \\
transformed data & 0 & 0 & 0 \\
\hline
\end{tabular}
Combined error models would need to be coded manually and are not recommended for dtbs approach.
\nextopt
\optdefault{tbs\_lambda}{string}
Default 1 if option -tbs is set. Initial value string, using NM-TRAN syntax, for lambda parameter in Transform Both Sides method, e.g. ``(-1, 0.5, 1)'' or 
``0 FIX''. The string must be enclosed in quotes, double quotes on Windows and either double or single on unix, and not include any comments. If tbs\_lambda is set then option -tbs will be set automatically, unless option -dtbs or -tbs\_zeta or -tbs\_delta is set. 
See option -tbs for more details. 
\nextopt
\optdefault{tbs\_zeta}{string}
Default 0.001 if W does not depend on IPRED, default 1 otherwise. Initial value string, using NM-TRAN syntax, for zeta parameter in Dynamic Transform Both Sides method, for example ``(-1, 0.5, 1)'' or ``0 FIX''.
The string must be enclosed in quotes, double quotes on Windows and either double or single on unix, and not include any comments.
If -tbs\_zeta is set then option -dtbs will be set automatically. See option -dtbs for more details. Options tbs\_zeta and tbs\_delta cannot be used in combination.  
\nextopt
\optdefault{tbs\_delta}{string}
Default not set. Initial value string, using NM-TRAN syntax, for delta parameter in Dynamic Transform Both Sides method, e.g. ``(-1, 0.5, 1)'' or ``0 FIX''. 
The string must be enclosed in quotes, double quotes on Windows and either double or single on unix, and not include any comments.
If tbs\_delta is set then option -dtbs will be set automatically. See option -dtbs for more details. Options tbs\_zeta and tbs\_delta cannot be used in combination.  
\nextopt
\end{optionlist}
