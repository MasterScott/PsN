\begin{optionlist}
\optname{h or -?}
Print a list of available options and exit. 
\nextopt
\optname{help}
With -help all programs will print a longer help message. If an option name is given as argument, help will be printed for this option. If no option is specified, help text for all options will be printed. 
\nextopt
\optdefault{clean}{integer}
Default is 1. The clean option can take four different values:  
\begin{description}
	\item[0] Nothing is removed 
	\item[1] NONMEM binary and intermediate files except INTER are removed, and files specified with option -extra\_files. 
	\item[2] model and output files generated by PsN restarts are removed, and data files in the NM\_run directory, and (if option -nmqual is used) the xml-formatted NONMEM output. 
	\item[3] All NM\_run directories are completely removed. If the PsN tool has created modelfit\_dir:s inside the main run directory, these  will also be removed. 
	\item[4] All NM\_run directories and all m1 directories are completely removed.
\end{description}
\nextopt
\optdefault{directory}{string}

Default \guidetoolname\_dirN,
where N will start at 1 and be increased by one each time you run the script. The directory option sets the directory in which PsN 

will run NONMEM and where PsN-generated output files will be stored. You do not have to create the directory,  it will be done for you. If you set -directory to a the name of a directory that already exists, PsN will run in the existing directory, except for scm, boot\_scm and xv\_scm that cannot be started in an existing directory.
\nextopt
\optname{model\_subdir}
	Use an alternative directory structure for PsN. An extra directory
    level unique to each model is introduced between the calling
    directory and the rundirectory. More information about this option can
    be found in PsN.pdf.
\nextopt
   
\optdefault{nm\_version}{string}
Default is 'default'. 
If you have more than one NONMEM version installed you can use option -nm\_version to choose which one to use, as long as it is 

defined in the [nm\_versions] section in psn.conf, see psn\_configuration.pdf for details. You can check which versions are defined, without opening psn.conf, using the command

\begin{verbatim}
psn -nm_versions
\end{verbatim}
\nextopt
\optdefault{seed}{string}
You can set your own random seed to make PsN runs reproducible. The random seed is a string, so both -seed=12345 and -seed=JustinBieber are valid. It is important to know that because of the way the Perl pseudo-random number generator works, for two similar string seeds the random sequences may be identical. This is the case e.g. with the two different seeds 123 and 122. 
Setting the same seed guarantees the same sequence, but setting two slightly different seeds does not guarantee two different random sequences, that must be verified.
\nextopt

\optdefault{threads}{integer}

Default is 5 (if the default psn.conf is used). Use the threads option to enable parallel execution of multiple models.

This option decides how many models PsN will run at the same time, and it is completely independent of whether the individual models are run with serial NONMEM or parallel NONMEM. If you want to run a single model in parallel you must use options -parafile and -nodes. On a desktop computer it is recommended to not set -threads higher the number of CPUs in the system plus one. 
You can specify more threads, but it will probably not increase the performance. If you are running on a computer cluster, you should consult your system administrator to find out how many threads you can specify. 
\nextopt
\optname{version}
Prints the PsN version number of the tool, and then exit. 
\nextopt
\end{optionlist}
