The following options control the messages printed to screen during the run. They do not affect the final results in any way.
\begin{optionlist}
\optdefault{debug}{integer}
Default is 0. The -debug option is mainly intended for developers who wish to debug PsN. You can set it to '1' to enable warning messages. If you run into problems that require support, you may have to set this and send the output to the developers. 
\nextopt
\optname{display\_iterations}
Default not set.  This option turns on display the iterations output from NONMEM during the model run. If the option is not set, the iterations output will be redirected to a file (nmfe\_output.txt). As with any option the user can choose to change the default by editing psn.conf, see the document psn\_configuration.pdf. The option can be disabled with -no-display\_iterations.

The template psn.conf distributed with the PsN installation package has this option set as the default for execute, but not for other tools.  
\nextopt
\optname{silent}
Default not set. The silent option redirects all messages that PsN normally prints to screen to the file run\_messages.txt. Note that messages from NONMEM, such as iterations, will not be printed to run\_messages.txt. Those messages will be in nmfe\_output.txt in the NM\_run subdirectories. Other results and log files are written to disk as usual. Nothing is printed to screen. 
\nextopt
\optname{stop\_motion}
Default not set. Used for debugging. Will cause PsN to pause its execution at certain predefined breakpoints, and only continue after the user hits enter.
\nextopt
\optname{verbose}
Default not set. With verbose set, PsN will print more details to screen about NONMEM runs. More precisely PsN will print the minimization message for each successful run and a R:X for each retry PsN makes of a failed run, where X is the run number. 
\nextopt
\optname{warn\_with\_trace}
Default not set. If -warn\_with\_trace is set, PsN will print a stack trace for all error and warning messages. This is only for developers. 
\nextopt
\end{optionlist}
