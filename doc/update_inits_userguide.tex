\documentclass[a4wide,12pt]{article}
%\setlength{\marginparwidth}{0pt}%35
%\setlength{\marginparsep}{0pt}%?
%\setlength{\evensidemargin}{0pt}
%\setlength{\oddsidemargin}{0pt}
\usepackage{lmodern}
\usepackage[utf8]{inputenc}
\usepackage[T1]{fontenc}
\usepackage{textcomp}
\usepackage{verbatim}
\usepackage{enumitem}
\usepackage{longtable}
\usepackage{alltt}
\usepackage{ifthen}
% Reduce the size of the underscore
\usepackage{relsize}
\renewcommand{\_}{\textscale{.7}{\textunderscore}}

\newcommand{\guidetitle}[1]{
\title{#1\\ \vspace{2 mm} {\large PsN 4.1.1}}
\date{2014-02-10}
}

\newcommand{\doctitle}[1]{
\title{#1}
\date{2014-02-10}
}


\newenvironment{optionlist}{
\renewcommand{\arraystretch}{1.1}
\setlength{\leftmargini}{2.5cm}
\begin{description}
%\setlength{\itemsep}{0ex}
}
{\end{description}}

\newcommand{\optname}[1]{\item{{\bfseries\texttt-#1}\newline}}
\newcommand{\optdefault}[2]{\item{{\bfseries\texttt-#1}{\mbox{ = \it #2}}\newline}}

\newcommand{\nextopt}{}

\setlength{\evensidemargin}{0pt}
\setlength{\oddsidemargin}{0pt}

\guidetitle{update\_inits user guide}{2015-09-08}

\begin{document}

\maketitle


\section{Overview}
The update\_inits script is used to update inital estimates in a model file with final
estimates from NONMEM output. It can also be used to, for example, update file names in \$TABLE, reparameterize
the \$OMEGAs or to ``flip comments''. The later operations can be performed in combination with 
or without updating initial estimates.

If updating initial estimates, the final estimates will either be taken from a lst-file given explicitly as the second 
command-line argument after the model file name, from another model file given with option -from\_model, 
or from the lst-file with the same file stem as the model file.
For example, if no lst-file is given as argument and the model file is called run1.mod then the program will try to read output from
run1.lst. If run1.ext also exists then final estimates with higher precision will be read from there, 
but only if run1.lst is also present.

If there are multiple \$PROBLEM then update\_inits will try to update \$PROBLEM by \$PROBLEM.

The command \verb|update| can be used as a synonym to \verb|update_inits|.

\noindent Examples:
\begin{itemize}
\item Update a copy of run1.mod with estimates from run1abc.lst:
\begin{verbatim}
update_inits run1.mod run1abc.lst -output_model=run2.mod
\end{verbatim}
Note: Option -output\_model can be abbreviated to -out.
\item Modify file run33.mod and copy original to run33.mod.org. This assumes that run33.lst exists.
\begin{verbatim}
update_inits run33.mod
\end{verbatim}
\item Update a copy of run22.mod and call the new file run23.mod. This assumes that run22.lst exists.
\begin{verbatim}
update_inits run22.mod -out=run23.mod
\end{verbatim}
\item Use parameters estimates from a model file instead of a lst-file:
\begin{verbatim}
update run3.mod -from_model=run4.mod -out=run6.mod
\end{verbatim}
\end{itemize}

\section{Renumbering}
\subsubsection*{Input run number for 'Based on' tag}
The program will use the number set with option -based\_on, if that option was used.
Otherwise PsN will try to extract an input run number as follows: If the input model file name
starts with run, Run or RUN followed by a digit and contains a dot 
somewhere after the number, then everthing from the first digit to the first dot will be used as the input run number. Examples:
\begin{itemize}
\item run1.mod gives 1
\item run1.2.mod gives 1
\item Run54a.ctl gives 54a
\item pheno.mod gives nothing
\item RUN55.mod gives number 55
\item Run54abc.ctl gives number 54abc
\item run55 gives nothing (since there is no dot anywhere after the number)
\end{itemize}
\subsubsection*{Output number for \$TABLE, \$ESTIMATION, \$MSFI}
If option -renumber is manually set to a number other than 0, the output run number will
be set to this number. If -renumber is not set but option -output\_model is set,
the program will try to extract an output run number using the same rules
as for the input number for the 'Based on' tag described above.

The output number will be used for the FILE option in \$TABLE, the MSFO option in
\$ESTIMATION, and \$MSFI file name for \$PROBLEMS after the first. 
If no output number is defined, either automatically or via option -renumber, 
then those options will not be changed.

The FILE option of all \$TABLE will get <any number up to optional dot> replaced with <output number>. 
Examples if the output number is 9:
\begin{itemize}
\item FILE=patab01 will be changed to FILE=patab9
\item FILE=mytab88.csv to FILE=mytab9.csv
\item FILE=output5abc.csv to FILE=output9.csv
\end{itemize}

\noindent The MSFO option of all \$ESTIMATION will get <any number up to optional dot> replaced with <output number>. 
Examples if the output number is 9:
\begin{itemize}
\item MSFO=msf8 will be changed to MSFO=msf9
\item MSFO=run50.msf to MSFO=run9.msf
\item MSFO=msf11abc to MSFO=msf9
\end{itemize}

\noindent The file name option of all \$MSFI \emph{after the first \$PROB} will get 
<any number up to optional dot> replaced with <output number>. \$MSFI in the first \$PROB
will not be changed.
Examples if the output number is 9:
\begin{itemize}
\item \$MSFI msf8 will be changed to \$MSFI msf9
\item \$MSFI run8.msf to \$MSFI run9.msf
\end{itemize}

\section{Cholesky reparameterization}
The script can perform an automatic Cholesky reparameterization of \$OMEGA and/or \$SIGMA records.
New THETAs, representing the standard deviations and correlations of ETAs and EPSs, will be
defined, and the original ETAs and EPSs will be replaced with linear combinations of new, uncorrelated
ETAs/EPSs with variance 1.

The reparameterization will be performed after updating of initial estimates, if any, but
reparameterization can be performed without updating estimates from e.g. a lst-file.

See also related option -correlation\_cutoff, which can be used to set correlations smaller than
a certain threshold to 0 FIX.

Correct handling of priors encoded using \$OMEGA/\$SIGMA is not implemented, priors must 
be encoded using special records \$OMEGAPD etc.

The script can also perform the inverse of the reparameterization on models that have previously been 
reparameterized using update\_inits. This however requires that
the auto-generated \$PK/\$PRED code and the \$THETA labels in the reparameterized model are unchanged,
otherwise the inversion will fail.

The option is called -cholesky, and it takes either a so-called record category input, a record list input,
or the inverse argument.
 
\subsubsection*{Examples using record list input}
Record list input cannot be combined with category input.\\
\begin{tabular}{ll}
	-cholesky=o2,o5,s1 &               Reparameterize \$OMEGA record number 2 and 5,\\
     &  \$SIGMA record number 1\\
\end{tabular}

\subsubsection*{Examples using category input}
\begin{tabular}{ll}
	-cholesky=all &                Reparameterize both \$OMEGA and \$SIGMA,\\
     & both BLOCK and DIAGONAL, both FIX and\\
     &  estimated\\
	-cholesky=omega &              Reparameterize only \$OMEGA, only BLOCK,\\
     & only estimated\\
	-cholesky=omega,sigma &              Reparameterize both \$OMEGA and \$SIGMA,\\
     &  only BLOCK, only estimated\\
	-cholesky=omega,fix &          Reparameterize only \$OMEGA, only BLOCK,\\
      &  both FIX and estimated\\
	-cholesky=omega,diagonal &     Reparameterize only \$OMEGA, both BLOCK\\
       &  and diagonal, only estimated\\
	-cholesky=omega,diagonal,fix & Reparameterize only \$OMEGA, both BLOCK\\
  &  and diagonal, both FIX and estimated\\
	-cholesky=sigma,diagonal &     Reparameterize only \$SIGMA, both BLOCK\\
      &  and diagonal, only estimated\\
	-cholesky=diagonal  &          Reparameterize only \$OMEGA (omitting\\
      &  parameter name implies omega), both BLOCK\\
       &  and diagonal, only estimated \\
\end{tabular}

\subsubsection*{Inversion of the reparameterization}
\begin{tabular}{ll}
    -cholesky=inverse &             Inverse reparameterization of a model that has previously\\
     &  been reparameterized with update\_inits\\
\end{tabular}

\subsection{Numerical considerations}
The correlations between ETAs and EPSs
will be larger than $-1$ and smaller than $1$. This is necessary \emph{but not sufficient}
for positive definiteness of a block \$OMEGA/\$SIGMA if the block size is larger than 2.

Example: If we have an omega block matrix
\[
A=
\left(
\begin{array}{cccc}
1          & 0          & \sqrt{0.5} & 0.1\\
0          & 1          & \sqrt{0.5} & 0\\
\sqrt{0.5} & \sqrt{0.5} & 1          & 0\\
0.1        & 0          & 0          & 1\\
\end{array}
\right)
\]
where $\sqrt{0.5}\approx 0.7071$
then the naive boundary conditions on the correlations and standard deviations are clearly fulfilled, but this
matrix is singular.
If a singular matrix is set as initial value for an \$OMEGA then NONMEM will
give a clear error message saying that omega is not positive definite, but NONMEM cannot make that check
in a model file where the \$OMEGAs have been Cholesky reparameterized.

In the Cholesky decomposition
a singular matrix leads to diagonal elements that are zero
when computing the Cholesky factors needed for linear combination of the new ETAs,
and there can also be division by zero.
Similarly, if we have a matrix that is \emph{close} to singular, as in
\[
\left(
\begin{array}{cccc}
1          & 0          & 0.7071     & 0.1\\
0          & 1          & 0.7071     & 0\\
0.7071     & 0.7071     & 1          & 0\\
0.1        & 0          & 0          & 1\\
\end{array}
\right)
\]
,
then we will get division by a very small number in one of the Cholesky factors,
which will give a very large factor for one of the ETAs, which can lead to problems.

\section{Input and options}

\subsection{Required input}
The name of a model file (control stream file) is required on the command-line.

\subsection{Optional input}
\begin{itemize}
\item The name of the lst-file to read final estimates from. It is given directly on 
the command-line \emph{after the model file} without an option name.
Cannot be used together with option -from\_model. 
If neither a lst-file or -from\_model is given, 
update\_inits will search for a lst-file with the same file stem as the model file, 
but with extension lst.
\end{itemize}
\begin{optionlist}
\optdefault{output\_model}{filename}
The name of the model file to create. If this options is omitted, a copy of
the original model file with extension .org is created, and 
the original file is modified.
\nextopt
\optname{ignore\_missing\_parameters}
Default not set. If set, update\_inits will not require 
a 1-to-1 matching of parameter names and indexes between the model to update and the source
of new estimates (lst-file or other model file).
\nextopt
\optdefault{from\_model}{filename}
The name of a model file to copy initial estimates from. 
Cannot be used together with a named lst-file on the command-line.
\nextopt
\optdefault{cholesky}{<categories>|<record list>|inverse}
Default not set.
If set, update\_inits will use cholesky reparameterization to replace the requested
\$SIGMA and \$OMEGA with FIXED identity matrices and a set of THETAs for estimation
of standard deviations and (for blocks only) correlations, OR, if -cholesky=inverse is set,
back-transformation of a previously reparameterized model will be performed. See details and
examples in section Cholesky reparameterization.
\nextopt
\optdefault{correlation\_cutoff}{number}
Default 0. Ignored unless option -cholesky is set. Value range 0-1.
If, during cholesky reparameterization, the absolute value of a correlation 
in a block \$OMEGA or \$SIGMA is lower than or equal to this cutoff, then set
the \$THETA for estimation of this correlation to 0 FIX.
\nextopt
\optdefault{comment}{text}
If the option is used, a new line with <comment> will be inserted 
directly following the \$PROBLEM row.
The comment text must be enclosed with quotes (double quotes on Windows) 
if it contains spaces.
\nextopt
\optdefault{degree}{fraction}
Default not set. 
After updating the initial estimates in the output file, randomly
perturb them by degree=X, i.e. change estimate to a value
randomly chosen in the range estimate +/- estimate*X while
respecting upper and lower boundaries, if set.
Degree is set to 0.1 when tweak\_inits is set in execute.
\nextopt
\optname{add\_tags}
Add all runrecord tags. update\_inits will not check if any tags 
are already present.
\nextopt
\optname{ensure\_posdef}
By default not set. 
NONMEM sometimes prints OMEGA or SIGMA matrices
in the lst-file which are not positive definite, and the 
ensure\_posdef option offers a way to fix this.
If option is set then PsN will make a cholesky decomposition of
each OMEGA and SIGMA block to check positive definiteness, and,if the cholesky fails,
inflate the diagonal elements of the block with 5\% before updating the model.
\nextopt
\optname{flip\_comments}
Default not set. Between each pair of tag lines
\begin{verbatim}
;Sim_start
\end{verbatim}
and
\begin{verbatim}
;Sim_end
\end{verbatim}
remove the first ; on each line that has ; as the first non-blank
character, and prepend with ; at each line that does not
have a ; as the first non-blank character.
This processing will be done as the very first step, which means lines
that are commented out by this procedure will not be updated in the following steps.
\nextopt
\optdefault{add\_prior}{df1,df2,...}
Default not used. Add \$PRIOR NWPRI based on output object. Will automatically read
estimates and covariances from output and use them to define the 
prior. df should be the degrees of freedom, a comma-separated list
with one integer per omega block.
This feature is highly experimental, and you must check \$PRIOR 
in the new model file manually before using it.
Option -add\_prior cannot be used together with option -from\_model. 
Also note that the informative record names \$THETAP, \$THETAPV etc are used,
so the resulting model can only be run with a NONMEM version that 
supports this.
\nextopt
\optdefault{seed}{string}
The random seed for pertubation if option -degree is set.
\nextopt
\optdefault{nm\_version}{string}
Default is 'default'. The formatting of the initial estimates will depend on the
major version number of NONMEM (e.g. 6 or 7) set in psn.conf for the
NM version chosen.
\nextopt
\optdefault{based\_on}{integer}
If the -based\_on option is used, update\_inits will set 
the runrecord 'Based on' tag (if present, or if option -add\_tags is used) 
to that number. 
If -based\_on is not used, update\_inits will by default try to extract 
a run number from the original model file name and use that instead.
If a number cannot be extracted then nothing will be set. 
\nextopt
\optdefault{renumber}{new number}
See details in section Renumbering.
Set option -renumber=0 to prevent automatic renumbering.
\nextopt
\optname{update\_fix}
Default false. If set, update\_inits will update parameters that are FIX in the model.
\nextopt
\optname{fix\_thetas}
Default false. If set, update\_inits will set all THETAs that are not part of a prior to FIX.
\nextopt
\optname{unfix\_thetas}
Default false. If set, update\_inits will remove FIX, if set, from all THETAs that are not part of a prior.
\nextopt
\optdefault{sigdig}{integer}
Default not set.
Only has effect with NONMEM7 and later and if set to a number <15. 
Print parameter
estimates with this many significant digits in the new model file,
with either scientific or decimal notation depending on which is more compact.
\nextopt
\end{optionlist}

\end{document}
