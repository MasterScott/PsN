\documentclass[a4wide,12pt]{article}
%\setlength{\marginparwidth}{0pt}%35
%\setlength{\marginparsep}{0pt}%?
%\setlength{\evensidemargin}{0pt}
%\setlength{\oddsidemargin}{0pt}
\usepackage{lmodern}
\usepackage[utf8]{inputenc}
\usepackage[T1]{fontenc}
\usepackage{textcomp}
\usepackage{verbatim}
\usepackage{enumitem}
\usepackage{longtable}
\usepackage{alltt}
\usepackage{ifthen}
% Reduce the size of the underscore
\usepackage{relsize}
\renewcommand{\_}{\textscale{.7}{\textunderscore}}

\newcommand{\guidetitle}[1]{
\title{#1\\ \vspace{2 mm} {\large PsN 4.1.1}}
\date{2014-02-10}
}

\newcommand{\doctitle}[1]{
\title{#1}
\date{2014-02-10}
}


\newenvironment{optionlist}{
\renewcommand{\arraystretch}{1.1}
\setlength{\leftmargini}{2.5cm}
\begin{description}
%\setlength{\itemsep}{0ex}
}
{\end{description}}

\newcommand{\optname}[1]{\item{{\bfseries\texttt-#1}\newline}}
\newcommand{\optdefault}[2]{\item{{\bfseries\texttt-#1}{\mbox{ = \it #2}}\newline}}

\newcommand{\nextopt}{}

\setlength{\evensidemargin}{0pt}
\setlength{\oddsidemargin}{0pt}
\usepackage{hyperref}
\guidetitle{UPDATE\_INITS user guide}{2018-03-02}

\begin{document}

\maketitle
\tableofcontents
\newpage

\section{Introduction}
The update\_inits tool is used to update inital estimates in a model file with final estimates from NONMEM output. It can also be used to, for example, update file names in \$TABLE, reparameterize the \$OMEGAs or to ``flip comments''. The later operations can be performed in combination with or without updating initial estimates.

If updating initial estimates, the final estimates will either be taken from a lst-file given explicitly as the second command line argument after the model file name, from another model file given with option -from\_model, or from the lst-file with the same file stem as the model file. For example, if no lst-file is given as argument and the model file is called run1.mod then the program will try to read output from run1.lst. If run1.ext also exists then final estimates with higher precision will be read from there, 
but only if run1.lst is also present.

If there are multiple \$PROBLEM then update\_inits will try to update \$PROBLEM by \$PROBLEM.

The command \verb|update| can be used as a synonym to \verb|update_inits|.\\

\noindent Examples:\\
Update a copy of run1.mod with estimates from run1abc.lst.
\begin{verbatim}
update_inits run1.mod run1abc.lst -output_model=run2.mod
\end{verbatim}
\newpage
Update a copy of run22.mod and call the new file run23.mod. This requires that run22.lst exists.
\begin{verbatim}
update_inits run22.mod -out=run23.mod
\end{verbatim}
Modify file run33.mod and copy original to run33.mod.org. This requires that run33.lst exists.
\begin{verbatim}
update_inits run33.mod
\end{verbatim}
Use parameters estimates from a model file instead of a lst-file:
\begin{verbatim}
update run3.mod -from_model=run4.mod -out=run6.mod
\end{verbatim}

\section{Input and options}

\subsection{Required input}
The name of a model file (a control stream file) is required on the command line.

\subsection{Optional input}
The name of the lst-file to read final estimates from is optional. It is given directly on the command line \emph{after the model file} without an option name. Cannot be used together with option -from\_model or -rawres\_input. If neither a lst-file, -rawres\_input or -from\_model is given, update\_inits will search for a lst-file with the same file stem as the model file, but with extension lst.

\begin{optionlist}
\optdefault{add\_prior}{df1,df2,...}
Default not set. Add \$PRIOR NWPRI based on output object. Option will automatically read estimates and covariances from output and use them to define the prior. df should be the degrees of freedom, a comma-separated list with one integer per omega block. This feature is highly experimental, and you must check \$PRIOR in the new model file manually before using it. Option -add\_prior cannot be used together with option -from\_model. Also note that the informative record names \$THETAP, \$THETAPV etc are used,
so the resulting model can only be run with a NONMEM version that supports this.
\nextopt
\optname{add\_tags}
Default not set. Add all runrecord tags. update\_inits will not check if any tags are already present.
\nextopt
\optdefault{based\_on}{integer}
If the -based\_on option is set, update\_inits will set the runrecord 'Based on' tag (if present, or if option -add\_tags is used) 
to that number. If -based\_on is not set, update\_inits will by default try to extract a run number from the original model file name and use that instead. If a number cannot be extracted then nothing will be set. 
\nextopt	
\optname{bounded\_theta}
Default set. Ignored unless option -cholesky is set. If the option is set, then the standard deviations and correlations will be equal to bounded untransformed THETAs. If the option is unset with -no-bounded\_theta, the THETAs introduced during Cholesky reparameterization will be unbounded (natural logarithm of standard deviation, logit of shifted and scaled correlation), and transformations will be used to restrict standard deviations to positive values and correlations between -1 and 1.
\nextopt
\optdefault{cholesky}{<categories>|<record list>|inverse}
Default not set. If set, update\_inits will use cholesky reparameterization to replace the requested \$SIGMA and \$OMEGA with FIXED identity matrices and a set of THETAs for estimation of standard deviations and (for blocks only) correlations, OR, if -cholesky=inverse is set, back-transformation of a previously reparameterized model will be performed. See details and examples in the section Cholesky reparameterization.
\nextopt
\optdefault{comment}{text}
If the option is set, a new line with <comment> will be inserted directly following the \$PROBLEM row. The comment text must be enclosed with quotes (double quotes on Windows) if it contains spaces.
\nextopt
\optdefault{correlation\_cutoff}{N}
Default is 0. Ignored unless option -cholesky is set. Value range 0-1. If, during cholesky reparameterization, the absolute value of a correlation in a block \$OMEGA or \$SIGMA is lower than or equal to this cutoff, then set the \$THETA for estimation of this correlation to 0 FIX.
\nextopt
\optdefault{degree}{fraction}
Default not set. After updating the initial estimates in the output file, randomly perturb them by degree=fraction, i.e. change estimate to a value	randomly chosen in the range estimate +/- estimate*fraction while respecting upper and lower boundaries, if set in the model file. Degree is set to 0.1, a 10\% perturbation, when option tweak\_inits is set in execute.
EOF
\nextopt
\optname{ensure\_posdef}
Default not set. NONMEM sometimes prints OMEGA or SIGMA matrices in the lst-file which are not positive definite, and the 
ensure\_posdef option offers a way to fix this. If option is set then PsN will make a cholesky decomposition of each OMEGA and SIGMA block to check positive definiteness, and, if the cholesky fails, inflate the diagonal elements of the block with 5\% before updating the model.
\nextopt
\optname{etas}
Default not set. Add \$ETAS to the model using the phi file of the model. Also add the necessary MCETA=1 to \$EST.
\nextopt
\optname{fix\_thetas}
Default false. If set, update\_inits will set all THETAs that are not part of a prior to FIX. 
\nextopt
\newpage
\optname{flip\_comments}
Default not set. Between each pair of tag lines
\begin{verbatim}
;Sim_start
\end{verbatim}
and
\begin{verbatim}
;Sim_end
\end{verbatim}
remove the first ; on each line that has ; as the first non-blank character, and prepend with ; at each line that does not	have a ; as the first non-blank character. This processing will be done as the very first step, so lines that are commented out by this procedure will not be updated any more.
\nextopt
\optdefault{from\_model}{filename}
Default not set. The name of a model file to copy initial estimates from, instead of a lst-file. Cannot be used together with a named lst-file or rawres\_input on the command line.
\nextopt
\optname{ignore\_missing\_parameters}
Default not set. If set, update\_inits will not require a 1-to-1 matching of parameter names and indexes between the model to update and the source of new estimates (lst-file or other model file).
\nextopt
\optdefault{in\_filter}{list of conditions}
Default not set. Option is only relevant in combination with rawres\_input. The parameter estimates lines in the file can be filtered on values in the different columns. When specifying which column(s) the filtering should be based on, the exact column name must be used, e.g. minimization\_successful. Filtering can only be based on columns with numeric values. The allowed relations are .gt. (greater than), .lt. (less than) and .eq. (equal to). Conditions are separated with commas. \\
Example: -in\_filter=model.eq.2
\nextopt
\optdefault{nm\_version}{string}
Default is 'default'. The formatting of the initial estimates will depend on the major version number of NONMEM (e.g. 6 or 7) set in psn.conf for the NONMEM version chosen.
\nextopt
\optdefault{offset\_rawres}{N}
Default is 0. Option is only relevant in combination with rawres\_input. The number of lines to skip in the input raw results file before reading final parameter estimates.
\nextopt
\optdefault{output\_model}{filename}
The name of the model file to create. If this options is omitted, a copy of the original model file with extension .org is created, and the original file is modified.
\nextopt
\optdefault{rawres\_input}{filename}
The name of a raw results file to read parameter estimates from. The first line in the file after -offset\_rawres that passes the in\_filter criteria will be used when updating initial estimates. Cannot be used together with a named lst-file on the command line or -from\_model.
\nextopt
\optdefault{renumber}{new number}
Default extracted from the -output\_model file name. 

If -output\_model=runY.mod is set where Y is a number then -renumber=Y will be set automatically. Provided that 'new number' is not 0, the FILE option of all \$TABLE will get <any number up to optional dot> replaced with <new number>, the MSFO option of all \$ESTIMATION will get <any number up to optional dot> replaced with <new number>, and \$MSFI filename of second and later \$PROB will get <any number up to optional dot> replaced with <new number>. Set option -renumber=0 to prevent automatic renumbering in \$TABLE and \$EST and \$MSFI.
\nextopt
\optdefault{seed}{string}
The random seed for pertubation if option -degree is set.
\nextopt
\optdefault{sigdig}{integer}
Default not set. Option has only effect with NONMEM 7 and later and if set to a number <15. Print parameter estimates with this many significant digits in the new model file, with either scientific or decimal notation depending on which is more compact.
\nextopt
\optname{silent}
Default not set. If set, all log messages from PsN are suppressed.
\nextopt
\optname{unfix\_thetas}
Default not set. If set, update\_inits will remove FIX, if set, from all THETAs that are not part of a prior.
\nextopt
\optname{update\_fix}
Default not set. If set, update\_inits will update parameters that are FIX in the model. The default is to not update parameters that are fixed.
\nextopt
\end{optionlist}

\section{Renumbering}
\subsubsection*{Input run number for 'Based on' tag}
The program will use the number set with option -based\_on, if that option was set. Otherwise PsN will try to extract an input run number as follows: If the input model file name starts with run, Run or RUN followed by a digit and contains a dot somewhere after the number, then everthing from the first digit to the first dot will be used as the input run number.\\
Examples:
\begin{itemize}
\item run1.mod gives 1
\item run1.2.mod gives 1
\item Run54a.ctl gives 54a
\item pheno.mod gives nothing
\item RUN55.mod gives number 55
\item Run54abc.ctl gives number 54abc
\item run55 gives nothing (since there is no dot anywhere after the number)
\end{itemize}
The 'Based on' tag will not be updated if the input and output model file is the same. For example none of the following will update the 'Based on' tag.
\begin{verbatim}
update_inits run2.mod -out=run2.mod
update_inits run3.mod
\end{verbatim}

\subsubsection*{Output number for \$TABLE, \$ESTIMATION, \$MSFI}
If option -renumber is manually set to a number other than 0, the output run number will be set to this number. If -renumber is not set but option -output\_model is set,the program will try to extract an output run number using the same rules as for the input number for the 'Based on' tag described above.

The output number will be used for the FILE option in \$TABLE, the MSFO option in \$ESTIMATION, and \$MSFI file name for \$PROBLEMS after the first. If no output number is defined, either automatically or via option -renumber, then those options will not be changed.

The FILE option of all \$TABLE will get <any number up to optional dot> replaced with <output number>.\\
Examples if the output number is 9:
\begin{itemize}
\item FILE=patab01 will be changed to FILE=patab9
\item FILE=mytab88.csv to FILE=mytab9.csv
\item FILE=output5abc.csv to FILE=output9.csv
\end{itemize}

\noindent The MSFO option of all \$ESTIMATION will get <any number up to optional dot> replaced with <output number>. 
Examples if the output number is 9:
\begin{itemize}
\item MSFO=msf8 will be changed to MSFO=msf9
\item MSFO=run50.msf to MSFO=run9.msf
\item MSFO=msf11abc to MSFO=msf9
\end{itemize}

\noindent The file name option of all \$MSFI \emph{after the first \$PROB} will get <any number up to optional dot> replaced with <output number>. \$MSFI in the first \$PROB will not be changed.\\
Examples if the output number is 9:
\begin{itemize}
\item \$MSFI msf8 will be changed to \$MSFI msf9
\item \$MSFI run8.msf to \$MSFI run9.msf
\end{itemize}

\section{Cholesky reparameterization}
The update\_inits tool can perform an automatic Cholesky reparameterization of \$OMEGA and/or \$SIGMA records. New THETAs, representing the standard deviations and correlations of ETAs and EPSs, will be defined. The standard deviation THETAs will be labelled SD\_<letter><number>, for example SD\_A2, where <letter> is the ``index'' of the \$OMEGA/\$SIGMA record and the number is the row number in that record. 
The correlation THETAs will be labelled COR\_<letter><number><number>, for example COR\_A21,  where <letter> is the ``index'' of the \$OMEGA/\$SIGMA record and the two numbers are the row and column numbers in that record. A set of variables for the Cholesky factors will be defined in the code, and these will be named CH\_<letter><number><number> analogously to the correlation variables.
Using the new variables, the original ETAs and EPSs will be replaced with linear combinations of new, uncorrelated ETAs/EPSs with variance 1.\\
Example (option -bounded\_theta is set as per default):

\begin{verbatim}
SD_A1=THETA(5)
SD_A2=THETA(6)
COR_A21=THETA(7)
SD_A3=THETA(8)
COR_A31=THETA(9)
COR_A32=THETA(10)
SD_A4=THETA(11)
COR_A41=THETA(12)
COR_A42=THETA(13)
COR_A43=THETA(14)
;Comments below show CH variables for 1st column,
too simple to need new variables
;CH_A11=1
;CH_A21=COR_A21
;CH_A31=COR_A31
;CH_A41=COR_A41
CH_A22=SQRT(1-(COR_A21**2))
CH_A32=(COR_A32-COR_A21*COR_A31)/CH_A22
CH_A42=(COR_A42-COR_A21*COR_A41)/CH_A22
CH_A33=SQRT(1-(COR_A31**2+CH_A32**2))
CH_A43=(COR_A43-(COR_A31*COR_A41+CH_A32*CH_A42))/CH_A33
CH_A44=SQRT(1-(COR_A41**2+CH_A42**2+CH_A43**2))
ETA_1=ETA(1)*SD_A1
ETA_2=ETA(1)*COR_A21*SD_A2+ETA(2)*CH_A22*SD_A2
ETA_3=ETA(1)*COR_A31*SD_A3+ETA(2)*CH_A32*SD_A3+ETA(3)*CH_A33*SD_A3
ETA_4=ETA(1)*COR_A41*SD_A4+ETA(2)*CH_A42*SD_A4+ETA(3)*CH_A43*SD_A4+ETA(4)*CH_A44*SD_A4

\end{verbatim}

The reparameterization will be performed after updating of initial estimates, if any, but reparameterization can be performed without updating estimates from e.g. a lst-file.

See also related option -correlation\_cutoff, which can be used to set correlations smaller than a certain threshold to 0 FIX.

Correct handling of priors encoded using \$OMEGA/\$SIGMA is not implemented, priors must be encoded using special records \$OMEGAPD etc.

The script can also perform the inverse of the reparameterization on models that have previously been reparameterized using update\_inits. This however requires that the auto-generated \$PK/\$PRED code and the \$THETA labels in the reparameterized model are unchanged, otherwise the inversion will fail.

The option is called -cholesky, and it takes either a so-called record category input, a record list input, or the inverse argument.

It is possible to set option -no-bounded\_theta. Then transformations will be used to avoid boundaries on the new THETAs. Then
a standard deviation will be, for example, 
\begin{verbatim}
SD_A1=EXP(THETA(5))
\end{verbatim}
which is always positive, and a correlation
\begin{verbatim}
COR_A21=EXP(THETA(7))*2/(EXP(THETA(7))+1) -1
\end{verbatim}
which is constrained between -1 and 1. %always start with EXP for regexp for inverse
\subsubsection*{Examples using record list input}
Record list input cannot be combined with category input.\\
\begin{tabular}{ll}
	-cholesky=o2,o5,s1 &               Reparameterize \$OMEGA record number 2 and 5,\\
     &  \$SIGMA record number 1\\
\end{tabular}

\subsubsection*{Examples using category input}
\begin{tabular}{ll}
	-cholesky=all &                Reparameterize both \$OMEGA and \$SIGMA,\\
     & both BLOCK and DIAGONAL, both FIX and\\
     &  estimated\\
	-cholesky=omega &              Reparameterize only \$OMEGA, only BLOCK,\\
     & only estimated\\
	-cholesky=omega,sigma &              Reparameterize both \$OMEGA and \$SIGMA,\\
     &  only BLOCK, only estimated\\
	-cholesky=omega,fix &          Reparameterize only \$OMEGA, only BLOCK,\\
      &  both FIX and estimated\\
	-cholesky=omega,diagonal &     Reparameterize only \$OMEGA, both BLOCK\\
       &  and diagonal, only estimated\\
	-cholesky=omega,diagonal,fix & Reparameterize only \$OMEGA, both BLOCK\\
  &  and diagonal, both FIX and estimated\\
	-cholesky=sigma,diagonal &     Reparameterize only \$SIGMA, both BLOCK\\
      &  and diagonal, only estimated\\
	-cholesky=diagonal  &          Reparameterize only \$OMEGA (omitting\\
      &  parameter name implies omega), both BLOCK\\
       &  and diagonal, only estimated \\
\end{tabular}

\subsubsection*{Inversion of the reparameterization}
\begin{tabular}{ll}
    -cholesky=inverse &             Inverse reparameterization of a model that has previously\\
     &  been reparameterized with update\_inits\\
\end{tabular}

\subsection{Numerical considerations}
The correlations between ETAs and EPSs will be larger than $-1$ and smaller than $1$. This is necessary \emph{but not sufficient}
for positive definiteness of a block \$OMEGA/\$SIGMA if the block size is larger than 2.\\
Example: If we have an omega block matrix
\[
A=
\left(
\begin{array}{cccc}
1          & 0          & \sqrt{0.5} & 0.1\\
0          & 1          & \sqrt{0.5} & 0\\
\sqrt{0.5} & \sqrt{0.5} & 1          & 0\\
0.1        & 0          & 0          & 1\\
\end{array}
\right)
\]
where $\sqrt{0.5}\approx 0.7071$
then the naive boundary conditions on the correlations and standard deviations are clearly fulfilled, but this matrix is singular.
If a singular matrix is set as initial value for an \$OMEGA then NONMEM will give a clear error message saying that omega is not positive definite, but NONMEM cannot make that check in a model file where the \$OMEGAs have been Cholesky reparameterized.

In the Cholesky decomposition a singular matrix leads to diagonal elements that are zero, and there can also be division by zero during computation of the CH factors. Similarly, if we have a matrix that is \emph{close} to singular, as in
\[
\left(
\begin{array}{cccc}
1          & 0          & 0.7071     & 0.1\\
0          & 1          & 0.7071     & 0\\
0.7071     & 0.7071     & 1          & 0\\
0.1        & 0          & 0          & 1\\
\end{array}
\right)
\]
then we will get division by a very small number in one of the CH factors, which will give a very large multiplier for an ETA, which can lead to numerical problems.
\end{document}
