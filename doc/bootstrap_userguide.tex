\documentclass[a4wide,12pt]{article}
%\setlength{\marginparwidth}{0pt}%35
%\setlength{\marginparsep}{0pt}%?
%\setlength{\evensidemargin}{0pt}
%\setlength{\oddsidemargin}{0pt}
\usepackage{lmodern}
\usepackage[utf8]{inputenc}
\usepackage[T1]{fontenc}
\usepackage{textcomp}
\usepackage{verbatim}
\usepackage{enumitem}
\usepackage{longtable}
\usepackage{alltt}
\usepackage{ifthen}
% Reduce the size of the underscore
\usepackage{relsize}
\renewcommand{\_}{\textscale{.7}{\textunderscore}}

\newcommand{\guidetitle}[1]{
\title{#1\\ \vspace{2 mm} {\large PsN 4.1.1}}
\date{2014-02-10}
}

\newcommand{\doctitle}[1]{
\title{#1}
\date{2014-02-10}
}


\newenvironment{optionlist}{
\renewcommand{\arraystretch}{1.1}
\setlength{\leftmargini}{2.5cm}
\begin{description}
%\setlength{\itemsep}{0ex}
}
{\end{description}}

\newcommand{\optname}[1]{\item{{\bfseries\texttt-#1}\newline}}
\newcommand{\optdefault}[2]{\item{{\bfseries\texttt-#1}{\mbox{ = \it #2}}\newline}}

\newcommand{\nextopt}{}

\guidetitle{Bootstrap user guide}{2018-03-02}

\usepackage{hyperref}

\begin{document}
	
\maketitle
\newcommand{\guidetoolname}{bootstrap}
\tableofcontents
\newpage

\section{Introduction}

The bootstrap tool can be used to calculate bias, standard errors and confidence intervals. 
It does so by resampling with replacement from the data. To compute standard errors for 
all parameters of a model using the non-parametric bootstrap implemented here, roughly 200 
mmdel fits are necessary. To assess 95\% confidence intervals approximatly 2000 runs will suffice.
\cite{Efron}. See also \cite{Niebecker} for diagnostics using option -dofv.

Example:
\begin{verbatim}
bootstrap run12.mod -samples=500 -seed=12345 -threads=5
\end{verbatim}

\section{Input and options}

\subsection{Required input}
A model file is required on the command line.

\subsection{Optional input}
	
\begin{optionlist}
\optname{allow\_ignore\_id}
Default not set. When not used, bootstrap will print a message and terminate if an IGNORE or ACCEPT statement based on the ID column is found in the \$DATA record. This is done because it would interfere with the internal renumbering of individuals that the script does, producing errors since the script renumbers individuals before sampling. If -allow\_ignore\_id is used (not recommended), a warning is printed but the program continues execution. 

Note: The IGNORE statement can safely be used in conjunction with any other column than ID.
\nextopt
\optname{bca}
Default not set. When used, the bootstrap utility will calculate the confidence intervals through the BCa method \cite{Efron}. The BCa is intended for calculation of second-order correct confidence intervals. Warning: Using bca is very time-consuming. 
\nextopt
\optname{copy\_data}
Default set. If option is set, the original data file will be copied to the run directory if the input model is to be run. If option is unset using -no-copy\_data, the absolute path to the original data file will be used in \$DATA when the input model is run, and the data file will not be copied which saves disk space.
\nextopt
\optname{dofv}
Default not set. Compute delta-ofv by doing a MAXEVAL=0 run on orginal dataset with the bootstrap parameter estimates.
\nextopt
\optname{keep\_covariance}
By default the \$COVARIANCE will be deleted from the bootstrap models, to save run time. If option -keep\_covariance is set, PsN will instead keep \$COVARIANCE.
\nextopt
\optname{keep\_tables}
By default all \$TABLE will be deleted from the bootstrap models, to save disk space. If option -keep\_tables is set, PsN will instead keep \$TABLE, which can require much disk space.
\nextopt
\optdefault{mceta}{N}
Default not set. If option -dofv is set and NONMEM 7.3 or later is used, setting this option will make PsN set MCETA=N in \$ESTIMATION. It is up to the user to check that the estimation method used can in NONMEM be combined with option MCETA, PsN will not do that.
\nextopt
\optname{run\_base\_model}
By default, the input model will be run unless the lst-file for the input model is found in the same directory as the input model. If -no-run\_base\_model  is set, the input model will not be run even if the lst-file is missing.
\nextopt
\optdefault{sample\_size}{M}
The number of subjects in each bootstrap data set. The default value is set to the number of individuals in the original data set. When the resampling is stratified, the sample\_size option can be used to specify the exact number of samples that should be drawn from each strata. Below follows an example of the syntax that should be used in such a case. Stratification is here done based on the study number, STUD, with the values 1001, 1002 and 1003. -sample\_size=''1001$=>$12,1002$=>$24,1003$=>$10''                          
Note the double quotes, single quotes will not work in Windows. This example specifies that the bootstrap should use 12 samples from study 1001, 24 samples from 1002 and 10 from study 1003. If only one sample size is used together with stratified resampling (the default case; sample\_size=number of individuals in the data set), the strata are assigned samples in proportion to their size in the data set. Please note that this usage of the sample\_size option does not guarantee that the sum of the samples of the strata is equal to the given sample\_size since PsN needs to round the figures to the closest integer. For a sample size equal to the number of individuals in the data set, the sum will however always be correct. 
\nextopt
\optdefault{samples}{N}
The number of bootstrapped datasets to generate. Default is 200. 
\nextopt
\optname{skip\_covariance\_step\_terminated}
Default not set. With this option enabled, the bootstrap will skip all samples where the NONMEM run terminated the covariance step.\\
Force disabling with -no\-skip\_covariance\_step\_terminated.
\nextopt
\optname{skip\_estimate\_near\_boundary}
Default set. With this option enabled, the bootstrap will skip all samples where the NONMEM run signal that some estimates are near its boundary.\\
Force disabling with -no-skip\_estimate\_near\_boundary.
\nextopt
\optname{skip\_minimization\_terminated}
Default set. The bootstrap will skip all samples where the NONMEM run terminated the minimization step.\\ 
Force disabling with -no-skip\_minimization\_terminated. 
\nextopt
\optname{skip\_with\_covstep\_warnings}
Default not set. With this option enabled, the bootstrap will skip all samples
where the NONMEM run had warnings from the covariance step.\\
Force disabling with -no\-skip\_with\_covstep\_warnings.
\nextopt
\optdefault{stratify\_on}{column name}
It may be necessary to use stratification in the resampling procedure. For example, if the original data consists of two groups of patients - say 10 patients with full pharmacokinetic profiles and 90 patients with sparse steady state concentration measurements - it may be wise to restrict the resampling procedure to resample within the two groups, producing bootstrap data sets that all contain 10 rich + 90 sparse data patients but with different compositions. The default is not to use stratification. Set -stratify\_on to the column (the name in \$INPUT) that defines the two groups. Note that the option sample\_size has a different behaviour when stratified resampling is used.\\
Bootstrapping is always done on entire individuals, so for each ID the data records are either all included or all excluded from a particular bootstrapped data set. The algorithm requires that an individual can unambigously be categorized according to the stratification variable. In data file terms, it means that the variable used for stratification must have one and only one value for each individual, otherwise the program will stop with an error message saying that at least one individual 
has multiple values in the stratification column, and therefore this column cannot be used for stratification of the resampling.
\nextopt
\optname{summarize}
Default not set. Only allowed when -directory is set to an existing bootstrap run directory and
the raw\_results file exist in the directory. Recompute bootstrap\_results based on the existing raw\_results, possibly using a different set of exclusion criteria.
\nextopt
\optname{update\_inits}
By default, the initial estimates of the bootstrap models will be set to the final estimates of the input model if final estimates are available. If the user sets -no-update\_inits, the initial estimates of the bootstrap models will be the same as the initial estimates set in the input model, even if the final estimates from the input model are available.
\nextopt
\end{optionlist}

\subsection{Diagnostic R plots}
\newcommand{\rplotsconditions}{
See section Output, subsections Basic and Extended R plots,
for descriptions of the default bootstrap R plots.
The default bootstrap template 
requires the xpose4 R library of at least version 4.5.0,
and that R libraries ggplot2, plyr, dplyr are installed.
If the conditions are not fulfilled then no pdf will be generated,
see the .Rout file in the main run directory for error messages.
}
PsN can automatically generate R plots to visualize results for \guidetoolname, using a default template found in the R-scripts subdirectory of the installation directory. The user can also create a custom template, see more details in the section Auto-generated R-plots from PsN in common\_options.pdf.

\rplotsconditions

\begin{optionlist}
\optdefault{rplots}{level}
-rplots<0 means R script is not generated\\ 
-rplots=0 (default) means R script is generated but not run\\ 
-rplots=1 means basic plots are generated\\													  
-rplots=2 means basic and extended plots are generated\\													  
\nextopt
\end{optionlist}

\subsubsection*{Troubleshooting}
If no .pdf was generated even if a template file is available and the appropriate options were set, check the .Rout-file in the main run directory for error messages. If no .Rout-file exists, then check that R is properly installed, and that either command 'R' is available or that R is configured in psn.conf.


\subsection{PsN common options}
For a complete list see common\_options.pdf or type psn\_options -h on the command line.


\section{Output}
The file bootstrap\_results.csv contains statistics and summaries specific for the bootstrap:
\begin{itemize}
    \item diagnostic.means - Means of all the diagnostics (such as estimate.near.boundary and subproblem.est.time) for the bootstrap runs.
    \item means - The means of parameter estimates, SEs etc. of the bootstrap distribution.
    \item bias - The difference between the bootstrap mean and the value for the original model
    \item standard.error.confidence.intervals - The confidence intervals of the estimates of the base model assuming that it is normally distributed with the standard deviation equal to that of the bootstrap distribution. 
    \item standard.errors - The standard deviation of parameter estimates, SEs etc. of the bootstrap distribution
    \item medians - The median values of parameter estimates, SEs etc. of the bootstrap distribution
    \item percentile.confidence.intervals - Confidence intervals of the bootstrapped distribution. The intervals are calculated using the weighted average at $x_{(n+1)p}$ method:
    \begin{enumerate}
        \item Set $n$ = the number of observations + 1
        \item Set p = number of the percentile to find / 100
        \item Set i = The integer part of $n\cdot p$
        \item Set f = The decimal part of $n\cdot p$
        \item Percentile = $(1-f)x_i + fx_{i+1}$
    \end{enumerate}
\end{itemize}

The details of which results are reported depend on the number of successful samples after filtering according to
option -skip\_minimization\_terminated etc. The minimum count successful samples required to obtain
percentile confidence intervals is % formula is  (200/limit) -1, i.e. 
19 for interval 5\%-95\%, 39 for interval 2.5\%-97.5\%, 199 for interval 0.5\%-99.5\% and 1999 for interval 0.05\%-99.95\%.

The raw\_results csv-file, where the exact name depends on the name of the model file, 
is a standard PsN file containing raw result data for termination status, parameter estimates, uncertainty estimates etc. for all model estimations. 
The first row is for the original dataset.

A file delta\_ofv.csv is created if option -dofv is set, see section Computing delta-ofv.

included\_individuals1.csv: One row per bootstrapped dataset. Each row consists of the ID:s of the individuals in that dataset. Note that different individuals may have the same ID number in a NONMEM dataset. The numbers appear in the order the individuals appear in the datafile.

included\_keys1.csv: One row per bootstrapped dataset. Each row consists of the internal and unique order numbers (1-N) of the individuals in that dataset. The numbers appear in the order the individuals appear in the bootstrapped dataset. 

sample\_keys1.csv:  One row per bootstrapped dataset. Each row consists of one number C per individual in the dataset, in the order the individuals appear in the original datafile, where C is the number of times the individual is included in the bootstrapped dataset. 

The row order is consistent between the files raw\_results, included\_individuals, included\_keys and sample\_keys so that row j (excluding headers if any) in each of the files concerns the same bootstrapped dataset.

all\_individuals1.csv: A list of all the individuals of the original dataset in the original order.

\subsection{Basic R plots}
A basic bootstrap R plot will be generated in file PsN\_bootstrap\_plots.pdf
if option -rplots is set >0, and the general R plots conditions fulfilled, see above.
The plot is created using the xpose4 boot.hist function
and shows histograms of each parameter in the model.

If option -dofv is set, there will also be a dOFV plot.
The dOFV are computed as the difference between the OFV of the 
bootstrap parameter vector evaluated on the original dataset minus the OFV 
of the final parameter estimates on the original dataset. 
This plot shows 3 dOFV distributions:  
\begin{enumerate}
\item that of all bootstrap samples, whether minimization on the bootstrap dataset was successful or not ('yes+no', red dotted line), 
\item that of bootstrap samples for which minimization was successful ('yes', red full line) and 
\item that of a chi-square distribution with degrees of freedom equal to the total number of estimated parameters in the 
model (blue full line). 
\end{enumerate}
In theory the bootstrap dOFV distribution should be at or below the reference chi-square. If it is not, another method for estimating parameter uncertainty, such as SIR, should be considered. Differences between the 2 red curves indicate that bootstrap results are likely to differ depending on whether the termination status is taken into account or not. 

\subsection{Extended R plots}
Extended bootstrap R plots will be generated in file PsN\_bootstrap\_plots.pdf
if option -rplots is set >1,
and the general R plots conditions fulfilled, see above.
The R plots are created using the xpose4 boot.hist function
and show histograms of standard errors, ofv and, if available
from NONMEM output, eigenvalues.
Be aware of the difference between OFV in the extended ofv plot,
that correspond to OFV obtained by estimation on the bootstrap datasets,
and the dOFV in Basic R plots.


\subsection{Computing a covariance matrix}
It is possible to use the covmat program to compute a covariance matrix based on the bootstrap samples. The -in\_filter option of covmat must be set to match the settings of bootstrap options -skip\_covariance\_step\_terminated,
-skip\_estimate\_near\_boundary, -skip\_minimization\_terminated and\\
-skip\_with\_covstep\_warnings. The settings of those options can be checked in the version\_and\_option\_info.txt file in
the bootstrap run directory.

Option -in\_filter is a comma-separated list of conditions.
If bootstrap option -skip\_minimization\_terminated was true, then add\\ 
\verb|minimization_successful.eq.1|
to -in\_filter. 

If bootstrap option -skip\_estimate\_near\_boundary was true, then add 
\verb|estimate_near_boundary.eq.0| 
to -in\_filter. 
If the bootstrap model has \$COV \emph{and} bootstrap option -skip\_covariance\_step\_terminated was true, then add
\verb|covariance_step_successful.eq.1| 
to -in\_filter. 
If the bootstrap model has \$COV \emph{and} bootstrap option -skip\_with\_covstep\_warnings was true, then add
\verb|covariance_step_warnings.eq.0| 
to -in\_filter. 
If the bootstrap model does \emph{not} have \$COV, then -in\_filter should not include conditions on covariance\_step\_successful or
covariance\_step\_warnings.

Example: If the bootstrap was run with
\begin{verbatim}
bootstrap run12.mod -samples=500 -dir=boot_run12 -skip_estimate_
near_boundary -skip_minimization_terminated
\end{verbatim}
and run12.mod does \emph{not} have \$COV, then a covariance matrix can be obtained with (all in one command)

\begin{verbatim}
covmat run12.mod -rawres_input=boot_run12/raw_results_run12.csv 
-offset=1 -in_filter=minimization_successful.eq.1,estimate_near_
boundary.eq.0
\end{verbatim}

By default the matrix is printed to screen, but it can be redirected to a file using the > operator.


\section{Known bugs and problems}

It is recommended to remove PRINT options from \$ESTIMATION, to save disk space.

PsN renumbers the individuals during bootstrapping. Therefore ACCEPT/IGNORE statements based on ID and also abbreviated code using ID will cause errors. 
There is an input check for this, but it is recommended to not rely on this check. Also, model code that relies on ID numbers will lead to errors.

The dataset is not filtered on IGNORE/ACCEPT statements before bootstrapping,
and all IGNORE/ACCEPT are left in the control stream, so some bootstrap
datasets may in practice be smaller than 'sample\_size' after NONMEM applies
IGNORE/ACCEPT. 

The results of two runs will be different even if the seed is the same if the lst-file of the base model is present at the start of one run but not the other. Running the base model changes the state of the random number generator, and therefore the bootstrapped datasets will be different depending on if the base model is run or not before generating the  new datasets. 

\section{Technical overview of algorithm}

PsN will rerun the base model if the lst-file of the input model (run1.lst if the input model file is called run1.mod) is not present in the same directory as the model file 
and the user has not set -no-run\_base\_model. If the .lst-file is present then PsN will simply read the estimates from that file.
The tool creates N (N=number of samples), datasets of size M (M=sample\_size) by randomly drawing individuals with replacement from the original dataset. The original dataset is not filtered on IGNORE/ACCEPT before bootstrapping. The tool creates N new NONMEM modelfiles which are identical to the original modelfile with the exception that each uses a different  bootstrapped dataset and, unless the user has set -no-update\_inits, that the initial parameter estimates are the final estimates from the original run. 
The model parameters are estimated with each dataset, including the original, resulting in N+1 estimates for each parameter.

\section{Computing delta-ofv}

The delta-ofv distribution can be used as a 
diagnostic of the bootstrap \cite{Niebecker}.
If option -dofv is set, PsN will perform a MAXEVAL=0 run for each set of bootstrap parameter estimates using the original model and data. This is done after all other runs and computations are completed, including any Bca step. The results will be printed to a csv file called delta\_ofv.csv where the first column is bootstrap data id and the second delta-ofv which is computed as 
$\mathrm{ofv}_{bs,maxev=0}$ – $\mathrm{ofv}_{original}$. 
If option -rplots is set >0, a plot with the delta-ofv distribution is
generated, see Basic R plots in section Auto-generated R plots from PsN. 

If option -mceta=N is set and NM7.3 or later is used and the estimation method is classical then option MCETA=N will be set in \$ESTIMATION. There will be up to 200 \$PROBLEMs defined in each model file (control stream) to reduce PsN overhead when calling NONMEM.

It is possible to restart a previously run bootstrap to compute delta-ofv without rerunning all bootstrap estimations, provided that option -clean was set to 2 or less in the original run. Simply run bootstrap again with option -directory set to the existing run directory. PsN will intiate reruns of all bootstrap samples, but reread existing output instead of really starting NONMEM. Then the delta-ofv models will be run. If option -bca was set in the original bootstrap run it is important to set -bca also in the restart, otherwise the jackknife results will be overwritten. 

\section{Recovering a crashed bootstrap}

All the model files and bootstrapped datasets are in the m1 subdirectory. Models that have finished will also have a 
lst-file and other NONMEM output files.  
\subsubsection*{Restarting crashed samples only}
If there is a computer crash that either interrupts the main bootstrap process or causes a number of individual samples to fail, or the bootstrap is manually killed, you can still resume the bootstrap and reuse the samples that did finish.
\begin{enumerate}
    \item Remove the entire modelfit\_dir1 subdirectory of the bootstrap run directory.
	\item Rerun the same bootstrap command (command.txt) and set option -directory=name\_of\_existing\_directory. 
\end{enumerate}

\subsubsection*{If you realize additional options such as -retries would have been needed}
If too many samples terminated due to e.g. rounding errors and you want to use -retries or -picky or similar for only those samples:
\begin{enumerate}
	\item Go to the m1 subdirectory and remove the lst-files for the runs that you wish to rerun. Output files from successful samples must be kept.
    \item Remove the entire modelfit\_dir1 subdirectory of the bootstrap run directory.
	\item Rerun the same bootstrap command 
(command.txt) but set option -directory=name\_of\_existing\_directory 
plus the new options governing retries.
\end{enumerate}

\subsubsection*{If too many samples are filtered out}
If too many samples were discarded due to e.g. minimization terminated and you wish to recompute the bootstrap results without rerunning any samples: 
Rerun the same bootstrap command (command.txt) but set options -summarize
and -directory=name\_of\_existing\_directory and change the
run selection options such as -no-skip\_minimization\_terminated. 

\references

\end{document}
