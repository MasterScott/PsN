\documentclass[a4paper,12pt]{article}
\title{MIMP user guide\\ \vspace{2 mm} {\large PsN 3.6.2}}
\date{2013-05-28}

\usepackage[utf8]{inputenc}
\usepackage{verbatim}
\usepackage{longtable}
\begin{document}

\maketitle


\section{Overview}
The mimp program implements the multiple imputation method\cite{Johansson}.
\\
\\
Example
\begin{verbatim}
mimp -base_model=base.mod -reg_model=reg.mod -mi_model=mi.mod
\end{verbatim}

\section{Input and options}

\subsection{Required input}
Unlike other PsN scripts mimp does not take a bare model file name as input. All model file names are given as options. The dataset needs to contain a column that depicts for each row in the dataset if the covariate that is partly missing is missing or observed.

\begin{longtable}{p{1in}p{4in}}
\verb|-base_model=<filename>| & \\
\nopagebreak
 & Base model. This is the model without the code snippet describing the effect of the covariate (that is partly missing) on the parameters. The individual parameter (empirical Bayes) estimates, of the parameter(s) on which the covariate has an effect, contains information about the individual response. The individual parameter estimates should be printed in \$TABLE of the base model. The model must contain a complete \$TABLE containing items exactly matching \$INPUT of the next model in the sequence. PsN does not check that \$TABLE is correctly defined.  \\
\\
\verb|-reg_model=<filename>| & \\
\nopagebreak
 & Regression model.  The regression model estimates the relationship between the partly missing covariate, the response (individual parameter estimates from the base model) and other (completely observed) covariates that carry information about the missing covariate. The partly missing covariate will be the dependent variable (DV) in this model and the column in the dataset that depicts if the covariate is missing or observed will be the missing dependent variable (MDV). The regression model must contain a complete \$TABLE containing items exactly matching \$INPUT of the multiple imputation model. \\
\\
\verb|-mi_model=<filename>| & \\
\nopagebreak
 & Multiple imputation model. This model must contain one simulation part and one estimation part, followed by both a \$SIM and a \$EST record. The simulation part should contain the code for imputation of the missing covariate while the estimation part should be the final model including the code for the effect of the covariate on the parameters. \\
\\
\end{longtable}


\pagebreak

\subsection{Optional input}

\begin{longtable}{p{1in}p{4in}}
\verb|-imputations=M| & \\
\nopagebreak
 & Default 6. The number of imputations to perform for each dataset. \\
\\
\verb|-sim_model=<filename>| & \\
\nopagebreak
 & Simulation model. This model must contain a correct \$SIM record and a complete \$TABLE containing items exactly matching \$INPUT of the base model. PsN does not check that \$TABLE is correctly defined. \\
\\
\verb|-samples=N| & \\
\nopagebreak
 & Number of datasets to simulate. Only relevant together with option -sim\_model, required if sim\_model is given. \\
\\
\verb|-chain_models= <file1,file2,...>| & \\
\nopagebreak
 & An ordered list of chain models. These models can be used if there is a need for additional estimation models in between the base model and the regression model. The models must be listed in the order in which they are to be estimated. Each model must contain a complete \$TABLE containing items exactly matching \$INPUT of the next model in the sequence (the next chain model or the regression model if it is the last chain model). PsN does not check that \$TABLE is correctly defined.   \\
\\
\verb|-alt_models=<model_1, model_2,...>| & \\
\nopagebreak
 & A comma-separated list of filenames of alternative models. \\
\\
\end{longtable}

\subsection{Some common PsN-options useful with mimp}

For a complete list of common options see common\_options\_defaults\_versions.pdf, or psn\_options -h on the commandline.

\begin{longtable}{p{1in}p{4in}}
\verb|-directory=mimp_dirN| & \\
\nopagebreak
 & The run directory can be named. \\
\\
\verb|-seed=X| & \\
\nopagebreak
 & Random seed.
\\
\verb|-help| & \\
\nopagebreak
 & With -help mimp will print a longer help message. \\
\end{longtable}


\section{Output}
The results are in the raw\_results file of mi\_dirJ, J=1:'imputations', (the multiple imputation subdirectories).

\section{Algorithm overview}

\begin{enumerate}

\item If option -sim\_model is used: Create 'samples' copies of the simulation model and set a unique  seed in each of them. Set FILE in \$TABLE to a unique name (using order number of simulation file). Run the simulation models.
\item If option -sim\_model is used: Create 'samples' copies of the base model. In each copy set filename in \$DATA to a new simulated dataset.
\item In the base model set FILE in \$TABLE to a unique name (repeat this 'samples' times, once for each copy of the base model, if -sim\_model was used).
\item Run the base model (or 'samples' copies of the base model).
\item Repeat for each model given via option chain\_models, if any: set filename in \$DATA to the filename in \$TABLE from the previous model (base model or previous chain model). Set FILE in \$TABLE to a unique name. Run the chain model (or 'samples' copies of the chain model, each with different \$DATA filename).
\item Set filename in \$DATA of the regression model to the filename in \$TABLE from the previous model in the sequence (either from the base model or the last chain model). Set FILE in \$TABLE to a unique name. Run the regression model (or 'samples' copies of the regression model, each with different \$DATA filename).
\item For each alternative model, if defined: Set filename in \$DATA of the alternative model to the filename in \$TABLE from the regression model. Set FILE in \$TABLE to a unique name. Run the alternative model (or 'samples' copies of the alternative model, each with different \$DATA filename).
\item Create 'imputations' copies of the multiple imputation model (or 'imputations' times 'samples' copies if -sim\_model was used). Set filename in \$DATA of the multiple imputation model to the filename in \$TABLE from the regression model. Each filename from regression \$TABLE is set in 'imputations' copies of the multiple imputation model. Set a unique seed in the \$SIM of each multiple imputation model. Run the multiple imputation models.
\end{enumerate}



\begin{thebibliography}{1}

\bibitem{Johansson} Åsa Johansson, {\em Comparison of methods for handling missing covariate data}, www.page-meeting.org/?abstract=1982, PAGE 20 Abstract 1982, 2011
\end{thebibliography}

\end{document}
