\documentclass[a4wide,12pt]{article}
%\setlength{\marginparwidth}{0pt}%35
%\setlength{\marginparsep}{0pt}%?
%\setlength{\evensidemargin}{0pt}
%\setlength{\oddsidemargin}{0pt}
\usepackage{lmodern}
\usepackage[utf8]{inputenc}
\usepackage[T1]{fontenc}
\usepackage{textcomp}
\usepackage{verbatim}
\usepackage{enumitem}
\usepackage{longtable}
\usepackage{alltt}
\usepackage{ifthen}
% Reduce the size of the underscore
\usepackage{relsize}
\renewcommand{\_}{\textscale{.7}{\textunderscore}}

\newcommand{\guidetitle}[1]{
\title{#1\\ \vspace{2 mm} {\large PsN 4.1.1}}
\date{2014-02-10}
}

\newcommand{\doctitle}[1]{
\title{#1}
\date{2014-02-10}
}


\newenvironment{optionlist}{
\renewcommand{\arraystretch}{1.1}
\setlength{\leftmargini}{2.5cm}
\begin{description}
%\setlength{\itemsep}{0ex}
}
{\end{description}}

\newcommand{\optname}[1]{\item{{\bfseries\texttt-#1}\newline}}
\newcommand{\optdefault}[2]{\item{{\bfseries\texttt-#1}{\mbox{ = \it #2}}\newline}}

\newcommand{\nextopt}{}

\guidetitle{MIMP user guide}{2017-11-29}

\begin{document}

\maketitle
\newcommand{\guidetoolname}{mimp}


\section{Overview}
The mimp tool implements the multiple imputation method \cite{Johansson1}, \cite{Johansson2}.\\
Example:
\begin{verbatim}
mimp -base_model=run1.mod -reg_model=run2.mod -mi_model=run3.mod
\end{verbatim}

\section{Input and options}

\subsection{Required input}
Unlike other PsN tools mimp does not take a bare model file name as input. All model file names are given as options. The dataset needs to contain a column that depicts for each row in the dataset if the covariate that is partly missing is missing or observed.

\begin{optionlist}
\optdefault{base\_model}{filename}
Base model. This is the model without the code snippet describing the effect of the covariate (that is partly missing) on the parameters. The individual parameter (empirical Bayes) estimates, of the parameter(s) on which the covariate has an effect, contains information about the individual response. The individual parameter estimates should be printed in \$TABLE of the base model.\\ The model must contain a complete \$TABLE containing items exactly matching \$INPUT of the next model in the sequence. PsN does not check that \$TABLE is correctly defined.  
\nextopt
\optdefault{reg\_model}{filename}
Regression model.  The regression model estimates the relationship between the partly missing covariate, the response (individual parameter estimates from the base model) and other (completely observed) covariates that carry information about the missing covariate. The partly missing covariate will be the dependent variable (DV) in this model and the column in the dataset that depicts if the covariate is missing or observed will be the missing dependent variable (MDV). The regression model must contain a complete \$TABLE containing items exactly matching \$INPUT of the multiple imputation model. 
\nextopt
\optdefault{mi\_model}{filename}
Multiple imputation model. This model must contain one simulation part and one estimation part, followed by both a \$SIM and a \$EST record. The simulation part should contain the code for imputation of the missing covariate while the estimation part should be the final model including the code for the effect of the covariate on the parameters. 
\nextopt
\end{optionlist}

\subsection{Optional input}

\begin{optionlist}
\optdefault{alt\_models}{model\_1, model\_2,...}
A comma-separated list of filenames of alternative models. 
\nextopt
\optdefault{chain\_models}{model1,model2,...}
An ordered list of chain models. These models can be used if there is a need for additional estimation models in between the base model and the regression model. The models must be listed in the order in which they are to be estimated. Each model must contain a complete \$TABLE containing items exactly matching \$INPUT of the next model in the sequence (the next chain model or the regression model if it is the last chain model). PsN does not check that \$TABLE is correctly defined.   
\nextopt
\optdefault{frac\_model}{string}
Fraction model.
\nextopt
\optdefault{imputations}{N}
Default is 6. The number of imputations to perform for each dataset. 
\nextopt
\optdefault{samples}{N}
Number of datasets to simulate. Only relevant together with option -sim\_model, required if sim\_model is given. 
\nextopt
\optdefault{sim\_model}{filename}
Simulation model. This model must contain a correct \$SIM record and a complete \$TABLE containing items exactly matching \$INPUT of the base model. PsN does not check that \$TABLE is correctly defined. 
\nextopt
\end{optionlist}

\subsection{Some important common PsN options}
For a complete list see common\_options.pdf, 
or psn\_options -h on the commandline.
\begin{optionlist}
\optname{h or -?}
Print the list of available options and exit. 
\nextopt
\optname{help}
With -help all programs will print a longer help message. 
If an option name is given as argument, help will be printed for this option. 
If no option is specified, help text for all options will be printed. 
\nextopt
\optdefault{directory}{'string'}
Default \guidetoolname\_dirN,
where N will start at 1 and
be increased by one each time you run the script. The directory option sets the directory in which PsN 
will run NONMEM and where PsN-generated output files will be stored. 
You do not have to create the directory,  it will be done for you. If you set
-directory to a the name of a directory that already exists, PsN will run in the existing directory.
\nextopt
\optdefault{seed}{'string'}
You can set your own random seed to make PsN runs reproducible.
The random seed is a string, so both -seed=12345 and -seed=JustinBieber are valid.
It is important to know that because of the way the Perl pseudo-random
number generator works, for two similar string seeds the random sequences may be identical. 
This is the case e.g. with the two different seeds 123 and 122. 
Setting the same seed guarantees the same sequence, but setting two slightly different 
seeds does not guarantee two different random sequences, that must be verified.
\nextopt
\optdefault{clean}{'integer'}
Default 1. The clean option can take four different values:  
\begin{description}
\item[0] Nothing is removed 
\item[1] NONMEM binary and intermediate files except INTER are removed, and files specified with option -extra\_files. 
\item[2] model and output files generated by PsN restarts are removed, and data files in the NM\_run directory, and (if option -nmqual is used) the xml-formatted NONMEM output. 
\item[3] All NM\_run directories are completely removed. If the PsN tool has created modelfit\_dir:s inside the main run directory, these  will also be removed. 
\end{description}
\nextopt
\optdefault{nm\_version}{'string'}
Default is 'default'. 
If you have more than one NONMEM version installed you can use option
-nm\_version to choose which one to use, as long as it is 
defined in the [nm\_versions] section in psn.conf, see psn\_configuration.pdf for details. 
You can check which versions are defined, without opening psn.conf, using the command
\begin{verbatim}
psn -nm_versions
\end{verbatim}
\nextopt
\optdefault{threads}{'integer'}
Default 1. Use the threads option to enable parallel execution of multiple models.
This option decides how many models PsN will run at the same time, and it is completely
independent of whether the individual models are run with serial NONMEM or parallel NONMEM.
If you want to run a single model in parallel you must use options -parafile and -nodes.
On a desktop computer it 
is recommended to not set -threads higher the number of CPUs in the system plus one. 
You can specify more threads, 
but it will probably not increase the performance. If you are running on a computer cluster, 
you should consult your 
system administrator to find out how many threads you can specify. 
\nextopt
\optname{version}
Prints the PsN version number of the tool, and then exit. 
\nextopt
\end{optionlist}



\section{Output}
The results are in the raw\_results file of mi\_dirJ, J=1:'imputations', (the multiple imputation subdirectories).

\section{Algorithm overview}

\begin{enumerate}

\item If option -sim\_model is used: Create 'samples' copies of the simulation model and set a unique  seed in each of them. Set FILE in \$TABLE to a unique name (using order number of simulation file). Run the simulation models.
\item If option -sim\_model is used: Create 'samples' copies of the base model. In each copy set filename in \$DATA to a new simulated dataset.
\item In the base model set FILE in \$TABLE to a unique name (repeat this 'samples' times, once for each copy of the base model, if -sim\_model was used).
\item Run the base model (or 'samples' copies of the base model).
\item Repeat for each model given via option chain\_models, if any: set filename in \$DATA to the filename in \$TABLE from the previous model (base model or previous chain model). Set FILE in \$TABLE to a unique name. Run the chain model (or 'samples' copies of the chain model, each with different \$DATA filename).
\item Set filename in \$DATA of the regression model to the filename in \$TABLE from the previous model in the sequence (either from the base model or the last chain model). Set FILE in \$TABLE to a unique name. Run the regression model (or 'samples' copies of the regression model, each with different \$DATA filename).
\item For each alternative model, if defined: Set filename in \$DATA of the alternative model to the filename in \$TABLE from the regression model. Set FILE in \$TABLE to a unique name. Run the alternative model (or 'samples' copies of the alternative model, each with different \$DATA filename).
\item Create 'imputations' copies of the multiple imputation model (or 'imputations' times 'samples' copies if -sim\_model was used). Set filename in \$DATA of the multiple imputation model to the filename in \$TABLE from the regression model. Each filename from regression \$TABLE is set in 'imputations' copies of the multiple imputation model. Set a unique seed in the \$SIM of each multiple imputation model. Run the multiple imputation models.
\end{enumerate}

\references

\end{document}
