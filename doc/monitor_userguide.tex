\documentclass[a4wide,12pt]{article}
%\setlength{\marginparwidth}{0pt}%35
%\setlength{\marginparsep}{0pt}%?
%\setlength{\evensidemargin}{0pt}
%\setlength{\oddsidemargin}{0pt}
\usepackage{lmodern}
\usepackage[utf8]{inputenc}
\usepackage[T1]{fontenc}
\usepackage{textcomp}
\usepackage{verbatim}
\usepackage{enumitem}
\usepackage{longtable}
\usepackage{alltt}
\usepackage{ifthen}
% Reduce the size of the underscore
\usepackage{relsize}
\renewcommand{\_}{\textscale{.7}{\textunderscore}}

\newcommand{\guidetitle}[1]{
\title{#1\\ \vspace{2 mm} {\large PsN 4.1.1}}
\date{2014-02-10}
}

\newcommand{\doctitle}[1]{
\title{#1}
\date{2014-02-10}
}


\newenvironment{optionlist}{
\renewcommand{\arraystretch}{1.1}
\setlength{\leftmargini}{2.5cm}
\begin{description}
%\setlength{\itemsep}{0ex}
}
{\end{description}}

\newcommand{\optname}[1]{\item{{\bfseries\texttt-#1}\newline}}
\newcommand{\optdefault}[2]{\item{{\bfseries\texttt-#1}{\mbox{ = \it #2}}\newline}}

\newcommand{\nextopt}{}

\guidetitle{Monitor user guide}{2020-09-15}

\begin{document}

\maketitle
\newcommand{\guidetoolname}{monitor}
\tableofcontents
\newpage

\section{Introduction and example usage}
The monitor script is used for monitoring an active scm or scmplus run.
Below are three snapshots from continuous monitoring of an scm run.
\begin{verbatim}
monitor phenobarbital.1 -interval=1
\end{verbatim}
{\tiny
\begin{verbatim}
Model              Forward_3      ofvdrop exceed_eval  N_test    N_ok N_localmin N_failed StepSelected
CLCV1-5                   ok         4.68           0       2       2          0        0       -
VCVD1-2              started         1.51                   2       2          0        0       -
VCV2-5                    ok         0.96           0       2       2          0        0       -
CLCV2-5                   ok         0.08           0       2       2          0        0       -
CLAPGR-2             started         0.02                   2       2          0        0       -
CLWGT-5                                 -                   2       2          0        0       2
VCVD2-2           notStarted            -                   2       2          0        0       -
VWGT-5                                  -                   1       1          0        0       1

Model              Forward_3      ofvdrop exceed_eval  N_test    N_ok N_localmin N_failed StepSelected
CLAPGR-2             started         9.99                   2       2          0        0       -
CLCV1-5                   ok         4.68           0       2       2          0        0       -
VCVD1-2                   ok         1.51           0       2       2          0        0       -
VCV2-5                    ok         0.96           0       2       2          0        0       -
CLCV2-5                   ok         0.08           0       2       2          0        0       -
VCVD2-2              started         0.01                   2       2          0        0       -
CLWGT-5                                 -                   2       2          0        0       2
VWGT-5                                  -                   1       1          0        0       1

Model             Backward_1      ofvdrop exceed_eval  N_test    N_ok N_localmin N_failed StepSelected BackstepRemoved
CLWGT-5              started      -134.49                   2       2          0        0       2               -
VWGT-5               started      -134.53                   1       1          0        0       1               -
CLAPGR-2                                -                   3       3          0        0       -               -
CLCV1-5                                 -                   3       3          0        0       -               -
CLCV2-5                                 -                   3       3          0        0       -               -
VCV2-5                                  -                   3       3          0        0       -               -
VCVD1-2                                 -                   3       3          0        0       -               -
VCVD2-2                                 -                   3       3          0        0       -               -

\end{verbatim}
}

\section{Output columns}
{\bfseries Model}\\
The name of the model in the scm log file, which is composed of the parameter, covariate and state number.

\subsection{Columns summarizing the current iteration}
These columns only appear in the output from monitor when there is a current iteration. 
When the last iteration has finished, these columns are not present.

{\bfseries Forward/Backward\_N} \\
The column header is a unique identifier of the current iteration, composed of the search direction and the step number N in that direction. The column contents is the state in the current iteration 
for each model, for example notStarted, started or ok. 
The column is empty for models that are not run in the current iteration, for example models
representing already selected parameter-covariate relations.

{\bfseries ofvdrop}\\
The drop in ofv for each model. For a started but not finished model the current ofv is read from 
the psn.ext in the NM\_run subdirectory, and the drop relative the base model for the current iteration
is computed. For continuous monitoring this value will be updated several
times before the model is finished, as NONMEM writes new lines to psn.ext. 
Often ofvdrop can be negative at the start of a run, i.e. the current ofv is 
worse than the ofv of the base model. 
The column contains a - for models that are not run in the current iteration.

{\bfseries exceed\_eval}\\
This column will indicate whether a finished model had a NONMEM message about maximum number
of evaluations exceeded in the termination message. 0 means no, 1 means yes, and empty means
model either not finished or not run.

\subsection{Columns summarizing all iterations except the current}
These columns summarize the contents of the scm log file. 
They are not present until at least one iteration has been finished and written by PsN
to the scm log file.
NONMEM files are not parsed.

{\bfseries N\_test}\\
The number of times, i.e. in how many iterations, 
the parameter-covariate relation has been tested.
%fixme backward counted?

{\bfseries N\_ok}\\
How many times, out of N\_test, does the scm log file list an ofv that such that extended model
has lower ofv than the base model?
%fixme

{\bfseries N\_localmin}\\
How many times, out of N\_test, did the run end in a local minimum, i.e. the log file lists a
higher ofv for the extended model than for the base model?

{\bfseries N\_failed}\\
How many times, out of N\_test, did the scm log file state that the model failed?

{\bfseries StepSelected} \\
In which forward step, if any, was the parameter-covariate relation selected?

{\bfseries StepStashed} \\
Only for scmplus: In which forward step, if any, was the parameter-covariate relation stashed, i.e. removed from the set of relations tested again?

{\bfseries StepReadded} \\
Only for scmplus: In which forward step, if any, was a previously stashed
parameter-covariate relation retested?

{\bfseries BackstepRemoved}\\ 
Both for scm and scmplus: 
In which backward step, if any, was a previously selected parameter-covariate relation removed?

\section{Input and options}

\subsection{Required input}
The name of an scm or scmplus run directory is a required argument.

\subsection{Optional input}

\begin{optionlist}
\optdefault{interval}{N}
Invoke continuous monitoring by specifying the number of seconds
between each summary of the run status.
Without option -interval, continuous monitoring is turned off.
\nextopt
\optname{help}
Print help text and exit.
\nextopt
\optname{version}
Print version information and exit.
\nextopt
\end{optionlist}

\end{document}


{\tiny
\begin{verbatim}
MODEL            TEST     BASE OFV     NEW OFV         TEST OFV (DROP)    GOAL     dDF    SIGNIFICANT PVAL
CLAPGR-2         PVAL    742.05105   721.58702             20.46402  >  18.30700   10        YES!  0.025157 
CLWGT-2          PVAL    742.05105    FAILED                 FAILED  >   3.84150    1                    999
VAPGR-2          PVAL    742.05105   726.38092             15.67013  >  18.30700   10              0.109470 
VWGT-2           PVAL    742.05105    FAILED                 FAILED  >   3.84150    1                    999

Parameter-covariate relation chosen in this forward step: CL-APGR-2
CRITERION              PVAL < 0.05
BASE_MODEL_OFV         742.05105
CHOSEN_MODEL_OFV       721.58702
Relations included after this step:
CL      APGR-2  
V       
--------------------
\end{verbatim}
}
