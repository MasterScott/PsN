\documentclass[a4wide,12pt]{article}
%\setlength{\marginparwidth}{0pt}%35
%\setlength{\marginparsep}{0pt}%?
%\setlength{\evensidemargin}{0pt}
%\setlength{\oddsidemargin}{0pt}
\usepackage{lmodern}
\usepackage[utf8]{inputenc}
\usepackage[T1]{fontenc}
\usepackage{textcomp}
\usepackage{verbatim}
\usepackage{enumitem}
\usepackage{longtable}
\usepackage{alltt}
\usepackage{ifthen}
% Reduce the size of the underscore
\usepackage{relsize}
\renewcommand{\_}{\textscale{.7}{\textunderscore}}

\newcommand{\guidetitle}[1]{
\title{#1\\ \vspace{2 mm} {\large PsN 4.1.1}}
\date{2014-02-10}
}

\newcommand{\doctitle}[1]{
\title{#1}
\date{2014-02-10}
}


\newenvironment{optionlist}{
\renewcommand{\arraystretch}{1.1}
\setlength{\leftmargini}{2.5cm}
\begin{description}
%\setlength{\itemsep}{0ex}
}
{\end{description}}

\newcommand{\optname}[1]{\item{{\bfseries\texttt-#1}\newline}}
\newcommand{\optdefault}[2]{\item{{\bfseries\texttt-#1}{\mbox{ = \it #2}}\newline}}

\newcommand{\nextopt}{}

\guidetitle{SIR user guide}{2015-08-31}
\usepackage{amsmath}

\begin{document}

\maketitle

\newcommand{\guidetoolname}{sir}

\section{Introduction}
The sir program calculates uncertainty
on model parameters for the input model using the Sampling Importance Resampling (SIR) procedure.
The SIR procedure is described in\\
\emph{Application of Sampling Importance Resampling to estimate parameter uncertainty distributions}, 
PAGE 22 (2013) Abstr 2907, Dosne AG, Bergstrand M, Karlsson MO.\\
and in\\
\emph{Determination of Appropriate Settings in the Assessment of Parameter Uncertainty Distributions 
using Sampling Importance Resampling (SIR)}, 
PAGE 24 (2015) Abstr 3546, Dosne AG, Bergstrand M, Karlsson MO.
 
Each iteration typically consists of the following three phases: First, parameter vectors will be simulated from the 
proposal density. The proposal density is either computed based on 
the (optionally Box-Cox transformed) resampled vectors from phase 3 of the previous iteration,
or, if there is no previous iteration, the truncated multivariate normal distribution 
given by the covariance matrix.
Second, each of the simulated parameter vectors will be evaluated on the original data (MAXEVAL=0).
Third, based on these evaluations, weights will be calculated for each of the parameter vectors and the vectors 
will be resampled according to these weights. 

Example command:
\begin{verbatim}
sir -samples=1000,2000 -resamples=500,1000 run10.mod
\end{verbatim}

\section{Obtaining a proposal density}
There are a number of different ways to obtain the proposal density for the first iteration.
\begin{enumerate}
\item The covariance matrix from a successful NONMEM covariance step. If the covariance step was successful, sir will automatically
read the covariance matrix from NONMEM, e.g. run10.cov,  and use it for the sampling in the first iteration.
\item If the covariance step was not successful, the precond script, see the precond\_userguide.pdf, 
is recommended. It can often recover the covariance step even if it failed when running the original model.
%\item If precond fails, then the user can compute a proposal covariance matrix $A$ based on the $R$ matrix given by NONMEM,
%e.g. run10.rmt, using $A=2R^{+}$ where $R^{+}$ is the pseudo-inverse of $R$. If precond has been run, the rmt-file is found in the 
%m1 subdirectory of the precond run directory. The $A$ matrix needs to be computed outside of PsN, using e.g. R, and
%then given as input to sir using option -covmat\_input.
\item If a bootstrap has been run, the file with simulated parameter vectors can be given as input, instead of 
a covariance matrix, with option -rawres\_input. 
Then the sir procedure starts with a 0th iteration where those vectors are treated as the resampled 
vectors from the third phase (Box-Cox transformation etc). This approach is useful when it is difficult to obtain a (faked) covariance matrix. 
If there are too few bootstrap vectors for full rank, sir will automatically fill in extra vectors using the
tweak\_inits functionality.
\item If none of the above alternatives are possible, the -auto\_rawres option can be used.
Then sir will create a complete rawres\_input file using the tweak\_inits functionality.
\end{enumerate}

\section{Input and options}
A model file is required on the command-line. 
If output file (lst) exists, then it is important that the control stream copy at the top of the lst-file matches the actual input model file
in terms of which parameters are present and which parameters are FIXED or SAME.
\begin{optionlist}
\optdefault{samples}{M1,M2}
Required. A comma-separated list of (usually two) integers, the number of parameter vectors to 
generate in each iteration. In each iteration the number needs to be greater than the number 
of resamples. A good starting point may be 5 times the number of resamples.
\nextopt
\optdefault{resamples}{m1,m2}
Required. A comma-separated list of (usually two) integers, the number of parameter vectors to resample 
in each iteration based on the weights
computed from delta ofv and the pdf. 
Only the resampled vectors will be used to compute the uncertainty (confidence intervals) around the model parameters. 
A good starting point may be 1000 resamples.
The list length must be equal to list 'samples'.
\nextopt
\optdefault{rawres\_input}{filename}
If rawres\_input is given, sir will run an intial 0th iteration in which sir will read all parameter
vectors from this file, starting on line offset\_rawres+1 and skipping any that does not fulfill the filter rules, if set.
These vectors will be Box-Cox transformed and used to create a proposal density for the next iteration.
This option is not allowed together with covmat\_input. 

If the number of vectors read is smaller than 
the number of estimated parameters (also counting off-diagonals),
it is not possible to obtain an empirical covariance matrix with full rank.
In this case sir will append copies of the existing vectors, perturbed using the tweak\_inits
functionality, until the resulting set of vectors has full rank.
If option -auto\_rawres=degree is set on the command-line, tweak\_inits will be done by 'degree',
otherwise the default degree 0.1 will be used.
If $N$ parameter vectors are read from the rawres file,
the generation of perturbed parameter vector $i$, element $j$, is performed according to 
init\_ij = init\_kj + rand\_uniform(-degree*init\_kj,+degree*init\_kj) where k=i modulo N and init\_kj is the value of 
parameter $j$ in original rawres vector $k$. 

The labels for  THETA/OMEGA/SIGMA 
in the file must match the labels in the model given as input 
to sir, the theta columns must be directly followed by the omega columns 
which must be directly followed by the sigma columns, and the first or
second column must have header model. Note that is is 
possible to generate a file with initial parameter estimates outside 
of PsN, as long as the file follows the format rules.
\nextopt
\optdefault{offset\_rawres}{N}
Only relevant in combination with -rawres\_input. Default 1. The number of result lines to skip in the input raw results file before starting to read final parameter estimates. In a regular bootstrap raw\_results file, and also in an initial\_estimates.csv file from an sse run, 
the first line of estimates refers to the input model with the full dataset, so therefore the default offset is 1. 
\nextopt
\optdefault{in\_filter}{condition on numerical col}
Only relevant in combination with -rawres\_input. Default not used. The parameter estimates lines in the file can be filtered on values in the different columns. When specifying which column(s) the filtering should be based on, the exact column name must be used, e.g. minimization\_successful. Filtering can only be based on columns with numeric values. The allowed relations are .gt. (greater than), .lt. (less than) and .eq. (equal to). If the value in the filter column is 'NA' then that parameter set will be skipped, regardless of the defined filter relation. Conditions are separated with commas. If the remaining number of lines after filtering is smaller than -samples, sse will stop with an error message. Then the user must either change the filtering rules or change -samples. If the user has created a file with parameter estimates outside of PsN, filtering can be done on any numeric column in that file. Do not set column headers containing .eq. or .lt. or .gt.in the user-generated file as this would interfere with the in\_filter option syntax.

Example (there must be no linebreaks in the actual command):
\begin{verbatim}
-in_filter=minimization_successful.eq.1,
significant_digits.gt.3.5
\end{verbatim} \\
\nextopt
\optdefault{auto\_rawres}{degree}
Default not set. Not allowed in combination with covmat\_input. If rawres\_input is not set,
this option will make sir use the tweak\_inits functionality to automatically create a 
rawres\_input file with perturbed parameter vectors, and these vectors will then be used
as if read from a real raw\_results file. 
The generation of perturbed parameter vector $i$, element $j$, is performed according to 
init\_ij = init\_0j + rand\_uniform(-degree*init\_0j,+degree*init\_0j)
where init\_0j is the final estimate of 
parameter $j$ in the input model. 
The tweaking procedure makes sure that boundary conditions, including positive definiteness of 
\$OMEGA and \$SIGMA, are respected.

If option -auto\_rawres is used
in combination with rawres\_input, tweak\_inits by 'degree' will be used 
to augment the given rawres\_input file until the matrix has full rank, see help text for -rawres\_input.
\nextopt
\optname{with\_replacement}
Default not set. By default, resampling is performed without replacement, but setting this option gives resampling with replacement.
It is not possible to cap replacement at e.g. 5, replacement will be unlimited.
\nextopt
\optdefault{covmat\_input}{filename}
Not allowed together with rawres\_input or auto\_rawres. If given, this covariance matrix is
used in iteration 1
for sampling parameter vectors and for computing the PDF.
The matrix will not be used in any but the first iteration. 
The format of the file is similar to a NONMEM-generated .cov-file except 
that the \verb|TABLE NO.| line should be omitted.
Fixed parameters do not have to be omitted, sir will filter them out. 
The covmat\_input file must be formatted to contain a space- or tab-separated  $N\times N$ symmetric covariance matrix.
The first line in the file must be a header with labels for THETA/OMEGA/SIGMA written as in a regular NONMEM .cov-file 
and a leading column called NAME: 
\begin{verbatim}
NAME THETA1 THETA2 ... SIGMA(1,1) ... OMEGA(1,1) ... 
\end{verbatim}
The NAME column contains the same parameter labels (to identify the rows).
The rows and columns must be sorted with THETAs first.

Please note that there is a PsN script called covmat that can be used to generate a covariance matrix from a raw\_results file.
\nextopt
\optdefault{inflation}{X}
Default is 1, which is the same as no inflation. If given, the covariance
matrix will in the first iteration be multiplied with a scalar $X$ multiple of the identity matrix
before the parameter vectors
are sampled from the truncated multivariate normal distribution.
The weights will also be computed based the inflated covariance matrix. 
Inflation is not used in any but the first iteration.
\nextopt
\optdefault{problems\_per\_file}{N}
Optional, default 100. The number of \$PROBLEM per model file when evaluating the parameter vectors
on the original data (running
MAXEVAL=0 or similar to get ofv:s for parameter vectors). Setting a higher value
decreases the overhead involved in running each control stream, but increases the 
risk of losing many samples in case a model file crashes. Setting -problems\_per\_file=1
gives maximum robustness to individual crashes, but also maximum overhead cost.
\nextopt
\optdefault{mceta}{N}
Optional. Sets option MCETA=N in \$ESTIMATION. Only allowed with NM7.3 and classical estimation methods.
\nextopt
\optname{copy\_data}
Default set. If option is set, the original data file
will be copied to the run directory.
If option is unset using -no-copy\_data, the absolute path to the original data file will be used in
\$DATA, and the data file will not be copied. This saves disk space.
\nextopt
\optname{recenter}
Default set. If option is set and any sampled parameter vector has a smaller ofv than the original
vector $\mu$ of final parameter estimates of the input model, then vector $\mu$ will be replaced with
the sampled parameter vector with the smallest ofv.
If option is unset using -no-recenter, the original parameter vector will be kept
and a warning will be printed that some vectors had a negative delta-ofv.
\nextopt
\optname{boxcox}
Default set. 
If option is set, sir will Box-Cox transform parameter vectors before computing
the covariance matrix used for sampling.
If option is unset using -no-boxcox, no transformation will be performed before computing covariance matrix
used for sampling.
\nextopt
\end{optionlist}


\section{sir program workflow}
\subsubsection*{Setup} 
The sir program will run the input model unless the lst-file with results is already present. If the lst-file is present, it
is important that the control stream copy at the top of the lst-file matches the
actual input model file in terms of which parameters are estimated, 
FIXED or SAME, otherwise there will be a mis-match between which parameter estimates are read from the
lst-file and which estimates are needed for the sir procedure.

Unless option -rawres\_input and/or -auto\_rawres is used, read the covariance matrix to be used
in iteration 1.
The covariance matrix file name is either explicitly given with option -covmat\_input
or by default the .cov-file of the input model run.

\subsubsection*{For each iteration $i$ (each item in list 'samples'):}
(NB: If -rawres\_input and/or -auto\_rawres is used there first an iteration 0 that starts at \underline{Step 5})
\begin{itemize}
\item[\underline{Step 1}] 
Simulate 'samples item $i$' parameter vectors from the 
covariance matrix, using Perl function Math::Random::random\_multivariate\_normal.

\noindent If there was a previous iteration, i.e $i>1$ or option -rawres\_input or -auto\_rawres was used, 
the covariance matrix is the un-inflated
empirical covariance matrix of (optionally Box-Cox transformed) resampled parameter vectors from
Step 6 in the previous iteration. If the vectors were Box-Cox transformed then
sampling is done on Box-Cox scale, and 
then the samples are back-transformed for boundary checks and evaluation.

\noindent If there was no previous iteration, the covariance matrix is either from
the .cov-file given by NONMEM or given via option -covmat\_input.
The matrix is inflated if option -inflation is set, and the sampling is done on the original scale. 

\noindent If a vector (on the original scale) does not fulfill the constraints from \$THETA boundaries
and positive definiteness of \$OMEGA and
\$SIGMA blocks (as judged by a PsN-implemented Cholesky decomposition) 
then that vector is 
discarded and a new one is drawn.
\item[\underline{Step 2}] 
Calculate each vector $x$’s probability 
given the covariance matrix based on the formula for the probability 
density function (PDF) of a multivariate normal distribution:\\
\begin{math}
\frac{1}{\left(2\pi\right)^{k/2}\left(det\left(A\right)\right)^{1/2}} exp\left(-\frac{1}{2}(\left(x-\mu\right)A^{-1} \left(x-\mu\right)^T\right)
\end{math}
\\
where $k$ is the number of dimensions, 
$\mu$ is the vector of expectations and $A$ is the (optionally inflated) covariance matrix.
The values are normalized with the PDF for the vector of expectations $\mu$, giving\\
\begin{math}
relPDF=exp\left(-\frac{1}{2}(\left(x-\mu\right)A^{-1} \left(x-\mu\right)^T\right)
\end{math}
\\
NB 1: When the NONMEM covariance matrix is used as input, PsN will compute $relPDF$ without reading the inverse covariance matrix
output by NONMEM.
NB 2: If $i>0$ or -rawres\_input is used, the vectors (including $\mu$) and covariance matrix are on Box-Cox scale if option -boxcox was set (the default).
\item[\underline{Step 3}] 
Evaluate the (original scale) parameter vectors on the original data.
Create model files with up to 100 \$PROBLEM based on the original model file but setting MAXEVAL=0
and replacing inits of nth \$PROBLEM with nth parameter vector on original scale, and also set MCETA if option -mceta was used. Compute dOFV 
(delta-OFV, reference is input model ofv) and store in raw results file for this iteration.
\item[\underline{Step 4}] 
Calculate the weights as the ratio $\frac{e^{-0.5\cdot dOFV}}{relPDF}$ and store in raw results file for this iteration. 
Resample 'resamples item $i$' parameter vectors based on these weights, with or without replacement depending on option 
-with\_replacement. 
Store the number of times each vector was resampled in raw results file for this iteration.

\noindent If any sample(s) had a negative dOFV and option -recenter is set (the default), reset the $mu$-vector
to be used in the next iteration to the parameter vector with the lowest ofv.
\item[\underline{Step 5}]
If this is iteration $0$, i.e. rawres\_input is used and the procedure starts here: Read all parameter vectors
from file rawres\_input, excluding offset\_rawres and vectors not passing in\_filter (if set).

\noindent Otherwise read the 'resamples item $i$' resampled vectors from Step 4, using multiple copies for vectors that were 
resampled more than once (only possible if -with\_replacement was set). 
\item[\underline{Step 6}]
If \emph{not} last iteration \emph{and} option -boxcox is set (the default): 
For each parameter individually, find the Box-Cox transformation that maximizes the correlation with the normal density.
Do not allow $\lambda$ less than -3 or larger than +3. The  $\lambda$ used is allowed to differ from the true optimal  $\lambda$
by at most $0.2$.
Compute the covariance matrix for the Box-Cox transformed resampled parameter vectors, and go to Step 1.

\noindent If \emph{not} last iteration \emph{and} option -boxcox is unset with -no-boxcox: 
Compute the covariance matrix for the resampled parameter vectors on original scale, and go to Step 1.

\noindent If last iteration: 
Compute the covariance matrix for the resampled parameter vectors on original scale, 
store in sir\_results.csv. 
Compute univariate confidence intervals and store in sir\_results.csv.
\end{itemize}

\section{Adjusting samples and resamples}
Sometimes the evaluation run fails for some samples, resulting in fewer successful samples (samples with defined ofv)
than the number of attempted samples. PsN tries to compensate for this loss, and then
adjust again if the success rate goes up. If PsN has to compensate for loss, option -problems\_per\_file will 
automatically be reset to 1.
\subsection{Adjusting samples}
In iteration 1 the number of attempted samples is always the number of
samples requested on the commandline with option -samples.
When the number of successful samples in the previous iteration was less than
$95\%$ of the attempted samples, the number of attempted samples in the current 
iteration is adjusted according to\\
$current.attempted.samples=\frac{current.requested.samples \cdot previous.attempted.samples}{previous.successful.samples}$\\
The number of attempted samples can be greater, but never smaller, than requested samples.

\subsection{Adjusting resamples}
When the number of successful samples for the current iteration 
differs more than $5\%$ from the original samples count requested on the commandline with option -samples,
PsN will try to maintain the requested sample-to-resample ratio by 
adjusting the actual number of resamples for the current iteration 
according to \\
$actual.resamples=\frac{requested.resamples\cdot successful.samples}{requested.samples}$\\
Note that the number of actual resamples can be both greater and smaller than the 
requested samples, because the attempted samples adjustment can make 
the number of successful samples both greater and smaller
than requested samples.
\subsection{Auto-generated R-plots from PsN}
\newcommand{\rplotsconditions}{The default sir template 
requires 
that R libraries
ggplot2, plyr, dplyr, caTools, reshape and tidyr are installed.
If the conditions are not fulfilled then no pdf will be generated,
see the .Rout file in the main run directory for error messages.}
PsN can automatically generate R plots to visualize results for \guidetoolname, using a default template found in the R-scripts subdirectory of the installation directory. The user can also create a custom template, see more details in the section Auto-generated R-plots from PsN in common\_options.pdf.

\rplotsconditions

\begin{optionlist}
\optdefault{rplots}{level}
-rplots<0 means R script is not generated\\ 
-rplots=0 (default) means R script is generated but not run\\ 
-rplots=1 means basic plots are generated\\													  
-rplots=2 means basic and extended plots are generated\\													  
\nextopt
\end{optionlist}

\subsubsection*{Troubleshooting}
If no .pdf was generated even if a template file is available and the appropriate options were set, check the .Rout-file in the main run directory for error messages. If no .Rout-file exists, then check that R is properly installed, and that either command 'R' is available or that R is configured in psn.conf.


\subsubsection*{Diagnostic sir plots}
Diagnostic rplots for the last iteration of sir will be generated if option -rplots is set >0.

The first plot shows 3 comparative dOFV distributions: SIR, reference chi-square and original covariance matrix.
If on this plot the SIR curve is close to the reference chi-square curve it means that SIR results are good.

The second plot shows the number of resampled values in each bin of the parameter space. 
Each bin contains the same number of parameter values, which is set to 1/10 of the number of initial samples.
That is, Bin 1 corresponds to the lowest 10\% of simulated parameter values and Bin 10 to the highest 10\%.
If on this plot the number of resamples is around the dashed line, it means that the original covariance-matrix 
is a good approximation of the uncertainty and that SIR results are good.
If some bins have much more resamples than others, it means that the covariance matrix is different from the true uncertainty. 
SIR results remain an improvement over the original covariance-matrix but they could potentially be improved by modifying the 
original covariance matrix.
For example, one could inflate the variances of parameters that have more resamples than expected in the "outside" bins 
(diagonal or u-shaped lines).

The third plot shows the 95\% confidence interval for each parameter as determined by SIR and the original covariance matrix.
If on this plot the 95\% confidence intervals for one parameter are identical, it means that the covariance matrix 
is a good approximation of the uncertainty for this parameter.
If the 95\% confidence intervals differ between SIR and the covariance matrix, it means that the covariance matrix is 
different from the true uncertainty for this parameter.

\section{Output}
\begin{itemize}
\item raw\_results for each iteration with added columns for\\
\begin{tabular}{ll}
\bf{sample.id} & A unique number\\
\bf{deltaofv} & $\mathrm{ofv} - \mathrm{original}\mbox{\tt\string_}\mathrm{model}\mbox{\tt\string_}\mathrm{ofv}$\\
\bf{likelihood\_ratio} & $e^{-0.5\cdot \mathrm{deltaofv}}$ \\
\bf{relPDF} & relative PDF \\
\bf{importance\_ratio} & $\mathrm{likelihood}\mbox{\tt\string_}\mathrm{ratio}/\mathrm{relPDF}$\\
\bf{probability\_resample} & $\mathrm{importance}\mbox{\tt\string_}\mathrm{ratio}/\sum{\mathrm{importance}\mbox{\tt\string_}\mathrm{ratio}}$ \\
\bf{resamples} & number of actual resamples with random seed used\\
\end{tabular}
NB: If -rawres\_input is used there is no raw\_results file for the first iteration.
\item sir\_results.csv with percentiles and empirical sd-correlation matrix
\item <modelname>\_sir.cov with the empirical covariance matrix in NONMEM-like format, i.e. 
      fixed width space separated with generic headers in order THETA, SIGMA, OMEGA.
\item if option -rplots>0: A file PsN\_sir\_plots.pdf with diagnostic sir plots.
\item For all but the last iteration: A file with the Box-Cox $\lambda$ for each parameter, the $\delta$ used to shift
the values to positive before Box-Cox transformation, the original estimate and the transformed estimate. 
If the $\lambda$ column is empty it means that no transformation was
needed because the parameter was already close to normally distributed.
\item For all but the last iteration: A file boxcox\_covmatrix\_iterationi.cov with 
the Box-Cox space covariance matrix for iteration $i$,
or, if -no-boxcox was set on commandline, a file untransformed\_covmatrix\_iterationi.cov.
\item A file summary\_iterations.csv with, for each iteration, 
the requested samples, attempted samples, successful samples, requested resamples,
actual resamples, requested samples-resamples ratio, actual samples-resamples ratio, count negative delta-ofv and minimum iteration sample ofv.
\end{itemize}


\section{Troubleshooting}
\subsection*{Mismatch between input files}
If you use the covmat\_input option and get an error message similar to
\begin{verbatim}
Number of parameters 122 in covmat_input does not match 
number 123 of estimated parameters in input model
\end{verbatim}
then the most likely cause is that there is a mismatch between the input files. Make sure that the number of
estimated parameters (the number of parameters, the FIXED and SAME settings, the DIAGONAL or BLOCK omega/sigma)
match between the control stream copy at the beginning of the pre-existing lst-file and the input model file.
\subsection*{raw\_results file with \$PRIOR}
If the input model has \$PRIOR NWPRI, and the priors are encoded with \$THETA, \$OMEGA and \$SIGMA instead of the
prior-specific records \$THETAP, \$THETAPV, \$OMEGAP etc, then PsN will not be able to handle the parameter column
headers correctly in the raw\_results file. The solution is to always use the prior-specific records for
encoding the prior information.
\subsection*{Error in iteration 1 that covariance matrix not positive definite}
The sir program will use a Cholesky decomposition without pivoting for processing of the covariance matrix. 
This works in most cases, but in some rare case this algorithm can fail for a matrix that is mathematically positive definite
but requires pivoting for Cholesky decomposition to work.
When this happens, sir will stop with an error message saying that the covariance matrix is numerically not positive definite.
If this happens in iteration 1 when option -covmat\_input was used, or the default NONMEM covariance matrix, 
then the user must manually modify the covariance matrix to improve its numerical properties (for example increase the diagonal elements
and/or change the order of the parameters so that pivoting is not necessary),
and then use option -covmat\_input to give the modified matrix as input to sir. 
If this happens in iteration 1 after a 0th iteration based on -auto\_rawres, or -rawres\_input with augmenting using -auto\_rawres, 
then it is recommended to either run a new sir with -no-boxcox or a different random seed, or 
try to run sir with -covmat\_input=tweak\_inits.cov where tweak\_inits.cov is found in the failed sir run directory and
is the empirical covariance matrix from iteration 0 without Box-Cox transformation.
\end{document}

\section{Suggested new workflow}
NB: if -rawres\_input is specified, then the process starts
at step 6 using those vectors as the 'resamples\_1' parameter vectors. 
\begin{itemize}
\item[\underline{Setup}] The sir program will run the input model unless the lst-file with results is already present. 
If the lst-file is present, it
is important that the control stream copy at the top of the lst-file matches the
actual input model file in terms of which parameters are estimated, 
FIXED or SAME, otherwise there will be a mis-match between which parameter estimates are read from the
lst-file and which estimates are needed for the sir procedure.
\item[\underline{Step 1}] Simulate 'samples\_1' parameter vectors from the 
(possibly inflated)
covariance matrix, i.e. the .cov-file given by NONMEM 
or the matrix given via option -covmat\_input. PsN uses the Perl function\\ 
Math::Random::random\_multivariate\_normal
for sampling. If a vector does no fulfill the constraints from \$THETA boundaries
and positive definiteness of \$OMEGA and
\$SIGMA blocks (as judged by a PsN-implemented Cholesky decomposition) 
then that vector is 
discarded and a new one is drawn.
\item[\underline{Step 2}] Calculate each vector $x$’s probability 
given the covariance matrix based on the formula for the probability 
density function (PDF) of a multivariate normal distribution:\\
\begin{math}
\frac{1}{\left(2\pi\right)^{k/2}\left(det\left(A\right)\right)^{1/2}} exp\left(-\frac{1}{2}(\left(x-\mu\right)A^{-1} \left(x-\mu\right)^T\right)
\end{math}
\\
where $k$ is the number of dimensions, 
$\mu$ is the vector of expectations and $A$ is the (possibly inflated) covariance matrix.
The values are normalized with the PDF for the vector of expectations $\mu$, giving\\
\begin{math}
relPDF=exp\left(-\frac{1}{2}(\left(x-\mu\right)A^{-1} \left(x-\mu\right)^T\right)
\end{math}
\\
NB: NONMEM covariance matrix is used as input, the inverse covariance matrix
output by NONMEM is not used.
\item[\underline{Step 3}] evaluate the parameter vectors on the original data.
Create model files with up to 100 \$PROBLEM based on the original model file but setting MAXEVAL=0
and replacing inits of nth \$PROBLEM with nth parameter vector, and also set MCETA if option -mceta was used. Compute dOFV 
(delta-OFV, reference is input model ofv) and store in raw\_results\_1.csv.

\item[\underline{Step 4}] calculate the weights as the ratio $\frac{e^{-0.5\cdot dOFV}}{relPDF}$ and store in 
raw\_results\_1.csv. 
\item[\underline{Step 5}] Resample 'resamples\_1' parameter vectors based on weights from above step, 
with or without replacement depending on option -with\_replacement. 
Store the number of times each vector was resampled in raw\_results\_1.csv.
\item[\underline{Step 6}] Box-Cox transform each parameter separately based on the samples from Step 5. Do not allow
$\lambda$ less than -3 or larger than +3. Use the $\lambda$ that maximizes the correlation between the normal distribution and
the distribution of the transformed parameter. The  $\lambda$ used is allowed to differ from the true optimal  $\lambda$
by at most $0.2$.
\item[\underline{Step 7}] Determine the means and empirical variance-covariance matrix of 
transformed parameter vectors.
\item[\underline{Step 8}] (Repeat Step 1) Sample 'samples\_2' parameter vectors 
with the new covariance matrix and transformed parameters.
Discard samples that, after back-transformation to original scale, do not
fulfill \$THETA boundary conditions and \$OMEGA/\$SIGMA positive definiteness, and instead draw new sample.
\item[\underline{Step 9}] (Repeat Step 2) Calculate each vector's probability, 
use new covariance matrix and Box-Cox transformed parameter vectors for computing each vector's relPDF.
\item[\underline{Step 10}] (Repeat Step 3) Evaluate the sampled parameter vectors, in original scale (back-transformed from Box-Cox), on the original data.
\item[\underline{Step 11}] (Repeat Step 4) Calculate the weights, 
where weights calculation uses relPDF from step 9 and dOFV from Step 10. 
Store in raw\_results\_2.csv. 
\item[\underline{Step 12}] (Repeat Step 5) Resample 'resamples\_2' original scale parameter vectors based on weights from above step, 
with or without replacement depending on option -with\_replacement. 
Store the number of times each vector was resampled in raw\_results\_2.csv.
\item[\underline{Step 13}] Compute final results from Step 12 (parameter vectors on original scale).
\end{itemize}

\end{document}

\subsubsection*{setup}
\begin{enumerate}
\item Count items in parameter vector to get $N$.
\item Unless have quantile vector for $N$ stored from before:\\ Compute vector $Q$ for $i=1\ldots N$,
\begin{math}
\left(
Q_i=\Phi^{-1}\left(\frac{i-0.5}{N}\right)
\right)
\end{math}
using built-in Perl functions. Also compute $\frac{\sum_i{Q}}{N}$ and 
$\sqrt{\sum_i{Q^2}-\frac{\left(\sum_i{Q}\right)^2}{N}}$\\ Store results.
\item Sort parameter vector from smallest to largest to get vector $sorted$.
\item If $sorted[0]\leq 0$ then compute $\Delta=abs\left(sorted[0]+\delta\right)$ and add scalar $\Delta$ to $sorted$.
Store $\Delta$.
\item Evaluate $r(\lambda)$ for starting values for secant algorithm
\item Run secant algorithm until stops, find optimal value of $\lambda$ and corresponding transformed sorted positive
vector. Store $\lambda$
\end{enumerate}

\subsubsection*{Box-Cox}
Must have $x>0$, so if any $p$ leq 0 then add $abs(p_{min})+\delta$ to all $p$ before Box-Cox.
\noindent
\begin{math}
\lambda = 0: x_{\lambda}=log(p)\\
\lambda\neq 0: x_{\lambda}=\frac{p^{\lambda}-1}{\lambda}
\end{math}

\begin{math}
\frac{dx}{d\lambda}, \lambda=0: undef\\
\frac{dx}{d\lambda}, \lambda\neq 0: \frac{d}{d\lambda}\frac{p^{\lambda}-1}{\lambda}=\frac{d}{d\lambda}\frac{e^{\lambda log{p}}-1}{\lambda}=
\frac{\lambda e^{\lambda log{p}}log{p} -(e^{\lambda log{p}}-1) }{\lambda^2}=\frac{1}{\lambda}\left(p^{\lambda}log{p} - \frac{p^{\lambda}-1}{\lambda}\right)
\end{math}

\subsubsection*{q-q-plot}
Form N pairs
\begin{math}
\left(
\Phi^{-1}\left(\frac{i-0.5}{N}\right), x_i
\right)
\end{math}
where $\Phi^{-1}$ is the inverse CDF of the normal density and $x_i$ denotes the $i$th sorted value
of the Box-Cox transformed data.
For inverse CDF use Perl Math::CDF::qnorm function.
Compare with Statistics::Distributions::udistr(1-z) where $z=\frac{i-0.5}{N}$

\subsubsection*{Pearson's r}
x and y do not need to have mean 0. x could Box-Cox transformed (g), y could be inverse CDF.
\begin{math}
r=\frac{\sum{xy}-\frac{\sum{x}\sum{y}}{N}}
{\sqrt{\sum{x^2}-\frac{\left(\sum{x}\right)^2}{N}}\sqrt{\sum{y^2}-\frac{\left(\sum{y}\right)^2}{N}}}
\end{math}
When y is inverse CDF then $\sum_iy_i$ is 0, giving 
\begin{math}
r=\frac{\sum{xy}}
{\sqrt{\sum{x^2}-\frac{\left(\sum{x}\right)^2}{N}}\sqrt{\sum{y^2}}}\\
=\frac{f(\lambda)}{g(\lambda)}
\end{math}

\begin{math}
f(\lambda)=\sum{xy}=\sum_i \left(y_i\frac{p_i^{\lambda}-1}{\lambda}\right)
\end{math}

\begin{math}
\frac{df(\lambda)}{d\lambda}
=\sum_i\left(y_i\frac{1}{\lambda}\left(p_i^{\lambda}log{p_i}-\frac{p_i^{\lambda}-1}{\lambda}\right)\right)
\end{math}


\begin{math}
g(\lambda)=
\left(\sum_i\left(\frac{p_i^{\lambda}-1}{\lambda}\right)^2-\frac{1}{N}\left(\sum_i \frac{p_i^{\lambda}-1}{\lambda}\right)^2  \right)^{0.5} 
\sqrt{\sum{y^2}}
\end{math}

\begin{math}
\frac{dg(\lambda)}{d\lambda}=
\sqrt{\sum{y^2}}\cdot
0.5\cdot\left(\sum_i\left(\frac{p_i^{\lambda}-1}{\lambda}\right)^2-\frac{1}{N}\left(\sum_i \frac{p_i^{\lambda}-1}{\lambda}\right)^2  \right)^{-0.5} 
\cdot\\
\left(
\sum_i2\left(\frac{p_i^{\lambda}-1}{\lambda} \right)\frac{1}{\lambda}\left(p_i^{\lambda}log{p_i}-\frac{p_i^{\lambda}-1}{\lambda}\right)
-\frac{2}{N}\left(\sum_i \frac{p_i^{\lambda}-1}{\lambda}\right) \left(\sum_i \frac{1}{\lambda}\left(p_i^{\lambda}log{p_i}-\frac{p_i^{\lambda}-1}{\lambda}\right) \right)
\right)
\end{math}

\begin{math}
=
\sqrt{\sum{y^2}}\cdot
\left(\sum_i\left(\frac{p_i^{\lambda}-1}{\lambda}\right)^2-\frac{1}{N}\left(\sum_i \frac{p_i^{\lambda}-1}{\lambda}\right)^2  \right)^{-0.5} 
\cdot\\
\left(
\sum_i\left(\frac{p_i^{\lambda}-1}{\lambda} \right)\frac{1}{\lambda}\left(p_i^{\lambda}log{p_i}-\frac{p_i^{\lambda}-1}{\lambda}\right)
-\frac{1}{N}\left(\sum_i \frac{p_i^{\lambda}-1}{\lambda}\right) \left(\sum_i \frac{1}{\lambda}\left(p_i^{\lambda}log{p_i}-\frac{p_i^{\lambda}-1}{\lambda}\right) \right)
\right)
\end{math}

\begin{math}
\frac{dr}{d\lambda}=\frac{f'g-fg'}{g^2}=g^{-2}\cdot\left(f'g-fg'\right)
\end{math}

\begin{multline}
\frac{dr}{d\lambda}=
\left(\sum_i\left(\frac{p_i^{\lambda}-1}{\lambda}\right)^2-\frac{1}{N}\left(\sum_i \frac{p_i^{\lambda}-1}{\lambda}\right)^2  \right)^{-1} 
\left(\sum{y^2}\right)^{-1}\cdot  \\
[
\left[
\sum_i\left(\frac{1}{\lambda}\left(p_i^{\lambda}log{p_i}-\frac{p_i^{\lambda}-1}{\lambda}\right)y_i\right)
\right]  \\
\left(\sum_i\left(\frac{p_i^{\lambda}-1}{\lambda}\right)^2-\frac{1}{N}\left(\sum_i \frac{p_i^{\lambda}-1}{\lambda}\right)^2  \right)^{0.5} 
\sqrt{\sum{y^2}}  \\
-
\left[\sum_i \left(y_i\frac{p_i^{\lambda}-1}{\lambda}\right)\right]\cdot\\
\sqrt{\sum{y^2}}\cdot
\left(\sum_i\left(\frac{p_i^{\lambda}-1}{\lambda}\right)^2-\frac{1}{N}\left(\sum_i \frac{p_i^{\lambda}-1}{\lambda}\right)^2  \right)^{-0.5} 
\cdot\\
\left[
\sum_i\left(\frac{p_i^{\lambda}-1}{\lambda} \right)\frac{1}{\lambda}\left(p_i^{\lambda}log{p_i}-\frac{p_i^{\lambda}-1}{\lambda}\right)
-\frac{1}{N}\left(\sum_i \frac{p_i^{\lambda}-1}{\lambda}\right) \left(\sum_i \frac{1}{\lambda}\left(p_i^{\lambda}log{p_i}-\frac{p_i^{\lambda}-1}{\lambda}\right) \right)
\right]
]
\end{multline}


\begin{multline}
\frac{dr}{d\lambda}=
\left(\sum_i\left(\frac{p_i^{\lambda}-1}{\lambda}\right)^2-\frac{1}{N}\left(\sum_i \frac{p_i^{\lambda}-1}{\lambda}\right)^2  \right)^{-0.5} 
\left(\sum_i{y_i^2}\right)^{-0.5}\cdot  \\
[
\left[
\sum_i\left(\frac{1}{\lambda}\left(p_i^{\lambda}log{p_i}-\frac{p_i^{\lambda}-1}{\lambda}\right)y_i\right)
\right]  \\
-
\left[\sum_i \left(y_i\frac{p_i^{\lambda}-1}{\lambda}\right)\right]\cdot\\
\left(\sum_i\left(\frac{p_i^{\lambda}-1}{\lambda}\right)^2-\frac{1}{N}\left(\sum_i \frac{p_i^{\lambda}-1}{\lambda}\right)^2  \right)^{-1} 
\cdot\\
\left[
\sum_i\left(\frac{p_i^{\lambda}-1}{\lambda} \right)\frac{1}{\lambda}\left(p_i^{\lambda}log{p_i}-\frac{p_i^{\lambda}-1}{\lambda}\right)
-\frac{1}{N}\left(\sum_i \frac{p_i^{\lambda}-1}{\lambda}\right) \left(\sum_i \frac{1}{\lambda}\left(p_i^{\lambda}log{p_i}-\frac{p_i^{\lambda}-1}{\lambda}\right) \right)
\right]
]
\end{multline}

\begin{math}
r=\left(\sum_i{x_i^2}-\frac{\left(\sum_i{x_i}\right)^2}{N}\right)^{-0.5} 
\left(\sum_i{y_i^2}\right)^{-0.5}\cdot
\left(\sum_i{y_ix_i}\right)
\end{math}

\begin{multline}
\frac{dr}{d\lambda}=
\left(\sum_i{x_i^2}-\frac{\left(\sum_i{x_i}\right)^2}{N}\right)^{-0.5} 
\left(\sum_i{y_i^2}\right)^{-0.5}\cdot\frac{1}{\lambda}\cdot  \\
[
\sum_i\left(y_ip_i^{\lambda}log{p_i}\right) -  
\sum_i\left(x_iy_i\right)
-\\
\frac{\left(\sum_i{y_ix_i}\right) \cdot
\left[
\sum_i \left(x_ip_i^{\lambda}log{p_i}\right) - \sum_i \left(x_i^2\right)
-\left( \sum_i \left(p_i^{\lambda}log{p_i}\right) -\sum_i \left(x_i\right) \right)\frac{\left(\sum_i{x_i}\right)}{N} 
\right]}
{\left(\sum_i{x_i^2}-\frac{\left(\sum_i{x_i}\right)^2}{N}\right)}
]
\end{multline}

\subsubsection*{Intermediate sums needed}
\begin{enumerate}
\item sum\_x: $\sum_i{x_i}$
\item sum\_xsquare: $\sum_i{x_i^2}$
%\item sum\_y: $\sum_i{y_i}$
\item sum\_ysquare: $\sum_i{y_i^2}$
\item sum\_xy: $\sum_i{y_ix_i}$
\item sum\_logterm: $\sum_i{p_i^{\lambda}log{p_i}}$
\item sum\_xlogterm: $\sum_i{x_ip_i^{\lambda}log{p_i}}$
\item sum\_ylogterm: $\sum_i{y_ip_i^{\lambda}log{p_i}}$

\end{enumerate}

\subsubsection*{secant method}
Algorithm secant method (p 366 mathematics handbook):\\
\begin{math}
\lambda_{n+1}=\lambda_n-
r^{'}\left(\lambda_{n}\right)\cdot\frac{\lambda_n-\lambda_{n-1}}{r^{'}\left(\lambda_n\right)-r^{'}\left(\lambda_{n-1}\right)}
\end{math}\\
Stop when either
\begin{enumerate}
\item $\lambda_{n+1}\geq\lambda_{max}$
\item $\lambda_{n+1}\leq-\lambda_{max}$
\item \begin{math}abs\left(r\left(\lambda_n\right)-r\left(\lambda_{n-1}\right)\right)\leq \delta_{r}
\end{math}
\item
\begin{math}
abs\left(r\left(\lambda_{n}\right)\cdot\frac{\lambda_n-\lambda_{n-1}}{r\left(\lambda_n\right)-r\left(\lambda_{n-1}\right)}\right)\leq\delta_{\lambda}
\end{math}
\end{enumerate}
After stopping criteria met, choose $\lambda$ that gave largest $r$.

\end{document}

\subsubsection*{General y}
\begin{math}
r=\frac{\sum{xy}-\frac{\sum{x}\sum{y}}{N}}
{\sqrt{\sum{x^2}-\frac{\left(\sum{x}\right)^2}{N}}\sqrt{\sum{y^2}-\frac{\left(\sum{y}\right)^2}{N}}}=\frac{f}{g}
\end{math}


\begin{math}
f(\lambda)=\sum{xy}-\frac{\sum{x}\sum{y}}{N}
=
\sum_i \left(y_i\frac{p_i^{\lambda}-1}{\lambda}\right) - \left(\sum_i\frac{p_i^{\lambda}-1}{\lambda}\right)\frac{\sum_iy_i}{N}
\end{math}

\begin{math}
\frac{df(\lambda)}{d\lambda}
=\sum_i\left(y_i\frac{1}{\lambda}\left(p_i^{\lambda}log{p_i}-\frac{p_i^{\lambda}-1}{\lambda}\right)\right)
-\left( \sum_i\frac{1}{\lambda}\left(p_i^{\lambda}log{p_i}-\frac{p_i^{\lambda}-1}{\lambda}\right)\right)
\frac{\sum_iy_i}{N} \\
\end{math}


\begin{math}
g(\lambda)=
\left(\sum_i\left(\frac{p_i^{\lambda}-1}{\lambda}\right)^2-\frac{1}{N}\left(\sum_i \frac{p_i^{\lambda}-1}{\lambda}\right)^2  \right)^{0.5} 
\sqrt{\sum{y^2}-\frac{\left(\sum{y}\right)^2}{N}}
\end{math}

\begin{math}
\frac{dg(\lambda)}{d\lambda}=
\sqrt{\sum{y^2}-\frac{\left(\sum{y}\right)^2}{N}}\cdot
0.5\cdot\left(\sum_i\left(\frac{p_i^{\lambda}-1}{\lambda}\right)^2-\frac{1}{N}\left(\sum_i \frac{p_i^{\lambda}-1}{\lambda}\right)^2  \right)^{-0.5} 
\cdot\\
\left(
\sum_i2\left(\frac{p_i^{\lambda}-1}{\lambda} \right)\frac{1}{\lambda}\left(p_i^{\lambda}log{p_i}-\frac{p_i^{\lambda}-1}{\lambda}\right)
-\frac{2}{N}\left(\sum_i \frac{p_i^{\lambda}-1}{\lambda}\right) \left(\sum_i \frac{1}{\lambda}\left(p_i^{\lambda}log{p_i}-\frac{p_i^{\lambda}-1}{\lambda}\right) \right)
\right)
\end{math}

\begin{math}
=
\sqrt{\sum{y^2}-\frac{\left(\sum{y}\right)^2}{N}}\cdot
\left(\sum_i\left(\frac{p_i^{\lambda}-1}{\lambda}\right)^2-\frac{1}{N}\left(\sum_i \frac{p_i^{\lambda}-1}{\lambda}\right)^2  \right)^{-0.5} 
\cdot\\
\left(
\sum_i\left(\frac{p_i^{\lambda}-1}{\lambda} \right)\frac{1}{\lambda}\left(p_i^{\lambda}log{p_i}-\frac{p_i^{\lambda}-1}{\lambda}\right)
-\frac{1}{N}\left(\sum_i \frac{p_i^{\lambda}-1}{\lambda}\right) \left(\sum_i \frac{1}{\lambda}\left(p_i^{\lambda}log{p_i}-\frac{p_i^{\lambda}-1}{\lambda}\right) \right)
\right)
\end{math}

\begin{math}
\frac{dr}{d\lambda}=\frac{f'g-fg'}{g^2}=g^{-2}\cdot\left(f'g-fg'\right)
\end{math}

\begin{multline}
\frac{dr}{d\lambda}=
\left(\sum_i\left(\frac{p_i^{\lambda}-1}{\lambda}\right)^2-\frac{1}{N}\left(\sum_i \frac{p_i^{\lambda}-1}{\lambda}\right)^2  \right)^{-1} 
\left(\sum{y^2}-\frac{\left(\sum{y}\right)^2}{N}\right)^{-1}\cdot  \\
[
\left[
\sum_i\left(\frac{1}{\lambda}\left(p_i^{\lambda}log{p_i}-\frac{p_i^{\lambda}-1}{\lambda}\right)y_i\right) -  
\left(\sum_i\frac{1}{\lambda}\left(p_i^{\lambda}log{p_i}-\frac{p_i^{\lambda}-1}{\lambda}\right)\right)\left(\sum_iy_i\right)\frac{1}{N}
\right]  \\
\left(\sum_i\left(\frac{p_i^{\lambda}-1}{\lambda}\right)^2-\frac{1}{N}\left(\sum_i \frac{p_i^{\lambda}-1}{\lambda}\right)^2  \right)^{0.5} 
\sqrt{\sum{y^2}-\frac{\left(\sum{y}\right)^2}{N}}  \\
-
\left[\sum_i \left(y_i\frac{p_i^{\lambda}-1}{\lambda}\right) - \left(\sum_i\frac{p_i^{\lambda}-1}{\lambda}\right)\frac{\sum_iy_i}{N}\right]\cdot\\
\sqrt{\sum{y^2}-\frac{\left(\sum{y}\right)^2}{N}}\cdot
\left(\sum_i\left(\frac{p_i^{\lambda}-1}{\lambda}\right)^2-\frac{1}{N}\left(\sum_i \frac{p_i^{\lambda}-1}{\lambda}\right)^2  \right)^{-0.5} 
\cdot\\
\left[
\sum_i\left(\frac{p_i^{\lambda}-1}{\lambda} \right)\frac{1}{\lambda}\left(p_i^{\lambda}log{p_i}-\frac{p_i^{\lambda}-1}{\lambda}\right)
-\frac{1}{N}\left(\sum_i \frac{p_i^{\lambda}-1}{\lambda}\right) \left(\sum_i \frac{1}{\lambda}\left(p_i^{\lambda}log{p_i}-\frac{p_i^{\lambda}-1}{\lambda}\right) \right)
\right]
]
\end{multline}

\begin{multline}
\frac{dr}{d\lambda}=
\left(\sum_i\left(\frac{p_i^{\lambda}-1}{\lambda}\right)^2-\frac{1}{N}\left(\sum_i \frac{p_i^{\lambda}-1}{\lambda}\right)^2  \right)^{-0.5} 
\left(\sum_i{y_i^2}-\frac{\left(\sum_i{y_i}\right)^2}{N}\right)^{-0.5}\cdot  \\
[
\left[
\sum_i\left(\frac{1}{\lambda}\left(p_i^{\lambda}log{p_i}-\frac{p_i^{\lambda}-1}{\lambda}\right)y_i\right) -  
\left(\sum_i\frac{1}{\lambda}\left(p_i^{\lambda}log{p_i}-\frac{p_i^{\lambda}-1}{\lambda}\right)\right)\frac{\left(\sum_iy_i\right)}{N}
\right]  \\
-
\left[\sum_i \left(y_i\frac{p_i^{\lambda}-1}{\lambda}\right) - \left(\sum_i\frac{p_i^{\lambda}-1}{\lambda}\right)\frac{\left(\sum_iy_i\right)}{N}\right]\cdot\\
\left(\sum_i\left(\frac{p_i^{\lambda}-1}{\lambda}\right)^2-\frac{1}{N}\left(\sum_i \frac{p_i^{\lambda}-1}{\lambda}\right)^2  \right)^{-1} 
\cdot\\
\left[
\sum_i\left(\frac{p_i^{\lambda}-1}{\lambda} \right)\frac{1}{\lambda}\left(p_i^{\lambda}log{p_i}-\frac{p_i^{\lambda}-1}{\lambda}\right)
-\frac{1}{N}\left(\sum_i \frac{p_i^{\lambda}-1}{\lambda}\right) \left(\sum_i \frac{1}{\lambda}\left(p_i^{\lambda}log{p_i}-\frac{p_i^{\lambda}-1}{\lambda}\right) \right)
\right]
]
\end{multline}

\newpage

\begin{math}
r=\left(\sum_i{x_i^2}-\frac{\left(\sum_i{x_i}\right)^2}{N}\right)^{-0.5} 
\left(\sum_i{y_i^2}-\frac{\left(\sum_i{y_i}\right)^2}{N}\right)^{-0.5}\cdot
\left(\sum_i{y_ix_i} - \frac{\left(\sum_ix_i\right)\left(\sum_iy_i\right)}{N}\right)
\end{math}

\begin{multline}
\frac{dr}{d\lambda}=
\left(\sum_i{x_i^2}-\frac{\left(\sum_i{x_i}\right)^2}{N}\right)^{-0.5} 
\left(\sum_i{y_i^2}-\frac{\left(\sum_i{y_i}\right)^2}{N}\right)^{-0.5}\cdot\frac{1}{\lambda}\cdot  \\
[
\left[
\sum_i\left(y_ip_i^{\lambda}log{p_i}\right) -  
\sum_i\left(x_iy_i\right) -  
\left( \sum_i\left(p_i^{\lambda}log{p_i}\right) - \sum_i\left(x_i\right)  \right)\frac{\left(\sum_iy_i\right)}{N}
\right]  \\
-
\left(\sum_i{y_ix_i} - \frac{\left(\sum_ix_i\right)\left(\sum_iy_i\right)}{N}\right)\cdot
\left(\sum_i{x_i^2}-\frac{\left(\sum_i{x_i}\right)^2}{N}\right)^{-1} 
\cdot\\
\left[
\sum_i \left(x_ip_i^{\lambda}log{p_i}\right) - \sum_i \left(x_i^2\right)
-\left( \sum_i \left(p_i^{\lambda}log{p_i}\right) -\sum_i \left(x_i\right) \right)\frac{\left(\sum_i{x_i}\right)}{N} 
\right]
]
\end{multline}

\end{document}
