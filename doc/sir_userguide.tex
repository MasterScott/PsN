\documentclass[a4wide,12pt]{article}
%\setlength{\marginparwidth}{0pt}%35
%\setlength{\marginparsep}{0pt}%?
%\setlength{\evensidemargin}{0pt}
%\setlength{\oddsidemargin}{0pt}
\usepackage{lmodern}
\usepackage[utf8]{inputenc}
\usepackage[T1]{fontenc}
\usepackage{textcomp}
\usepackage{verbatim}
\usepackage{enumitem}
\usepackage{longtable}
\usepackage{alltt}
\usepackage{ifthen}
% Reduce the size of the underscore
\usepackage{relsize}
\renewcommand{\_}{\textscale{.7}{\textunderscore}}

\newcommand{\guidetitle}[1]{
\title{#1\\ \vspace{2 mm} {\large PsN 4.1.1}}
\date{2014-02-10}
}

\newcommand{\doctitle}[1]{
\title{#1}
\date{2014-02-10}
}


\newenvironment{optionlist}{
\renewcommand{\arraystretch}{1.1}
\setlength{\leftmargini}{2.5cm}
\begin{description}
%\setlength{\itemsep}{0ex}
}
{\end{description}}

\newcommand{\optname}[1]{\item{{\bfseries\texttt-#1}\newline}}
\newcommand{\optdefault}[2]{\item{{\bfseries\texttt-#1}{\mbox{ = \it #2}}\newline}}

\newcommand{\nextopt}{}

\guidetitle{SIR user guide}{2015-11-27}
\usepackage{amsmath}

\begin{document}

\maketitle

\newcommand{\guidetoolname}{sir}

%TODO cholesky block check in sampling when reserved labels for sd/corr

\section{Introduction}
The sir program calculates uncertainty
on model parameters for the input model using the Sampling Importance Resampling (SIR) procedure 
\cite{Dosne2013},\cite{Dosne2015}.
The output of the SIR procedure is a set of parameter vectors representing the uncertainty around 
model parameters, just like a bootstrap. 
Confidence intervals around the parameters are also computed, based on this set of parameter vectors.
 
The SIR procedure gets to the final set of parameter vectors via N successive iterations. 
Each iteration consists of 3 steps. 
First, M parameter vectors are simulated from a “proposal” uncertainty. Then, for each parameter vector, 
an “importance ratio” IR (representing how well the vector represents the uncertainty) is calculated. 
Finally, m parameter vectors (m<M) are then resampled from the M simulated vectors based on these IR.  
They are called the “posterior” or SIR uncertainty. This is the end of an iteration. In the next iteration, 
the m parameter vectors are then used as the new proposal uncertainty, and a new set of M parameter vectors 
are simulated from this new proposal. 
IR are evaluated for the new set of parameter vectors and m vectors are resampled based on their IR. 
In practice, the number of samples (M) and resamples (m) can vary between iterations. The procedure should be 
repeated until there is no change between the proposal uncertainty and the posterior uncertainty, meaning that 
the SIR procedure has converged to the true uncertainty.
Because the number of iterations is chosen by the user, diagnostics plots have been developed to assess whether 
this number of iterations was enough for SIR to converge, and these can be generated via the -rplots option,
see section Diagnostic sir plots.

Example command for two iterations:
\begin{verbatim}
sir -samples=1000,2000 -resamples=500,1000 -rplots=1 run10.mod
\end{verbatim}

\section{Obtaining a proposal uncertainty}
There are a number of different ways to obtain the proposal density for the first iteration.
\begin{enumerate}
\item The covariance matrix from a successful NONMEM covariance step. If the covariance step was successful, sir will automatically
read the covariance matrix from NONMEM, e.g. run10.cov,  and use it for the sampling in the first iteration.
\item If the covariance step was not successful, the precond script, see the precond\_userguide.pdf, 
is recommended. It can often recover the covariance step even if it failed when running the original model.
%\item If precond fails, then the user can compute a proposal covariance matrix $A$ based on the $R$ matrix given by NONMEM,
%e.g. run10.rmt, using $A=2R^{+}$ where $R^{+}$ is the pseudo-inverse of $R$. If precond has been run, the rmt-file is found in the 
%m1 subdirectory of the precond run directory. The $A$ matrix needs to be computed outside of PsN, using e.g. R, and
%then given as input to sir using option -covmat\_input.
\item If a bootstrap has been run, the file with simulated parameter vectors can be given as input, instead of 
a covariance matrix, with option -rawres\_input. 
Then the sir procedure starts with a 0th iteration where those vectors are treated as the resampled 
vectors from the third phase (Box-Cox transformation etc). This approach is useful when it is difficult to obtain a (faked) covariance matrix. 
If there are too few bootstrap vectors for full rank, sir will automatically fill in extra vectors using the
tweak\_inits functionality.
\item If the user can guess a reasonable CV for each parameter, then 
options -cv\_theta, -cv\_omega and -cv\_sigma can be combined
to obtain a proposal density without having to create the .cov-file.
\item If the user can guess a reasonable variance for each parameter, then 
options -covmat\_input=diagonal,
-theta\_inflation, -omega\_inflation and -sigma\_inflation can be combined to obtain any diagonal covariance matrix
without having to create the .cov-file.
\item If none of the above alternatives are possible, the -auto\_rawres option can be used.
Then sir will create a complete rawres\_input file using the tweak\_inits functionality.
\end{enumerate}

\section{Input and options}
A model file is required on the command-line. 
If output file (lst) exists, then it is important that the control stream copy at the top of the lst-file matches the actual input model file
in terms of which parameters are present and which parameters are FIXED or SAME.
\begin{optionlist}
\optdefault{samples}{M1,M2}
Default 1000,1000,1000,2000,2000.
A comma-separated list of integers, the number of parameter vectors to 
generate in each iteration. In each iteration the number needs to be greater than the number 
of resamples, unless sampling is with replacement.
\nextopt
\optdefault{resamples}{m1,m2}
Default 200,400,500,1000,1000.
A comma-separated list of integers, the number of parameter vectors to resample 
in each iteration based on the weights
computed from delta ofv and the pdf. The list length must be equal to list 'samples'.
\nextopt
\optdefault{rawres\_input}{filename}
If rawres\_input is given, sir will run an intial 0th iteration in which sir will read all parameter
vectors from this file, starting on line offset\_rawres+1 and skipping any that does not fulfill the filter rules, if set.
These vectors will be Box-Cox transformed and used to create a proposal density for the next iteration.
This option is not allowed together with covmat\_input. 

If the number of vectors read is smaller than 
the number of estimated parameters (also counting off-diagonals),
it is not possible to obtain an empirical covariance matrix with full rank.
In this case sir will append copies of the existing vectors, perturbed using the tweak\_inits
functionality, until the resulting set of vectors has full rank.
If option -auto\_rawres=degree is set on the command-line, tweak\_inits will be done by 'degree',
otherwise the default degree 0.1 will be used.
If $N$ parameter vectors are read from the rawres file,
the generation of perturbed parameter vector $i$, element $j$, is performed according to 
init\_ij = init\_kj + rand\_uniform(-degree*init\_kj,+degree*init\_kj) where k=i modulo N and init\_kj is the value of 
parameter $j$ in original rawres vector $k$. 

The labels for  THETA/OMEGA/SIGMA 
in the file must match the labels in the model given as input 
to sir, the theta columns must be directly followed by the omega columns 
which must be directly followed by the sigma columns, and the first or
second column must have header model. Note that is is 
possible to generate a file with initial parameter estimates outside 
of PsN, as long as the file follows the format rules.
\nextopt
\optdefault{offset\_rawres}{N}
Only relevant in combination with -rawres\_input. Default 1. The number of result lines to skip in the input raw results file before starting to read final parameter estimates. In a regular bootstrap raw\_results file, and also in an initial\_estimates.csv file from an sse run, 
the first line of estimates refers to the input model with the full dataset, so therefore the default offset is 1. 
\nextopt
\optdefault{in\_filter}{condition on numerical col}
Only relevant in combination with -rawres\_input. Default not used. The parameter estimates lines in the file can be filtered on values in the different columns. When specifying which column(s) the filtering should be based on, the exact column name must be used, e.g. minimization\_successful. Filtering can only be based on columns with numeric values. The allowed relations are .gt. (greater than), .lt. (less than) and .eq. (equal to). If the value in the filter column is 'NA' then that parameter set will be skipped, regardless of the defined filter relation. Conditions are separated with commas. If the remaining number of lines after filtering is smaller than -samples, sse will stop with an error message. Then the user must either change the filtering rules or change -samples. If the user has created a file with parameter estimates outside of PsN, filtering can be done on any numeric column in that file. Do not set column headers containing .eq. or .lt. or .gt.in the user-generated file as this would interfere with the in\_filter option syntax.

Example (there must be no linebreaks in the actual command):
\begin{verbatim}
-in_filter=minimization_successful.eq.1,
significant_digits.gt.3.5
\end{verbatim} \\
\nextopt
\optdefault{auto\_rawres}{degree}
Default not set. Not allowed in combination with covmat\_input. If rawres\_input is not set,
this option will make sir use the tweak\_inits functionality to automatically create a 
rawres\_input file with perturbed parameter vectors, and these vectors will then be used
as if read from a real raw\_results file. 
The generation of perturbed parameter vector $i$, element $j$, is performed according to 
init\_ij = init\_0j + rand\_uniform(-degree*init\_0j,+degree*init\_0j)
where init\_0j is the final estimate of 
parameter $j$ in the input model. 
The tweaking procedure makes sure that boundary conditions, including positive definiteness of 
\$OMEGA and \$SIGMA, are respected.

If option -auto\_rawres is used
in combination with rawres\_input, tweak\_inits by 'degree' will be used 
to augment the given rawres\_input file until the matrix has full rank, see help text for -rawres\_input.
\nextopt
%\optname{with\_replacement}
%Default not set. By default, resampling is performed without replacement, but setting this option gives resampling 
%with unlimited replacement.
%To cap resampling at e.g. 5, use option -cap\_resampling.
\nextopt
\optdefault{cap\_resampling}{N}
Default 1. The maximum number of times a single parameter vector can be resampled in each iteration. 
Value 1 means regular sampling without replacement. 
A value equal to the maximum value of the 'resamples' array will give
unlimited replacement, but this is not recommended. Any value in between gives limited replacement. 
\nextopt
\optdefault{covmat\_input}{filename}
Not allowed together with rawres\_input or auto\_rawres. If given, this covariance matrix is
used in iteration 1
for sampling parameter vectors and for computing the PDF.
The matrix will not be used in any but the first iteration. 

If the user wants to input an identity matrix, it is not necessary to create the .cov-file.
Instead set option -covmat\_input=identity. Option -covmat\_input=identity can be combined
with inflation to get any diagonal covariance matrix without creating the .cov-file.

If a file is used, the format of the file is similar to a NONMEM-generated .cov-file except 
that the \verb|TABLE NO.| line should be omitted.
Fixed parameters do not have to be omitted, sir will filter them out. 
The covmat\_input file must be formatted to contain a space- or tab-separated  $N\times N$ symmetric covariance matrix.
The first line in the file must be a header with labels for THETA/OMEGA/SIGMA written as in a regular NONMEM .cov-file 
and a leading column called NAME: 
\begin{verbatim}
NAME THETA1 THETA2 ... SIGMA(1,1) ... OMEGA(1,1) ... 
\end{verbatim}
The NAME column contains the same parameter labels (to identify the rows).
The rows and columns must be sorted with THETAs first.

Please note that there is a PsN script called covmat that can be used to generate a covariance matrix from a raw\_results file.
\nextopt
\optdefault{theta\_inflation}{X}
Either a positive vector with length equal to the number of \emph{estimated} (not FIX) \$THETAs, or
a positive scalar which will be interpreted as a vector of the required length and all equal values.
Default is the scalar 1, which is the same as no inflation. 
If given, the variance
of each estimated \$THETA 
will in the first iteration be inflated with the factor from the inflation vector
before the parameter vectors
are sampled from the truncated multivariate normal distribution.
The weights will also be computed based the inflated covariance matrix. 
Inflation is not used in any but the first iteration.
\nextopt
\optdefault{omega\_inflation}{X}
Either a positive vector with length equal to the number of \emph{estimated} (not FIX) 
%\$OMEGA elements, both diagonal
\$OMEGA \emph{diagonal} elements
%, both diagonal
%and off-diagonal, 
or
a positive scalar which will be interpreted as a vector of the required length and all equal values.
Default is the scalar 1, which is the same as no inflation. 
If given, the variance
of each estimated \$OMEGA element
will in the first iteration be inflated with the factor from the inflation vector
before the parameter vectors
are sampled from the truncated multivariate normal distribution. Inflation for off-diagonal
omega elements will be automatically computed based on inflation for diagonals so that the correlation is unchanged.
For example, the variance of OM(2,1) will be inflated by 
\[
\sqrt{inflation(OM(1,1))}\cdot\sqrt{inflation(OM(2,2))}
\].
The weights will also be computed based the inflated covariance matrix. 
Inflation is not used in any but the first iteration.
\nextopt
\optdefault{sigma\_inflation}{X}
See help text for omega\_inflation.
\nextopt
\optdefault{cv\_theta}{X}
Default not used. Used for giving the proposal density as CV\%, e.g. 30 for 30\% CV.
Not allowed in combination with inflation, -covmat\_input, -rawres\_input or -auto\_rawres.
Either a positive vector with length equal to the number of estimated (not FIX) \$THETAs, or
a positive scalar which will be interpreted as a vector of the appropriate length and all equal values.
When this option is used, the variances of theta(i) in the proposal density will be computed
as 
\[
\left(theta(i)*cv_{theta}(i)/100\right)^2
\]
\nextopt
\optdefault{cv\_omega}{X}
Default not used. Used for giving the proposal density as CV\%, e.g. 30 for 30\% CV.
Not allowed in combination with inflation, -covmat\_input, -rawres\_input or -auto\_rawres.
Either a positive vector with length equal to the number of estimated (not FIX) \$OMEGA 
diagonal elements or a positive scalar which will be interpreted as a vector of the required 
length and all equal values. When this option is used, the variances of omega(i,i) in the 
proposal density will be computed as 
\[
var_{om(i,i)}=\left(om(i,i)*cv_{om}(i)/100\right)^2
\]
%variance (\hat Cov(X,Y)) = q(n)/n * (Cov(X,Y)^2 + Var(X) Var(Y))
%Jonas  q(n) = 1 if ML-estiamtes for \hat Cov(X,Y), and converge to 1 when n-> 1 for other estimators.
% N=2\frac{var^2}{SE^2}+1  where var is variance est from NONMEM, SE is SE of variance from NONMEM
The variances of off-diagonal elements will be computed according to
\[
var_{om(i,j)}= \frac{1}{\hat{N}} \cdot \left(om(i,j)^2 + om(i,i)\cdot om(j,j) \right)
\]
where
\[
\hat{N} = \frac{om(i,i)^2}{var_{om(i,i)}}+\frac{om(j,j)^2}{var_{om(j,j)}}+1 =\left(\frac{100}{cv_{om}(i)}\right)^2+\left(\frac{100}{cv_{om}(j)}\right)^2+1
\]

%old version
%\[
%var_{omega(i,j)}=\sqrt{var_{omega(i,i)}}\cdot\sqrt{var_{omega(j,j)}}
%\cdot\frac{abs(omega(i,j))}{\sqrt{omega(i,i)\cdot omega(j,j)}}
%\]
If -cv\_theta is set to a scalar when option -cv\_omega is not given, -cv\_omega will 
automatically be set to the same scalar as -cv\_theta. If -cv\_theta is set to a vector and 
-cv\_omega is not set at all, or if -cv\_omega is set when -cv\_theta is not, then there will 
be an error message.
\nextopt
\optdefault{cv\_sigma}{X}
See help text for cv\_omega.
\nextopt
\optdefault{problems\_per\_file}{N}
Optional, default 100. The number of \$PROBLEM per model file when evaluating the parameter vectors
on the original data (running
MAXEVAL=0 or similar to get ofv:s for parameter vectors). Setting a higher value
decreases the overhead involved in running each control stream, but increases the 
risk of losing many samples in case a model file crashes. Setting -problems\_per\_file=1
gives maximum robustness to individual crashes, but also maximum overhead cost.
\nextopt
\optdefault{mceta}{N}
Optional. Sets option MCETA=N in \$ESTIMATION. Only allowed with NM7.3 and classical estimation methods.
\nextopt
\optname{copy\_data}
Default set. If option is set, the original data file
will be copied to the run directory.
If option is unset using -no-copy\_data, the absolute path to the original data file will be used in
\$DATA, and the data file will not be copied. This saves disk space.
\nextopt
\optname{recenter}
Default set. If option is set and any sampled parameter vector has a smaller ofv than the original
vector $\mu$ of final parameter estimates of the input model, then vector $\mu$ for the \emph{next} iteration will be replaced with
the sampled parameter vector with the smallest ofv.
If option is unset using -no-recenter, the original parameter vector will be kept
and a warning will be printed that some vectors had a negative delta-ofv.
\nextopt
\optname{boxcox}
Default set. 
If option is set, sir will Box-Cox transform parameter vectors before computing
the covariance matrix used for sampling.
If option is unset using -no-boxcox, no transformation will be performed before computing covariance matrix
used for sampling.
\nextopt
\optname{add\_iterations}
Default not set. 
Iterations can be added to a previous sir run, for example if the diagnostic R plots show that
convergence has not been achieved. When option -add\_iterations is set, option -directory 
(see the common options section below) is required,
and this directory must contain output from at least one complete sir iteration.

The options set on the commandline 
pertain to \emph{new} iterations with the last available iteration as starting point.
Options -covmat\_input and -rawres\_input are forbidden in combination with -add\_iterations, 
the model file does not have to be specified (it will be ignored if it is),
otherwise the same sir options are required/optional as in a regular sir run.

If some but not all iterations from the previous run had finished, the new run will start with the output
from the last complete iteration, and the old sample and resample counts for
the \emph{non-finished} iterations will be dropped. Example: if the old setting was -samples=1000,1000
and the new with -add\_iterations 
is -samples=2000,2000 and the old run was terminated after the first iteration,
then after the add\_iterations run has finished
there will be three iterations in total with sample sequence 1000,2000,2000 
after the new run has finished.

\nextopt
\end{optionlist}


\subsection{Diagnostic sir plots}
\newcommand{\rplotsconditions}{
See section Output, subsections Basic and Extended diagnostic plots,
for descriptions of the default sir plots.
The default sir template 
requires 
that R libraries
gplots, ggplot2, plyr, dplyr, caTools, reshape, gridExtra, RColorBrewer and tidyr are installed.
If the conditions are not fulfilled then no pdf will be generated,
see the .Rout file in the main run directory for error messages.
}

PsN can automatically generate R plots to visualize results for \guidetoolname, using a default template found in the R-scripts subdirectory of the installation directory. The user can also create a custom template, see more details in the section Auto-generated R-plots from PsN in common\_options.pdf.

\rplotsconditions

\begin{optionlist}
\optdefault{rplots}{level}
-rplots<0 means R script is not generated\\ 
-rplots=0 (default) means R script is generated but not run\\ 
-rplots=1 means basic plots are generated\\													  
-rplots=2 means basic and extended plots are generated\\													  
\nextopt
\end{optionlist}

\subsubsection*{Troubleshooting}
If no .pdf was generated even if a template file is available and the appropriate options were set, check the .Rout-file in the main run directory for error messages. If no .Rout-file exists, then check that R is properly installed, and that either command 'R' is available or that R is configured in psn.conf.


 

\subsection{Some important common PsN options}
There are many options that govern how PsN manages NONMEM runs, and
those options are common to all PsN programs that run NONMEM.
For a complete list of such options see common\_options.pdf, 
or psn\_options -h on the commandline. A selection of
the most important common options relevant for sir is found here.
\begin{optionlist}
\optname{h or -?}
Print the list of available options and exit. 
\nextopt
\optname{help}
With -help all programs will print a longer help message. 
If an option name is given as argument, help will be printed for this option. 
If no option is specified, help text for all options will be printed. 
\nextopt
\optdefault{directory}{'string'}
Default \guidetoolname\_dirN,
where N will start at 1 and
be increased by one each time you run the script. The directory option sets the directory in which PsN 
will run NONMEM and where PsN-generated output files will be stored. 
You do not have to create the directory,  it will be done for you. If you set
-directory to a the name of a directory that already exists, PsN will run in the existing directory.
\nextopt
\optdefault{seed}{'string'}
You can set your own random seed to make PsN runs reproducible.
The random seed is a string, so both -seed=12345 and -seed=JustinBieber are valid.
It is important to know that because of the way the Perl pseudo-random
number generator works, for two similar string seeds the random sequences may be identical. 
This is the case e.g. with the two different seeds 123 and 122. 
Setting the same seed guarantees the same sequence, but setting two slightly different 
seeds does not guarantee two different random sequences, that must be verified.
\nextopt
\optdefault{clean}{'integer'}
Default 1. The clean option can take four different values:  
\begin{description}
\item[0] Nothing is removed 
\item[1] NONMEM binary and intermediate files except INTER are removed, and files specified with option -extra\_files. 
\item[2] model and output files generated by PsN restarts are removed, and data files in the NM\_run directory, and (if option -nmqual is used) the xml-formatted NONMEM output. 
\item[3] All NM\_run directories are completely removed. If the PsN tool has created modelfit\_dir:s inside the main run directory, these  will also be removed. 
\end{description}
\nextopt
\optdefault{nm\_version}{'string'}
Default is 'default'. 
If you have more than one NONMEM version installed you can use option
-nm\_version to choose which one to use, as long as it is 
defined in the [nm\_versions] section in psn.conf, see psn\_configuration.pdf for details. 
You can check which versions are defined, without opening psn.conf, using the command
\begin{verbatim}
psn -nm_versions
\end{verbatim}
\nextopt
\optdefault{threads}{'integer'}
Default 1. Use the threads option to enable parallel execution of multiple models.
This option decides how many models PsN will run at the same time, and it is completely
independent of whether the individual models are run with serial NONMEM or parallel NONMEM.
If you want to run a single model in parallel you must use options -parafile and -nodes.
On a desktop computer it 
is recommended to not set -threads higher the number of CPUs in the system plus one. 
You can specify more threads, 
but it will probably not increase the performance. If you are running on a computer cluster, 
you should consult your 
system administrator to find out how many threads you can specify. 
\nextopt
\optname{version}
Prints the PsN version number of the tool, and then exit. 
\nextopt
\end{optionlist}



\section{sir program workflow}
\subsubsection*{Setup} 
The sir program will run the input model unless the lst-file with results is already present. If the lst-file is present, it
is important that the control stream copy at the top of the lst-file matches the
actual input model file in terms of which parameters are estimated, 
FIXED or SAME, otherwise there will be a mis-match between which parameter estimates are read from the
lst-file and which estimates are needed for the sir procedure.

Unless option -rawres\_input and/or -auto\_rawres is used, read the covariance matrix to be used
in iteration 1.
The covariance matrix file name is either explicitly given with option -covmat\_input
or by default the .cov-file of the input model run.

\subsubsection*{For each iteration $i$ (each item in list 'samples'):}
(NB: If -rawres\_input and/or -auto\_rawres is used there first an iteration 0 that starts at \underline{Step 5})
\begin{itemize}
\item[\underline{Step 1}] 
Simulate 'samples item $i$' parameter vectors from the 
covariance matrix, using Perl function Math::Random::random\_multivariate\_normal.
If this is iteration 1, $i==1$, 
the covariance matrix is inflated before sampling 
if either of options -theta\_inflation, -omega\_inflation or -sigma\_inflation is set. 

\noindent If there was a previous iteration, i.e $i>1$ or option -rawres\_input or -auto\_rawres was used, 
the covariance matrix used in this step is the
empirical covariance matrix of (optionally Box-Cox transformed) resampled parameter vectors from
Step 6 in the previous iteration. 
If the vectors were Box-Cox transformed then
sampling is done on Box-Cox scale, and 
then the samples are back-transformed for boundary checks and evaluation.

\noindent If there was no previous iteration, the covariance matrix is either from
the .cov-file given by NONMEM or given via option -covmat\_input.
The sampling is done on the original scale. 

\noindent If a vector (on the original scale) does not fulfill the constraints from \$THETA boundaries
and positive definiteness of \$OMEGA and
\$SIGMA blocks (as judged by a PsN-implemented Cholesky decomposition) 
then that vector is 
discarded and a new one is drawn.
\item[\underline{Step 2}] 
Calculate each vector $x$’s probability 
given the covariance matrix based on the formula for the probability 
density function (PDF) of a multivariate normal distribution:\\
\begin{math}
\frac{1}{\left(2\pi\right)^{k/2}\left(det\left(A\right)\right)^{1/2}} exp\left(-\frac{1}{2}(\left(x-\mu\right)A^{-1} \left(x-\mu\right)^T\right)
\end{math}
\\
where $k$ is the number of dimensions, 
$\mu$ is the vector of expectations and $A$ is the (optionally inflated) covariance matrix.
The values are normalized with the PDF for the vector of expectations $\mu$, giving\\
\begin{math}
relPDF=exp\left(-\frac{1}{2}(\left(x-\mu\right)A^{-1} \left(x-\mu\right)^T\right)
\end{math}
\\
NB 1: When the NONMEM covariance matrix is used as input, PsN will compute $relPDF$ without reading the inverse covariance matrix
output by NONMEM.
NB 2: If $i>0$ or -rawres\_input is used, the vectors (including $\mu$) and covariance matrix are on Box-Cox scale if option -boxcox was set (the default).
\item[\underline{Step 3}] 
Evaluate the (original scale) parameter vectors on the original data.
Create model files with up to 100 \$PROBLEM based on the original model file but setting MAXEVAL=0
and replacing inits of nth \$PROBLEM with nth parameter vector on original scale, and also set MCETA if option -mceta was used. Compute dOFV 
(delta-OFV, reference is input model ofv) and store in raw results file for this iteration.
\item[\underline{Step 4}] 
Calculate the weights as the ratio $\frac{e^{-0.5\cdot dOFV}}{relPDF}$ and store in raw results file for this iteration. 
Resample 'resamples item $i$' parameter vectors based on these weights, with or without limited replacement depending on option 
-cap\_resampling.
Store '-cap\_resampling' copies of each vector in the raw results file for this iteration, with a 1 or 0 in the 'resamples'
column indicating if the copy was resampled or not.

\noindent If any sample(s) had a negative dOFV and option -recenter is set (the default), reset the $mu$-vector
to be used in the next iteration to the parameter vector with the lowest ofv.
\item[\underline{Step 5}]
If this is iteration $0$, i.e. rawres\_input is used and the procedure starts here: Read all parameter vectors
from file rawres\_input, excluding offset\_rawres and vectors not passing in\_filter (if set).

\noindent Otherwise read the 'resamples item $i$' resampled vectors from the raw results file of Step 4.
%, using multiple copies for vectors that were 
%resampled more than once (only possible if -with\_replacement was set). 
\item[\underline{Step 6}]
If \emph{not} last iteration \emph{and} option -boxcox is set (the default): 
For each parameter individually, find the Box-Cox transformation that maximizes the correlation (Pearson's r) with the normal density.
Do not allow $\lambda$ less than -3 or larger than +3. The  $\lambda$ used is allowed to differ from the true optimal  $\lambda$
by at most $0.2$. The optimization is done with the DIRECT algorithm \cite{direct}.
Compute the covariance matrix for the Box-Cox transformed resampled parameter vectors, and go to Step 1.

\noindent If \emph{not} last iteration \emph{and} option -boxcox is unset with -no-boxcox: 
Compute the covariance matrix for the resampled parameter vectors on original scale, and go to Step 1.

\noindent If last iteration: 
Compute the covariance matrix for the resampled parameter vectors on original scale, 
store in sir\_results.csv. 
Compute univariate confidence intervals and store in sir\_results.csv.
\end{itemize}

\section{Recovering an interrupted run}
The sir program can automatically recover an interrupted sir run
if at least one iteration was completed. This process is invoked automatically
if option -directory is set to an existing sir run directory containing output from
at least one complete iteration and option -add\_iterations is \emph{not} set.

When recovering an old run, the model file and all sir-specific options plus common options 
-nm\_version and -seed from
the new commandline \emph{will be ignored}, instead the model and options
set in the original run will be used.

\section{Adjusting samples and resamples}
Sometimes the evaluation run fails for some samples, resulting in fewer successful samples (samples with defined ofv)
than the number of attempted samples. PsN tries to compensate for this loss, and then
adjust again if the success rate goes up. If PsN has to compensate for loss, option -problems\_per\_file will 
automatically be reset to 1.
\subsection{Adjusting samples}
In iteration 1 the number of attempted samples is always the number of
samples requested on the commandline with option -samples.
When the number of successful samples in the previous iteration was less than
$95\%$ of the attempted samples, the number of attempted samples in the current 
iteration is adjusted according to\\
$current.attempted.samples=\frac{current.requested.samples \cdot previous.attempted.samples}{previous.successful.samples}$\\
The number of attempted samples can be greater, but never smaller, than requested samples.

\subsection{Adjusting resamples}
When the number of successful samples for the current iteration 
differs more than $5\%$ from the original samples count requested on the commandline with option -samples,
PsN will try to maintain the requested sample-to-resample ratio by 
adjusting the actual number of resamples for the current iteration 
according to \\
$actual.resamples=\frac{requested.resamples\cdot successful.samples}{requested.samples}$\\
Note that the number of actual resamples can be both greater and smaller than the 
requested samples, because the attempted samples adjustment can make 
the number of successful samples both greater and smaller
than requested samples.



\section{Output}
\begin{itemize}
\item raw\_results for each iteration with added columns for\\
\begin{tabular}{ll}
\bf{sample.id} & A unique number for each simulated vector\\
\bf{deltaofv} & $\mathrm{ofv} - \mathrm{original}\mbox{\tt\string_}\mathrm{model}\mbox{\tt\string_}\mathrm{ofv}$\\
\bf{likelihood\_ratio} & $e^{-0.5\cdot \mathrm{deltaofv}}$ \\
\bf{relPDF} & relative PDF \\
\bf{importance\_ratio} & $\mathrm{likelihood}\mbox{\tt\string_}\mathrm{ratio}/\mathrm{relPDF}$\\
\bf{probability\_resample} & $\mathrm{importance}\mbox{\tt\string_}\mathrm{ratio}/\sum{\mathrm{importance}\mbox{\tt\string_}\mathrm{ratio}}$ \\
\bf{resamples} & number of actual resamples \\
\bf{sample\_order} & order number of resample \\
\end{tabular}

The first line is the result for vector $\mu$ of the current iteration. Normally this is the final estimates of the
input model, but if option -recenter was set and a parameter vector had a lower ofv, then $\mu$ for the \emph{next} iteration
will be that new vector.

In the raw results file there will be '-cap\_resampling' (default 1) almost identical line(s) for each parameter vector,
the only differences will be in the resamples and sample\_order columns.

\item sir\_results.csv with summary statistics, percentiles and empirical sd-correlation matrix
\item <modelname>\_sir.cov with the empirical covariance matrix in NONMEM-like format, i.e. 
      fixed width space separated with generic headers in order THETA, SIGMA, OMEGA.
\item if option -rplots>0: A file PsN\_sir\_plots.pdf with diagnostic sir plots.
\item For all but the last iteration: A file with the Box-Cox $\lambda$ for each parameter, the $\delta$ used to shift
the values to positive before Box-Cox transformation, the original estimate and the transformed estimate. 
If the $\lambda$ column is empty it means that no transformation was
needed because the parameter was already close to normally distributed.
\item For all but the last iteration: A file boxcox\_covmatrix\_iterationi.cov with 
the Box-Cox space covariance matrix for iteration $i$,
or, if -no-boxcox was set on commandline, a file untransformed\_covmatrix\_iterationi.cov.
\item A file summary\_iterations.csv with, for each iteration, 
the requested samples, attempted samples, successful samples, requested resamples,
actual resamples, requested samples-resamples ratio, actual samples-resamples ratio, count negative delta-ofv and minimum iteration sample ofv.
\end{itemize}

\subsection{Basic diagnostic plots}
Basic diagnostic rplots for the last iteration of sir will be generated 
in file  PsN\_plots\_base.pdf 
if option -rplots is set >0.

\subsubsection{dOFV plot}
\begin{description}
\item[Purpose] Determine whether the SIR procedure has converged, i.e. whether SIR results are final.
\item[Description] This plot shows the distribution of the dOFV for the N proposal (PROPOSAL, dotted colored lines) and N posterior uncertainties (SIR, full colored lines) of the SIR procedure. The distributions are represented by plotting the value of the dOFV (on the y-axis) for the different quantiles (on the x-axis). The colors correspond to the iterations. A reference distribution corresponding to a chi-square distribution with degrees of freedom equal to the total number of estimated parameters is displayed (full grey line). The shaded areas correspond to resampling noise over the last 2 iterations. The estimated degree of freedom for each distribution is displayed at the bottom right.
\item[Interpretation] SIR results are final if the two last SIR posterior uncertainties are similar, i.e. within resampling noise. If this is not the case, further iterations should be performed until it is the case. The dOFV curve of the final posterior uncertainty should resemble a chi-square distribution with degree of freedom equal or less than the total number of estimated parameters, i.e. be below the reference chi-square.
\end{description}
\subsubsection{CI plot}
\begin{description}
\item[Purpose] Visualize the estimated uncertainty of the model parameters over all SIR iterations. 
\item[Description] This plot displays the confidence intervals (colored error bars on the y-axis, default CI95\%) 
of all model parameters over all iterations of the SIR procedure (on the x-axis). It also includes the CI of the first proposal density, i.e. the SIR starting point (NONMEM’s covariance matrix for example). The points correspond to the median and the horizontal dashed lines to the final estimates of the model (from NONMEM’s .ext file). The final results of the SIR procedure are the last CI on the right in each panel.
\item[Interpretation] Final median and outer percentiles of a parameter’s CI can be read off this plot. Be \emph{very} 
careful when looking at trends over iterations (stabilization/increase/decrease), as the number of resamples from which the CI are computed can be very variable. The outer bounds of wide CI (such as the default 95\% here) 
can vary a lot because of random noise, so this pot should in general not be used to diagnose SIR convergence.
\end{description}


\subsection{Extended diagnostic plots}
Extended diagnostic plots will be generated
in file PsN\_plots\_extended.pdf if option -rplots is set >1.

\subsubsection{Adequacy of proposal density}
\begin{description}
\item[Purpose] Evaluate how good the proposal uncertainties describe the true parameter uncertainty (by iteration).
\item[Description] This plot shows separately for each parameter and each iteration the proportion of resampled parameters (on the y-axis) over different parts of the parameter space defined by the simulated parameters, which is divided into 10 equally sized bins ordered by increasing parameter values on the x-axis. 
The horizontal dotted line at $\frac{m_{total}}{M_{total}}$ 
corresponds to the expected value of this proportion were the proposal uncertainty the true uncertainty. The grey shaded area corresponds to stochastic noise, computed as the CI95\% 
around the expected proportion.
\item[Interpretation] 4 types of trends can be observed in this plot. If the observed proportion is within the stochastic noise around the expected proportion for all percentile bins (horizontal trend), it means that the proposal density of the last iteration is close to the true uncertainty. If the observed proportion is higher in the center and lower at the ends (bell-shaped trend), it means that the proposal density is too wide. Oppositely, if the observed proportion is lower in the center and higher at the ends (u-shaped trend), it means that the proposal density is too narrow. Lastly, if the proportion is higher at one end and lower at the other (diagonal trend), it means that the proposal density is not asymmetric enough. No action should be taken based on this plot as it does not tell us whether the resampling was able to compensate for these trends.
\end{description}

\subsubsection{Exhaustion of samples (2 plots: by parameter, and over all parameters)}
\begin{description}
\item[Purpose] Support determination of whether the SIR procedure has converged, i.e. whether SIR results are final.
\item[Description] These 2 plots show the number of resampled parameters (on the y-axis) that belong to “sample bin X” over different parts of the parameter space defined by the resampled parameters (on the x-axis). The resampled parameter space is divided into 5 equally sized “resample” bins ordered by resampling order. For each parameter, the focus is only on the sample bin with the highest proportion identified in the previous plot (“sample bin X”) as this is the limiting factor for SIR to converge. The first plot looks at the exhaustion of samples separately for each parameter and the second plot at the exhaustion of samples over all parameters. In the first plot, the horizontal dotted line corresponds to the observed number of resamples in resample bin X divided by 5 (mbinX/5) and the grey shaded area corresponds to stochastic noise, computed as the CI95\% around the number of resamples. In the second plot, the blue line corresponds to a smooth of the number of parameters resampled over all parameters.
\item[Interpretation] Only 1 type of trend needs to be detected in this plot. If the number of resampled vectors decreases over the resample bins (diagonal downward trend), it means that there were not enough samples in the SIR procedure to fully correct the proposal uncertainty. Further SIR iterations should thus be performed.
\end{description}

\section{Troubleshooting}
\subsection*{Mismatch between input files}
If you use the covmat\_input option and get an error message similar to
\begin{verbatim}
Number of parameters 122 in covmat_input does not match 
number 123 of estimated parameters in input model
\end{verbatim}
then the most likely cause is that there is a mismatch between the input files. Make sure that the number of
estimated parameters (the number of parameters, the FIXED and SAME settings, the DIAGONAL or BLOCK omega/sigma)
match between the control stream copy at the beginning of the pre-existing lst-file and the input model file.
%\subsection*{Numerical error during Box-Cox transformation}
%If there is a message ``Numerical error when searching for optimal lambda for Box Cox transformation...''
%it is recommended to inspect the raw results file to see if parameter vectors very similar
\subsection*{raw\_results file with \$PRIOR}
If the input model has \$PRIOR NWPRI, and the priors are encoded with \$THETA, \$OMEGA and \$SIGMA instead of the
prior-specific records \$THETAP, \$THETAPV, \$OMEGAP etc, then PsN will not be able to handle the parameter column
headers correctly in the raw\_results file. The solution is to always use the prior-specific records for
encoding the prior information.
\subsection*{Error in iteration 1 that covariance matrix not positive definite}
The sir program will use a Cholesky decomposition without pivoting for processing of the covariance matrix. 
This works in most cases, but sometimes this algorithm can fail for a matrix that is mathematically positive definite
but requires pivoting for Cholesky decomposition to work. Another cause of failure can be rounding. 
Since NONMEM performs rounding before writing 
the covariance matrix to the .cov-file, it can happen that the matrix in the cov-file is not positive definite
although NONMEM's internally stored covariance matrix was invertible. A third cause of failure
is when NONMEM writes something other than the covariance matrix, for example the S-matrix, to the .cov-file.

When the sir program's Cholesky decomposition of the covariance matrix fails, 
sir will stop with an error message saying that the covariance matrix is numerically not positive definite. 
If this happens in iteration 1 when option -covmat\_input or the default NONMEM covariance matrix was used, 
then the user must manually modify the covariance matrix to improve its numerical properties (for example increase the diagonal elements
and/or change the order of the parameters so that pivoting is not necessary),
and then use option -covmat\_input to give the modified matrix as input to sir. 
If this happens in iteration 1 after a 0th iteration based on -auto\_rawres, or -rawres\_input with augmenting using -auto\_rawres, 
then it is recommended to either run a new sir with -no-boxcox or a different random seed, or 
try to run sir with -covmat\_input=tweak\_inits.cov where tweak\_inits.cov is found in the failed sir run directory and
is the empirical covariance matrix from iteration 0 without Box-Cox transformation.


\begin{thebibliography}{99}
\bibitem{Dosne2013} A. G. Dosne, M. Bergstrand and M. O. Karlsson, 
{\em Application of Sampling Importance Resampling to estimate parameter uncertainty distributions},
PAGE 22 (2013) Abstr 2907 \mbox{www.page-meeting.org/?abstract=2907}.
\bibitem{Dosne2015} A. G. Dosne, M. Bergstrand and M. O. Karlsson, 
{\em Determination of Appropriate Settings in the Assessment of Parameter Uncertainty Distributions 
using Sampling Importance Resampling (SIR)},
PAGE 24 (2015) Abstr 3546 \mbox{www.page-meeting.org/?abstract=3546}.
\bibitem{direct} D. R. Jones, C. D. Perttunen and B. E. Stuckman, {\em Lipschitzian optimization without the Lipschitz constant},
Journal of Optimization Theory and Application: 79:1, 1993.
\end{thebibliography}

\end{document}

\section{Suggested new workflow}
NB: if -rawres\_input is specified, then the process starts
at step 6 using those vectors as the 'resamples\_1' parameter vectors. 
\begin{itemize}
\item[\underline{Setup}] The sir program will run the input model unless the lst-file with results is already present. 
If the lst-file is present, it
is important that the control stream copy at the top of the lst-file matches the
actual input model file in terms of which parameters are estimated, 
FIXED or SAME, otherwise there will be a mis-match between which parameter estimates are read from the
lst-file and which estimates are needed for the sir procedure.
\item[\underline{Step 1}] Simulate 'samples\_1' parameter vectors from the 
(possibly inflated)
covariance matrix, i.e. the .cov-file given by NONMEM 
or the matrix given via option -covmat\_input. PsN uses the Perl function\\ 
Math::Random::random\_multivariate\_normal
for sampling. If a vector does no fulfill the constraints from \$THETA boundaries
and positive definiteness of \$OMEGA and
\$SIGMA blocks (as judged by a PsN-implemented Cholesky decomposition) 
then that vector is 
discarded and a new one is drawn.
\item[\underline{Step 2}] Calculate each vector $x$’s probability 
given the covariance matrix based on the formula for the probability 
density function (PDF) of a multivariate normal distribution:\\
\begin{math}
\frac{1}{\left(2\pi\right)^{k/2}\left(det\left(A\right)\right)^{1/2}} exp\left(-\frac{1}{2}(\left(x-\mu\right)A^{-1} \left(x-\mu\right)^T\right)
\end{math}
\\
where $k$ is the number of dimensions, 
$\mu$ is the vector of expectations and $A$ is the (possibly inflated) covariance matrix.
The values are normalized with the PDF for the vector of expectations $\mu$, giving\\
\begin{math}
relPDF=exp\left(-\frac{1}{2}(\left(x-\mu\right)A^{-1} \left(x-\mu\right)^T\right)
\end{math}
\\
NB: NONMEM covariance matrix is used as input, the inverse covariance matrix
output by NONMEM is not used.
\item[\underline{Step 3}] evaluate the parameter vectors on the original data.
Create model files with up to 100 \$PROBLEM based on the original model file but setting MAXEVAL=0
and replacing inits of nth \$PROBLEM with nth parameter vector, and also set MCETA if option -mceta was used. Compute dOFV 
(delta-OFV, reference is input model ofv) and store in raw\_results\_1.csv.

\item[\underline{Step 4}] calculate the weights as the ratio $\frac{e^{-0.5\cdot dOFV}}{relPDF}$ and store in 
raw\_results\_1.csv. 
\item[\underline{Step 5}] Resample 'resamples\_1' parameter vectors based on weights from above step, 
with or without replacement depending on option -with\_replacement. 
Store the number of times each vector was resampled in raw\_results\_1.csv.
\item[\underline{Step 6}] Box-Cox transform each parameter separately based on the samples from Step 5. Do not allow
$\lambda$ less than -3 or larger than +3. Use the $\lambda$ that maximizes the correlation between the normal distribution and
the distribution of the transformed parameter. The  $\lambda$ used is allowed to differ from the true optimal  $\lambda$
by at most $0.2$.
\item[\underline{Step 7}] Determine the means and empirical variance-covariance matrix of 
transformed parameter vectors.
\item[\underline{Step 8}] (Repeat Step 1) Sample 'samples\_2' parameter vectors 
with the new covariance matrix and transformed parameters.
Discard samples that, after back-transformation to original scale, do not
fulfill \$THETA boundary conditions and \$OMEGA/\$SIGMA positive definiteness, and instead draw new sample.
\item[\underline{Step 9}] (Repeat Step 2) Calculate each vector's probability, 
use new covariance matrix and Box-Cox transformed parameter vectors for computing each vector's relPDF.
\item[\underline{Step 10}] (Repeat Step 3) Evaluate the sampled parameter vectors, in original scale (back-transformed from Box-Cox), on the original data.
\item[\underline{Step 11}] (Repeat Step 4) Calculate the weights, 
where weights calculation uses relPDF from step 9 and dOFV from Step 10. 
Store in raw\_results\_2.csv. 
\item[\underline{Step 12}] (Repeat Step 5) Resample 'resamples\_2' original scale parameter vectors based on weights from above step, 
with or without replacement depending on option -with\_replacement. 
Store the number of times each vector was resampled in raw\_results\_2.csv.
\item[\underline{Step 13}] Compute final results from Step 12 (parameter vectors on original scale).
\end{itemize}

\end{document}

\subsubsection*{setup}
\begin{enumerate}
\item Count items in parameter vector to get $N$.
\item Unless have quantile vector for $N$ stored from before:\\ Compute vector $Q$ for $i=1\ldots N$,
\begin{math}
\left(
Q_i=\Phi^{-1}\left(\frac{i-0.5}{N}\right)
\right)
\end{math}
using built-in Perl functions. Also compute $\frac{\sum_i{Q}}{N}$ and 
$\sqrt{\sum_i{Q^2}-\frac{\left(\sum_i{Q}\right)^2}{N}}$\\ Store results.
\item Sort parameter vector from smallest to largest to get vector $sorted$.
\item If $sorted[0]\leq 0$ then compute $\Delta=abs\left(sorted[0]+\delta\right)$ and add scalar $\Delta$ to $sorted$.
Store $\Delta$.
\item Evaluate $r(\lambda)$ for starting values for secant algorithm
\item Run secant algorithm until stops, find optimal value of $\lambda$ and corresponding transformed sorted positive
vector. Store $\lambda$
\end{enumerate}

\subsubsection*{Box-Cox}
Must have $x>0$, so if any $p$ leq 0 then add $abs(p_{min})+\delta$ to all $p$ before Box-Cox.
\noindent
\begin{math}
\lambda = 0: x_{\lambda}=log(p)\\
\lambda\neq 0: x_{\lambda}=\frac{p^{\lambda}-1}{\lambda}
\end{math}

\begin{math}
\frac{dx}{d\lambda}, \lambda=0: undef\\
\frac{dx}{d\lambda}, \lambda\neq 0: \frac{d}{d\lambda}\frac{p^{\lambda}-1}{\lambda}=\frac{d}{d\lambda}\frac{e^{\lambda log{p}}-1}{\lambda}=
\frac{\lambda e^{\lambda log{p}}log{p} -(e^{\lambda log{p}}-1) }{\lambda^2}=\frac{1}{\lambda}\left(p^{\lambda}log{p} - \frac{p^{\lambda}-1}{\lambda}\right)
\end{math}

\subsubsection*{q-q-plot}
Form N pairs
\begin{math}
\left(
\Phi^{-1}\left(\frac{i-0.5}{N}\right), x_i
\right)
\end{math}
where $\Phi^{-1}$ is the inverse CDF of the normal density and $x_i$ denotes the $i$th sorted value
of the Box-Cox transformed data.
For inverse CDF use Perl Math::CDF::qnorm function.
Compare with Statistics::Distributions::udistr(1-z) where $z=\frac{i-0.5}{N}$

\subsubsection*{Pearson's r}
x and y do not need to have mean 0. x could Box-Cox transformed (g), y could be inverse CDF.
\begin{math}
r=\frac{\sum{xy}-\frac{\sum{x}\sum{y}}{N}}
{\sqrt{\sum{x^2}-\frac{\left(\sum{x}\right)^2}{N}}\sqrt{\sum{y^2}-\frac{\left(\sum{y}\right)^2}{N}}}
\end{math}
When y is inverse CDF then $\sum_iy_i$ is 0, giving 
\begin{math}
r=\frac{\sum{xy}}
{\sqrt{\sum{x^2}-\frac{\left(\sum{x}\right)^2}{N}}\sqrt{\sum{y^2}}}\\
=\frac{f(\lambda)}{g(\lambda)}
\end{math}

\begin{math}
f(\lambda)=\sum{xy}=\sum_i \left(y_i\frac{p_i^{\lambda}-1}{\lambda}\right)
\end{math}

\begin{math}
\frac{df(\lambda)}{d\lambda}
=\sum_i\left(y_i\frac{1}{\lambda}\left(p_i^{\lambda}log{p_i}-\frac{p_i^{\lambda}-1}{\lambda}\right)\right)
\end{math}


\begin{math}
g(\lambda)=
\left(\sum_i\left(\frac{p_i^{\lambda}-1}{\lambda}\right)^2-\frac{1}{N}\left(\sum_i \frac{p_i^{\lambda}-1}{\lambda}\right)^2  \right)^{0.5} 
\sqrt{\sum{y^2}}
\end{math}

\begin{math}
\frac{dg(\lambda)}{d\lambda}=
\sqrt{\sum{y^2}}\cdot
0.5\cdot\left(\sum_i\left(\frac{p_i^{\lambda}-1}{\lambda}\right)^2-\frac{1}{N}\left(\sum_i \frac{p_i^{\lambda}-1}{\lambda}\right)^2  \right)^{-0.5} 
\cdot\\
\left(
\sum_i2\left(\frac{p_i^{\lambda}-1}{\lambda} \right)\frac{1}{\lambda}\left(p_i^{\lambda}log{p_i}-\frac{p_i^{\lambda}-1}{\lambda}\right)
-\frac{2}{N}\left(\sum_i \frac{p_i^{\lambda}-1}{\lambda}\right) \left(\sum_i \frac{1}{\lambda}\left(p_i^{\lambda}log{p_i}-\frac{p_i^{\lambda}-1}{\lambda}\right) \right)
\right)
\end{math}

\begin{math}
=
\sqrt{\sum{y^2}}\cdot
\left(\sum_i\left(\frac{p_i^{\lambda}-1}{\lambda}\right)^2-\frac{1}{N}\left(\sum_i \frac{p_i^{\lambda}-1}{\lambda}\right)^2  \right)^{-0.5} 
\cdot\\
\left(
\sum_i\left(\frac{p_i^{\lambda}-1}{\lambda} \right)\frac{1}{\lambda}\left(p_i^{\lambda}log{p_i}-\frac{p_i^{\lambda}-1}{\lambda}\right)
-\frac{1}{N}\left(\sum_i \frac{p_i^{\lambda}-1}{\lambda}\right) \left(\sum_i \frac{1}{\lambda}\left(p_i^{\lambda}log{p_i}-\frac{p_i^{\lambda}-1}{\lambda}\right) \right)
\right)
\end{math}

\begin{math}
\frac{dr}{d\lambda}=\frac{f'g-fg'}{g^2}=g^{-2}\cdot\left(f'g-fg'\right)
\end{math}

\begin{multline}
\frac{dr}{d\lambda}=
\left(\sum_i\left(\frac{p_i^{\lambda}-1}{\lambda}\right)^2-\frac{1}{N}\left(\sum_i \frac{p_i^{\lambda}-1}{\lambda}\right)^2  \right)^{-1} 
\left(\sum{y^2}\right)^{-1}\cdot  \\
[
\left[
\sum_i\left(\frac{1}{\lambda}\left(p_i^{\lambda}log{p_i}-\frac{p_i^{\lambda}-1}{\lambda}\right)y_i\right)
\right]  \\
\left(\sum_i\left(\frac{p_i^{\lambda}-1}{\lambda}\right)^2-\frac{1}{N}\left(\sum_i \frac{p_i^{\lambda}-1}{\lambda}\right)^2  \right)^{0.5} 
\sqrt{\sum{y^2}}  \\
-
\left[\sum_i \left(y_i\frac{p_i^{\lambda}-1}{\lambda}\right)\right]\cdot\\
\sqrt{\sum{y^2}}\cdot
\left(\sum_i\left(\frac{p_i^{\lambda}-1}{\lambda}\right)^2-\frac{1}{N}\left(\sum_i \frac{p_i^{\lambda}-1}{\lambda}\right)^2  \right)^{-0.5} 
\cdot\\
\left[
\sum_i\left(\frac{p_i^{\lambda}-1}{\lambda} \right)\frac{1}{\lambda}\left(p_i^{\lambda}log{p_i}-\frac{p_i^{\lambda}-1}{\lambda}\right)
-\frac{1}{N}\left(\sum_i \frac{p_i^{\lambda}-1}{\lambda}\right) \left(\sum_i \frac{1}{\lambda}\left(p_i^{\lambda}log{p_i}-\frac{p_i^{\lambda}-1}{\lambda}\right) \right)
\right]
]
\end{multline}


\begin{multline}
\frac{dr}{d\lambda}=
\left(\sum_i\left(\frac{p_i^{\lambda}-1}{\lambda}\right)^2-\frac{1}{N}\left(\sum_i \frac{p_i^{\lambda}-1}{\lambda}\right)^2  \right)^{-0.5} 
\left(\sum_i{y_i^2}\right)^{-0.5}\cdot  \\
[
\left[
\sum_i\left(\frac{1}{\lambda}\left(p_i^{\lambda}log{p_i}-\frac{p_i^{\lambda}-1}{\lambda}\right)y_i\right)
\right]  \\
-
\left[\sum_i \left(y_i\frac{p_i^{\lambda}-1}{\lambda}\right)\right]\cdot\\
\left(\sum_i\left(\frac{p_i^{\lambda}-1}{\lambda}\right)^2-\frac{1}{N}\left(\sum_i \frac{p_i^{\lambda}-1}{\lambda}\right)^2  \right)^{-1} 
\cdot\\
\left[
\sum_i\left(\frac{p_i^{\lambda}-1}{\lambda} \right)\frac{1}{\lambda}\left(p_i^{\lambda}log{p_i}-\frac{p_i^{\lambda}-1}{\lambda}\right)
-\frac{1}{N}\left(\sum_i \frac{p_i^{\lambda}-1}{\lambda}\right) \left(\sum_i \frac{1}{\lambda}\left(p_i^{\lambda}log{p_i}-\frac{p_i^{\lambda}-1}{\lambda}\right) \right)
\right]
]
\end{multline}

\begin{math}
r=\left(\sum_i{x_i^2}-\frac{\left(\sum_i{x_i}\right)^2}{N}\right)^{-0.5} 
\left(\sum_i{y_i^2}\right)^{-0.5}\cdot
\left(\sum_i{y_ix_i}\right)
\end{math}

\begin{multline}
\frac{dr}{d\lambda}=
\left(\sum_i{x_i^2}-\frac{\left(\sum_i{x_i}\right)^2}{N}\right)^{-0.5} 
\left(\sum_i{y_i^2}\right)^{-0.5}\cdot\frac{1}{\lambda}\cdot  \\
[
\sum_i\left(y_ip_i^{\lambda}log{p_i}\right) -  
\sum_i\left(x_iy_i\right)
-\\
\frac{\left(\sum_i{y_ix_i}\right) \cdot
\left[
\sum_i \left(x_ip_i^{\lambda}log{p_i}\right) - \sum_i \left(x_i^2\right)
-\left( \sum_i \left(p_i^{\lambda}log{p_i}\right) -\sum_i \left(x_i\right) \right)\frac{\left(\sum_i{x_i}\right)}{N} 
\right]}
{\left(\sum_i{x_i^2}-\frac{\left(\sum_i{x_i}\right)^2}{N}\right)}
]
\end{multline}

\subsubsection*{Intermediate sums needed}
\begin{enumerate}
\item sum\_x: $\sum_i{x_i}$
\item sum\_xsquare: $\sum_i{x_i^2}$
%\item sum\_y: $\sum_i{y_i}$
\item sum\_ysquare: $\sum_i{y_i^2}$
\item sum\_xy: $\sum_i{y_ix_i}$
\item sum\_logterm: $\sum_i{p_i^{\lambda}log{p_i}}$
\item sum\_xlogterm: $\sum_i{x_ip_i^{\lambda}log{p_i}}$
\item sum\_ylogterm: $\sum_i{y_ip_i^{\lambda}log{p_i}}$

\end{enumerate}

\subsubsection*{secant method}
Algorithm secant method (p 366 mathematics handbook):\\
\begin{math}
\lambda_{n+1}=\lambda_n-
r^{'}\left(\lambda_{n}\right)\cdot\frac{\lambda_n-\lambda_{n-1}}{r^{'}\left(\lambda_n\right)-r^{'}\left(\lambda_{n-1}\right)}
\end{math}\\
Stop when either
\begin{enumerate}
\item $\lambda_{n+1}\geq\lambda_{max}$
\item $\lambda_{n+1}\leq-\lambda_{max}$
\item \begin{math}abs\left(r\left(\lambda_n\right)-r\left(\lambda_{n-1}\right)\right)\leq \delta_{r}
\end{math}
\item
\begin{math}
abs\left(r\left(\lambda_{n}\right)\cdot\frac{\lambda_n-\lambda_{n-1}}{r\left(\lambda_n\right)-r\left(\lambda_{n-1}\right)}\right)\leq\delta_{\lambda}
\end{math}
\end{enumerate}
After stopping criteria met, choose $\lambda$ that gave largest $r$.

\end{document}

\subsubsection*{General y}
\begin{math}
r=\frac{\sum{xy}-\frac{\sum{x}\sum{y}}{N}}
{\sqrt{\sum{x^2}-\frac{\left(\sum{x}\right)^2}{N}}\sqrt{\sum{y^2}-\frac{\left(\sum{y}\right)^2}{N}}}=\frac{f}{g}
\end{math}


\begin{math}
f(\lambda)=\sum{xy}-\frac{\sum{x}\sum{y}}{N}
=
\sum_i \left(y_i\frac{p_i^{\lambda}-1}{\lambda}\right) - \left(\sum_i\frac{p_i^{\lambda}-1}{\lambda}\right)\frac{\sum_iy_i}{N}
\end{math}

\begin{math}
\frac{df(\lambda)}{d\lambda}
=\sum_i\left(y_i\frac{1}{\lambda}\left(p_i^{\lambda}log{p_i}-\frac{p_i^{\lambda}-1}{\lambda}\right)\right)
-\left( \sum_i\frac{1}{\lambda}\left(p_i^{\lambda}log{p_i}-\frac{p_i^{\lambda}-1}{\lambda}\right)\right)
\frac{\sum_iy_i}{N} \\
\end{math}


\begin{math}
g(\lambda)=
\left(\sum_i\left(\frac{p_i^{\lambda}-1}{\lambda}\right)^2-\frac{1}{N}\left(\sum_i \frac{p_i^{\lambda}-1}{\lambda}\right)^2  \right)^{0.5} 
\sqrt{\sum{y^2}-\frac{\left(\sum{y}\right)^2}{N}}
\end{math}

\begin{math}
\frac{dg(\lambda)}{d\lambda}=
\sqrt{\sum{y^2}-\frac{\left(\sum{y}\right)^2}{N}}\cdot
0.5\cdot\left(\sum_i\left(\frac{p_i^{\lambda}-1}{\lambda}\right)^2-\frac{1}{N}\left(\sum_i \frac{p_i^{\lambda}-1}{\lambda}\right)^2  \right)^{-0.5} 
\cdot\\
\left(
\sum_i2\left(\frac{p_i^{\lambda}-1}{\lambda} \right)\frac{1}{\lambda}\left(p_i^{\lambda}log{p_i}-\frac{p_i^{\lambda}-1}{\lambda}\right)
-\frac{2}{N}\left(\sum_i \frac{p_i^{\lambda}-1}{\lambda}\right) \left(\sum_i \frac{1}{\lambda}\left(p_i^{\lambda}log{p_i}-\frac{p_i^{\lambda}-1}{\lambda}\right) \right)
\right)
\end{math}

\begin{math}
=
\sqrt{\sum{y^2}-\frac{\left(\sum{y}\right)^2}{N}}\cdot
\left(\sum_i\left(\frac{p_i^{\lambda}-1}{\lambda}\right)^2-\frac{1}{N}\left(\sum_i \frac{p_i^{\lambda}-1}{\lambda}\right)^2  \right)^{-0.5} 
\cdot\\
\left(
\sum_i\left(\frac{p_i^{\lambda}-1}{\lambda} \right)\frac{1}{\lambda}\left(p_i^{\lambda}log{p_i}-\frac{p_i^{\lambda}-1}{\lambda}\right)
-\frac{1}{N}\left(\sum_i \frac{p_i^{\lambda}-1}{\lambda}\right) \left(\sum_i \frac{1}{\lambda}\left(p_i^{\lambda}log{p_i}-\frac{p_i^{\lambda}-1}{\lambda}\right) \right)
\right)
\end{math}

\begin{math}
\frac{dr}{d\lambda}=\frac{f'g-fg'}{g^2}=g^{-2}\cdot\left(f'g-fg'\right)
\end{math}

\begin{multline}
\frac{dr}{d\lambda}=
\left(\sum_i\left(\frac{p_i^{\lambda}-1}{\lambda}\right)^2-\frac{1}{N}\left(\sum_i \frac{p_i^{\lambda}-1}{\lambda}\right)^2  \right)^{-1} 
\left(\sum{y^2}-\frac{\left(\sum{y}\right)^2}{N}\right)^{-1}\cdot  \\
[
\left[
\sum_i\left(\frac{1}{\lambda}\left(p_i^{\lambda}log{p_i}-\frac{p_i^{\lambda}-1}{\lambda}\right)y_i\right) -  
\left(\sum_i\frac{1}{\lambda}\left(p_i^{\lambda}log{p_i}-\frac{p_i^{\lambda}-1}{\lambda}\right)\right)\left(\sum_iy_i\right)\frac{1}{N}
\right]  \\
\left(\sum_i\left(\frac{p_i^{\lambda}-1}{\lambda}\right)^2-\frac{1}{N}\left(\sum_i \frac{p_i^{\lambda}-1}{\lambda}\right)^2  \right)^{0.5} 
\sqrt{\sum{y^2}-\frac{\left(\sum{y}\right)^2}{N}}  \\
-
\left[\sum_i \left(y_i\frac{p_i^{\lambda}-1}{\lambda}\right) - \left(\sum_i\frac{p_i^{\lambda}-1}{\lambda}\right)\frac{\sum_iy_i}{N}\right]\cdot\\
\sqrt{\sum{y^2}-\frac{\left(\sum{y}\right)^2}{N}}\cdot
\left(\sum_i\left(\frac{p_i^{\lambda}-1}{\lambda}\right)^2-\frac{1}{N}\left(\sum_i \frac{p_i^{\lambda}-1}{\lambda}\right)^2  \right)^{-0.5} 
\cdot\\
\left[
\sum_i\left(\frac{p_i^{\lambda}-1}{\lambda} \right)\frac{1}{\lambda}\left(p_i^{\lambda}log{p_i}-\frac{p_i^{\lambda}-1}{\lambda}\right)
-\frac{1}{N}\left(\sum_i \frac{p_i^{\lambda}-1}{\lambda}\right) \left(\sum_i \frac{1}{\lambda}\left(p_i^{\lambda}log{p_i}-\frac{p_i^{\lambda}-1}{\lambda}\right) \right)
\right]
]
\end{multline}

\begin{multline}
\frac{dr}{d\lambda}=
\left(\sum_i\left(\frac{p_i^{\lambda}-1}{\lambda}\right)^2-\frac{1}{N}\left(\sum_i \frac{p_i^{\lambda}-1}{\lambda}\right)^2  \right)^{-0.5} 
\left(\sum_i{y_i^2}-\frac{\left(\sum_i{y_i}\right)^2}{N}\right)^{-0.5}\cdot  \\
[
\left[
\sum_i\left(\frac{1}{\lambda}\left(p_i^{\lambda}log{p_i}-\frac{p_i^{\lambda}-1}{\lambda}\right)y_i\right) -  
\left(\sum_i\frac{1}{\lambda}\left(p_i^{\lambda}log{p_i}-\frac{p_i^{\lambda}-1}{\lambda}\right)\right)\frac{\left(\sum_iy_i\right)}{N}
\right]  \\
-
\left[\sum_i \left(y_i\frac{p_i^{\lambda}-1}{\lambda}\right) - \left(\sum_i\frac{p_i^{\lambda}-1}{\lambda}\right)\frac{\left(\sum_iy_i\right)}{N}\right]\cdot\\
\left(\sum_i\left(\frac{p_i^{\lambda}-1}{\lambda}\right)^2-\frac{1}{N}\left(\sum_i \frac{p_i^{\lambda}-1}{\lambda}\right)^2  \right)^{-1} 
\cdot\\
\left[
\sum_i\left(\frac{p_i^{\lambda}-1}{\lambda} \right)\frac{1}{\lambda}\left(p_i^{\lambda}log{p_i}-\frac{p_i^{\lambda}-1}{\lambda}\right)
-\frac{1}{N}\left(\sum_i \frac{p_i^{\lambda}-1}{\lambda}\right) \left(\sum_i \frac{1}{\lambda}\left(p_i^{\lambda}log{p_i}-\frac{p_i^{\lambda}-1}{\lambda}\right) \right)
\right]
]
\end{multline}

\newpage

\begin{math}
r=\left(\sum_i{x_i^2}-\frac{\left(\sum_i{x_i}\right)^2}{N}\right)^{-0.5} 
\left(\sum_i{y_i^2}-\frac{\left(\sum_i{y_i}\right)^2}{N}\right)^{-0.5}\cdot
\left(\sum_i{y_ix_i} - \frac{\left(\sum_ix_i\right)\left(\sum_iy_i\right)}{N}\right)
\end{math}

\begin{multline}
\frac{dr}{d\lambda}=
\left(\sum_i{x_i^2}-\frac{\left(\sum_i{x_i}\right)^2}{N}\right)^{-0.5} 
\left(\sum_i{y_i^2}-\frac{\left(\sum_i{y_i}\right)^2}{N}\right)^{-0.5}\cdot\frac{1}{\lambda}\cdot  \\
[
\left[
\sum_i\left(y_ip_i^{\lambda}log{p_i}\right) -  
\sum_i\left(x_iy_i\right) -  
\left( \sum_i\left(p_i^{\lambda}log{p_i}\right) - \sum_i\left(x_i\right)  \right)\frac{\left(\sum_iy_i\right)}{N}
\right]  \\
-
\left(\sum_i{y_ix_i} - \frac{\left(\sum_ix_i\right)\left(\sum_iy_i\right)}{N}\right)\cdot
\left(\sum_i{x_i^2}-\frac{\left(\sum_i{x_i}\right)^2}{N}\right)^{-1} 
\cdot\\
\left[
\sum_i \left(x_ip_i^{\lambda}log{p_i}\right) - \sum_i \left(x_i^2\right)
-\left( \sum_i \left(p_i^{\lambda}log{p_i}\right) -\sum_i \left(x_i\right) \right)\frac{\left(\sum_i{x_i}\right)}{N} 
\right]
]
\end{multline}

\end{document}
