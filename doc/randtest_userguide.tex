\documentclass[a4wide,12pt]{article}
%\setlength{\marginparwidth}{0pt}%35
%\setlength{\marginparsep}{0pt}%?
%\setlength{\evensidemargin}{0pt}
%\setlength{\oddsidemargin}{0pt}
\usepackage{lmodern}
\usepackage[utf8]{inputenc}
\usepackage[T1]{fontenc}
\usepackage{textcomp}
\usepackage{verbatim}
\usepackage{enumitem}
\usepackage{longtable}
\usepackage{alltt}
\usepackage{ifthen}
% Reduce the size of the underscore
\usepackage{relsize}
\renewcommand{\_}{\textscale{.7}{\textunderscore}}

\newcommand{\guidetitle}[1]{
\title{#1\\ \vspace{2 mm} {\large PsN 4.1.1}}
\date{2014-02-10}
}

\newcommand{\doctitle}[1]{
\title{#1}
\date{2014-02-10}
}


\newenvironment{optionlist}{
\renewcommand{\arraystretch}{1.1}
\setlength{\leftmargini}{2.5cm}
\begin{description}
%\setlength{\itemsep}{0ex}
}
{\end{description}}

\newcommand{\optname}[1]{\item{{\bfseries\texttt-#1}\newline}}
\newcommand{\optdefault}[2]{\item{{\bfseries\texttt-#1}{\mbox{ = \it #2}}\newline}}

\newcommand{\nextopt}{}

\guidetitle{RANDTEST user guide}{2015-09-16}

\begin{document}

\maketitle
\newcommand{\guidetoolname}{randtest}


\section{Introduction}

Randtest is a tool for computing actual significance levels \cite{Wahlby}, \cite{Deng}. 
The program will shuffle the values in the randomization 
column 'samples' times and run the model with each of the new datasets. The results will be summarized in the csv-format 
raw\_results-file and randtest\_results.csv.
Example command
\begin{verbatim}
randtest run1.mod -samples=1000 -randomization_col=DGRP -base_model=run0.mod
\end{verbatim}

\section{Input and options}
\subsection{Input specific to randtest}
A model file for the full model is required on the command-line.
\begin{optionlist}
\optdefault{samples}{N}
The number of randomized datasets to generate, required. 
\nextopt
\optdefault{randomization\_column}{col}
The name of the column to randomize, required. The column name is taken from the model file's \$INPUT record. Column names in the data file are ignored. 
\nextopt
\optdefault{stratify\_on}{column}
Optional, default not used. Restrict randomization of the data to within stratification groups, so that individuals in one stratification group cannot be assigned randomization column values from another stratification group. 
\nextopt
\optdefault{base\_model}{filename}
Optional. Name of model to estimate with original data and use as reference when computing delta-ofv. 
\nextopt
\optname{match\_transitions}
Not used by default. Method for copying randomization column values from one individual to another during shuffling. See details in section Shuffling the randomization column.     
\nextopt
\optname{full\_model\_inits}
Default false. Only relevant when -update\_inits is true. If true, use final estimates from
the full model as initial estimates for the randomized data models. If false,
final estimates from the base model are used instead.
\nextopt
\optname{update\_inits}
Default true. Update the initial estimates of the 
full model to the final estimates from the estimation (lst-file) of either the base model,
or, if option -full\_model\_inits is set, of the full model.
If estimates from the base model is used (option -full\_model\_inits is not set)
only update for those parameters that are estimated (non-fix) in the base model AND are found,
based on THETA/OMEGA/SIGMA numbering, both in the base and full model.
\nextopt
\optname{copy\_data}
Default set. If option is set, the data file
will be copied to the run directory if the input and/or base model is run.
If option is unset using -no-copy\_data, the absolute path to the original data file(s) will be used in
\$DATA, and the data file will not be copied. This saves disk space.
\nextopt
\end{optionlist}

\subsection{Some important common PsN options}
For a complete list see common\_options.pdf, 
or psn\_options -h on the commandline.
\begin{optionlist}
\optname{h or -?}
Print the list of available options and exit. 
\nextopt
\optname{help}
With -help all programs will print a longer help message. 
If an option name is given as argument, help will be printed for this option. 
If no option is specified, help text for all options will be printed. 
\nextopt
\optdefault{directory}{'string'}
Default \guidetoolname\_dirN,
where N will start at 1 and
be increased by one each time you run the script. The directory option sets the directory in which PsN 
will run NONMEM and where PsN-generated output files will be stored. 
You do not have to create the directory,  it will be done for you. If you set
-directory to a the name of a directory that already exists, PsN will run in the existing directory.
\nextopt
\optdefault{seed}{'string'}
You can set your own random seed to make PsN runs reproducible.
The random seed is a string, so both -seed=12345 and -seed=JustinBieber are valid.
It is important to know that because of the way the Perl pseudo-random
number generator works, for two similar string seeds the random sequences may be identical. 
This is the case e.g. with the two different seeds 123 and 122. 
Setting the same seed guarantees the same sequence, but setting two slightly different 
seeds does not guarantee two different random sequences, that must be verified.
\nextopt
\optdefault{clean}{'integer'}
Default 1. The clean option can take four different values:  
\begin{description}
\item[0] Nothing is removed 
\item[1] NONMEM binary and intermediate files except INTER are removed, and files specified with option -extra\_files. 
\item[2] model and output files generated by PsN restarts are removed, and data files in the NM\_run directory, and (if option -nmqual is used) the xml-formatted NONMEM output. 
\item[3] All NM\_run directories are completely removed. If the PsN tool has created modelfit\_dir:s inside the main run directory, these  will also be removed. 
\end{description}
\nextopt
\optdefault{nm\_version}{'string'}
Default is 'default'. 
If you have more than one NONMEM version installed you can use option
-nm\_version to choose which one to use, as long as it is 
defined in the [nm\_versions] section in psn.conf, see psn\_configuration.pdf for details. 
You can check which versions are defined, without opening psn.conf, using the command
\begin{verbatim}
psn -nm_versions
\end{verbatim}
\nextopt
\optdefault{threads}{'integer'}
Default 1. Use the threads option to enable parallel execution of multiple models.
This option decides how many models PsN will run at the same time, and it is completely
independent of whether the individual models are run with serial NONMEM or parallel NONMEM.
If you want to run a single model in parallel you must use options -parafile and -nodes.
On a desktop computer it 
is recommended to not set -threads higher the number of CPUs in the system plus one. 
You can specify more threads, 
but it will probably not increase the performance. If you are running on a computer cluster, 
you should consult your 
system administrator to find out how many threads you can specify. 
\nextopt
\optname{version}
Prints the PsN version number of the tool, and then exit. 
\nextopt
\end{optionlist}

\begin{optionlist}
\optdefault{retries}{N}
Default 0. The number of times to try running each sample with slightly perturbed initial estimates if the first attempt is not successful.
The randtest program will, if -base\_model is used, check that the ofv of each sample is at least as good as the ofv of
the base model, within 'accepted\_ofv\_difference'. If not then a retry is initiated. The default value of the number of
retries is 0, so to enable the retry feature for randtest the option -retries must be set larger than 0. 
\nextopt
\optdefault{accepted\_ofv\_difference}{number}
Default 0.5. A retry will not be initiated if the ofv of the current sample is only 'accepted\_ofv\_difference' worse than
the base model ofv.
\nextopt
\end{optionlist}


\subsection{Auto-generated R-plots from PsN}
\newcommand{\rplotsconditions}{The default randtest template 
requires that option -base\_model was used, and also 
that R libraries
ggplot2, reshape2, gridExtra, scales, MASS and plotrix are installed.
It is assumed that input (full) model has more THETAs than base (reduced) model,
and that the additional THETAs are the ones relevant for the
randomization column.
If the conditions are not fulfilled then no pdf will be generated,
see the .Rout file in the main run directory for error messages.}
PsN can automatically generate R plots to visualize results for \guidetoolname, using a default template found in the R-scripts subdirectory of the installation directory. The user can also create a custom template, see more details in the section Auto-generated R-plots from PsN in common\_options.pdf.

\rplotsconditions

\begin{optionlist}
\optdefault{rplots}{level}
-rplots<0 means R script is not generated\\ 
-rplots=0 (default) means R script is generated but not run\\ 
-rplots=1 means basic plots are generated\\													  
-rplots=2 means basic and extended plots are generated\\													  
\nextopt
\end{optionlist}

\subsubsection*{Troubleshooting}
If no .pdf was generated even if a template file is available and the appropriate options were set, check the .Rout-file in the main run directory for error messages. If no .Rout-file exists, then check that R is properly installed, and that either command 'R' is available or that R is configured in psn.conf.


\subsubsection*{Basic plots}
Basic rplots will be generated if option -rplots is set >0,
and the general rplots conditions fulfilled, see above.
The basic plots are
histograms of delta-ofv and the extra THETAs in the input (full) model.
Panel 1 excludes all positive valued deltaOFV (if any) 
and then plots the distribution of deltaOFV. 
Panel 2 converts all positive valued deltaOFV (if any) to 0, 
hence modified deltaOFV or mod.delOFV, 
and then plots the distribution of mod.deltaOFV. 
So, if there is no positive deltaOFV, these two panels are the same.
Panel 3 and above are histograms of the extra THETAs in the input (full) model.
\subsubsection*{Extended plots}
The extended plots, will be generated if -rplots>1, and are on the
second page of the pdf-output.

Top left panel: comparison of the cumulative distribution functions (CDF) 
of the empirical distribution (absolute values of deltaOFV after randtest) in 
orange and the theoretical distribution (Chi-sq distribution closest to the 
empirical distribution) in olive green. The vertical dashed blue and solid 
green lines represent 95 percentile cutoff for the empirical and theoretical 
distributions, respectively. The vertical dashed purple line shows the absolute
value of the true deltaOFV.

Bottom left panel: Comparison of the probability distribution functions (PDF) 
of the empirical distribution (absolute values of deltaOFV after randtest) in 
red and the theoretical distribution (Chi-sq distribution closest to the 
empirical distribution) in blue. The vertical dashed red and blue lines 
represent 95 percentile cutoff for the empirical and theoretical distributions,
respectively.

The top and bottom panels on the right represent the similar comparisons for 
the modified deltaOFV (mod.deltaOFV).

\section{Shuffling the randomization column}

The only column that will change during shuffling is the randomization column. All other columns will be intact, and the order and numbering of individuals will be preserved. Shuffling can only be done on the level of individuals. If stratification is requested via option -stratify\_on, then shuffling will be done separately within the groups defined by unique values in the stratification column. PsN will only look at the value of the stratification column for the first record of each individual. If other records for the same individual have a different stratification value, PsN will print a warning and then ignore the new value.
During shuffling the sequence of randomization column values from the records of one individual will be copied to another individual. PsN will handle copying also when the number of data set records differs between individuals.

PsN will not filter the data set based on any IGNORE/ACCEPT statements in the NONMEM model file before randomization. This means that
during shuffling the
values in the randomization column of an IGNOREd data record can be copied to a record that is not IGNOREd, and vice versa.

\subsection{If -match\_transitions is not set, the default}

By default, when option -match\_transitions is not set, PsN will copy values record by record. If the sequence of 'copy-from' values is shorter than 'copy-to' then PsN will do 'last-observation-carry-forward'. Example: 
Copying [0,1,1] to [0,0,2,2] will give result [0,1,1,1] when -match\_transitions is not set. Missing data marked by e.g. -99 or values that are '.' (a dot) will be handled like any other value. 

\subsection{If match\_transitions is set}

If option -match\_transitions is set, PsN will match transitions instead of records in the sequence of values. The 'copy-from' sequence determines the values after each transition, while the 'copy-to' sequence determines where the transitions should be. Example: 
\begin{itemize}
\item Copying [0,1,1] to [0,0,2,2] will give result [0,0,1,1], i.e. the resulting sequence has 0 before the first transition and 1 after the first transition. 
\end{itemize}

\noindent If the 'copy-from' sequence has fewer transitions than 'copy-to' then PsN will do 'last-observation-carry-forward'. Example:
\begin{itemize}
\item Copying [0,1,1,1] to [0,0,2,3] will give result [0,0,1,1]. 
\end{itemize}
\noindent If the dataset has missing data marked with something numeric, e.g. -99, then going from an observation to -99 will not be counted as a transition, and -99 will not be copied. Examples:
\begin{itemize}
\item Copying [1,-99,1,0] to [0,0,2,2] will give result [1,1,0,0]. 
\item Copying [1,1,1,0] to [0,-99,2,2] will give result [1,1,0,0].
\item Copying [-99,1,1,0] to [0,0,2,2] will give result [1,1,0,0]. 
\end{itemize}

\noindent Non-numeric values, e.g. . (a dot) will be lumped together and treated as any non-missing value, so going from an observation to a dot will be considered a transition. Example:
\begin{itemize}
\item Copying [0,0,1,1] to  [.,0,0,0] will give result [0,1,1,1] 
\end{itemize}

\section{Output}

The raw\_results file is almost a standard PsN file containing raw result data for termination status, parameter estimates, uncertainty estimates etc. for all model estimations. The first row is for the base model, if any, estimated with the original dataset. The next row is the input model with the original dataset. The following are the input model with the randomized datasets. If a base model was given an extra column with header deltaofv is inserted next to the ofv-column. This column contains the ofv-value of the current model minus ofv for the base model.

If a base-model was given, a file randtest\_results.csv will also be produced, 
containing summary information about the empirical distribution of the
delta-ofv values.


\section{Known bugs and problems}

It is recommented to remove all \$TABLE from the modelfile, otherwise there will be much extra output produced. For the same reason it is recommended to remove PRINT options from \$ESTIMATION. 
The results of two runs will be different even if the seed is the same if the lst-file of the input model and base modle are present at the start of one run but not the other. Running the input model changes the state of the random number generator, and therefore the datasets will be different depending on if the input model is run or not before generating the  new datasets.

\section{Technical overview of algorithm}

PsN will rerun the input (full) model if the lst-file of the input model (run1.lst if the input model file is called run1.mod) is not present in the same directory as the model file. 
If the model file has the extension .mod and the lst-file is present then PsN will simply read the estimates from that file. Similar for the base model.
The program creates N (N=number of samples), datasets of size M (M=sample\_size) by shuffling the values in the randomization column between individuals. The program creates N new NONMEM modelfiles which are identical to the original modelfile with the exception that each uses a different randomized dataset and that the initial parameter estimates are the final estimates from the base model run. The model parameters are estimated with each dataset.


\begin{thebibliography}{99}
\bibitem{Wahlby}  U .Wählby U, E. N. Jonsson and M. O. Karlsson, 
\emph{Assessment of Actual Significance levels for Covariate Effects in NONMEM}, 
J PKPD 28(3):231-252. 
\bibitem{Deng} C. Deng, E. L. Plan and M. O. Karlsson, {\em Influence of clinical trial design to detect drug effect in systems with within subject variability}, PAGE 24 (2015) Abstr 3549 \mbox{www.page-meeting.org/?abstract=3549} 
\end{thebibliography}


\end{document}
