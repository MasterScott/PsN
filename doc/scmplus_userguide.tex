\documentclass[a4wide,12pt]{article}
%\setlength{\marginparwidth}{0pt}%35
%\setlength{\marginparsep}{0pt}%?
%\setlength{\evensidemargin}{0pt}
%\setlength{\oddsidemargin}{0pt}
\usepackage{lmodern}
\usepackage[utf8]{inputenc}
\usepackage[T1]{fontenc}
\usepackage{textcomp}
\usepackage{verbatim}
\usepackage{enumitem}
\usepackage{longtable}
\usepackage{alltt}
\usepackage{ifthen}
% Reduce the size of the underscore
\usepackage{relsize}
\renewcommand{\_}{\textscale{.7}{\textunderscore}}

\newcommand{\guidetitle}[1]{
\title{#1\\ \vspace{2 mm} {\large PsN 4.1.1}}
\date{2014-02-10}
}

\newcommand{\doctitle}[1]{
\title{#1}
\date{2014-02-10}
}


\newenvironment{optionlist}{
\renewcommand{\arraystretch}{1.1}
\setlength{\leftmargini}{2.5cm}
\begin{description}
%\setlength{\itemsep}{0ex}
}
{\end{description}}

\newcommand{\optname}[1]{\item{{\bfseries\texttt-#1}\newline}}
\newcommand{\optdefault}[2]{\item{{\bfseries\texttt-#1}{\mbox{ = \it #2}}\newline}}

\newcommand{\nextopt}{}

\guidetitle{scmplus user guide}{2020-09-15}

\begin{document}

\maketitle
\newcommand{\guidetoolname}{scmplus}
\tableofcontents
\newpage


\section{Overview}
The scmplus program is a wrapper around the scm script that reduces the number of times
that insignificant parameter-covariate relations are retested. The insignificant runs are often
the most unstable and difficult to estimate, and they take a considerable amount of computing time.
The method used is called Adaptive Scope Reduction.

\subsection{Overview of basic scm algorithm}

The basic scm algorithm proceeds in a Forward inclusion phase and a
Backward elimination phase. In the forward phase each parameter-covariate (par-cov) combination 
in the scope are tested one
at a time. The par-cov that yields the largest drop in OFV, provided it is
significant, is retained in the model. The remaining par-cov are then
tested again in the updated model and the one with the largest drop in
OFV (assuming significance) is retained. This stepwise inclusion
proceeds until no more significant par-cov can be identified. The
Backward elimination removes par-cov from the Forward model in a
stepwise fashion at a stricter significance level. Once all remaining par-cov 
are significant, the final model has been established.


\subsection{Summary of improvements in scmplus}
The scmplus algorithm increases the efficiency of the legacy scm
algorithm through the following changes:
\subsubsection{Adaptive scope reduction} 
Parameter-covariate relations that have
not shown promise in early steps are not tested in later steps. However,
before proceeding to the backward elimination phase, the 
parameter-covariate relations are re-tested. 

Specifically, after each step
set with scmplus option -scope\_reduction\_steps, scmplus will stash all relations that are insignificant
according to option -p\_cutoff. Option -keep\_local\_min can be used to further control which relations
are stashed.
When the reduced scope forward search is finished, either because all relations have been included or
because no more relations are significant, scmplus will retest the stashed relations provided that option
-retest\_stashed\_relations is set. If any of the retested relations is found to be significant,
the forward inclusion phase will proceed as with the regular scm.

\subsubsection{Limit the number of function evaluations}
To avoid excessive
iterations without improvement in OFV, the max no of function
evaluations are limited (default to 3.1 times the function evaluations
used in the base model). This feature is invoked with option -maxevals.
\subsubsection{Only run NONMEM estimation step}
By default, scmplus will drop \$COVARIANCE and \$TABLE from the model.
Options -keep\_covariance and -keep\_tables can be used to turn off this feature.
\subsubsection{Alternative termination criteria}
Since the scm is only concerned with
changes in OFV, the CTYPE=4 criteria is used instead of the default.
This feature is invoked with option -ctype4.
\subsubsection{Appropriate numerical settings}
With ADVAN 6, 8, 9 and 13, scmplus requires SIGL to be set.
This feature can be turned off with option -ignore\_no\_sigl.


\section{Input and options}

The scmplus program depends on the scm tool of PsN, and all scm options apply also to scmplus,
with the exception of -linearize.
Please refer to scm\_userguide.pdf for help on scm options.

Example:
\begin{verbatim}
scmplus run101.scm -scope_reduction_steps=1,2,3 -p_cutoff=0.05
\end{verbatim}

\subsection{Required input}
An scm configuration file is required on the command-line. The format of the configuration file follows the format of 
the scm configuration file exactly. 
The input model must be set in the configuration file, it cannot be given on the scmplus commandline. 

No other option is required.

\subsection{Optional input}
These options are specific to scmplus, and they can only be given on the command-line, not in the
scm configuration file.

\begin{optionlist}
\optdefault{scope\_reduction\_steps}{comma-separated list of integers}
Comma-separated list of steps after which to perform scope reduction. Default 1 (after first step only).
To perform scope reduction after every forward step set -scope\_reduction\_steps=all 
To never perform scope reduction set -scope\_reduction\_steps=none 
\nextopt
\optname{retest\_stashed\_relations}
Default set. Unset with -no-retest\_stashed\_relations.
If set then parameter-covariate relations that were stashed
in scope reduction will be retested at the end of the forward search.
\nextopt
\optdefault{p\_cutoff}{number}
Cutoff p-value in scope reduction. Default is equal to p\_forward,
which in turn has default value 0.05 if not set in the scm config file.
\nextopt
\optname{keep\_local\_min}
Default set. Unset with -no-keep\_local\_min. If set then candidate models that
terminated with local minimum are kept in scope reduction.
\optname{keep\_covariance}
By default the \$COVARIANCE record will be deleted from the models, to save run time.
If option -keep\_covariance is set, the \$COVARIANCE record will be kept.
\nextopt
\optname{keep\_tables}
By default all \$TABLE will be deleted from the models.
If option -keep\_tables is set, all \$TABLE will be kept.
\nextopt
\optdefault{maxevals}{number}
Default maxevals in scmplus. 
If set to a decimal number, and the estimation method is classical,
\$EST MAXEVALS will be set to that number times 
the actual number of function evaluations in the base model,
rounded down to the nearest integer and capped at 9999. 
If set to an integer smaller than 10000 will be interpreted as the new value
of \$EST MAXEVALS. A warning will be printed if the new integer value is 
less than or equal to 10. 
If set to an integer equal to or larger than 9999 then this option
will be ignored by scmplus and passed on to common\_options maxevals
(see documentation for common options).
\optname{ctype4}
Default set, unset with -no-ctype4.
If -ctype4 is set, and a classical estimation method is used,
scmplus will ensure CTYPE=4 is set in \$EST
\nextopt
\optname{ignore\_no\_sigl}
Default not set.
If not set, scmplus will stop with an error message if
SIGL is not set in \$EST and ADVAN 6,8,9 or 13 is used.
If set, and SIGL is not set, a warning will be printed but scmplus will run.
\nextopt
\optname{fast}
Default not set. 
If -fast is set, the scmplus defaults for options -ctype4, -retest\_stashed\_relations,
-scope\_reduction\_steps and -maxevals will change to -ctype4 set, 
-retest\_stashed\_relations unset, -scope\_reduction\_steps=all and -maxevals=3.1\\ 
If any of those options are set individually, the individual setting will override the
-fast defaults.
\nextopt
\optname{etas}
Default not set. 
If -etas is set, and a classical estimation method is used, scmplus will ensure record \$ETAS is used
in the model, that FILE is set to the phi-file of the base model of the current iteration,
and that MCETA in \$EST is set to at least 1.
\nextopt
\optdefault{base\_ofv}{number}
Only applicable when included\_relations is set in the scm config file.
Use this value as the ofv of the base model with included relations,
i.e. do not run the base model with included relations but use
this value instead.
\nextopt
\optname{setup\_only}
Default not set. 
If -setup\_only is set, scmplus will setup everything but not start the scm run.
\nextopt
\end{optionlist}


\end{document}

\section{Setup steps}

\subsection{Filter data}
No estimation or maxeval, only \$TABLE step. Ensure no ACCEPT or IGNORE needed.

\subsection{Update model}
\begin{itemize}
\item Update \$DATA to filtered file and remove ACCEPT, IGNORE
\item Drop \$COV.
%\item Remove poorly supported ETAs
\item Warn if OMEGA blocks.
\item Warn if IOV.
\item Warn if eta on epsilon, AR1, L2, etc
\item Set MAXEVAL=1000 (or lower). After that number of evaluations the OFV will be very close to the minima.
\item Set CTYPE=4. It will detect when the OFV hasn’t changed for many iterations. OFV are what we are basing our decisions on.
%\item Make sure boundaries are as tight as possible.
\item Warn if option retries is on
\end{itemize}

\section{config file}

\subsection{start of script}
\begin{itemize}
\item Parse using standard code
\item run extra input checks
\end{itemize}

\subsection{after each sub-run}
parse log-file. Get set of included relations after final step in run. get path to model+lst in m1 or final models

pass file reference, scan using set of functions? fields can be empty (significant YES! or empty).
split on '>' first
split on whitespace. check header
grep FAILED what does it mean? save for later

check example backward search

{\tiny
\begin{verbatim}
MODEL            TEST     BASE OFV     NEW OFV         TEST OFV (DROP)    GOAL     dDF    SIGNIFICANT PVAL
CLAPGR-2         PVAL    742.05105   721.58702             20.46402  >  18.30700   10        YES!  0.025157 
CLWGT-2          PVAL    742.05105    FAILED                 FAILED  >   3.84150    1                    999
VAPGR-2          PVAL    742.05105   726.38092             15.67013  >  18.30700   10              0.109470 
VWGT-2           PVAL    742.05105    FAILED                 FAILED  >   3.84150    1                    999

Parameter-covariate relation chosen in this forward step: CL-APGR-2
CRITERION              PVAL < 0.05
BASE_MODEL_OFV         742.05105
CHOSEN_MODEL_OFV       721.58702
Relations included after this step:
CL      APGR-2  
V       
--------------------
\end{verbatim}
}

\subsection{Before each sub-run}
If any included relations then look up included-relations-base-model.lst. If not exist check option base\_ofv
If exist then parse lst. Get ofv, final estimates, upper lower bounds, number of iterations
Update/add new entry in config file object for inits, lowerbounds, upperbounds for each included relation.

Beware of wildcards. Wildcards are replaced by all valid alternatives by config\_file.pm.
Probably ok to first set wildcard and then on following line set exception, but
not in opposite order. If working with object then just set new value.

Test: hockey-stick (two-theta relation) inits and bounds.
override wildcard inits and bounds.
Use data-dumper for config file?

labeling of relation thetas in code is \verb|$parameter.$covariate.$i| where i is 1 to number
of thetas in that parcov relation.
do update\_inits first to get final estimates, then
Always the last thetas, so loop from beginning and make hash name->upper,lower,init.
In model->new we have ensure\_unique\_labels when parsing new file. If so then drop underscore in hash

in logfile the model column is parameter.covariate-state
filename is parameter.covariate.state.'.mod'
number of thetas is a field in relations hash, just like inits and bounds
\begin{verbatim}
$self -> relations->{$par}{$cov}{'nthetas'}{$state},
$self -> relations->{$par}{$cov}{'inits'}{$state},
$self -> relations->{$par}{$cov}{'bounds'}{$state} 
\end{verbatim}

change in logfile format since 4.7.0? diff

\section{Types of scm sub-runs}


\subsection{Single iteration scm sub-run}

\paragraph{-max\_steps=1}
Only run one iteration even if additional relations left to test after
first iteration.

\paragraph{-base\_ofv=X}
Set base ofv of base model so that included relations model not rerun.

\paragraph{-parallel\_states}
If multiple states on par-cov combo then test all in parallel.



\subsection{Full iteration scm sub-run}



\section{Reduce scope}\label{}

\subsection{Q: Is next step a scope reduction step}

\end{document}
