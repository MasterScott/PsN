\documentclass[a4wide,12pt]{article}
%\setlength{\marginparwidth}{0pt}%35
%\setlength{\marginparsep}{0pt}%?
%\setlength{\evensidemargin}{0pt}
%\setlength{\oddsidemargin}{0pt}
\usepackage{lmodern}
\usepackage[utf8]{inputenc}
\usepackage[T1]{fontenc}
\usepackage{textcomp}
\usepackage{verbatim}
\usepackage{enumitem}
\usepackage{longtable}
\usepackage{alltt}
\usepackage{ifthen}
% Reduce the size of the underscore
\usepackage{relsize}
\renewcommand{\_}{\textscale{.7}{\textunderscore}}

\newcommand{\guidetitle}[1]{
\title{#1\\ \vspace{2 mm} {\large PsN 4.1.1}}
\date{2014-02-10}
}

\newcommand{\doctitle}[1]{
\title{#1}
\date{2014-02-10}
}


\newenvironment{optionlist}{
\renewcommand{\arraystretch}{1.1}
\setlength{\leftmargini}{2.5cm}
\begin{description}
%\setlength{\itemsep}{0ex}
}
{\end{description}}

\newcommand{\optname}[1]{\item{{\bfseries\texttt-#1}\newline}}
\newcommand{\optdefault}[2]{\item{{\bfseries\texttt-#1}{\mbox{ = \it #2}}\newline}}

\newcommand{\nextopt}{}

\guidetitle{NCA user guide}

\begin{document}

\maketitle

\section{Introduction}
The PsN NCA can feed the nca R-script\cite{Chayan} with simulated data. It is based on the simulation part of the vpc/npc.

\section{Running NCA}
Example:

\begin{verbatim}
nca run1.mod -samples=1000
\end{verbatim}

Input: a modelfile is required.

\begin{optionlist}
\optdefault{samples}{number}
The number of simulated datasets to generate. This option in mandatory and must be 20 or higher. 
\nextopt
\optdefault{msfo}{file}
is optional, unless the model has \$NONP record then an msfo-file	is required. Use final parameter estimates from msfo-file instead of initial estimates from modelfile when simulating.
\nextopt
\optdefault{lst}{file}
is optional, but forbidden together with -msfo. Use final parameter estimates from this lst-file for the simulation. By default PsN will look for a file with the same name as the regular input model, or the simulation model if option -sim\_model is used, but with suffix .lst instead of .mod. If such a file is found then option -lst=$<$modelfile$>$.lst is set automatically.
\nextopt
\optdefault{columns}{var1,var2,...}
is optional. The columns option is a comma-separated list of variable names. All listed variables will be added to the nca table output.
\nextopt
\optdefault{sim\_model}{file}
is optional. Cannot be used together with -flip\_comments, -keep\_estimation or -noprediction. By default PsN will create a simulation model based on the required input model, but by using option -sim\_model it is possible to use a separate input model for the simulations. PsN will remove \$COV and \$TABLE, change SEED and NSUBS in \$SIM, add a new \$TABLE and  update initial estimates if option -lst is set or add \$MSFI if option -msfo is used, but otherwise no changes will be made to the user defined simulation model. See section Modified models. Note that -lst will be set automatically if a file with the same name as the regular input model but with suffix lst intead of mod is found. 
\nextopt
\optname{flip\_comments}
is optional. Cannot be used together with -sim\_model, -keep\_estimation or -noprediction. By default PsN will create a simulation model based on the required input model, but option -flip\_comments invokes a method for handling user-defined simulation code in the required input model. If option is set, PsN will create the simulation model by flipping comments (commented lines will be uncommented and vice versa) between the tags


\begin{verbatim}
;Sim_start
\end{verbatim}
and 
\begin{verbatim}
;Sim_end
\end{verbatim}
For example, if the required input model has lines
\begin{verbatim}
;Sim_start 
IGNORE(TYPE.EQ.1)
;ACCEPT(TYPE.EQ.1) 
;Sim_end
\end{verbatim}
then the MAXEVAL=0 model will be run as such and the simulation model will instead have lines
\begin{verbatim}
;IGNORE(TYPE.EQ.1)
ACCEPT(TYPE.EQ.1) 
\end{verbatim}
The tags may appear multiple times. Note that the tags must look exactly  as above or the editing will fail. When creating the simulation model PsN will remove \$COV and \$TABLE, change SEED and NSUBS in \$SIM, add a new \$TABLE and  update initial estimates if option -lst is set or add \$MSFI if option -msfo is used, but otherwise no changes will be made to the code. See section Modified models.\\
\nextopt
\optdefault{dv}{variable}
is optional, default is DV. If a synonym for DV is set in \$INPUT, the synonym must be set as the dependent variable on the commandline, -dv=$<$synonym$>$. 
\nextopt
\optname{keep\_estimation}
is optional, by default not set. If this option is set, a post-hoc evaluation step is performed for each simulated dataset (\$ESTIMATION is kept and MAXEVALS is set to 0). Note that in this case variables such as IPRED(F) are based on the re-estimated post-hoc parameters. Also note that in earlier program versions keep\_estimation was set or unset automatically, see section Additional rules and logic 3). 
\nextopt
\optname{noprediction}
is optional, by default not set. If set, NOPREDICTION will be added to the \$SIMULATION record of the simulation model, in addition to ONLYSIMULATION. This option is generally recommended with likelihood models for odd type data (i.e. -2LOGLIKELIHOOD or LIKELIHOOD in \$ESTIMATION). It is not allowed to use -noprediction in combination with the option \mbox{-keep\_estimation}. 
\nextopt
\optdefault{idv}{variable}
optional, default TIME, the independent variable. 
\nextopt
\optdefault{rawres\_input}{filename}
A simple way to simulate with uncertainty. Instead of using identical parameter estimates for simulation of each new dataset, take parameter estimates from a raw\_results.csv file, e.g. from a bootstrap run or the intial\_estimates.csv file from a previous sse run with \$PRIOR in the simulation model. The raw results file must be comma-separated and contain at least as many samples as the input -samples to sse, the labels for THETA/OMEGA/SIGMA in the file must match the labels in the simulation model given as input to sse, the theta columns must be directly followed by the omega columns which must be directly followed by the sigma columns, and the column header must be model either in the first column just as in a bootstrap raw\_results file or in the second or third column as in a sse raw\_results file. If a column header contains a comma, e.g. OMEGA(2,2), then that header must be enclosed in double quotes. This is done automatically in PsN raw results files. Note that is is possible to generate a file with initial parameter estimates outside of PsN, as long as the file follows the format rules. 
\nextopt
\optdefault{offset\_rawres}{N}
Only relevant in combination with -rawres\_input. Default 1. The number of result lines to skip in the input raw results file before starting to read final parameter estimates. In a regular bootstrap raw\_results file the first line of estimates refers to the input model with the full dataset, so therefore the default offset is 1. 
\nextopt
\optname{nca\_options}
This option takes a string that will be copied verbatim to the R script created to invoke nca.R. Strings containing spaces or double quotes can be enclosed within single quotes on Linux or MacOS. Example: \verb|nca run1.mod -nca_options='blq = "yes", blqNm = "BLQ"'|
On Windows strings containing spaces must be enclosed within double quotes and to have a double quote in the string this must be preceeded with a backslash.
Example: \verb|nca run1.mod -nca_options="blq = \"yes\", blqNm = \"BLQ\""|
\nextopt
\optname{include\_all\_columns}
If set all columns in \$INPUT, that are not marked as skipped or dropped, will be included in the tables.
\nextopt
\end{optionlist}



\subsection{Simulation input details}
The option -samples is required. The scripts does not allow -samples to be smaller than 20, but in order for the analysis to produce meaningful results samples needs to be much larger. No model estimation is performed for the simulated datasets, except for a post-hoc estimation step in case -keep\_estimation is set (then MAXEVALS=0 in \$ESTIMATION). There are five ways of choosing which parameter values are to be used in the simulations: 

\begin{enumerate}
	\item Default: the initial estimates from the lst-file with the same name as the modelfile but with .mod replaced with .lst, e.g. run123.lst if the modelfile is run123.mod. If no such lst-file exists the estimates from the modelfile are used.
	\item the final estimates from a lst-file whose name is not the modelfile name with .mod replaced with .lst: use command-line option -lst=$\langle$filename$\rangle$
	\item the final estimates from an msfo-file: use command-line option -msfo=$<$filename$>$
	\item final estimates from a raw\_results file, e.g. from a bootstrap. This method implies using a different set of estimates for each sample. 
	\item parameter estimates drawn from a prior distribution defined using \$PRIOR in the input/simulation model.
\end{enumerate}

Alternatives 4) and 5) result in simulation with uncertainty.

The user may either skip the \$SIMULATION record entirely and let the program produce it according to the rules specified in section Modified models – model for simulated data. Or the user can include a complete \$SIMULATION record, for example when using special random distributions. Inclusion of a \$SIMULATION record is needed when simulating categorical data is intended (vpc with -levels option). In this case the model file must be equipped with IF(ICALL.EQ.4) coding to separate the simulation model from the estimation model, and the \$SIMULATION record must contain both a normal seed and a uniform seed (1234 UNIFORM). If there is a user-defined \$SIMULATION record, the program will replace the random seeds, set NSUBPROBLEMS to the number of samples requested on the input line, check that TRUE=FINAL is set in the case when the option -msfo is used, and set/unset ONLYSIMULATION to be compatible with the option keep\_estimation. If option rawres\_input is used, the program will create one simulation model per sample.

An additional alternative is to supply the program with a complete simulation model, either via option -flip\_comments or option -sim\_model. In both cases the program will only change seeds and NSUBPROBLEMS in \$SIMULATION. No other changes to that particular record will be made.

For more information about the details of the simulation see the relevant sections of the vpc documentation.

\section{Output}
Two modified modelfiles are created, one for generating original data table output and one for generating simulated data table output. The model parameters will never be reestimated. Unless an lst- or msfo-file is given as input or an lst-file is found by replacing .mod with .lst in the modelfile name, the initial parameter estimates from the modelfile will be used for the simulations. The two modified models are run using PsN.
An R-script snippet called runnca.R is created. It will contain a call to the nca R-script with some parameters filled in.

\begin{thebibliography}{99}
	\bibitem{Chayan} Chayan Acharya, Andrew C. Hooker, Siv Jönsson, Mats O. Karlsson, {\em A diagnostic tool for population models using non-compartmental analysis: The nca functionality in R}, 2014, PAGE. Abstracts of the Annual Meeting of the Population Approach Group in Europe, www.page-meeting.org/?abstract=3103 
\end{thebibliography}

\end{document}
