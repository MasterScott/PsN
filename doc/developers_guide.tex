\documentclass[a4wide,12pt]{article}
%\setlength{\marginparwidth}{0pt}%35
%\setlength{\marginparsep}{0pt}%?
%\setlength{\evensidemargin}{0pt}
%\setlength{\oddsidemargin}{0pt}
\usepackage{lmodern}
\usepackage[utf8]{inputenc}
\usepackage[T1]{fontenc}
\usepackage{textcomp}
\usepackage{verbatim}
\usepackage{enumitem}
\usepackage{longtable}
\usepackage{alltt}
\usepackage{ifthen}
% Reduce the size of the underscore
\usepackage{relsize}
\renewcommand{\_}{\textscale{.7}{\textunderscore}}

\newcommand{\guidetitle}[1]{
\title{#1\\ \vspace{2 mm} {\large PsN 4.1.1}}
\date{2014-02-10}
}

\newcommand{\doctitle}[1]{
\title{#1}
\date{2014-02-10}
}


\newenvironment{optionlist}{
\renewcommand{\arraystretch}{1.1}
\setlength{\leftmargini}{2.5cm}
\begin{description}
%\setlength{\itemsep}{0ex}
}
{\end{description}}

\newcommand{\optname}[1]{\item{{\bfseries\texttt-#1}\newline}}
\newcommand{\optdefault}[2]{\item{{\bfseries\texttt-#1}{\mbox{ = \it #2}}\newline}}

\newcommand{\nextopt}{}

\guidetitle{Developer's guide}{2016-09-09}
\usepackage{hyperref}

\begin{document}

\maketitle

  
\section{Introduction}

This document is mainly intended for PsN developers and testers. It describes how to install development versions of PsN, how to run the tests, how to work with source control and coding guidelines.

\section{Installing the current development version of PsN for running and testing}
\begin{itemize}
\item Install git \cite{git}. 
\item Do git clone in a suitable place on your computer.
\begin{verbatim}
git clone git://git.code.sf.net/p/psn/PsN4 PsN4
\end{verbatim}
\item Change directory to PsN4.
\begin{verbatim}
cd PsN4
\end{verbatim}
\item Run commands 
\begin{verbatim}
make clean
make release
\end{verbatim}
\noindent which gives you a directory PsN-Source.
\item Change directory to PsN-Source.
\begin{verbatim}
cd PsN-Source
\end{verbatim}
\item Run the setup script, as you would with if you had downloaded an installation package from psn.sf.net
\begin{verbatim}
perl setup.pl
\end{verbatim}
\end{itemize}

\section{Running the development version of PsN without installation to make contributions to the code}
\begin{itemize}
\item Install git \cite{git}. 
\item Do git clone in a suitable place on your computer. On linux:
\begin{verbatim}
cd ~
git clone git://git.code.sf.net/p/psn/PsN4 PsN4

\end{verbatim}
\noindent Then do the following (assuming linux):
\item Add the PsN4/bin directory to your \$PATH variable, so that you can call PsN scripts without including the path.
\item Unless you already have a psn.conf file in the location set in PsN.pm, put one there. 
You can use PsN4/lib/psn.conf\_template as a template and just fill in the [nm\_version] section,
according to psn\_configuration.pdf.
\item Make sure you have the required Perl modules listed in PsN4/README.txt installed.
\end{itemize}

\section{External contributions to PsN}
If you have followed the instructions in the previous chapter and have made changes to PsN that you wish to contribute
to the official version of PsN you should follow the instructions below:

\begin{itemize}
    \item Only have one feature per commit. This makes it easier for the core developers to review changes and to choose to add some features and leave out features deemed not suitable for inclusion.
    \item Make test cases for your features and include them in the same commits.
    \item Add or update the documentation for your changes.
    \item Follow the coding style of the PsN style guide below.
    \item Place your repository in a public location.
    \item Ask the core developers to fetch from your repository.
\end{itemize}

The core developers are grateful for contributions but please keep in mind that we don't have much time to spend on support to external developers.
The more important the feature you are contributing is the more help with the above bullets you can expect to get.


\section{Testing}
When installing PsN the user has the option to install the test suite in a specified directory. The default directory is in the lib directory. If working from a cloned git repository all tests are in the PsN4/test directory. Unit tests are in unit subdirectory and system tests in the system subdirectory.  
Unit tests are defined as tests that do not involve \emph{running} NONMEM, while system tests are tests involving running NONMEM.  To be able to run the tests the package Test::Exception has to be installed on some systems.
Tests can be run by using the command \verb|prove| (which is bundled with perl). 

The unit tests can be run by issuing the command (if the current directory is the test directory).
\begin{verbatim}
prove -r unit
\end{verbatim}

and the system tests can be run with
\begin{verbatim}
prove -r system
\end{verbatim}

To suppress PsN log messages during system tests, use option -silent:
\begin{verbatim}
prove -r system :: -silent
\end{verbatim}

Tests can be run from any directory by specifying a path:
\begin{verbatim}
prove -r PsN4/test/unit
\end{verbatim}

Individual test scripts can be run with e.g. 
\begin{verbatim}
prove bootstrap.t
\end{verbatim}

It is possible to run multiple tests in parallel to speed up the testing. This can be done with the option -j n to prove, where n is the number of parallel jobs e.g.
\begin{verbatim}
prove -j 4 -r system
\end{verbatim}

If you want to run the system tests on a cluster using the cluster submission options of PsN 
you must set the relevant options in psn.conf, for example
\begin{verbatim}
[default_options]
run_on_slurm=1
slurm_account=admin
slurm_partition=development
\end{verbatim}

If a test fails, you can get more detailed output information by running it
with perl instead of prove, e.g.
\begin{verbatim}
perl bootstrap.t
\end{verbatim}

\subsection{System tests for multiple NONMEM versions}
Multiple versions of NONMEM can be tested automatically with the \verb|runsystem| perl script. This script takes
an array of NONMEM versions as arguments and all arguments staring with "-" will be tunnelled through to prove.
The following example will run all system tests once for nm72 and once for nm73 using 4 parallel threads.
\begin{verbatim}
./runsystem nm72 nm73 -j4
\end{verbatim}
There is also an option to run the system tests for all NONMEM versions available in psn.conf:
\begin{verbatim}
./runsystem all
\end{verbatim}
The set of NONMEM version run by \verb|./runsystem all| can be changed by setting\\
\verb|PsN_test_all = list,of,versions| \\
in the top section of psn.conf, i.e. before any [bracket section].
Note that PsN\_test\_all is case sensitive.

The log messages from PsN will be redirected to a file in the user's home directory or, if on Windows, the
Desktop folder. The location of the log file can be changed by setting\\
\verb|PsN_test_logfile_dir = /full/path/to/existing/directory| \\
in the top section of psn.conf, i.e. before any [bracket section].
Note that PsN\_test\_logfile\_dir is case sensitive. Note also that if prove option -j is used, log messages from
parallel tests will be intermixed.
\subsection{Test coverage}
In the current release we have excellent test coverage for execute (parsing NONMEM control streams, 
running models, tweaking initial estimates, and reading of final estimates from NONMEM output and reporting 
them in the raw\_results file). 
The same well tested code for running NONMEM is used by all other tools. 
There is also good tests for parsing of data files, which is done by e.g. bootstrap and scm but not by execute or
vpc.
Tests are excellent for tool-specific features of precond, sir, parallel\_retries, randtest, sse and sumo. 
For vpc we have good coverage for the basic vpc features (extracting numbers from NONMEM tables, binning data and 
computing intervals without censoring or prediction correction), but we have only very basic crash tests for the 
more advanced features. The bootstrap program has tests for dataset generation and calculation of results.
%bootstrap needs more detailed dataset unit tests, with and without stratification
scm has many crash tests, and tests to ensure that the computed means/medians/max/min of covariates are correct, 
but not of control stream manipulation.

\subsection{Test issues}
\begin{itemize}
    \item The sse unit tests will fail unless File::Copy::Recursive is installed.
    \item The setup unit tests will fail unless the PsN version tested is the default PsN version on the system.
	\item To speed up testing it is advised to set the nmfe option -prdefault in the psn.conf (add the line nmfe\_options=-prdefault).
	\item If the tests are to be run on a cluster or on a Mac a setup of a temporary directory is needed. Set\\
    \verb|PsN_test_tempdir = /full/path/to/existing/empty/directory/| \\
    in the top section of psn.conf, i.e. before any [bracket section].
     If testing on a cluster the directory must be reachable from all nodes.
     Note that PsN\_test\_tempdir is case sensitive.
	\item For NONMEM 7.1.0 and 7.1.2 the system tests will print out a lot of warnings like this:
\begin{verbatim}
Warning: Line truncated at (1)
FSUBS.f90:10.132:
\end{verbatim}
These warnings are from NONMEM and can be ignored.
	\item	The PsN developers have not run the test suite with NONMEM 6. There are serveral tests that are known to fail with NONMEM 6. If you are a user of NONMEM 6 and wish to run the test suite please give it a try and send your findings to any of the developers or the mailing list.
	\item If any test fails using any version of NONMEM 7 this is most likely a bug. Please report this so that it can be fixed.
    \item If the PsN installation directory is moved after the initial installation of the test suite, the file includes.pm in the top level of the test directory
    must be edited to include the new path to PsN core and toolkit. 
\end{itemize} 

\section{Help scripts for SourceForge webpage}

\begin{description}
\item[putdoc] Run at every internal release, i.e. every time new default
version on Uppsala cluster. Must be run from the directory where it is
(some local file paths in the script). This script will upload doc/*.pdf to \\
\verb|http://psn.sourceforge.net/internal_release_pdfdocs/| \\
and the .tar.gz and .zip install packages to\\
\verb|http://psn.sourceforge.net/internal_release_install_packages/|
The wiki links for documentation and install package for latest
release are pointing to these web places, so this is important to
do for each internal release.
\item[putwebpage] Run whenever web page php files are updated in between
external releases (when putrelease is run instead).
Will upload home.php, docs.php, download.php, install.php, list.php and
buglist.php to SourceForge, where they are immediately visible.
\item[putrelease] Run at external release. Not well tested since run very
infrequently. Will upload same php-files as putwebpage. Then will
upload doc/*.pdf and zippped/packed documentation packages (from PsN-Source)
to \\
\verb|http://psn.sourceforge.net/pdfdocs/|\\
Currently does not upload bug\_list\_at\_next\_release.
\end{description}
\section{R plots}
In the R-scripts folder there is one main template file per PsN tool that supports the R plots option.
The template files have names toolname\_default.R, for example bootstrap\_default.R and randtest\_default.R.
Each template file shall be as concise as possible. All R function definitions shall be in a subfolder of the R-scripts folder, and the subfolder shall be named after the PsN tool, for example pvar.

\begin{description}
\item[TODO]
  Do a simple clean-up of each template file toolname\_default.R so that already existing  R function definitions, if any, are moved
  to new files in subfolders with correct names, and the template file instead sources the new files, in the same way as pvar\_default.R
\item[TODO] Do more extensive clean-up of each template file, rewriting scripts to easily testable functions that return
  data frames etc, plus functions that return plot objects. Like pvar.
\item[TODO] Delete TODO items from the documentation when they are done.
\end{description}





\section{nmoutput2so}
Clone the PsN4 git repository and cd into your clone directory (See chapter 2). A standalone version of nmoutput2so can be built with \verb|make nmoutput2so|. A zip file called nmoutput2so.zip will be created. This can be unzipped
to any directory and is runnable from that directory.
Perl module XML::LibXML must be installed before running nmoutput2so.

\section{Build environment}
The documentation is made in \LaTeX and pdfs can be generated with \verb|make doc|. The
packages texlive-extra-util, texlive-bibtex-extra, biber, texlive-science and texlive-pictures are needed to build all documents.

\section{Revision control}
PsN use git as revision control system. This chapter give some best practices.

\subsection{Tagging}
All releases of PsN should be tagged with its version number with an annotated tag. Remember
that tags need to be pushed separately.
\begin{verbatim}
git tag -a 4.0.0 -m "My tag message"
git push origin --tags
\end{verbatim}

\section{Coding style guide}
PsN is written in Perl using the Moose object system \cite{Moose}. For a gentle introduction to Perl and Moose see for example Modern Perl \cite{modern}. This chapter will give some pointers on how to make best use of Perl and Moose in PsN. 

\subsection{Object construction}

Remember the construction order in Moose. If we have a superclass called parent and a subclass called child and call \verb|child->new| the constructors will be executed in the following order:

\begin{enumerate}
	\item child->BUILDARGS
	\item parent->BUILDARGS
	\item Moose internal construction
	\item parent->BUILD
	\item child->BUILD
\end{enumerate}


\subsection{Methods}

\subsubsection{Parameter validation}
All methods should validate their parameters with validated\_hash in \mbox{MooseX::Params::Validate}

\begin{verbatim}
use MooseX::Params::Validate;

sub my_method
{
  my ($self, %parm) = validated_hash(@_,
    parameter1 => { isa => 'Str', optional => 0 },
    parameter2 => { isa => 'Num', default => '28' },
  );

  # ...
}
\end{verbatim}

See the MooseX::Params::Validate documentation \cite{params} and Moose::Types documentation \cite{types} for more information.

Remember to turn off the parameter caching if you give default values that can vary between calls.

\begin{verbatim}
my ($self, %parm) = validated_hash(@_,
  parameter1 => { isa => 'Str', optional => 0 },
  parameter2 => { isa => 'Num', default => $self->some_method },
  MX_PARAMS_VALIDATE_NO_CACHE => 1
);
\end{verbatim}

There are cases when MooseX::Params::Validate have caused Perl to crash (see bug \#91211 at \url{https://rt.cpan.org/Public/Bug/Display.html?id=91211})
These cases have been when isa was set to 'ArrayRef[Str]' or 'ArrayRef[Int]' and the crashes stopped when the bracketed type was removed. When crashing on windows the error was either "Perl interpreter has stopped" or "Bizarre copy of ... in list assignment in MooseX/Params/Validate.pm line 63". For error messages under Linux see the bug report.


\subsubsection{Private methods}

Names of private methods should start with an underscore.
\begin{verbatim}
sub _this_is_a_private_method
{

}
\end{verbatim}


\subsection{Attributes}
triggers should be named \_attribute\_set
\begin{verbatim}
has 'filename' => (is => 'rw', isa => 'Str', trigger => \&_filename_set );

sub _filename_set
{
  my $self = shift;
  my $parm = shift;
  my $old_parm = shift;

  if ($parm ne $old_parm) {
    print "We are changing the filename!\n";
  }
}
\end{verbatim}

clearers should be named clear\_attribute
\begin{verbatim}
has 'filenames' => (is => 'rw', isa => 'ArrayRef[Str]', clearer => 'clear_filenames' );

if ($need_to_delete) {
  $self->clear_filenames;
}
\end{verbatim}

predicates should be named has\_attribute
\begin{verbatim}
has 'filename' => (is => 'rw', isa => 'Str', predicate => 'has_filename' );

if ($self->has_filename) {
  print "The filename attribute exists\n";
}
\end{verbatim}

\subsection{External modules}

As a workaround for a bug in the Carp package \verb|use include_modules;| should be used instead of \verb|use Carp;|. All the ordinary Carp methods will be exported.


\section{Data class}
As of PsN version 4.3, no data object is created automatically when a model object is created.
The model object has no (list of) data object(s) attribute, nor does any other object.
When a data object is needed it must be created explicitly. If a data object is created, the contents of the
data file is always parsed. A number of new methods, many of them
static, have been written for creation of new data sets for e.g. cdd, bootstrap, randtest. 
These methods handle writing of the new data sets to disk, and return a list of file names to the
new data files. Data objects are not returned, they are cleared from memory. A number of unit tests have been written for
the data generation methods.

The programmer should never overwrite 
an existing data file.
The data->\_write method will croak if the file to write to already exists. If built-in Perl sub cp(old,new) is
used, the programmer must ensure a data file is not overwritten. A possible improvement is to write a special subroutine
for copying data files, that does checking that the new file does not exist.

All data class attributes and methods relating to flush, sync, synchronize and target have been removed. 

\subsection{Data class attributes and methods removed}
skip\_parsing, synced, target, mdv\_column, dv\_column, individual\_ids.

target\_set, diff, flush, single\_valued\_data.

\subsubsection{Old methods renamed}
The following methods have been made private ( underscore added in sub name): 
\_bootstrap, \_bootstrap\_from\_keys, \_case\_deletion, \_renumber\_ascending, \_randomize\_data.

\subsection{New data class methods}
\begin{description} 
\item[add\_randomized\_input\_data] For boot\_scm if option -dummy\_covariates set
\item[bootstrap\_create\_datasets] standard bootstrap data generation, uses private \_bootstrap, \_renumber\_ascending methods
\item[bootstrap\_create\_datasets\_from\_keys] special bootstrap data generation, uses private \_bootstrap\_from\_keys, \_renumber\_ascending
\item[frem\_compute\_covariate\_properties]
\item[cdd\_create\_datasets] uses private \_case\_deletion
\item[create\_randomized\_data] for randtest, uses private \_randomize\_data
\item[lasso\_calculate\_covariate\_statistics]
\item[lasso\_get\_categories]
\item[scm\_calculate\_covariate\_statistics] main covariate statistics method
\item[scm\_calculate\_categorical\_statistics] helper method to scm\_calculate\_covariate\_statistics
\item[scm\_calculate\_continuous\_statistics] helper method to scm\_calculate\_covariate\_statistics
\end{description}
\section{model::problem::data record class}
In PsN 4.3 the data record class has been extended. It handles the data file of the model, in the sense that it keeps track
of where the data file is located and makes sure a correct path is written in \$DATA when a model file
is written to disk. The data record class has no data object.

The class has the following attributes, in addition to the inherited record attributes:
\begin{description}
\item[filename] This is the filename, without path, of the physical data file on disk. It has a public reader and a private writer. To change the
value of the attribute, a
separate public set method must be used, that in turn calls the private writer.  
\item[directory] This is the absolute path to the physical data file on disk. It has a public reader and a private writer. 
To change the value of the attribute, a
separate public set method must be used, that in turn calls the private writer.  
\item[model\_directory] The absolute path to the original model file when a new model object is created. 
This attribute is only used in BUILD, to figure
out the absolute path to the data file when the file name in \$DATA is relative, which it very often is.
\item[ignoresign] A single character used to ignore header lines in data files, for example @. Set with IGNORE=character
in \$DATA. To be distinguished from ignore lists, set with IGNORE=(some logical expression), in \$DATA. This attribute
is used when writing \$DATA to disk, see the overloaded \_format\_record method, 
which means that it is important to use the ignoresign setter/getter and 
not the data record options when changing the ignorecharacter.
\item[ignore\_list] A helper attribute, a parsed version of the ignorelist. Not used when writing \$DATA to disk.
\end{description}

It is \emph{not} possible to access the data file name via the options of the data record class. After parsing of
the input \$DATA record lines from the original model file, the data file info is stored in the special attributes
(see the BUILD code). The stored filename and directory are used when writing a model to disk, see
the overloaded \_format\_record method.

The programmer must not set the ignoresign via the data record options. The ignoresign accessor must be used instead. However,
for ignore lists (logical expressions enclosed by parentheses) the record options must be used.
The typical case when ignoresign should be set is when a NONMEM \$TABLE file is used as a data file for 
a new model. Then the programmer can for example set \verb|$model->problems->[0]->datas->[0]->ignoresign('@')|

The \_format\_record method, and the helper method format\_filename, take mandatory input options Bool relative\_data\_path and
string write\_directory. 
The write\_directory option is the path to the directory where the model file needing \$DATA is being written.
If relative\_data\_path is true, the filename written in \$DATA will use the relative path from the write\_directory to the
directory holding the data file. If relative\_data\_path is false, the filename written in \$DATA will use the absolute
path to the data file, and the write\_directory will not matter.

NONMEM will crash if the filename set in \$DATA is longer than 80 characters. The \_format\_record method can
be set to croak if the filename is too long, this helps during code development. It is a good idea to run system tests in
a temporary directory with a path longer than 80 characters, to make sure PsN handles this correctly without
writing too long paths that would make nmfe crash. Also, try putting the test directory containing test\_files
etc in a very long path, this can also reveal bugs.
However, not all model files written to disk are actually run by nmfe, so in some cases
a check in modelfit.pm that the path is not too long would be enough. 
If filename is too long then attribute copy\_datafile (see section below) can be set to true when creating psn.mod in NM\_run
subfolders.

\section{Model class}

All attributes and methods relating to flush, sync, synchronize and target have been removed. (The output class still hash a flush method.)

The model object has no data objects. 
The following attributes have been removed: datas (the array of data objects), skip\_data\_parsing, synced, target.

Getting or changing the data file in \$DATA shall always be done using the datafiles method.

Getting data file names: datafiles method returns an arrayref with filenames given for the problem numbers specified in option arrayref problem\_numbers,
default all problems of the model. Optional Bool absolute\_path, default false. If absolute\_path is true, then the data file names
are given with absolute path, otherwise just the filename is given. 

Changing data file names: datafiles method takes an arrayref new\_names with data file name strings. If problem\_numbers is
given then new\_names must have the same length as problem\_numbers, otherwise new\_names must have one string
per problem in the model.

\subsection{New model class attributes}
\begin{description}
\item[relative\_data\_path] Bool, default true, decides the default value for this model object when
deciding whether absolute path to data file or relative path should be used when the model file is written to disk.
\item[is\_dummy] Bool, false always except when sub create\_dummy\_model is used to create dummy model object. This attribute
gives a safe way to determine whether a model object is a dummy, better than checking the model file name.
\end{description}

\subsection{model->copy method}
Method copy has been rewritten. The following options are new and/or especially important:
\begin{description}
\item[write\_copy] A boolean, by default option write\_copy is true, meaning that the new model will be written to disk 
directly, before the copy method returns. But if the new model object is to be modified directly after the copy,
which is very often the case, option write\_copy should be set to false and a 'manual' \_write should be done
after all modification is done.
\item[copy\_datafile] Boolean, default is false. If set to true, then the data file of the original model object (filename 
retrieved using datafiles method) will be copied to the directory of the new model copy, and attribute 
relative\_data\_path will be set to true for the copy even if the original model object had
relative\_data\_path set to false.
Whenever a model object is created this way with copy\_datafile true, an absolute data path must never
be used when writing the model object to disk later.
Setting copy\_datafile to true increases the number of files on disk, so it should be used restrictively.
\item[copy\_output] Boolean, normally this should be false, which is the default. Copy the output object or not.
If set to true then running the copy can affect the pre-existing output on disk of the original model. 
\end{description}

\subsection{model->\_write method}
Method \_write has been rewritten. 
It can be set to croak if a file with the name to write to already exists and the overwrite option is not set. This
helps in cleaning up the code during development
to remove redundant writes to disk.
The data file is never written as part of model->\_write.
Option relative\_data\_path to \_write method can be used to override the model attribute with same name,
deciding whether relative or absolute data file path should be used in \$DATA.
Normally the default, decided by the model attribute, should be used for relative\_data\_path.

\section{copy\_data command-line option and modelfit/tool subclass attribute}
A few scripts has copy\_data as a command-line option (execute, bootstrap, vpc, npc, nca, sir, randtest). 
By default copy\_data is always true. On the
command-line copy\_data shall mean, to be consistent across scripts: 
use a copy of the \emph{original} data when models needing the \emph{original}
data file are run. If command-line option copy\_data is unset with -no-copy\_data, this means: always use the 
original data file with absolute path in \$DATA when running models that use the original data set. Setting -no-copy\_data
is popular for studies with high requirements on traceability, it is easier to show that the data is not corrupted 
if the same file is always used, rather than copies.

The command-line meaning of copy\_data was for bootstrap, sir and randtest in PsN 4.2, then it was related to using a relative path 
to data in m1 or not, but this option was not widely known or used. In PsN 4.3 the meaning is changed, for consistency across scripts.

If command-line copy\_data was unset using -no-copy\_data, 
then the bin script should ensure that using an absolute path is not too long for NONMEM (there is a subroutine for this, see example in bin/bootstrap)
and then that the input model objects have attribute relative\_data\_path set to false (see bin/bootstrap).
Also the modelfit object(s) used to run these models (or copies of these models using the same data) 
should have copy\_data set to the same as the user gave on the commandline, otherwise it is a programming error. See bootstrap.pm, 
modelfit that runs the base model and the dofv-models: 
\begin{verbatim}
copy_data => $self->copy_data
\end{verbatim}

The copy\_data option set on the commandline shall always be passed on to the 
tool subclass object created in the bin script, but inside the various tool subclass objects (bootstrap, npc, randtest... objects),
the command\_line setting of copy\_data (\$self->copy\_data) is not always passed on again to modelfit or other tool objects
that are created. See bootstrap example.

Attribute copy\_data to the modelfit class is used in the following way:
If attribute copy\_data is true, modelfit will copy the data file
to NM\_run directories (by setting option copy\_datafile => 1 when doing model->copy to create psn.mod),
and then psn.mod will have the datafile name without any path in \$DATA (model attribute relative\_data\_path is true, with
the path being . ). If modelfit attribute copy\_data is not true, 
modelfit will set option copy\_datafile => 0 when doing model->copy to create psn.mod.
In the later case, it will be the model objects own setting of relative\_data\_path that decides
whether an absolute or relative path will be used in \$DATA in psn.mod.

Model objects using data files that were generated inside the run directory (normally in the m1 subdirectory), 
for example bootstrap samples, should have attribute relative\_data\_path set to true.
When a modelfit object is created for running such models, e.g. bootstrap samples,
the copy\_data attribute of the new modelfit object should always be false, otherwise you get a serious waste of disk space
when data files are copied from m1 to NM\_run subdirectories.
See bootstrap.pm, modelfit object that runs the bootstrap models:
\begin{verbatim}
copy_data => 0
\end{verbatim}


\begin{thebibliography}{1}
	\bibitem{git} {\em Pro Git}, \url{http://git-scm.com/book}, Apress
	\bibitem{modern} {\em Modern Perl} \url{http://onyxneon.com/books/modern_perl/}, Onyx Neon Press
	\bibitem{Moose} {\em Moose}, \url{https://metacpan.org/pod/Moose::Manual}, CPAN
	\bibitem{params} {\em MooseX::Params::Validate}, \url{http://search.cpan.org/\~drolsky/MooseX-Params-Validate-0.18/lib/MooseX/Params/Validate.pm}, CPAN
	\bibitem{types} {\em Moose::Manual::Types}, \url{http://search.cpan.org/~ether/Moose-2.1005/lib/Moose/Manual/Types.pod}, CPAN
\end{thebibliography}


\end{document}
