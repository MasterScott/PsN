\documentclass[a4wide,12pt]{article}
%\setlength{\marginparwidth}{0pt}%35
%\setlength{\marginparsep}{0pt}%?
%\setlength{\evensidemargin}{0pt}
%\setlength{\oddsidemargin}{0pt}
\usepackage{lmodern}
\usepackage[utf8]{inputenc}
\usepackage[T1]{fontenc}
\usepackage{textcomp}
\usepackage{verbatim}
\usepackage{enumitem}
\usepackage{longtable}
\usepackage{alltt}
\usepackage{ifthen}
% Reduce the size of the underscore
\usepackage{relsize}
\renewcommand{\_}{\textscale{.7}{\textunderscore}}

\newcommand{\guidetitle}[1]{
\title{#1\\ \vspace{2 mm} {\large PsN 4.1.1}}
\date{2014-02-10}
}

\newcommand{\doctitle}[1]{
\title{#1}
\date{2014-02-10}
}


\newenvironment{optionlist}{
\renewcommand{\arraystretch}{1.1}
\setlength{\leftmargini}{2.5cm}
\begin{description}
%\setlength{\itemsep}{0ex}
}
{\end{description}}

\newcommand{\optname}[1]{\item{{\bfseries\texttt-#1}\newline}}
\newcommand{\optdefault}[2]{\item{{\bfseries\texttt-#1}{\mbox{ = \it #2}}\newline}}

\newcommand{\nextopt}{}

\guidetitle{Developer's guide}
\usepackage{hyperref}

\begin{document}

\maketitle

  
\section{Introduction}

This document is intended for PsN developers. It is work in progress.

\section{Installing development version of PsN}
\begin{itemize}
\item Install git \cite{git}. 
\item Do git clone in a suitable place on your computer, and checkout a suitable branch. On linux:
\begin{verbatim}
cd ~
git clone git://git.code.sf.net/p/psn/PsN4 PsN4
git checkout -b NLP origin/NLP 

\end{verbatim}
\pagebreak
\noindent Then do the following (assuming linux):
\item Add the PsN4/bin directory to your \$PATH variable, so that you can call PsN scripts without including the path.
\item Copy the file PsN4/lib/PsN\_template.pm to PsN4/lib/PsN.pm. Open PsN.pm and add the following lines at the very top, before
use ext::Carp (of course change paths from /home/kajsa/kod-psn/ to what you have):
\begin{verbatim}
package PsN;
use lib '/home/kajsa/kod-psn/PsN4/lib';
$lib_dir = '/home/kajsa/kod-psn/PsN4/lib';
$config_file = '/home/kajsa/psn.conf';
$version = 'development';
\end{verbatim} 
Save and close PsN.pm
\item Unless you already have a psn.conf file in the location set in PsN.pm, put one there. 
You can use PsN4/lib/psn.conf\_template as a template and just fill in the [nm\_version] section,
according to psn\_configuration.pdf.
\item Make sure you have the required Perl modules listed in PsN4/README.txt installed.
\item To be able to run the tests in PsN4/test you need to edit PsN4/test/includes.pm to set the right use statements for your user. 
\item A few
tests in PsN4/test/system/sse.t are guaranteed to fail since they use a model and datafile that is not public and therefore not included in the
repository. To skip these tests you can comment out the commands using private\_test\_files in PsN4/test/system/sse.t.
\item To be able to push your changes on the git server you need create a SourceForge account, 
and then ask the PsN team to give you permission to push on PsN4.
Stay on the NLP branch so that git pull and git push will be on that branch. 
\end{itemize}

\section{Testing}
The PsN4/test directory contains the tests. Unit tests are in PsN4/test/unit and system tests in PsN4/test/system. 
Unit tests are defined as tests that do not involve \emph{running} NONMEM, while system tests is all the rest. 
Scripts unit\_tests.pl and system\_tests.pl in 
PsN4/test will run all tests (i.e. all .t scripts) in the unit and system subdirectories, respectively. 
Individual test scripts are run with e.g. perl bootstrap.t. 
Currently a few
tests in PsN4/test/system/sse.t are guaranteed to fail since they use a model and datafile that is not public and therefore not included in the
repository. To skip these tests you can comment out the commands using private\_test\_files in PsN4/test/system/sse.t.
\section{Build environment}
The documentation is made in \LaTeX and pdfs can be generated with \verb|make doc|. The
packages texlive-extra-util and texlive-science are needed to build all documents.

\section{Revision control}
PsN use git as revision control system. This chapter give some best practices.

\subsection{Tagging}
All releases of PsN should be tagged with its version number with an annotated tag. Remember
that tags need to be pushed separately.
\begin{verbatim}
git tag -a 4.0.0 - m "My tag message"
git push origin --tags
\end{verbatim}

\section{Style guide}

\subsection{Object construction}

Remember the construction order in Moose. If we have a superclass called parent and a subclass called child and call \verb|child->new| the constructors will be executed in the following order:

\begin{enumerate}
	\item child->BUILDARGS
	\item parent->BUILDARGS
	\item Moose internal construction
	\item parent->BUILD
	\item child->BUILD
\end{enumerate}


\subsection{Methods}

\subsubsection{Parameter validation}
All methods should validate their parameters with validated\_hash in MooseX::Params::Validate

\begin{verbatim}
use MooseX::Params::Validate;

sub my_method
{
  my ($self, %parm) = validated_hash(@_,
    parameter1 => { isa => 'Str', optional => 0 },
    parameter2 => { isa => 'Num', default => '28' },
  );

  # ...
}
\end{verbatim}

See the MooseX::Params::Validate documentation \cite{params} and Moose::Types documentation \cite{types} for more information.

Remember to turn off the parameter caching if you give default values that can vary between calls.

\begin{verbatim}
my ($self, %parm) = validated_hash(@_,
  parameter1 => { isa => 'Str', optional => 0 },
  parameter2 => { isa => 'Num', default => $self->some_method },
  MX_PARAMS_VALIDATE_NO_CACHE => 1
);
\end{verbatim}

\subsubsection{Private methods}

Names of private methods should start with an underscore.
\begin{verbatim}
sub _this_is_a_private_method
{

}
\end{verbatim}


\subsection{Attributes}
triggers should be named \_attribute\_set
\begin{verbatim}
has 'filename' => (is => 'rw', isa => 'Str', trigger => \&_filename_set );

sub _filename_set
{
  my $self = shift;
  my $parm = shift;
  my $old_parm = shift;

	if ($parm ne $old_parm) {
    print "We are changing the filename!\n";
  }
}
\end{verbatim}

clearers should be named clear\_attribute
\begin{verbatim}
has 'filenames' => (is => 'rw', isa => 'ArrayRef[Str]', clearer => 'clear_filenames' );

if ($need_to_delete) {
  $self->clear_filenames;
}
\end{verbatim}

predicates should be named has\_attribute
\begin{verbatim}
has 'filename' => (is => 'rw', isa => 'Str', predicate => 'has_filename' );

if ($self->has_filename) {
  print "The filename attribute exists\n";
}
\end{verbatim}

\subsection{External modules}

As a workaround for a bug in the Carp package \verb|use include_modules;| should be used instead of \verb|use Carp;|. All the ordinary Carp methods will be exported.


\begin{thebibliography}{1}
	\bibitem{git} {\em Pro Git}, \url{http://git-scm.com/book}, Apress
	\bibitem{params} {\em MooseX::Params::Validate}, \url{http://search.cpan.org/\~drolsky/MooseX-Params-Validate-0.18/lib/MooseX/Params/Validate.pm}, CPAN
	\bibitem{types} {\em Moose::Manual::Types}, \url{http://search.cpan.org/~ether/Moose-2.1005/lib/Moose/Manual/Types.pod}, CPAN
\end{thebibliography}


\end{document}
