\documentclass[a4wide,12pt]{article}
%\setlength{\marginparwidth}{0pt}%35
%\setlength{\marginparsep}{0pt}%?
%\setlength{\evensidemargin}{0pt}
%\setlength{\oddsidemargin}{0pt}
\usepackage{lmodern}
\usepackage[utf8]{inputenc}
\usepackage[T1]{fontenc}
\usepackage{textcomp}
\usepackage{verbatim}
\usepackage{enumitem}
\usepackage{longtable}
\usepackage{alltt}
\usepackage{ifthen}
% Reduce the size of the underscore
\usepackage{relsize}
\renewcommand{\_}{\textscale{.7}{\textunderscore}}

\newcommand{\guidetitle}[1]{
\title{#1\\ \vspace{2 mm} {\large PsN 4.1.1}}
\date{2014-02-10}
}

\newcommand{\doctitle}[1]{
\title{#1}
\date{2014-02-10}
}


\newenvironment{optionlist}{
\renewcommand{\arraystretch}{1.1}
\setlength{\leftmargini}{2.5cm}
\begin{description}
%\setlength{\itemsep}{0ex}
}
{\end{description}}

\newcommand{\optname}[1]{\item{{\bfseries\texttt-#1}\newline}}
\newcommand{\optdefault}[2]{\item{{\bfseries\texttt-#1}{\mbox{ = \it #2}}\newline}}

\newcommand{\nextopt}{}

\guidetitle{NPFIT user guide}{2016-09-20}

\newcommand{\guidetoolname}{npfit}

\begin{document}

\maketitle


\section{Introduction}
Nonparametric estimation using a range of NPSUPP values.

Examples
\begin{verbatim}
npfit run1.mod -npsupp=50,100,200
\end{verbatim}

\section{Input and options}

\subsection{Required input}
A model file is required, and a list of npsupp-values.
NONMEM requires that the \$ESTIMATION record is present with the conditional method or POSTHOC.
%\subsection{Optional input}

\subsection{Options}
\begin{optionlist}
\optdefault{npsupp}{50,100,200}
Required. A comma-separated list of non-negative integers.
For each value N a new copy of the input model will be run with
\$NONPARAMETRIC UNCONDITIONAL NPSUPP=N
See the NONMEM documentation on \$NONPARAMETRIC for interpretation of NPSUPP.
All values in the list should be equal to or greater than the number
of individuals in the data set.
\nextopt
%\optname{copy\_data}
%\nextopt
%\optname{keep\_tables}
%\nextopt
%\optdefault{rplots}{level}
%\nextopt
\end{optionlist}

\subsection{Some important common PsN options}
There are many options that govern how PsN manages NONMEM runs, and
those options are common to all PsN programs that run NONMEM.
For a complete list of such options see common\_options.pdf, 
or psn\_options -h on the commandline. A selection of
the most important common options relevant for npfit is found here.
\begin{optionlist}
\optname{h or -?}
Print the list of available options and exit. 
\nextopt
\optname{help}
With -help all programs will print a longer help message. 
If an option name is given as argument, help will be printed for this option. 
If no option is specified, help text for all options will be printed. 
\nextopt
\optdefault{directory}{'string'}
Default \guidetoolname\_dirN,
where N will start at 1 and
be increased by one each time you run the script. The directory option sets the directory in which PsN 
will run NONMEM and where PsN-generated output files will be stored. 
You do not have to create the directory,  it will be done for you. If you set
-directory to a the name of a directory that already exists, PsN will run in the existing directory.
\nextopt
\optdefault{seed}{'string'}
You can set your own random seed to make PsN runs reproducible.
The random seed is a string, so both -seed=12345 and -seed=JustinBieber are valid.
It is important to know that because of the way the Perl pseudo-random
number generator works, for two similar string seeds the random sequences may be identical. 
This is the case e.g. with the two different seeds 123 and 122. 
Setting the same seed guarantees the same sequence, but setting two slightly different 
seeds does not guarantee two different random sequences, that must be verified.
\nextopt
\optdefault{clean}{'integer'}
Default 1. The clean option can take four different values:  
\begin{description}
\item[0] Nothing is removed 
\item[1] NONMEM binary and intermediate files except INTER are removed, and files specified with option -extra\_files. 
\item[2] model and output files generated by PsN restarts are removed, and data files in the NM\_run directory, and (if option -nmqual is used) the xml-formatted NONMEM output. 
\item[3] All NM\_run directories are completely removed. If the PsN tool has created modelfit\_dir:s inside the main run directory, these  will also be removed. 
\end{description}
\nextopt
\optdefault{nm\_version}{'string'}
Default is 'default'. 
If you have more than one NONMEM version installed you can use option
-nm\_version to choose which one to use, as long as it is 
defined in the [nm\_versions] section in psn.conf, see psn\_configuration.pdf for details. 
You can check which versions are defined, without opening psn.conf, using the command
\begin{verbatim}
psn -nm_versions
\end{verbatim}
\nextopt
\optdefault{threads}{'integer'}
Default 1. Use the threads option to enable parallel execution of multiple models.
This option decides how many models PsN will run at the same time, and it is completely
independent of whether the individual models are run with serial NONMEM or parallel NONMEM.
If you want to run a single model in parallel you must use options -parafile and -nodes.
On a desktop computer it 
is recommended to not set -threads higher the number of CPUs in the system plus one. 
You can specify more threads, 
but it will probably not increase the performance. If you are running on a computer cluster, 
you should consult your 
system administrator to find out how many threads you can specify. 
\nextopt
\optname{version}
Prints the PsN version number of the tool, and then exit. 
\nextopt
\end{optionlist}


\section{Results}

The file raw\_nonparametric\_modelname.csv, for example raw\_nonparametric\_run1.csv,
contains the parametric and nonparametric ofv, the npsupp value used,
and the nonparametric ETAs.

\section{Internal npfit workflow}

%check that in bin/npfit tool::nonparametric->new you have top_tool => 1 

\begin{enumerate}
\item Read the input model into memory. % in npfit/bin via model->new
\item Input checking, i.e. verify that required options are set and valid and that requirements on the input model
are satisfied.%in input_checking.pm for npfit
\begin{enumerate}
\item The nonmem version must be 7.4 or later.
% $PsN::nm_major_version > 7 or ($PsN::nm_major_version ==7 and $PsN::nm_minor_version > 3)
\item The model must have \$ESTIMATION in at least one \$PROBLEM. The first \$PROBLEM with \$ESTIMATION will be the
only one being updated in the following procedure.
% find $probnum, i.e. start at $i=1 and find first where following is true:
%(defined $model->problems->[$i-1]->estimations and scalar(@{$model->problems->[$i-1]->estimations})>0)
% if none found then this is an input error
\item If METHOD is not conditional then \$ESTIMATION must specify POSTHOC.
% my $method = $model->problems->[$probnum-1]->estimations->[-1]->get_method;
% unless ($method eq '1' or $method =~ /^CON/ or $method eq 'FOCE')
%  use $model->is_option_set  for estimation record for problem number $probnum, record number -1, option POSTHOC with fuzzy match
% if not set then error 
\end{enumerate}
\item Create (or reopen) a run directory according to the usual PsN conventions.
%automatically by tool->new
\item If the run directory already existed, read any models/results that are already present and skip
to the step below where the old process stopped. %later, not now
\item If there is an lst-file linked to the input model then read the estimation results. %automatically by model->new
\item If there are no estimation results available for the input model then run the input model and read the estimation results.
% unless $model->is_run
%			my $orig_fit = 
%				tool::modelfit->new( %{common_options::restore_options(@common_options::tool_options)},
%									 base_directory	 => $self ->directory(),
%									 directory		 => undef,
%									 models		 => [$model],
%									 raw_results           => undef,
%									 top_tool              =>0 );
%		tool::add_to_nmoutput(run => $orig_fit, extensions => ['ext','cov']);		
%			ui -> print( category => 'all',		 message => 'Running input model' );
%			$orig_fit -> run;

\item Find $N_{ind}$, the number of individuals in the data set as reported in the estimation lst-file,
i.e. the number of individuals after any IGNORE/ACCEPT. If $N_{ind}$ is larger than any of the npsupp values
then print a warning.
% my $N_individuals = $model->outputs->[0]->nind();
\item Update initial estimates with the final estimates from the estimation.
%find $probnum again, this is not passed on from input_checking via any option
% $model->update_inits(from_output => $model->outputs->[0], problem_number => $probnum);
\item If the estimation method is classical set MAXEVAL=1 in \$ESTIMATION 
% if ($model->problems->[$probnum-1]->estimations->[-1]->is_classical)
%  $model->set_option MAXEVAL=1 in estimation for problem number $probnum, record number -1, with fuzzy match
\item For each value i of npsupp, copy the updated model and set \$NONPARAMETRIC UNCONDITIONAL NPSUPP=i
Any pre-existing \$NONPARAMETRIC will be removed.
Write the model copy in the m1 subfolder of the run directory.
% loop as in bin/npfit, except use m1 subdirectory, which is automatically created
\item Run models with NONMEM
%create modelfit object and push to $self->tools as in tool/proseval.pm
% in bin/npfit replace modefit new with nonparametric new and do run on that object instead.
\item In the raw\_nonparametric results file, add a column with the npsupp values. 
%first in sub modelfit_analyze, later if have time move to raw_results_callback
\item Create, and possibly run, R script. %later
\end{enumerate}

%\references

\end{document}
