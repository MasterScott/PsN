\documentclass[a4wide,12pt]{article}
%\setlength{\marginparwidth}{0pt}%35
%\setlength{\marginparsep}{0pt}%?
%\setlength{\evensidemargin}{0pt}
%\setlength{\oddsidemargin}{0pt}
\usepackage{lmodern}
\usepackage[utf8]{inputenc}
\usepackage[T1]{fontenc}
\usepackage{textcomp}
\usepackage{verbatim}
\usepackage{enumitem}
\usepackage{longtable}
\usepackage{alltt}
\usepackage{ifthen}
% Reduce the size of the underscore
\usepackage{relsize}
\renewcommand{\_}{\textscale{.7}{\textunderscore}}

\newcommand{\guidetitle}[1]{
\title{#1\\ \vspace{2 mm} {\large PsN 4.1.1}}
\date{2014-02-10}
}

\newcommand{\doctitle}[1]{
\title{#1}
\date{2014-02-10}
}


\newenvironment{optionlist}{
\renewcommand{\arraystretch}{1.1}
\setlength{\leftmargini}{2.5cm}
\begin{description}
%\setlength{\itemsep}{0ex}
}
{\end{description}}

\newcommand{\optname}[1]{\item{{\bfseries\texttt-#1}\newline}}
\newcommand{\optdefault}[2]{\item{{\bfseries\texttt-#1}{\mbox{ = \it #2}}\newline}}

\newcommand{\nextopt}{}

\guidetitle{nmoutput2so user guide}

\begin{document}

\maketitle

\section{Introduction}
The nmoutput2so script creates and populates a DDMoRe standardized output xml file given the output from one or more NONMEM runs. It is necessary that the format of the model file and other files follow a set of requirements (see below) for this to work. The Perl module XML::LibXML is required by nmoutput2so, but not PsN in general. Make sure that it is installed before using nmoutput2so. See the PsN installation guide for intructions.


Examples
\begin{verbatim}
nmoutput2so model.lst
\end{verbatim}

\section{Input and options}

\subsection{Required input}
Required argument is a list of nonmem results files (.lst). Wildcards will work (i.e. *.lst)


\subsection{Optional input}

\begin{optionlist}
\optdefault{precision}{10}
The number of significant digits to use for all numerical data in the standard output file.
\nextopt
\optname{bootstrap\_results}
A PsN bootstrap\_results.csv file to use for filling the Bootstrap element.
\nextopt
\optname{use\_tables}
Default set. Will use table files (sdtab and patab) to populate the elements of the SO that needs these. Use -no-use\_tables to not use any table files.
\nextopt
\optname{exclude\_elements}
Set a comma separated list of simple XPaths relative the SOBlock to exclude from the SO.
For example:
\begin{verbatim}
-exclude_elements=Estimation/PopulationEstimates
\end{verbatim}
\nextopt
\optname{only\_include\_elements}
Set a comma separated list of simple XPaths relative the SOBlock. These elements will be the only ones used.
For example: 
\begin{verbatim}
-only_include_elements=Estimation/Likelihood
\end{verbatim}
\nextopt
\optname{message}
Specify a string to be added as an information message in the TaskInformation of the first SOBlock.
\nextopt
\optname{toolname}
The toolname to use for messages. Default is 'NONMEM'
\nextopt
\optname{max\_replicates}
Set a maximum number of simulation replicates to add. Default is to add all replicates
\nextopt
\optname{pretty}
Set if SO should be output with indentations and newlines.
Default is to not add intentations and thus to generate as compact xml files as possible.
\nextopt
\optname{so\_filename}
Set a filename for the resulting SO xml file. If this is not set the file stem of the first .lst file will be used.
\nextopt
\end{optionlist}

\section{Requirements on the model file and other files}

\subsection{Requirements on the model file}
\begin{itemize}
    \item All NONMEM parameters (THETAs, OMEGAs and SIGMAs) must have labels. A label is a one word (and only one word) comment on the same line as the definition of the initial estimate of the parameter. This means that all block definitions must be written on different rows. Example:
        \begin{verbatim}
$OMEGA BLOCK(2)
 1.2     ; PPV_CL
 0.01    ; CORR_CL_V
 0.1     ; PPV_V
        \end{verbatim}
    \item To be able to populate the Residuals and the Predictions entries an sdtab table must be specified. The name but not the orders of the columns is important. ID and TIME must be in the table. Other columns are optional. The example below shows an sdtab with all columns. The name of the file must start with "sdtab" Example:
        \begin{verbatim}
$TABLE ID TIME DV PRED IPRED RES IRES WRES IWRES NOAPPEND FILE=sdtab2
        \end{verbatim}
    \item To be able to populate the IndividualEstimates entry a patab table must be specified. NOAPPEND must be used and the order of the columns is important. The first column must be ID, then comes all individual parameters followed by the ETAs (see below). Example:
        \begin{verbatim}
$TABLE ID CL V ETA_CL ETA_V NOAPPEND FILE=patab
        \end{verbatim}
    \item To be able to populate the RandomEffects entry all ETAs must be defined in the main code block (\$PK or \$PRED). One ETA should be defined per row with nothing more on the row (except for spaces) than one assignment. Example:
        \begin{verbatim}
 ETA_CL = ETA(1)
 ETA_V = ETA(2)
        \end{verbatim}
\end{itemize}


\subsection{Requirements on other files}
\begin{itemize}
    \item All nonmem result files must have the same filename (excluding extension) as the .lst file. If not it is impossible to pick up the results from the .ext, .cov, .cor files etc.
\end{itemize}

\section{Known limitations}
\begin{itemize}
    \item The full xml file is stored in memory at once. This means that big results, for example big simulations, can potentially use a lot or all of the available memory.
    \item Does not support multiple DVIDs in SimulatedProfiles. Sets all to 1.
    \item Dose rows are included in Residuals, Predictions and SimulatedProfiles. They have value set to 0.
\end{itemize}
\section{Output}

The resulting standardized output object xml will be written to the file called <name of .lst file>.so\_xml.

The following entries will be populated:

\begin{itemize}
    \item TaskInformation
        \begin{itemize}
            \item Message - Errors, warnings etc will be put here
            \item RunTime - From the .lst file
        \end{itemize}
    \item Estimation
        \begin{itemize}
            \item PopulationEstimates
            \begin{itemize}
                \item MLE - Results will be taken from the .ext file if present otherwise the .lst file
            \end{itemize}
            \item PrecisionPopulationEstimates
            \begin{itemize}
                \item MLE
                \begin{itemize}
                    \item CovarianceMatrix - If the covariance step was successful. Results will be taken from the .cov file if present otherwise from the .lst file.
                    \item CorrelationMatrix - If the covariance step was successful. Results will be taken from the .cor file if present otherwise from the .lst file.
                    \item StandardError - Results will be taken from the .ext file if present otherwise the .lst file
                    \item RelativeStandardError - Results will be taken from the .ext file if present otherwise the .lst file. The RSE is calculated has the ratio between the standard error and the estimated value of the parameter.
                \end{itemize}
                \item Bootstrap
                \begin{itemize}
                    \item PercentilesCI - From a PsN bootstrap\_results.csv file.
                \end{itemize}
            \end{itemize}
            \item IndividualEstimates
            \begin{itemize}
                \item Estimates
                \begin{itemize}
                    \item Median - Calculated from the patab if created
                    \item Mean - Calculated from the patab if created
                \end{itemize}
                \item RandomEffects
                \begin{itemize}
                    \item EffectMedian - Calculated from the patab if ETAs are named correctly
                    \item EffectMean - Calculated from the patab if ETAs are named correctly
                \end{itemize}
            \end{itemize}
            \item Residuals - Taken from the sdtab if present
            \item Predictions - Taken from the sdtab if present
            \item Likelihood
                \begin{itemize}
                    \item Deviance - This is the NONMEM ofv value taken from the .lst file
                \end{itemize}
        \end{itemize}
    \item Simulation
        \begin{itemize}
            \item SimulatedProfiles - Taken from the first table that contains all of ID, TIME and DV
        \end{itemize}
\end{itemize}


\end{document}
