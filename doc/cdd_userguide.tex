\documentclass[a4wide,12pt]{article}
%\setlength{\marginparwidth}{0pt}%35
%\setlength{\marginparsep}{0pt}%?
%\setlength{\evensidemargin}{0pt}
%\setlength{\oddsidemargin}{0pt}
\usepackage{lmodern}
\usepackage[utf8]{inputenc}
\usepackage[T1]{fontenc}
\usepackage{textcomp}
\usepackage{verbatim}
\usepackage{enumitem}
\usepackage{longtable}
\usepackage{alltt}
\usepackage{ifthen}
% Reduce the size of the underscore
\usepackage{relsize}
\renewcommand{\_}{\textscale{.7}{\textunderscore}}

\newcommand{\guidetitle}[1]{
\title{#1\\ \vspace{2 mm} {\large PsN 4.1.1}}
\date{2014-02-10}
}

\newcommand{\doctitle}[1]{
\title{#1}
\date{2014-02-10}
}


\newenvironment{optionlist}{
\renewcommand{\arraystretch}{1.1}
\setlength{\leftmargini}{2.5cm}
\begin{description}
%\setlength{\itemsep}{0ex}
}
{\end{description}}

\newcommand{\optname}[1]{\item{{\bfseries\texttt-#1}\newline}}
\newcommand{\optdefault}[2]{\item{{\bfseries\texttt-#1}{\mbox{ = \it #2}}\newline}}

\newcommand{\nextopt}{}

\guidetitle{CDD user guide}{2017-10-12}

\begin{document}

\maketitle
\newcommand{\guidetoolname}{cdd}

\section{Introduction}
The Case Deletion Diagnostics tool is run using the command from the command line with a few mandatory arguments. 
CDD is run as a diagnostic after a model is regarded finished or at least mature enough to run validation tool on. 
The Case Deletions Diagnostics (CDD) algorithm is a tool primarily used to identify influential components of the dataset, 
usually individuals. The CDD works by identifying groups in the data set and creating one new data set for each member 
of the group, where that member has been removed. The model is run once with each new data set. The PsN implementation 
of the CDD can take any column as base for the grouping and all rows with the same value in that column will be considered 
a group as long as no individual contain multiple values in that column. One should take care that the grouping creates sensible 
individual records. PsN will renumber the ID column so that two individuals with the same ID will not end up next to each other.

Examples
\begin{verbatim}
cdd run89.mod -case_column=10
\end{verbatim}
This will perform a Case Deletion Diagnostic on the model specified in run89.mod based on the factors in column ten. If for example, 
column ten holds the ids of the seven centers included in the study, this command will create seven copies of the dataset, each with 
individuals included in one specific center deleted. Say that the centers are numbered 1 to 7. Then dataset 1 will have individuals from 
center 1 excluded, dataset 2 individuals from center 2 and so on.

\section{Input and options}

\subsection{Required input}
A NONMEM model file with successful termination is required on the command-line.
\begin{optionlist}
\optdefault{case\_column}{name|number}
Through this flag you set the column on which the case-deletion is done. You can either use the name of the column as specified in the 
\mbox{\$INPUT} record in the model file or you can use the column number in the \mbox{\$INPUT} record. Numbering starts with 1. 
Default is ID.
\end{optionlist}

\subsection {optional input}
\begin{optionlist}
\optdefault{bins}{N}
Sets the number of databins, or cdd datasets, to use. If the number of unique values, or factors, in the case column is higher than N then one or more factors will be deleted in each cdd dataset. Specifying N as higher than the number of factors will have no effect. N is then reset to the number of factors. Default value = The number of unique values in the case column. 
\nextopt
\optname{etas}
Use the phi file of the original model as initial estimates for the etas.
\nextopt
\optdefault{outside\_n\_sd\_check}{X}
Default 2. Mark the runs with Cook score - Covariance ratio outside X standard deviations of the principal component analysis (PCA). 
\nextopt
\optdefault{selection\_method}{random | consecutive}
Default consecutive. Specifies whether the factors selected for exclusion should be drawn randomly or consecutively from the datafile. 
\nextopt
\optname{update\_inits}
By default, the initial estimates of the cdd models will be set to the final estimates of the input model if final
estimates are available. If the user sets -no-update\_inits, the initial estimates of the cdd models will be
the same as the initial estimates set in the input model, even if final estimates from the input model are available.
\nextopt
\optname{xv}
Default false. Run the cross-validation step (-xv) or not (-no-xv). 
\nextopt
\end{optionlist}
			
		\subsection{Some important common PsN options}
		There are many options that govern how PsN manages NONMEM runs, and
		those options are common to all PsN programs that run NONMEM.
		For a complete list see common\_options.pdf, 
		or psn\_options -h on the commandline.
		\subsection{Optional input}
		\begin{optionlist}
\optname{h or -?}
Print the list of available options and exit. 
\nextopt
\optname{help}
With -help all programs will print a longer help message. 
If an option name is given as argument, help will be printed for this option. 
If no option is specified, help text for all options will be printed. 
\nextopt
\optdefault{directory}{'string'}
Default \guidetoolname\_dirN,
where N will start at 1 and
be increased by one each time you run the script. The directory option sets the directory in which PsN 
will run NONMEM and where PsN-generated output files will be stored. 
You do not have to create the directory,  it will be done for you. If you set
-directory to a the name of a directory that already exists, PsN will run in the existing directory.
\nextopt
\optdefault{seed}{'string'}
You can set your own random seed to make PsN runs reproducible.
The random seed is a string, so both -seed=12345 and -seed=JustinBieber are valid.
It is important to know that because of the way the Perl pseudo-random
number generator works, for two similar string seeds the random sequences may be identical. 
This is the case e.g. with the two different seeds 123 and 122. 
Setting the same seed guarantees the same sequence, but setting two slightly different 
seeds does not guarantee two different random sequences, that must be verified.
\nextopt
\optdefault{clean}{'integer'}
Default 1. The clean option can take four different values:  
\begin{description}
\item[0] Nothing is removed 
\item[1] NONMEM binary and intermediate files except INTER are removed, and files specified with option -extra\_files. 
\item[2] model and output files generated by PsN restarts are removed, and data files in the NM\_run directory, and (if option -nmqual is used) the xml-formatted NONMEM output. 
\item[3] All NM\_run directories are completely removed. If the PsN tool has created modelfit\_dir:s inside the main run directory, these  will also be removed. 
\end{description}
\nextopt
\optdefault{nm\_version}{'string'}
Default is 'default'. 
If you have more than one NONMEM version installed you can use option
-nm\_version to choose which one to use, as long as it is 
defined in the [nm\_versions] section in psn.conf, see psn\_configuration.pdf for details. 
You can check which versions are defined, without opening psn.conf, using the command
\begin{verbatim}
psn -nm_versions
\end{verbatim}
\nextopt
\optdefault{threads}{'integer'}
Default 1. Use the threads option to enable parallel execution of multiple models.
This option decides how many models PsN will run at the same time, and it is completely
independent of whether the individual models are run with serial NONMEM or parallel NONMEM.
If you want to run a single model in parallel you must use options -parafile and -nodes.
On a desktop computer it 
is recommended to not set -threads higher the number of CPUs in the system plus one. 
You can specify more threads, 
but it will probably not increase the performance. If you are running on a computer cluster, 
you should consult your 
system administrator to find out how many threads you can specify. 
\nextopt
\optname{version}
Prints the PsN version number of the tool, and then exit. 
\nextopt
\end{optionlist}

		\begin{optionlist}
		\optname{last\_est\_complete}
		is optional and only applies with NONMEM 7 and cdd option -xv. 
		See common\_options.pdf for details.
		\nextopt
		\end{optionlist}

\subsection{cdd rplots}
\newcommand{\rplotsconditions}{
If option -rplots is set $>=1$, a plot with Covariance ratios
vs Cook scores for each case, e.g. ID, will be generated. 
The default cdd rplots template 
requires no special R libraries.
If no pdf is generated,
see the .Rout file in the main run directory for error messages.}
PsN can automatically generate R plots to visualize results for \guidetoolname, using a default template found in the R-scripts subdirectory of the installation directory. The user can also create a custom template, see more details in the section Auto-generated R-plots from PsN in common\_options.pdf.

\rplotsconditions

\begin{optionlist}
\optdefault{rplots}{level}
-rplots<0 means R script is not generated\\ 
-rplots=0 (default) means R script is generated but not run\\ 
-rplots=1 means basic plots are generated\\													  
-rplots=2 means basic and extended plots are generated\\													  
\nextopt
\end{optionlist}

\subsubsection*{Troubleshooting}
If no .pdf was generated even if a template file is available and the appropriate options were set, check the .Rout-file in the main run directory for error messages. If no .Rout-file exists, then check that R is properly installed, and that either command 'R' is available or that R is configured in psn.conf.


\section{Output}

Below, $P_{orig}$ denotes the vector of estimated parameters (theta, omega and sigma)
on the original dataset, $p_{orig,j}$ is the $j$th element of $P_{orig}$,
$P_{i}$ denotes the vector of estimated parameters (theta, omega and sigma)
on the dataset with case $i$ excluded, $p_{i,j}$ is the $j$th element of $P_{i}$,
$cov(Y)$ is the covariance matrix of $Y$,
$se(y)$ is the standard error of $y$,
$N$ is the number of cases in the dataset,
and
$det(Y)$ is the determinant of $Y$.

\[
p_{(\cdot),j}=\frac{1}{N}\sum_{i=1}^{N}p_{i,j} 
\]
is the mean of $p_{i,j}$ over all case deleted datasets, 
\[
cov^{jackknife}_{j,k} = \frac{N-1}{N}\sum_{i=1}^{N}\left(p_{i,j}-p_{(\cdot),j}\right)\left(p_{i,k}-p_{(\cdot),k}\right)
\]
is the jackknife estimate of the covariance between $p_{orig,j}$ and $p_{orig,k}$, and this is used
to compute the full matrix
$cov^{jackknife}(P_{orig})$. If any case deleted dataset fails to give parameter estimates $P_{i}$ then
jackknife estimates of the covariances will not be computed.

\subsubsection*{Raw results file}
The standard PsN raw results file has a set of columns with cdd-specific results appended
for each case deleted dataset $i$:
\begin{description}
\item[cook.scores] $\sqrt{\left(P_{i}-P_{orig}\right)^Tcov(P_{orig})^{-1}\left(P_{i}-P_{orig}\right)}$\\
Only computed if $cov(P_{orig})$ is available.
\item[jackknife.cook.scores] $\sqrt{\left(P_{i}-P_{orig}\right)^Tcov^{jackknife}(P_{orig})^{-1}\left(P_{i}-P_{orig}\right)}$
\item[cov.ratios] \[\sqrt{\frac{det(cov(P_{i}))}{det(cov(P_{orig}))}}\]
Only computed if $cov(P_{orig})$ is available.
Set to 0 if $cov(P_{i})$ not available or numerically singular.
\item[outside.n.sd] 1 or 0, only relevant if both Cook scores and Cov ratios available.
1 if Cook score - Covariance ratio is outside 'outside\_n\_sd\_check' standard deviations of the PCA. 
\item[cdd.delta.ofv] This value is computed for each case deleted dataset i and
contains the difference $OFV_{cdd-i,orig} - OFV_{cdd-i,est}$,
where $OFV_{cdd-i,orig}$ is the sum of individual OFV (iOFV) for the individuals in case deleted dataset $i$ (cdd-i) 
from an estimation where all subjects are included and
$OFV_{cdd-i,est}$ is the OFV from an estimation where only the subjects in cdd-i are included. 
This value is always positive.
\item[cook.par.<param j>] Individual Cook score for parameter <param j>,
\[\frac{abs\left(p_{i,j}-p_{orig,j}\right)}{se(p_{orig,j})} \]
\item[jack.cook.par.<param j>] Individual jackknife approximated Cook score for parameter <param j>,
\[\frac{abs\left(p_{i,j}-p_{orig,j}\right)}{\sqrt{cov^{jackknife}_{j,j}}}\]
\end{description}

If the cross-validation was requested with option -xv, there are $N$ extra rows in the raw results file.
Each line is the result from evaluation of the model on the remainder data using parameter values from
estimation of the corresponding case-deleted dataset.

\subsubsection*{jackknife.cov}
The file jackknife.cov contains $cov^{jackknife}(P_{orig})$. It is comma separated and has the same number of columns
as the number of estimated (not FIXED) parameters. The column order is theta, omega, sigma, and the
column/row labels are the parameter labels from the model file, if any.

\subsubsection*{cdd\_results.csv}
The Diagnostics section of
cdd\_results.csv contains the same computed cook.scores, jackknife.cook.scores, cov.ratios and
outside.n.sd as the raw results file.

The section relative.changes.percent lists, for each case deleted dataset, the relative change in percent
for ofv, estimates and standard errors.

The section Jackknife.bias.estimate
has two rows. The jackknife bias of parameter $j$ is \[bias=\left(N-1\right)\left(p_{(\cdot),j}-p_{orig,j}\right) \]
and the relative bias in percent is \[100\cdot\frac{bias}{p_{orig,j}}\].

\section{Known bugs/issues}

If NONMEM 6 is used with the cdd and the S matrix is algorithmically singular (message in lst-file, checked also by sumo script) the Cook 
scores cannot be trusted. The cdd does not check for this error. 

\end{document}
